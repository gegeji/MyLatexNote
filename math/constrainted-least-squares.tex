\chapter{Constrained Least Squares}

\section{Karush–Kuhn–Tucker Conditions}

\begin{problem}
    $$
\begin{array}{l}
\min _{x}\left\{f(x)=2 x_{1}^{2}+x_{2}^{2}\right\} \\
\text { s.t. } \quad h(x)=x_{1}+x_{2}-1=0
\end{array}
$$

直接利用无约束优化问题求解: 
$$ \nabla f(x)=\left[\begin{array}{l}4 x_{1} \\ 2 x_{2}\end{array}\right]=0 \Rightarrow x=\left[\begin{array}{l}0 \\ 0\end{array}\right] $$

显然不满足约束条件 $ x_{1}+x_{2}-1=0+0-1 \neq 0 $, 不是优化问题的解。
\end{problem}

由约束条件可得 $ x_{1}=1-x_{2} $ , 代入目标函数则有

$$ f(x)=3 x_{2}^{2}-4 x_{2}+2 $$
即当 $ \hat{x}_{2}=\frac{2}{3} $ 时,目标函数值最小,并有 $ \hat{x}_{1}=1-\hat{x}_{2}=\frac{1}{3} $ 。

\begin{FigureCenter}{The Geometry of the problem}
    

\tikzset{every picture/.style={line width=0.75pt}} %set default line width to 0.75pt        

\begin{tikzpicture}[x=0.75pt,y=0.75pt,yscale=-1,xscale=1]
%uncomment if require: \path (0,333); %set diagram left start at 0, and has height of 333

%Straight Lines [id:da5016924761517159] 
\draw    (240.45,208.11) -- (460.06,208.46) ;
\draw [shift={(463.06,208.46)}, rotate = 180.09] [fill={rgb, 255:red, 0; green, 0; blue, 0 }  ][line width=0.08]  [draw opacity=0] (8.93,-4.29) -- (0,0) -- (8.93,4.29) -- cycle    ;
%Straight Lines [id:da4228098452592868] 
\draw    (240.45,208.11) -- (240.26,50.64) ;
\draw [shift={(240.26,47.64)}, rotate = 89.93] [fill={rgb, 255:red, 0; green, 0; blue, 0 }  ][line width=0.08]  [draw opacity=0] (8.93,-4.29) -- (0,0) -- (8.93,4.29) -- cycle    ;
%Shape: Rectangle [id:dp4169105767030461] 
\draw  [dash pattern={on 0.84pt off 2.51pt}] (240.45,126.72) -- (279.2,126.72) -- (279.2,208.11) -- (240.45,208.11) -- cycle ;
%Straight Lines [id:da38972494470736385] 
\draw [color={rgb, 255:red, 74; green, 144; blue, 226 }  ,draw opacity=1 ][line width=2.25]    (240.91,88.18) -- (357.45,208.11) ;
%Straight Lines [id:da6148568898829554] 
\draw [color={rgb, 255:red, 152; green, 195; blue, 245 }  ,draw opacity=1 ][line width=2.25]    (271.9,120.08) -- (388.44,240) ;
%Straight Lines [id:da5253071245967558] 
\draw [color={rgb, 255:red, 152; green, 195; blue, 245 }  ,draw opacity=1 ][line width=2.25]    (205.39,51.63) -- (321.93,171.55) ;
%Straight Lines [id:da011458677930073602] 
\draw [color={rgb, 255:red, 184; green, 84; blue, 80 }  ,draw opacity=1 ][fill={rgb, 255:red, 184; green, 84; blue, 80 }  ,fill opacity=1 ][line width=1.5]    (240.91,88.18) -- (260.95,67.55) ;
\draw [shift={(263.74,64.68)}, rotate = 134.18] [fill={rgb, 255:red, 184; green, 84; blue, 80 }  ,fill opacity=1 ][line width=0.08]  [draw opacity=0] (11.61,-5.58) -- (0,0) -- (11.61,5.58) -- cycle    ;
%Straight Lines [id:da8455771801891709] 
\draw [color={rgb, 255:red, 245; green, 166; blue, 35 }  ,draw opacity=1 ][fill={rgb, 255:red, 184; green, 84; blue, 80 }  ,fill opacity=1 ][line width=1.5]    (279.2,126.72) -- (299.24,106.1) ;
\draw [shift={(302.03,103.23)}, rotate = 134.18] [fill={rgb, 255:red, 245; green, 166; blue, 35 }  ,fill opacity=1 ][line width=0.08]  [draw opacity=0] (11.61,-5.58) -- (0,0) -- (11.61,5.58) -- cycle    ;
%Shape: Ellipse [id:dp8656582545427394] 
\draw  [color={rgb, 255:red, 245; green, 166; blue, 35 }  ,draw opacity=1 ][dash pattern={on 5.63pt off 4.5pt}][line width=1.5]  (163.6,204.04) .. controls (163.6,150.27) and (195.41,106.68) .. (234.64,106.68) .. controls (273.87,106.68) and (305.68,150.27) .. (305.68,204.04) .. controls (305.68,257.81) and (273.87,301.4) .. (234.64,301.4) .. controls (195.41,301.4) and (163.6,257.81) .. (163.6,204.04) -- cycle ;

% Text Node
\draw (225.88,81.79) node [anchor=north west][inner sep=0.75pt]  [xscale=0.75,yscale=0.75]  {$1$};
% Text Node
\draw (185.43,107.67) node [anchor=north west][inner sep=0.75pt]  [xscale=0.75,yscale=0.75]  {$\hat{x}_{1} =\frac{2}{3}$};
% Text Node
\draw (254.9,212.59) node [anchor=north west][inner sep=0.75pt]  [xscale=0.75,yscale=0.75]  {$\hat{x}_{2} =\frac{1}{3}$};
% Text Node
\draw (451.96,210.8) node [anchor=north west][inner sep=0.75pt]  [xscale=0.75,yscale=0.75]  {$x_{1}$};
% Text Node
\draw (230.45,29.38) node [anchor=north west][inner sep=0.75pt]  [xscale=0.75,yscale=0.75]  {$x_{2}$};
% Text Node
\draw (347.29,210.72) node [anchor=north west][inner sep=0.75pt]  [xscale=0.75,yscale=0.75]  {$1$};
% Text Node
\draw (33.88,198.25) node [anchor=north west][inner sep=0.75pt]  [color={rgb, 255:red, 185; green, 126; blue, 27 }  ,opacity=1 ,xscale=0.75,yscale=0.75]  {$f( x) =2x_{1}^{2} +x_{2}^{2}$};
% Text Node
\draw (329.11,238.71) node [anchor=north west][inner sep=0.75pt]  [color={rgb, 255:red, 74; green, 144; blue, 226 }  ,opacity=1 ,xscale=0.75,yscale=0.75]  {$h( x) =x_{1} +x_{2} -1=0$};
% Text Node
\draw (305,84.4) node [anchor=north west][inner sep=0.75pt]  [xscale=0.75,yscale=0.75]  {$\textcolor[rgb]{0.96,0.65,0.14}{\nabla f}\textcolor[rgb]{0.96,0.65,0.14}{(}\textcolor[rgb]{0.96,0.65,0.14}{\hat{x}}\textcolor[rgb]{0.96,0.65,0.14}{)} =\lambda \textcolor[rgb]{0.72,0.33,0.31}{\nabla h}\textcolor[rgb]{0.72,0.33,0.31}{(}\textcolor[rgb]{0.72,0.33,0.31}{\hat{x}}\textcolor[rgb]{0.72,0.33,0.31}{)}$};
% Text Node
\draw (255,45.4) node [anchor=north west][inner sep=0.75pt]  [color={rgb, 255:red, 184; green, 84; blue, 80 }  ,opacity=1 ,xscale=0.75,yscale=0.75]  {$\nabla h(\hat{x} )$};


\end{tikzpicture}

\end{FigureCenter}

惩罚未能满足约束条件, 引入拉格朗日函数 (Lagrange Function)
$$
L(x, \lambda)=f(x)-\lambda h(x)=2 x_{1}^{2}+x_{2}^{2}+\lambda\left(1-x_{1}-x_{2}\right)
$$

$$ \left.\begin{array}{l}\frac{\partial L}{\partial x_{1}}=4 x_{1}-\lambda=0 \\ \frac{\partial L}{\partial x_{2}}=2 x_{2}-\lambda=0 \\ \frac{\partial L}{\partial \lambda}=1-x_{1}-x_{2}=0\end{array}\right \} \Rightarrow \hat{x}_{1}=\frac{1}{3}, \hat{x}_{2}=\frac{2}{3}, \hat{\lambda}=\frac{4}{3} $$

\begin{definition}[Lagrange Functions]
    $$\begin{aligned}
        &\min _{x} / \max_{x} f(x) \\
\text{ s.t. } & h_{i}(x)=0, i \in I \triangleq\{1, \cdots, p\} \\
&g_{j}(x) \leq 0, j \in J \triangleq\{1, \cdots, q\}
    \end{aligned}$$

$ \lambda_{i} \in \mathbb{R}, i \in \mathrm{I}, u_{j} \in \mathbb{R}^{+}, j \in J $称为拉格朗日乘子(Lagrange Multipliers) 。

引入拉格朗日函数 $$ L(x, \lambda, u)=f(x)-\sum_{i \in I} \lambda_{i} h_{i}(x)-\sum_{j \in J} u_{j} g_{j}(x),  \lambda=\left[\begin{array}{c}\lambda_{1} \\ \vdots \\ \lambda_{p}\end{array}\right], u=\left[\begin{array}{c}u_{1} \\ \vdots \\ u_{q}\end{array}\right] $$

对拉格朗日函数进行求导

$$ \nabla_{x} L(x, \lambda, u)=\nabla_{x} f(x)-\sum_{i \in I} \lambda_{i} \nabla_{x} h_{i}(x)-\sum_{j \in J} u_{j} \nabla_{x} g_{j}(x)=0 $$
\end{definition}

\begin{theorem}[Karush-Kuhn-Tucker Conditions] 
    对于优化问题

    $$\begin{aligned}
        &\min _{x} / \max f(x) \\
\text{ s.t. } & h_{i}(x)=0, i \in I \triangleq\{1, \cdots, p\} \\
&g_{j}(x) \leq 0, j \in J \triangleq\{1, \cdots, q\}
    \end{aligned}$$
    
    KKT条件包括:
\begin{itemize}
    \item $ h_{i}(x)=0, i \in I $
    \item $ \lambda_{i} h_{i}(x)=0, i \in I $
    \item $ g_{j}(x) \leq 0, j \in J $
    \item $ u_{j} g_{j}(x)=0, j \in J $
    \item $ u_{j} \geq 0, j \in J $
\end{itemize}

\end{theorem}


\subsection{An Example for Karush-Kuhn-Tucker Conditions}


    % todo (2021-11-19 09:08): figure

\begin{problem}
    求解
$$
\begin{aligned}
    &\max _{x}\left\{f(x)=20 x_{1}+10 x_{2}\right\} \\
  \text{ s.t. }  &   g_{1}(x)=x_{1}^{2}+x_{2}^{2} \leq 1 \\
  & g_{2}(x)=x_{1}+2 x_{2} \leq 2 \\
  &  g_{3}(x)=-x_{1} \leq 0 \\
    & g_{4}(x)=-x_{2} \leq 0 
\end{aligned}
$$
\end{problem}

对拉格朗日函数求导
$$ \nabla_{x} L(x, u)=\nabla_{x} f(x)-u_{1} \nabla_{x} g_{1}(x)-u_{2} \nabla_{x} g_{2}(x)-u_{3} \nabla_{x} g_{3}(x)-u_{4} \nabla_{x} g_{4}(x)=0, u_{j} \geq 0 $$

计算各个部分,可得
$$ \nabla_{x} f(x)=\left[\begin{array}{c}20 \\ 10\end{array}\right], \nabla_{x} g_{1}(x)=\left[\begin{array}{c}2 x_{1} \\ 2 x_{2}\end{array}\right], \nabla_{x} g_{2}(x)=\left[\begin{array}{l}1 \\ 2\end{array}\right], \nabla_{x} g_{3}(x)=\left[\begin{array}{l}-1 \\ 0\end{array}\right], \nabla_{x} g_{4}(x)=\left[\begin{array}{l}0 \\ -1\end{array}\right] . $$

检测边界条件,可知
$$ \nabla_{x} f(x)=u_{1} \nabla_{x} g_{1}(x), \nabla_{x} f(x) \neq u_{2} \nabla_{x} g_{2}(x), \nabla_{x} f(x) \neq u_{3} \nabla_{x} g_{3}(x), \nabla_{x} f(x) \neq u_{4} \nabla_{x} g_{4}(x) $$

即$\nabla_{x} f(x)$只可能是$u_{1} \nabla_{x} g_{1}(x)$的线性组合。

所以
$$ u_{2}=u_{3}=u_{4}=0, \quad u_{1} \neq 0 $$

即

$$
\begin{aligned}
    \nabla_{x} f(x)&=u_{1} \nabla_{x} g_{1}(x)\\
\left[\begin{array}{c}
20 \\
10
\end{array}\right]&=u_{1}\left[\begin{array}{c}
2 x_{1} \\
2 x_{2}
\end{array}\right]\\
x_{1}&=2 x_{2} 
\end{aligned}
$$

代入

$$g_{1}(x)=x_{1}^{2}+x_{2}^{2}-1=5 x_{2}^{2}-1=0$$


由于 $ x_{2} \geq 0 $, 可得 $ \left[\begin{array}{l}x_{1} \\ x_{2}\end{array}\right]=\frac{\sqrt{5}}{5}\left[\begin{array}{l}2 \\ 1\end{array}\right], u_{1}=5 \sqrt{5} $.

\section{优化问题:最小范数优化问题}

\begin{problem}

$$\begin{aligned}
    & \min _{x}\|x\|_{2}^{2}\\
   \text{s.t.} &C x=d 
\end{aligned}$$

矩阵C $ \in \mathbb{R}^{p \times n} $, 向量 $ d \in \mathbb{R}^{p} $
\end{problem}


在大多数应用中 $ p<n,  C x=d $ 是一个\textbf{欠定方程}。这时,$C x=d$表示为一条直线。它的几何意义是在直线 $ C x=d $ 中,寻找范数最小的解。

假如$p = n$,$C x=d$表示一个点。

% todo (2021-11-24 18:40): figure



假设矩阵$C$行向量线性无关时,有:
\begin{itemize}
    \item 对任意一个 $ \mathrm{d}, C x=d $ 至少有一个解;
    \item 矩阵$C$为宽的或者方的 $ (p \leq n) $;
    \item 当 $ p<n $,有无穷多个解。
\end{itemize}

\subsection{最小范数优化问题的例子}

矩阵 $ C \in \mathbb{R}^{2 \times 10} $, 向量 $ d \in \mathbb{R}^{2} $ :

$$Cx=d$$

$$\displaystyle \underbrace{\left[\begin{array}{ c c c c c }
    19/2 & 17/2 & 15/2 & \cdots  & 1/2\\
    1 & 1 & 1 & \cdots  & 1
    \end{array}\right]}_{C} x=\underbrace{\left[\begin{array}{ l }
    1\\
    0
    \end{array}\right]}_{d}$$

这个方程只有一组含有两个非0元素的解。

$$
x=\left[\begin{array}{c}
1 \\
-1 \\
0 \\
\vdots \\
0
\end{array}\right], \quad x=\left[\begin{array}{c}
0 \\
1 \\
-1 \\
\vdots \\
0
\end{array}\right], \cdots
$$

\section{优化问题:点到线的最短距离}

\begin{problem}
    给定点 $ a \neq 0 $ ,点到线最短距离问题:
$$
\begin{array}{ll}
\min _{x} & \|x-a\|_{2}^{2} \\
\text { s. t. } & C x=d
\end{array}
$$

\end{problem}

令 $ y=x-a $, 则点到线的最短距离问题, 可等价为最小范数问题:

\begin{problem}
    $$
\begin{array}{ll}
\min _{y} & \|y\|_{2}^{2} \\
\text { s. t. } & C y=d-C a
\end{array}
$$
\end{problem}

则 $ x=y+a $ 是点到线最短距离问题的解。

\section{最小范数优化问题}

\begin{problem}
    $$
\begin{array}{l}
\min _{x} \frac{1}{2}\|x\|_{2}^{2} \\
\text { s.t. } C x=d
\end{array}
$$
\end{problem}

引入拉格朗日函数
$$
L(x, \lambda)=\frac{1}{2}\|x\|_{2}^{2}-\lambda^{T}(C x-d)
$$

对拉格朗日函数求导
$$
\nabla_{x} L(x, \lambda)=x-C^{T} \lambda=0 \Rightarrow x=C^{T} \lambda
$$

由矩阵$C$行线性无关可得
$$
C x=C C^{T} \lambda=d, \lambda=\left(C C^{T}\right)^{-1} d
$$

则有
$$
\hat{x}=C^{T} \lambda=C^{T}\left(C C^{T}\right)^{-1} d=C^{\dagger} d
$$

以上得到的是$\hat{x}=C^{\dagger} d$是问题最优解的必要条件。

\begin{proposition}
    $\hat{x}=C^{\dagger} d$是问题最优解。
\end{proposition}

\begin{proof}
    1.首先证明解 $ \hat{x} $ 满足等式 $ \hat{x}=C^{T} \lambda=C^{T}\left(C C^{T}\right)^{-1} d=C^{\dagger} d $.

$$
C \hat{x}=C C^{T}\left(C C^{T}\right)^{-1} d=d
$$

2. 在 $ C x=d, x \neq \hat{x} $ 的情况下

$$
\begin{aligned}
\hat{x}^{T}(x-\hat{x}) &=d^{T}\left(C C^{T}\right)^{-1} C(x-\hat{x}) \\
&=d^{T}\left(C C^{T}\right)^{-1}(C x-C \hat{x}) \\
&=d^{T}\left(C C^{T}\right)^{-1}(d-d) \\
&=0
\end{aligned}
$$

3.在 $ C x=d, x \neq \hat{x} $ 的情况下,证明 $ \|x\|_{2}^{2}>\|\hat{x}\|_{2}^{2} $.

$$
\begin{aligned}
\|x\|_{2}^{2} &=\|\hat{x}+x-\hat{x}\|_{2}^{2} \\
&=\|\hat{x}\|_{2}^{2}+2 \hat{x}^{T}(x-\hat{x})+\|x-\hat{x}\|_{2}^{2} \\
&=\|\hat{x}\|_{2}^{2}+\|x-\hat{x}\|_{2}^{2} \\
&>\|\hat{x}\|_{2}^{2}
\end{aligned}
$$
\end{proof}

\section{QR分解求解最小范数优化问题}

对矩阵 $ C^{T} \in \mathbb{R}^{n \times p} $ 进行 $ Q R $ 分解, $ C^{T}=Q R $
$$
\begin{aligned}
\hat{x} &=C^{T}\left(C C^{T}\right)^{-1} d \\
&=Q R\left(R^{T} Q^{T} Q R\right)^{-1} d \\
&=Q R\left(R^{T} R\right)^{-1} d \\
&=Q R^{-T} d
\end{aligned}
$$

\subsection{QR分解求解最小范数优化问题的复杂度}

总复杂度: $ \approx 2 n p^{2} $ flops

\begin{itemize}
    \item 矩阵 $ C^{T}=Q R $ 分解, $ C^{T}=Q R\left(2 n p^{2}\right. $ flops $ ) $
    \item 回代法求解 $ R^{T} z=d\left(p^{2}\right. $ flops $ ) $
    \item 计算 $ \hat{x}=Q z(2 n p $ flops $ ) $
\end{itemize}



\subsection{QR分解求解最小范数优化问题的例子}

\begin{example}
    $$
C=\left[\begin{array}{cccc}
1 & -1 & 1 & 1 \\
1 & 0 & 1 / 2 & 1 / 2
\end{array}\right], \quad d=\left[\begin{array}{l}
0 \\
1
\end{array}\right]
$$
对矩阵 $ C^{T} $ 进行 $ Q R $ 分解, $ C^{T}=Q R $
$$
\left[\begin{array}{cc}
1 & 1 \\
-1 & 0 \\
1 & 1 / 2 \\
1 & 1 / 2
\end{array}\right]=\left[\begin{array}{cc}
1 / 2 & 1 / \sqrt{2} \\
-1 / 2 & 1 / \sqrt{2} \\
1 / 2 & 0 \\
1 / 2 & 0
\end{array}\right]\left[\begin{array}{cc}
2 & 1 \\
0 & 1 / \sqrt{2}
\end{array}\right]
$$
通过回代法求解 $ R^{T} Z=d $
$$
\left[\begin{array}{cc}
2 & 0 \\
1 & 1 / \sqrt{2}
\end{array}\right]\left[\begin{array}{l}
z_{1} \\
z_{2}
\end{array}\right]=\left[\begin{array}{l}
0 \\
1
\end{array}\right]
$$

解得$
 z_{1}=0, z_{2}=\sqrt{2}
$。

可得 $ \hat{x}=Q z=(1,1,0,0) $
\end{example}


\section{优化问题:分段多项式拟合问题}

\begin{problem}
    假设样本点 $ \left(x_{1}, y_{1}\right), \ldots,\left(x_{N}, y_{N}\right) \in \mathbb{R}^{2} $ ,有 $ x_{1}, \ldots, x_{M} \leq a $, $ x_{M+1}, \ldots, x_{N}>a $。

    设$ \mathrm{d} $ 阶多项式 

    $$
    \begin{aligned}
        f(x)&=\theta_{1}+\theta_{2} x+\cdots+\theta_{d} x^{d-1}\\
        g(x)&=\theta_{d+1}+\theta_{d+2} x \ldots+\theta_{2 d} x^{d-1}
    \end{aligned}
    $$

    两个多项式 $ f(x), g(x) $ 对于样本点 $ \left(x_{1}, y_{1}\right), \ldots,\left(x_{N}, y_{N}\right) $ 进行拟合

    $$
    \begin{array}{l}
    f\left(x_{i}\right) \approx y_{i}, x_{i} \leq a \\
    g\left(x_{i}\right) \approx y_{i}, x_{i}>a
    \end{array}
    $$

    拟合要求:函数值和导数值必须在分段位置 $ a $ 连续
    $$
    f(a)=g(a), f^{\prime}(a)=g^{\prime}(a)
    $$
\end{problem}

对于$
\begin{array}{l}
f\left(x_{i}\right) \approx y_{i}, x_{i} \leq a \\
g\left(x_{i}\right) \approx y_{i}, x_{i}>a
\end{array}
$,可以构造矩阵

$$ A=\left[\begin{array}{cccccccc}1 & x_{1} & \cdots & x_{1}^{d-1} & 0 & 0 & \cdots & 0 \\ \vdots & \vdots & & \vdots & \vdots & \vdots & & \vdots \\ 1 & x_{M} & \cdots & x_{M}^{d-1} & 0 & 0 & \cdots & 0 \\ 0 & 0 & \cdots & 0 & 1 & x_{M+1} & \cdots & x_{M+1}^{d-1} \\ \vdots & \vdots & & \vdots & \vdots & \vdots & & \vdots \\ 0 & 0 & \cdots & 0 & 1 & x_{N} & \cdots & x_{N}^{d-1}\end{array}\right], \theta=\left[\begin{array}{c}\theta_{1} \\ \vdots \\ \theta_{d} \\ \theta_{d+1} \\ \vdots \\ \theta_{2 d}\end{array}\right], b=\left[\begin{array}{c}y_{1} \\ \vdots \\ y_{M} \\ y_{M+1} \\ \vdots \\ y_{N}\end{array}\right] $$

则$f(a)=g(a)$可以转化为

$$ A \theta \approx b $$

对于$f(a)=g(a), f^{\prime}(a)=g^{\prime}(a)$,可以构造矩阵

$$ C=\left[\begin{array}{cccccccc}1 & a & \cdots & a^{d-1} & -1 & -a & \cdots & -a^{d-1} \\ 0 & 1 & \cdots & (d-1) a^{d-2} & 0 & -1 & \cdots & -(d-1) a^{d-2}\end{array}\right], d=\left[\begin{array}{l}0 \\ 0\end{array}\right] $$

$$ C \theta=d $$

\begin{problem}
     $$\begin{array}{ll}\min _{\theta} & \sum_{i=1}^{M}\left(f\left(x_{i}\right)-y_{i}\right)^{2}+\sum_{i=M+1}^{N}\left(g\left(x_{i}\right)-y_{i}\right)^{2} \\ \text { s.t. } & f(a)=g(a), f^{\prime}(a)=g^{\prime}(a)\end{array}$$
\end{problem}


\begin{problem}
    $$\begin{aligned}
        &\min _{\theta}\|A \theta-b\|_{2}^{2}\\
       \text{s.t.} &C \theta=d
    \end{aligned}$$

\end{problem}

\section{先验假设}

\begin{proposition}
    堆叠矩阵列线性无关
$$
\left[\begin{array}{l}
A \\
C
\end{array}\right] \in \mathbb{R}^{(m+p) \times n}
$$
\end{proposition}

\begin{proposition}
    矩阵 $ C \in \mathbb{R}^{p \times n} $行线性无关

\end{proposition}

假设 1 是一个比 $ A $ 可右逆更弱的条件。 

$ p \leq n \leq m+p $

\section{优化问题:最小二乘法约束KKT条件}

$ \min _{x} \frac{1}{2}\|A x-b\|_{2}^{2} $
s.t. $ \quad C x=d $

引入拉格朗日函数:
$$
L(x, z)=\frac{1}{2}\|A x-b\|_{2}^{2}-z^{T}(d-C x), z \in \mathbb{R}^{p}
$$
对拉格朗日函数求导:
$$
\begin{array}{l}
\nabla_{x} L(x, z)=A^{T}(A x-b)+C^{T} z=0 \\
\nabla_{z} L(x, z)=C x-d=0
\end{array}
$$

$$ \left[\begin{array}{cc}A^{T} A & C^{T} \\ \mathrm{C} & 0\end{array}\right]\left[\begin{array}{l}x \\ z\end{array}\right]=\left[\begin{array}{l}A^{T} b \\ d\end{array}\right] $$

\section{KKT最优条件}

\begin{theorem}[优化条件Karush-Kuhn-Tucker(KKT)等式]
    令 $ \hat{x} $ 是约束优化问题的解,则有
$$
\left[\begin{array}{cc}
A^{T} A & c^{T} \\
& 0
\end{array}\right]\left[\begin{array}{l}
\hat{x} \\
z
\end{array}\right]=\left[\begin{array}{l}
A^{T} b \\
d
\end{array}\right],\left[\begin{array}{l}
\hat{x} \\
z
\end{array}\right] \in \mathbb{R}^{n+p}
$$
\end{theorem}


特殊情况
- 最小二乘法问题:当 $ p=0 $ 时,即为正规方程 $ A^{T} A \hat{x}=A^{T} b $
- 最小范数问题: 当 $ A=I, b=0 $ 时,可以推导得到 $ C \hat{x}=b, \hat{x}+C^{T} z=0 $

\begin{proof}
    假设 $ x $ 满足 $ C x=d,(\hat{x}, z) $ 满足КKT等式.

    $$\begin{aligned}
        \| Ax-b\| _{2}^{2} & =\| A(x-\hat{x} )+A\hat{x} -b\| _{2}^{2}\\
         & =\| A(x-\hat{x} )\| _{2}^{2} +\| A\hat{x} -b\| _{2}^{2} +2(x-\hat{x} )^{T} A^{T} (A\hat{x} -b)\\
         & = \| A(x-\hat{x} )\| _{2}^{2} +\| A\hat{x} -b\| _{2}^{2} -2(\textcolor[rgb]{0.96,0.65,0.14}{x-\hat{x}} )^{T}\textcolor[rgb]{0.72,0.33,0.31}{C^{T} z\ }\textcolor[rgb]{0,0,0}{(}\textcolor[rgb]{0.96,0.65,0.14}{A^{T} A\hat{x} +C^{T} z=A^{T} b}\textcolor[rgb]{0,0,0}{)(}\textcolor[rgb]{0.72,0.33,0.31}{Cx=C\hat{x} =d}\textcolor[rgb]{0,0,0}{)}\\
         & =\| A(x-\hat{x} )\| _{2}^{2} +\| A\hat{x} -b\| _{2}^{2}\\
         & \geq \| A\hat{x} -b\| _{2}^{2}
        \end{aligned}$$

        $ \hat{x} $ 是唯一性,因为假设矩阵A列线性无关, C行线性无关,即
$$
\begin{aligned}
A^{T} A(\hat{x}-x) &=0 \Rightarrow x=\hat{x} \\
C^{T}(\hat{z}-z) &=0 \Rightarrow \hat{z}=z \\
A^{T} A \hat{x}+C^{T} z &=A^{T} b
\end{aligned}
$$
\end{proof}

\begin{theorem}
    如果矩阵A列线性无关, C行线性无关,则矩阵
$$
\left[\begin{array}{cc}
A^{T} A & C^{T} \\
\mathrm{C} & 0
\end{array}\right]
$$
为非奇异矩阵
\end{theorem}

\begin{proof}
    $$ \begin{aligned}\left[\begin{array}{cc}A^{T} A & C^{T} \\ \mathrm{C} & 0\end{array}\right]\left[\begin{array}{c}x \\ z\end{array}\right]=0 & \Rightarrow x^{T}\left(A^{T} A x+C^{T} z\right)=0, C x=0 \\ & \Rightarrow\|A x\|_{2}^{2}+(C x)^{T} z=\|A x\|_{2}^{2}=0, C x=0 \\ & \Rightarrow A x=0, C x=0 \\ & \Rightarrow x=0 \quad \text { A列线性无关 } \end{aligned} $$

    由于 $ A^{T} \mathrm{~A} x+C^{T} \mathrm{z}=0 $, 则当 $ x=0 $ 时,有 $ \mathrm{z}=0 $ 。 C行线性无关
\end{proof}

如果矩阵A列线性无关和 $ C $ 行线性无关不同时成立,则矩阵
$$
\left[\begin{array}{cc}
A^{T} A & C^{T} \\
\mathrm{C} & 0
\end{array}\right]
$$
为奇异矩阵
如果 $ \mathrm{A} $ 列线性相关,则存在 $ x \neq 0 $, 使得 $ A x=0 $, 则
$$
\left[\begin{array}{cc}
A^{T} A & C^{T} \\
\mathrm{C} & 0
\end{array}\right]\left[\begin{array}{l}
x \\
0
\end{array}\right]=0
$$
如果 $ C $ 行线性相关,则存在 $ z \neq 0 $, 使得 $ C^{T} Z=0 $, 则
$$
\left[\begin{array}{cc}
A^{T} A & C^{T} \\
\mathrm{C} & 0
\end{array}\right]\left[\begin{array}{l}
0 \\
\mathrm{z}
\end{array}\right]=0
$$
因此该矩阵为奇异矩阵

\section{LU分解求解}

$$
\left[\begin{array}{cc}
A^{T} A & C^{T} \\
C & 0
\end{array}\right]\left[\begin{array}{l}
x \\
z
\end{array}\right]=\left[\begin{array}{c}
A^{T} b \\
d
\end{array}\right]
$$
算法:
计算 $ H=A^{T} A\left(2 m n^{2}\right. $ flops $ ) $ 。
计算 $ c=A^{T} b(2 \mathrm{mn} $ flops $ ) $ 。
用LU分解法求解下列线性方程 $ \left((2 / 3)(p+n)^{3}\right. $ flops $ ) $ :
$$
\left[\begin{array}{cc}
H & C^{T} \\
C & 0
\end{array}\right]\left[\begin{array}{l}
x \\
z
\end{array}\right]=\left[\begin{array}{l}
c \\
d
\end{array}\right]
$$

复杂度为: $ 2 m n^{2}+(2 / 3)(p+n)^{3} $ flops 。

\section{QR分解求解}

由于 $ \hat{x} $ 满足 $ C \hat{x}=d $ ,则有 $ C^{\mathrm{T}} C \hat{x}=C^{\mathrm{T}} d $ ,可得
$$
L(x, z)=\frac{1}{2}\|A x-b\|_{2}^{2}-z^{T}(d-C x)
$$
KKT条件:
$$
\begin{aligned}
\nabla_{x} L(x, z) &=A^{T}(A \hat{x}-b)+C^{T} z+C^{T} C \hat{x}-C^{T} d=0 \\
& \Rightarrow\left(A^{T} A+C^{T} C\right) \hat{x}+C^{T}(z-d)=A^{T} b \\
\nabla_{z} L(x, z) &=C \hat{x}-d=0
\end{aligned}
$$

令 $ w=z-d $ , KKT条件写成:
$$
\left[\begin{array}{cc}
A^{T} A+C^{T} C & C^{T} \\
C & 0
\end{array}\right]\left[\begin{array}{l}
\hat{x} \\
w
\end{array}\right]=\left[\begin{array}{c}
A^{T} b \\
d
\end{array}\right]
$$
假设 1 保证了 $ A^{T} A+C^{T} C $ 是非奇异的,即存在以下 $ Q R $ 因子分解:
$ \left[\begin{array}{l}A \\ C\end{array}\right] $ 列向量无关,则 $ \left[\begin{array}{l}A \\ C\end{array}\right]=Q R=\left[\begin{array}{l}Q_{1} \\ Q_{2}\end{array}\right] R=\left[\begin{array}{l}Q_{1} R \\ Q_{2} R\end{array}\right] $

代入QR分解,可得:
$$
\left[\begin{array}{cc}
R^{T} R & R^{T} Q_{2}^{T} \\
Q_{2} R & 0
\end{array}\right]\left[\begin{array}{c}
\hat{x} \\
w
\end{array}\right]=\left[\begin{array}{c}
R^{T} Q_{1}^{T} b \\
d
\end{array}\right]
$$
将第一个方程两边乘 $ R^{-T} $ 和并令变量 $ y=R \hat{x} $ 相乘,可得:
$$
\left[\begin{array}{cc}
I & Q_{2}^{T} \\
Q_{2} & 0
\end{array}\right]\left[\begin{array}{l}
y \\
w
\end{array}\right]=\left[\begin{array}{c}
Q_{1}^{T} b \\
d
\end{array}\right]
$$

矩阵 $ C=Q_{2} R \Rightarrow Q_{2}=C R^{-1}, Q_{2} $ 具有行线性无关:
$$
Q_{2}^{T} u=R^{-T} C^{T} u=0 \quad \Rightarrow \quad C^{T} u=0 \quad \Rightarrow \quad u=0
$$
因为 $ C $ 行线性无关的(假设2)。

利用 $ Q_{2}^{T} $ 的 $ Q R $ 分解来求解:
$$
\left[\begin{array}{cc}
I & Q_{2}^{T} \\
Q_{2} & 0
\end{array}\right]\left[\begin{array}{l}
y \\
w
\end{array}\right]=\left[\begin{array}{c}
Q_{1}^{T} b \\
d
\end{array}\right]
$$
方程第1行可得 $ y=Q_{1}^{T} b-Q_{2}^{T} w $ ,并代入第二行:
$$
Q_{2} y=d \Rightarrow Q_{2} Q_{2}^{T} w=Q_{2} Q_{1}^{T} b-d
$$
用 $ \mathrm{QR} $ 分解 $ Q_{2}^{T}=\tilde{Q} \tilde{R} $ 来解这个关于 $ w $ 的方程:
$$
\tilde{R}^{T} \tilde{R} w=\tilde{R}^{T} \tilde{Q}^{T} Q_{1}^{T} b-d
$$

上式可以简化为:
$$
\tilde{R} w=\tilde{Q}^{T} Q_{1}^{T} b-\tilde{R}^{-T} d
$$

$$ \left[\begin{array}{cc}A^{T} A+C^{T} C & C^{T} \\ C & 0\end{array}\right]\left[\begin{array}{c}\hat{x} \\ w\end{array}\right]=\left[\begin{array}{c}A^{T} b \\ d\end{array}\right] \quad \tilde{R} w=\tilde{Q}^{T} Q_{1}^{T} b-\tilde{R}^{-T} d $$

算法过程:
1. 计算两个QR分解:
$$
\left[\begin{array}{l}
{ }_{C}^{A} \\
C
\end{array}\right]=\left[\begin{array}{l}
Q_{1} \\
Q_{2}
\end{array}\right] R, Q_{2}^{T}=\tilde{Q} \tilde{R}
$$
2. 用前代法求解 $ \tilde{R}^{T} u=d $ ,计算 $ c=\tilde{Q}^{T} Q_{1}^{T} b-u $ 。
3. 用回代法求解 $ \tilde{R} w=c $ ,计算 $ y=Q_{1}^{T} b-Q_{2}^{T} w $ 。
4. 用回代法计算 $ R \hat{x}=y_{\circ} $
复杂度: $ Q R $ 分解有 $ 2(p+m) n^{2}+2 n p^{2} $ 次flops。

\section{复杂度QR vs LU}

假设 $ p<n $ :
■ LU复杂度: $ 2 m n^{2}+(2 / 3)(p+n)^{3}<2 m n^{2}+(16 / 3) n^{3} $
QR复杂度: $ 2(p+m) n^{2}+2 n p^{2}<2 m n^{2}+4 n^{3} $
稳定性: $ \mathrm{QR} $ 分解避免直接计算 $ A^{T} \mathrm{~A}_{\circ} $

\section[Supplement Material: Karush-Kuhn-Tucker (KKT)条件]{Supplement Material: Karush-Kuhn-Tucker (KKT)条件\footnote{Cited from \url{https://zhuanlan.zhihu.com/p/38163970}.}}

\subsection{等式约束优化问题}

\begin{problem}[等式约束优化问题]
    给定一个目标函数 $ f: \mathbb{R}^{n} \rightarrow \mathbb{R} $, 我们希望找到 $ \mathbf{x} \in \mathbb{R}^{n} $, 在满足约束条件 $ g(\mathbf{x})=0 $ 的前提下, 使得 $ f(\mathbf{x}) $ 有最小值

    $$
\begin{array}{ll}
\min & f(\mathbf{x}) \\
\text { s.t. } & g(\mathbf{x})=0
\end{array}
$$
\end{problem}

为方便分析, 假设 $ f $ 与 $ g $ 是连续可导函数。 Lagrange乘数法是等式约束优化问题的典型解法。定义

\begin{definition}[Lagrangian函数]
    $$
L(\mathbf{x}, \lambda)=f(\mathbf{x})+\lambda g(\mathbf{x})
$$
\end{definition}

其中 $ \lambda $ 称为\term{Lagrange乘数}。 

\begin{theorem}
    Lagrange乘数法将原本的约束优化问题转换成等价的无约束优化问题
$$
\min _{\mathbf{x}, \lambda} L(\mathbf{x}, \lambda)
$$
\end{theorem}

\begin{theorem}[拉格朗日乘子法最优解必要条件]
    计算 $ L $ 对 $ \mathbf{x} $ 与 $ \lambda $ 的偏导数并设为零,可得最优解的必要条件:
$$
\begin{array}{l}
\nabla_{\mathbf{x}} L=\dfrac{\partial L}{\partial \mathbf{x}}=\nabla f+\lambda \nabla g=\mathbf{0} \\
\\
\nabla_{\lambda} L=\dfrac{\partial L}{\partial \lambda}=g(\mathbf{x})=0
\end{array}
$$

其中第一式为\term{定常方程式(stationary equation)}, 第二式为\term{约束条件}。
\end{theorem}


解开上面 $ n+1 $ 个方程式可 得 $ L(\mathbf{x}, \lambda) $ 的驻点(stationary point) $ \mathbf{x}^{\star} $ 以及 $ \lambda $ 的值(正负数皆可能)。

\subsection{不等式约束优化问题}

接下来我们将约束等式 $ g(\mathbf{x})=0 $ 推广为不等式 $ g(\mathbf{x}) \leq 0 $ 。考虑这个问题

\begin{problem}[不等式约束优化问题]

$$
\begin{array}{ll}
\min & f(\mathbf{x}) \\
\text { s.t. } & g(\mathbf{x}) \leq 0
\end{array}
$$

约束不等式 $ g(\mathbf{x}) \leq 0 $ 称为\term{原始可行性(primal feasibility)}, 据此我们定义\term{可行域(feasible region)} $ K=\{ \mathbf{x} \in \mathbb{R}^{n} \mid g(\mathbf{x}) \leq 0 \}$。
\end{problem}

假设 $ \mathbf{x}^{\star} $ 为满足约束条件的最佳解, 分开两种情况讨论:

\begin{itemize}
    \item $ g\left(\mathbf{x}^{*}\right)<0 $, 最佳解位于 $ K $ 的内部, 称为内部解(interior solution), 这时约束条件是\term{无效的 (inactive)};
    \item $ g\left(\mathbf{x}^{*}\right)=0 $, 最佳解落在 $ K $ 的边界, 称为边界解(boundary solution), 此时约束条件是\term{有效的 (active)}.
\end{itemize}

这两种情况的最佳解具有不同的必要条件。

\begin{itemize}
    \item 内部解:在约束条件无效的情形下, $ g(\mathbf{x}) $ 不起作用, 约束优化问题退化为无约束优化问题, 因此驻点 $ \mathbf{x}^{\star} $ 满足 $ \nabla f=\mathbf{0} $ 且 $ \lambda=0 $ 。
    \item 边界解:在约束条件有效的情形下, 约束不等式变成等式 $g(\mathbf{x})=0$, 这与前述Lagrange乘数法的情况相同。
\end{itemize}

对于边界解,我们可以证明

\begin{theorem}
    驻点 $\mathbf{x}^{\star}$ 发生于 $\nabla f \in \operatorname{span} \nabla g$。
\end{theorem}
换句话说, 

\begin{corollary}
    存在 $\lambda$ 使得 $\nabla f=-\lambda \nabla g$。

    注意这里 $\lambda$ 的正负号是有其意义的。
\end{corollary}

因为我们希望最小化 $f$, 梯度 $\nabla f$ (函数 $f$ 在 点 $\mathbf{x}$ 的最陡上升方向)应该指向可行域 $K$ 的内部(因为最优解最小值是在边界取得的), 但 $\nabla g$ 指向 $K$ 的外部(即 $g(\mathbf{x})>0$ 的区域, 因为你的约束是小于等于0), 因此 $\lambda \geq 0$, 称为\term{对偶可行性(dual feasibility)。}

因此, 不论是内部解或边界解, $\lambda g(\mathbf{x})=0$ 恒成立, 称为\term{互补松弛性(complementary slackness)}。

整合上述两种情况, 

\begin{theorem}[Karush-Kuhn-Tucker (KKT)条件]
    最佳解的必要条件包括:Lagrangian函数 $L(\mathbf{x}, \lambda)$ 的定常方程式、 原始可行性、对偶可行性,以及互补松弛性:
$$
\begin{aligned}
\nabla_{\mathbf{x}} L &=\nabla f+\lambda \nabla g=\mathbf{0} \\
g(\mathbf{x}) & \leq 0 \\
\lambda & \geq 0 \\
\lambda g(\mathbf{x}) &=0
\end{aligned}
$$
这些条件合称为\term{Karush-Kuhn-Tucker (KKT)条件}。

如果我们要最大化 $f(\mathbf{x})$ 且受限于 $g(\mathbf{x}) \leq 0$, 那么对偶可行性要改成 $\lambda \leq 0$ 。
\end{theorem}


上面结果可推广至多个约束等式与约束不等式的情况。考虑标准约束优化问题(或称非线性规划):

\begin{definition}[标准约束优化问题(非线性规划)]
    $$
\begin{array}{ll}
\min & f(\mathbf{x}) \\
\text { s.t. } & g_{j}(\mathbf{x})=0, \quad j=1, \ldots, m \\
& h_{k}(\mathbf{x}) \leq 0, \quad k=1, \ldots, p
\end{array}
$$
\end{definition}

\begin{theorem}[标准约束优化的KKT条件]
    定义Lagrangian 函数
$$
L\left(\mathbf{x},\left\{\lambda_{j}\right\},\left\{\mu_{k}\right\}\right)=f(\mathbf{x})+\sum_{j=1}^{m} \lambda_{j} g_{j}(\mathbf{x})+\sum_{k=1}^{p} \mu_{k} h_{k}(\mathbf{x})
$$
其中 $ \lambda_{j} $ 是对应 $ g_{j}(\mathbf{x})=0 $ 的Lagrange乘数, $ \mu_{k} $ 是对应 $ h_{k}(\mathbf{x}) \leq 0 $ 的Lagrange乘数(或称KKT 乘数)。 

KKT条件包括
$$
\begin{aligned}
\nabla_{\mathbf{x}} L &=\mathbf{0} \\
g_{j}(\mathbf{x}) &=0, \quad j=1, \ldots, m \\
h_{k}(\mathbf{x}) & \leq 0 \\
\mu_{k} & \geq 0 \\
\mu_{k} h_{k}(\mathbf{x}) &=0, \quad k=1, \ldots, p
\end{aligned}
$$

\end{theorem}

\subsection{拉格朗日乘子法的例子}

\begin{problem}

    考虑这个问题
$$
\begin{array}{ll}
\min & x_{1}^{2}+x_{2}^{2} \\
\text { s.t. } & x_{1}+x_{2}=1 \\
& x_{2} \leq \alpha
\end{array}
$$
其中 $ \left(x_{1}, x_{2}\right) \in \mathbb{R}^{2}, \alpha $ 为实数。
\end{problem}

写出Lagrangigan函数
$$
L\left(x_{1}, x_{2}, \lambda, \mu\right)=x_{1}^{2}+x_{2}^{2}+\lambda\left(1-x_{1}-x_{2}\right)+\mu\left(x_{2}-\alpha\right)
$$

KKT 方程组如下:
$$
\begin{aligned}
\frac{\partial L}{\partial x_{i}} &=0, \quad i=1,2 \\
x_{1}+x_{2} &=1 \\
x_{2}-\alpha & \leq 0 \\
\mu & \geq 0 \\
\mu\left(x_{2}-\alpha\right) &=0
\end{aligned}
$$
求偏导可得 $ \frac{\partial L}{\partial x_{1}}=2 x_{1}-\lambda=0 $ 且 $ \frac{\partial L}{\partial x_{2}}=2 x_{2}-\lambda+\mu=0 $, 分别解出 $ x_{1}=\frac{\lambda}{2} $ 且 $ x_{2}=\frac{\lambda}{2}-\frac{\mu}{2} $ 。代入约束等式 $ x_{1}+x_{2}=\lambda-\frac{\mu}{2}=1 $ 或 $ \lambda=\frac{\mu}{2}+1 $ 。合并上面结果,
$$
x_{1}=\frac{\mu}{4}+\frac{1}{2}, \quad x_{2}=-\frac{\mu}{4}+\frac{1}{2}
$$

最后再加入约束不等式 $ -\frac{\mu}{4}+\frac{1}{2} \leq \alpha $ 或 $ \mu \geq 2-4 \alpha $ 。分开三种情况讨论。

\begin{enumerate}
    \item $ \alpha>\frac{1}{2} $ : 不难验证 $ \mu=0>2-4 \alpha $ 满足所有的KKT条件, 约束不等式是无效的, $ x_{1}^{\star}=x_{2}^{\star}=\frac{1}{2} $ 是内部解,目标函数的极小值是 $ \frac{1}{2}$。
    \item $ \alpha=\frac{1}{2} $ : 如同 $ 1, \quad \mu=0=2-4 \alpha $ 满足所有的KKT条件, $ \quad x_{1}^{\star}=x_{2}^{\star}=\frac{1}{2} $ 是边界解, 因为 $ x_{2}^{\star}=\alpha $。
    \item $ \alpha<\frac{1}{2} $ : 这时约束不等式是有效的, $ \mu=2-4 \alpha>0 $, 则 $ x_{1}^{\star}=1-\alpha $ 且 $ x_{2}^{\star}=\alpha $, 目标函数的极小值是 $ (1-\alpha)^{2}+\alpha^{2} $。
\end{enumerate}

\section[Supplement Material: 浅谈最优化问题的KKT条件]{Supplement Material: 浅谈最优化问题的KKT条件\footnote{Cited from \url{https://zhuanlan.zhihu.com/p/26514613}.}}


\begin{theorem}[KKT条件]
    对于具有等式和不等式约束的一般优化问题
$$
\begin{aligned}
&\min f(\mathbf{x}) \\
\text { s.t. }& g_{j}(\mathbf{x}) \leq 0(j=1,2, \cdots, m) \\
&h_{k}(\mathbf{x})=0(k=1,2, \cdots, l)
\end{aligned}
$$
$ \mathrm{KKT} $ 条件给出了判断 $ \mathrm{x}^{*} $ 是否为最优解的\textbf{必要条件}, 即:
$$
\left\{\begin{array}{l}
\frac{\partial f}{\partial x_{i}}+\sum_{j=1}^{m} \mu_{j} \frac{\partial g_{j}}{\partial x_{i}}+\sum_{k=1}^{l} \lambda_{k} \frac{\partial h_{k}}{\partial x_{i}}=0,(i=1,2, \ldots, n) \\
h_{k}(\mathbf{x})=0,(k=1,2, \cdots, l) \\
\mu_{j} g_{j}(\mathbf{x})=0,(j=1,2, \cdots, m) \\
\mu_{j} \geq 0
\end{array}\right.
$$
\end{theorem}


\subsection{等式约束优化问题}

等式约束优化问题是指

\begin{problem}[等式约束优化问题]
    $$
\begin{array}{l}
\min f\left(x_{1}, x_{2}, \ldots, x_{n}\right) \\
\text { s.t. } h_{k}\left(x_{1}, x_{2}, \ldots, x_{n}\right)=0
\end{array}
$$
\end{problem}


我们令 $ L(\mathbf{x}, \lambda)=f(\mathbf{x})+\sum_{k=1}^{l} \lambda_{k} h_{k}(\mathbf{x}) $, 函数 $ L(x, y) $ 称为\term{Lagrange函数}, 参数 $ \lambda $ 称为\term{Lagrange乘子}.

再联立方程组: 
$$ \left\{\begin{array}{l}\frac{\partial L}{\partial x_{i}}=0(i=1,2, \cdots, n) \\ \frac{\partial L}{\partial \lambda_{k}}=0(k=1,2, \cdots, l)\end{array}\right. $$

得到的解为可能极值点,由于我们用的是必要条件,具体是否为极值点需根据问题本身的具体情况检验. 这个方程组称为\textbf{等式约束的极值必要条件}.

上式我们对 $ n $ 个 $ x_{i} $ 和 $ l $ 个 $ \lambda_{k} $ 分别求偏导, 回想一下在\term{无约束优化问题} 
$$ f\left(x_{1}, x_{2}, \ldots, x_{n}\right)=0 $$ 
中, 我们根据极值的必要条件, 分别令 $ \frac{\partial f}{\partial x_{i}}=0 $, 求出可能的极值点. 

因此可以联想到:等式约束下的 Lagrange乘数法引入了 $ l $ 个Lagrange乘子,或许我们可以\textbf{把 $ \lambda_{k} $ 也看作优化变量}( $ x_{i} $ 就叫做\term{优化变量}). 相当于将优化变量个数增加到 $ (n+l) $ 个, $ x_{i} $ 与 $ \lambda_{k} $ 一视同仁, 均为优化变量, 均对它们求偏导.

\subsection{不等式约束优化问题}

以上我们讨论了等式约束的情形,接下来我们来介绍不等式约束的优化问题.

我们先给出其主要思想:\textbf{转化}的思想——\textbf{将不等式约束条件变成等式约束条件}.具体做法是\textbf{引入}\term{松弛变量}.松弛变量也是优化变量,也需要一视同仁求偏导.

具体而言, 我们先看一个一元函数的例子:

\begin{example}
    $$\begin{aligned}
        &\min f(x)\\
    \text { s.t. }& g_{1}(x)=a-x \leq 0\\
    &g_{2}(x)=x-b \leq 0
    \end{aligned} $$
\end{example}

\begin{remark}
    优化问题中,我们必须求得一个确定的值,因此不妨令所有的不等式均取到等号,即 $ \leq $ 的情况.
\end{remark}


对于约束 $ g_{1} $ 和 $ g_{2} $, 我们分别引入两个松弛变量 $ a_{1}^{2} $ 和 $ b_{1}^{2} $, 得到 $ h_{1}\left(x, a_{1}\right)=g_{1}+a_{1}^{2}=0 $ 和 $ h_{2}\left(x, b_{1}\right)=g_{2}+b_{1}^{2}=0 $. 

\begin{remark}
    注意, 这里直接加上平方项 $ a_{1}^{2} 、 b_{1}^{2} $ 而非 $ a_{1} 、 b_{1} $, 是因为 $ g_{1} $ 和 $ g_{2} $ 这两个 不等式的左边必须加上一个正数才能使不等式变为等式. 若只加上 $ a_{1} $ 和 $ b_{1} $, 又会引入新的约束 $ a_{1} \geq 0 $ 和 $ b_{1} \geq 0 $, 这不符合我们的意愿.
\end{remark}


$$\begin{aligned}
    g_1(x) &= a - x \le 0 \\
    g_2(x) &= x - b \le 0
\end{aligned} \Rightarrow \begin{aligned}
    &h_{1}\left(x, a_{1}\right)=g_{1}(x)+a_{1}^{2}=a-x+a_{1}^{2}=0\\
    &h_{2}\left(x, b_{1}\right)=g_{2}(x)+b_{1}^{2}=x-b+b_{1}^{2}=0
\end{aligned}$$


由此我们将不等式约束转化为了等式约束, 并得到Lagrange函数
$$
L\left(x, a_{1}, b_{1}, \mu_{1}, \mu_{2}\right)=f(x)+\mu_{1}\left(a-x+a_{1}^{2}\right)+\mu_{2}\left(x-b+b_{1}^{2}\right)
$$
我们再按照等式约束优化问题(极值必要条件)对其求解, 联立方程
$$
\left\{\begin{array}{l}
\dfrac{\partial F}{\partial x}=\dfrac{\partial f}{\partial x}+\mu_{1} \dfrac{d g_{1}}{d x}+\mu_{2} \dfrac{d g_{2}}{d x}=\dfrac{d f}{d x}-\mu_{1}+\mu_{2}=0 \\
\dfrac{\partial F}{\partial \mu_{1}}=g_{1}+a_{1}^{2}=0, \\ 
\dfrac{\partial F}{\partial \mu_{2}}=g_{2}+b_{1}^{2}=0 \\
\dfrac{\partial F}{\partial a_{1}}=2 \mu_{1} a_{1}=0, \\ 
\dfrac{\partial F}{\partial b_{1}}=2 \mu_{2} b_{1}=0 \\
\mu_{1} \geq 0, \quad \mu_{2} \geq 0
\end{array}\right.
$$

\begin{remark}
    对于不等式约束前的乘子, 我们要求其大于等于 0. ($\mu_{1} \geq 0, \mu_{2} \geq 0$)
\end{remark}

得出方程组后, 便开始动手解它. 看到第3行的两式 $\mu_{1} a_{1}=0$ 和 $\mu_{1} a_{1}=0$ 比较简单, 我们就从它们入手吧.

对于 $\mu_{1} a_{1}=0$, 我们有两种情况:

情形1 $: \quad \mu_{1}=0, a_{1} \neq 0$

此时由于乘子 $\mu_{1}=0$, 因此 $g_{1}$ 与其相乘为零, 可以理解为约束 $g_{1}$ 不起作用, 且有 $g_{1}(x)=a-x<0 .$

情形2: $\quad \mu_{1} \geq 0, a_{1}=0$

此时 $g_{1}(x)=a-x=0$ 且 $\mu_{1}>0$, 可以理解为约束 $g_{1}$ 起作用, 且有 $g_{1}(x)=0$.

合并情形 1 和情形 2 得: \textbf{$\mu_{1} g_{1}=0$, 且在约束起作用时 $\mu_{1}>0, g_{1}(x)=0$; 约束不起作用时 $\mu_{1}=0, g_{1}(x)<0 .$}

同样地, 分析 $\mu_{2} b_{1}=0$, 可得出约束 $g_{2}$ 起作用和不起作用的情形, 并分析得到 $\mu_{2} g_{2}=0$.

由此, 
\begin{theorem}[一元一次优化式的KKT条件]
    方程组(极值必要条件)转化为
$$
\left\{\begin{array}{l}
\frac{d f}{d x}+\mu_{1} \frac{d g_{1}}{d x}+\mu_{2} \frac{d g_{2}}{d x}=0 \\
\mu_{1} g_{1}(x)=0, \mu_{2} g_{2}(x)=0 \\
\mu_{1} \geq 0, \mu_{2} \geq 0
\end{array}\right.
$$

\end{theorem}

 这是一元一次的情形.类似地, 

\begin{corollary}
    对于多元多次不等式约束问题
$$
\begin{array}{l}
\min f(\mathbf{x}) \\
\text { s.t. } g_{j}(\mathbf{x}) \leq 0(j=1,2, \cdots, m)
\end{array}
$$
有
$$
\left\{\begin{array}{l}
\frac{\partial f\left(x^{*}\right)}{\partial x_{i}}+\sum_{j=1}^{m} \mu_{j} \frac{\partial g_{j}\left(x^{*}\right)}{\partial x_{i}}=0(i=1,2, \ldots, n) \\
\mu_{j} g_{j}\left(x^{*}\right)=0(j=1,2, \ldots, m) \\
\mu_{j} \geq 0(j=1,2, \ldots, m)
\end{array}\right.
$$

上式便称为不等式约束优化问题的KKT(Karush-Kuhn-Tucker)条件. $ \mu_{j} $ 称为KKT乘子, 且约束 起作用时 $ \mu_{j} \geq 0, g_{j}(x)=0 $ ; 约束不起作用时 $ \mu_{j}=0, g_{j}(x)<0 $.
\end{corollary}

\subsection{KKT乘子必须大于等于0}

\begin{problem}
    还剩最后一个问题没有解决:为什么KKT乘子必须大于等于零?
\end{problem}

我将用几何性质来解释. 由于
$$
\frac{\partial f\left(x^{*}\right)}{\partial x_{i}}+\sum_{j=1}^{m} \mu_{j} \frac{\partial g_{j}\left(x^{*}\right)}{\partial x_{i}}=0(i=1,2, \ldots, n)
$$

用梯度表示 $$ \nabla f\left(\mathbf{x}^{*}\right)+\sum_{j \in J} \mu_{j} \nabla g_{j}\left(\mathbf{x}^{*}\right)=0$$

$J$ 为起作用约束的集合.

移项可得 $$ -\nabla f\left(\mathbf{x}^{*}\right)=\sum_{j \in J} \mu_{j} \nabla g_{j}\left(\mathbf{x}^{*}\right) $$

注意到梯度为向量. 

\begin{theorem}
    $$ -\nabla f\left(\mathbf{x}^{*}\right)=\sum_{j \in J} \mu_{j} \nabla g_{j}\left(\mathbf{x}^{*}\right) $$

    在约束极小值点 $ \mathbf{x}^{*} $ 处,函数 $ f\left(\mathbf{x}^{*}\right) $ 的负梯度一定可以表示成:所有起作用约束在该点的梯度(等值线的法向量)的线性组合.
\end{theorem}


\begin{corollary}[梯度的性质]
    复习课本中梯度的性质:某点梯度的方向就是函数等值线 $ f(\mathbf{x})=C $ 。(在这点的法线方向, 等值线就是地理的等高线。)
\end{corollary}

为方便作图, 假设现在\textbf{只有两个约束条件起约束作用}, 我们作出图形如图\ref{fig:kkt-1}.

\begin{FigureCenter}{}
    \label{fig:kkt-1}
    \includegraphics[width=0.5\textwidth]{KKT-geometry-1.jpg}
\end{FigureCenter}


注意我们上面推导过, 约束起作用时 $ g_{j}(\mathbf{x})=0 $, 所以此时约束在几何上应该是一簇\textbf{约束平面}.

我们假设在 $ \mathbf{x}^{*} $ 取得极小值点, 若同时满足 $ g_{1}(\mathbf{x})=0 $ 和 $ g_{2}(\mathbf{x})=0 $, 则 $ \mathbf{x}^{k} $ 一定在这两个平面的交线上, 且 $ -\nabla f\left(\mathbf{x}^{*}\right)=\sum_{j \in J} \mu_{j} \nabla g_{j}\left(\mathbf{x}^{*}\right) $, 即 $ -\nabla f\left(\mathbf{x}^{k}\right) 、 \nabla g_{1}\left(\mathbf{x}^{k}\right) $ 和 $ \nabla g_{2}\left(\mathbf{x}^{k}\right) $ 共面.

\begin{FigureCenter}{}
    \label{fig:kkt-2}
    \includegraphics[width=\textwidth]{KKT-geometry-2.jpg}
\end{FigureCenter}

图\ref{fig:kkt-2}是在点 $ \mathbf{x}^{k} $ 处沿 $ x_{1} O x_{2} $ 面的截面, 过点 $ \mathbf{x}^{k} $ 作目标函数的负梯度 $ -\nabla f\left(\mathbf{x}^{k}\right) $, 它垂直于目标函数 的等值线 $ f(\mathbf{x})=C $ ,且指向目标函数 $ f(\mathbf{x}) $ 的最速减小方向.

\begin{corollary}
    一点的梯度与等值线相互垂直。
\end{corollary}

再作约束函数 $ g_{1}(\mathbf{x})=0 $ 和 $ g_{2}(\mathbf{x})=0 $ 的梯度 $ \nabla g_{1}\left(\mathbf{x}^{k}\right) $ 和 $ \nabla g_{2}\left(\mathbf{x}^{k}\right) $, 它们分别垂直 $ g_{1}(\mathbf{x})=0 $ 和 $ g_{2}(\mathbf{x})=0 $ 两曲面在 $ \mathbf{x}^{k} $ 的切平面, 并形成一个雉形夹角区域.此时, 可能有 $ \mathrm{a} 、 \mathrm{~b} $ 两种情形:

\begin{FigureCenter}{}
    \label{fig:kkt-3}
    \includegraphics[width=0.5\textwidth]{KKT-geometry-3.jpg}
\end{FigureCenter}

我们先来看情形 $\mathrm{b}$ :若3个向量的位置关系如\ref{fig:kkt-3}所示, 即 $-\nabla f$ 落在 $\nabla g_{1}$ 和 $\nabla g_{2}$ 所形成的锥角区外的 一侧. 此时, 作等值面 $f(\mathbf{x})=C$ 在点 $\mathbf{x}^{k}$ 的切平面(它与 $-\nabla f\left(\mathbf{x}^{k}\right)$ 垂直), 我们发现:沿着与负 梯度 $-\nabla f$ 成锐角的方向移动(如下图红色箭头方向), 只要在红色区域取值, 目标函数 $f(\mathbf{x})$ 总 能减小.而红色区域是可行域 $(f(\mathbf{x})=C$, C取不同的常数能得到不同的等值线, 因此能取到红色 区域), 因此既可减小目标函数值, 又不破坏约束条件. 这说明 $\mathbf{x}^{k}$ 仍可沿约束曲面移动而不破坏约 束条件, 且目标函数值还能够减小.所以 $\mathbf{x}^{k}$ 不是稳定的最优点, 即不是局部极值点.

反过头来看情形a: $ -\nabla f $ 落在 $ \nabla g_{1} $ 和 $ \nabla g_{2} $ 形成的锥角内. 此时, 同样作 $ f(\mathbf{x})=C $ 在点 $ \mathbf{x}^{k} $ 与 $ -\nabla f $ 垂直的切平面. 当从 $ \mathbf{x}^{k} $ 出发沿着与负梯度 $ -\nabla f $ 成锐角的方向移动时, 虽然能使目标函数值减小, 但此时任何一点都不在可行区域内. 显然, 此时 $ \mathbf{x}^{k} $ 就是局部最优点 $ \mathbf{x}^{*} $, 再做任何移动都将破坏约 束条件, 故它是稳定点.

由于 $ -\nabla f\left(\mathbf{x}^{*}\right) $ 和 $ \nabla g_{1}\left(\mathbf{x}^{*}\right) 、 \nabla g_{2}\left(\mathbf{x}^{*}\right) $ 在一个平面内, 所以前者可看成是后两者的线性组合. 又由 上面的几何分析知, $ -\nabla f\left(\mathbf{x}^{*}\right) $ 在 $ \nabla g_{1}\left(\mathbf{x}^{*}\right) $ 和 $ \nabla g_{2}\left(\mathbf{x}^{*}\right) $ 的夹角之间, 所以线性组合的系数为正, 有
$$
-\nabla f\left(\mathbf{x}^{*}\right)=\mu_{1} \nabla g_{1}\left(\mathbf{x}^{*}\right)+\mu_{2} \nabla g_{2}\left(\mathbf{x}^{*}\right), \text { 且 } \mu_{1}>0, \mu_{2}>0 \text {. }
$$
这就是 $ \mu_{j}>0 $ 的原因. 类似地, 当有多个不等式约束同时起作用时, 要求 $ -\nabla f\left(\mathbf{x}^{*}\right) $ 处于 $ \nabla g_{j}\left(\mathbf{x}^{*}\right) $ 形成的超角锥(高维图形, 我姑且称之为 “超” )之内.

\subsection{总结:同时包含等式和不等式约束的一般优化问题}

\begin{theorem}[同时包含等式和不等式约束的一般优化问题的KKT条件]
    $$
\begin{array}{l}
\min f(\mathbf{x}) \\
\text { s.t. } g_{j}(\mathbf{x}) \leq 0(j=1,2, \cdots, m) \\
h_{k}(\mathbf{x})=0(k=1,2, \cdots, l)
\end{array}
$$
KKT条件 $ \left(\mathrm{x}^{*}\right. $ 是最优解的必要条件 $ ) $ 为
$$
\left\{\begin{array}{l}
\frac{\partial f}{\partial x_{i}}+\sum_{j=1}^{m} \mu_{j} \frac{\partial g_{j}}{\partial x_{i}}+\sum_{k=1}^{l} \lambda_{k} \frac{\partial h_{k}}{\partial x_{i}}=0,(i=1,2, \ldots, n) \\
h_{k}(\mathbf{x})=0,(k=1,2, \cdots, l) \\
\mu_{j} g_{j}(\mathbf{x})=0,(j=1,2, \cdots, m) \\
\mu_{j} \geq 0
\end{array}\right.
$$
\end{theorem}

\begin{remark}
    对于等式约束的Lagrange乘子,并没有非负的要求。
\end{remark}

\begin{remark}
    以后求其极值点,不必再引入松弛变量,直接使用KKT条件判断。
\end{remark}


\section[Supplement Material: 凸优化、拉格朗日乘子法和KKT条件]{Supplement Material: 凸优化、拉格朗日乘子法和KKT条件\footnote{Cited from \url{https://zhuanlan.zhihu.com/p/59928816}.}}

\subsection{凸集的概念}

\begin{definition}[点、(直)线、线段]
    $ x_{1} \neq x_{2} $ 是 $ \mathbb{R}^{n} $ 中的两\term{点}, $ \theta \in \mathbb{R} $, 那么 $ y=x_{2}+\theta\left(x_{1}-x_{2}\right) $ 表示穿过两点的\term{线}。

    当 $ 0 \leqslant \theta \leqslant 1 $ 时, $ y $ 是 $ x_{1} $ 到 $ x_{2} $ 的\term{线段},$ y $ 也可以表示成 $ y=\theta x_{1}+(1-\theta) x_{2} $。
\end{definition}

\begin{definition}[仿射集]
    一个集合 $ C \subseteq \mathbb{R}^{n} $ 是\term{仿射集}如果其中任意两个不同的点的连线仍包含于 $ C $ 。
\end{definition}

\begin{definition}[凸集]
    一个集合 $ C \subseteq \mathbb{R}^{n} $ 是凸集, 如果任意两点之间的线段仍包含于 $ C $ 。即 $ \forall x_{1}, x_{2} \in C $, 任意 $ 0 \leqslant \theta \leqslant 1 $, 有 $ \theta x_{1}+(1-\theta) x_{2} \in C $ 。
\end{definition}

\begin{theorem}
    两个凸集的交集仍是凸集。
\end{theorem}

\begin{definition}[凸函数]
    一个函数 $ f: \mathbb{R}^{n} \rightarrow \mathbb{R} $ 是\term{凸函数}如果定义域是凸集而且对任意定义域的 $ x, y, 0 \leqslant \theta \leqslant 1 $ 有
$$
f(\theta x+(1-\theta) y) \leqslant \theta f(x)+(1-\theta) f(y)
$$
\end{definition}

二维时候的几何意义是, 如果两点的线段总位于函数曲线之上, 那么该函数是凸函数。

\begin{theorem}
    $ f $ 是凸的, 那么 $ -f $ 是 凹的。
\end{theorem}

\begin{theorem}
    
仿射函数既凸又凹。
\end{theorem}


\begin{theorem}[$ \alpha- $sublevel 集]
    给定凸函数 $ f$,则$\{x \in D(f): f(x) \leqslant \alpha\} $是一个凸集。
\end{theorem}

\begin{proof}
    对任意 $ x, y \in D(f) $ 使得 $$ f(x) \leqslant \alpha, f(y) \leqslant \alpha $$
    
    有 $$ f(\theta x+(1-\theta) y) \leqslant \theta f(x)+(1-\theta) f(y) \leqslant \theta \alpha+(1-\theta) \alpha $$
    
    也就是说如果 $ x, y $ 属于 $ \alpha- $ sublevel 集,那么二者之间的线段上的点也可以使得 $ f \leqslant \alpha , $ 即二者之间的线段也包含于 $ \alpha- $ sublevel 集。
\end{proof}

\subsection{凸优化}

\begin{definition}[优化问题]
    \term{优化问题}有如下形式:

    $$
\begin{array}{ll}
\operatorname{minimize} & f_{0}(x) \\
\text { subject to } & f_{i}(x) \leqslant b_{i}, i=1, \ldots, m
\end{array}
$$

$ x=\left(x_{1}, \ldots, x_{n}\right) $ 是\term{优化变量}, 函数 $ f_{0}: \mathbb{R}^{n} \rightarrow \mathbb{R} $ 是\term{目标函数}, 函数 $ f_{i}: \mathbb{R}^{n} \rightarrow \mathbb{R} $ 是\term{约束函数}。 $ b_{i} $ 是约束。使得 $ f_{0}(x) $ 在约束条件下最小的 $ x^{*} $ 叫做\term{最优点}, 或问题的\term{解}。
\end{definition}

\begin{definition}[凸优化]
    \term{凸优化}问题有如下形式:

    $$
    \begin{array}{ll}
    \operatorname{minimize} & f(x) \\
    \text { subject to } & x \in C
    \end{array}
    $$
    
    $ f $ 是凸函数, $ C $ 是凸集。
\end{definition}

\begin{corollary}
    凸集可以表示成某些个凸集的交集。
\end{corollary}

所以凸优化问题一般表示为

\begin{definition}[凸优化问题]
    \label{def:convex-problem}

    $$
    \begin{array}{ll}
    \operatorname{minimize} & f(x) \\
    \text { subject to } & g_{i}(x) \leqslant 0, i=1, \ldots, m \\
    & h_{j}(x)=0, j=1, \ldots, p
    \end{array}
    $$

    其中 $ g_{i} $ 是\term{凸函数}, $ h_{j} $ 是\term{仿射函数}。 
\end{definition}

$ g_{i}(x) \leqslant 0 $叫做 \term{$ 0- $sublevel集}, 是凸集。

\begin{theorem}
    $ h_{j}(x)=0 $ 也是凸集。
\end{theorem}

\begin{proof}
    $$ h\left(\theta x_{1}+(1-\theta) x_{2}\right)=\theta h\left(x_{1}\right)+(1-\theta) h\left(x_{2}\right)=0 $$
\end{proof}

凸优化问题的特点是, \textbf{所有局部最优点都是全局最优点}。

\begin{definition}[二次规划]
    如果一个凸优化问题的 $ g_{i} $ 都是仿射函数, 且 $ f $ 是凸二次函数, 那么它叫做二次规划:
    $$
    \begin{array}{ll}
    \operatorname{minimize} & \frac{1}{2} x^{\top} P x+c^{\top} x+d \\
    \text { subject to } & g_{i}(x) \leqslant 0, i=1, \ldots, m \\
    & h_{j}(x)=0, j=1, \ldots, p
    \end{array}
    $$

    其中 $ P $ 是一个对称半正定矩阵(使得 $ \left.\frac{1}{2} x^{\top} P x \geqslant 0\right) $ 。
\end{definition}



\subsection{拉格朗日对偶性}

\begin{theorem}
    对于没有限制的凸函数, 最优点 $ x^{*} $ 一定满足 $ \nabla_{x} f\left(x^{*}\right)=0 $。
\end{theorem}

然而对于有限制条件的凸优化问题却不是这样。拉格朗日对偶性可以将有限制的凸优化问题转化为没有限制的问题, 来求解凸优化问题。

\begin{definition}[拉格朗日函数]
    给定一个凸优化问题, 拉格朗日算子是一个函数 $ \mathcal{L}: \mathbb{R}^{n} \times \mathbb{R}^{m} \times \mathbb{R}^{p} \rightarrow \mathbb{R} $, 定义为:
    $$
    \mathcal{L}(x, \alpha, \beta)=f(x)+\sum_{i=1}^{m} \alpha_{i} g_{i}(x)+\sum_{i=1}^{p} \beta_{i} h_{i}(x)
    $$

    $ x \in \mathbb{R}^{n} $ 叫做\term{主变量(primal variable)}。 $ \alpha \in \mathbb{R}^{m}, \beta \in \mathbb{R}^{p} $ 统称对偶变量或拉格朗日乘子。
\end{definition}

\begin{theorem}
    总存在一个拉格朗日乘子, 使得没有限制的拉格朗日算子相对于 $ x $ 的最小值, 等于原凸优化问题的最优值
\end{theorem}

后文会证明。

\subsubsection{主问题}

为了说明拉格朗日算子和原凸优化问题的关系,需要引入主问题和对偶问题。

考虑优化问题:

\begin{problem}
    \label{pbl:primal}
    $$
\min _{x}\left[\max _{\alpha, \beta, \alpha_{i} \geqslant 0} \mathcal{L}(x, \alpha, \beta)\right]=\min _{x} \theta_{\mathcal{P}}(x)
$$

括号内的 $ \theta_{\mathcal{P}}: \mathbb{R}^{n} \rightarrow \mathbb{R} $ 叫做\term{主目标(primal objective)}, 等号右边的没有限制的最小化问题叫做\term{主问题(primal problem)}。 用 $ x^{*} $ 表示\term{主问题的解},$ p^{*}=\theta_{\mathcal{P}}\left(x^{*}\right) $ 表示\term{主目标的最优值}。
\end{problem}

\begin{definition}[primal feasible]
    $ x $ 是\term{主可行的(primal feasible)}如果 $ g_{i}(x) \leqslant 0, h_{j}(x)=0 $ 。
\end{definition}


$$
\begin{aligned}
\theta_{\mathcal{P}}(x) &=\max _{\alpha, \beta, \alpha_{i} \geqslant 0} \mathcal{L}(x, \alpha, \beta) \\
&=\max _{\alpha, \beta, \alpha_{i} \geqslant 0}\left[f(x)+\sum_{i=1}^{m} \alpha_{i} g_{i}(x)+\sum_{i=1}^{p} \beta_{i} h_{i}(x)\right] \\
&=f(x)+\max _{\alpha, \beta, \alpha_{i} \geqslant 0}\left[\sum_{i=1}^{m} \alpha_{i} g_{i}(x)+\sum_{i=1}^{p} \beta_{i} h_{i}(x)\right]
\end{aligned}
$$

观察最后一个式子:

\begin{itemize}
    \item 如果任何一个 $ g_{i}(x)>0 $, 那么最大化 $ \theta_{\mathcal{P}}(x) $ 只需要将相应的 $ \alpha_{i} $ 设置为无限大。
    \item 如果 $ g_{i}(x) \leqslant 0 $, 因为 $ \alpha_{i} \geqslant 0 $, 所以 $ \theta_{\mathcal{P}}(x) $ 最大时, 必然有 $ \alpha_{i}=0 $ 。
\end{itemize}

相似地:

\begin{itemize}
    \item 如果 $ h_{i} \neq 0 $, 那么最大化 $ \theta_{\mathcal{P}}(x) $ 只需要将相应的 $ \beta_{i} $ 设置为 $ h_{i}(x) $ 的相同符号且绝对值无限大。
    \item 如果 $ h_{i}(x)=0 $, 那么 $ \sum_{i=1}^{p} \beta_{i} h_{i}(x) $ 项的最大值为0 。
\end{itemize}

综上, 有
$$
\theta_{\mathcal{P}}(x)=\left\{\begin{array}{ll}
f(x)+0 & x \text { 为主可行的 } \\
f(x)+\infty & x \text { 不是主可行的 }
\end{array}\right.
$$
因此当 $ x $ 主可行时, 主问题\ref{pbl:primal}的最优值等于原凸优化问题\ref{def:convex-problem}的最优值。

\subsubsection{对偶问题}

\begin{definition}[对偶问题]
    对调主问题的最大最小操作:
$$
\max _{\alpha, \beta, \alpha \geqslant 0}\left[\min _{x} \mathcal{L}(x, \alpha, \beta)\right]=\max _{\alpha, \beta, \alpha \geqslant 0} \theta_{\mathcal{D}}(\alpha, \beta)
$$

$ \theta_{\mathcal{D}}(\alpha, \beta): \mathbb{R}^{m} \times \mathbb{R}^{p} \rightarrow \mathbb{R} $ 是\term{对偶目标(dual objective)},等号右边的有限制的最大化问题叫做\term{对偶问题}。 用 $ \left(\alpha^{*}, \beta^{*}\right) $ 表示\term{对偶问题的解}, $ d^{*}=\theta_{\mathcal{D}}\left(\alpha^{*}, \beta^{*}\right) $ 表示\term{对偶目标的最优值}。
\end{definition}

\begin{definition}[对偶可行]
    如果 $ \alpha_{i}(x) \geqslant 0 $ ,$ (\alpha, \beta) $ 是对偶可行的。
\end{definition}


\begin{theorem}
    如果 $ (\alpha, \beta) $ 是对偶可行的,那么 $ \theta_{\mathcal{D}}(\alpha, \beta) \leqslant p^{*} $.
\end{theorem}

\begin{proof}
    $$ \begin{aligned} \theta_{\mathcal{D}}(\alpha, \beta) &=\min _{x} \mathcal{L}(x, \alpha, \beta) \\ & \leqslant \mathcal{L}\left(x^{*}, \alpha, \beta\right) \\ &=f\left(x^{*}\right)+\sum_{i=1}^{m} \alpha_{i} g_{i}\left(x^{*}\right)+\sum_{i=1}^{p} \beta_{i} h_{i}\left(x^{*}\right) \\ & \leqslant f\left(x^{*}\right)=p^{*} \end{aligned} $$
\end{proof}

\begin{theorem}[弱对偶性]
    对任意主问题和对偶问题,有 $ d^{*} \leqslant p^{*} $.
\end{theorem}

\begin{theorem}[强对偶性]
    对任意主问题和对偶问题, 如果满足某个条件(constraint qualifications), 那 么 $ d^{*}=p^{*} $ 。最常用的constraint qualification是Slater's condition: 所有的不等式限制都严格满 足 (即 $ g_{i}(x)<0 $ ) 。
\end{theorem}

\begin{theorem}[Slater 条件]
    设定义在 $ \mathcal{D} $ 上的函数 $ f_{i}(\cdot), i=1,2, \cdots, n $ 为凸函数, $ g_{j}(\cdot), j=1,2, \cdots, m $ 为 仿射函数, 考虑凸优化问题

    $$
    \min _{\mathbf{x}} f_{0}(\mathbf{x}), \quad \text { s.t. } f_{i}(\mathbf{x}) \leq 0, g_{i}(\mathbf{x}) \leq 0
    $$

    如果存在点 $ \mathrm{x} \in $ relint $ \mathcal{D} $ (即 $ \mathcal{D} $ 的相对内点), 则强对偶性成立.
\end{theorem}

实践中, 几乎所有的凸优化问题都满足某种constraint qualification, 所以 主问题和对偶问题有相同的最优值。

\begin{theorem}[互补松弛性 (complementary slackness, KKT complementarity)]
    如果强对偶性满足, 那么 $ \alpha_{i}^{*} g\left(x_{i}^{*}\right)=0, i=1, \ldots, m $
    
\end{theorem}

\begin{proof}
    $$
    \begin{aligned}
    p^{*}=d^{*}=\theta_{\mathcal{D}}\left(\alpha^{*}, \beta^{*}\right) &=\min _{x} \mathcal{L}\left(x, \alpha^{*}, \beta^{*}\right) \\
    & \leqslant \mathcal{L}\left(x^{*}, \alpha^{*}, \beta^{*}\right) \\
    &=f\left(x^{*}\right)+\sum_{i=1}^{m} \alpha_{i}^{*} g_{i}\left(x^{*}\right)+\sum_{i=1}^{p} \beta_{i}^{*} h_{i}\left(x^{*}\right) \\
    & \leqslant f\left(x^{*}\right)=p^{*}
    \end{aligned}
    $$

    因为第一个和最后一个表达式相等, 所以中间所有的小于等于号都是等号, 有 $$ \sum_{i=1}^{m} \alpha_{i}^{*} g_{i}\left(x^{*}\right)+\sum_{i=1}^{p} \beta_{i}^{*} h_{i}\left(x^{*}\right)=0 $$

    因为 $ \alpha_{i}^{*} \geqslant 0 , h_{i}\left(x^{*}\right)=0 $, 所以 $ \alpha_{i}^{*} $ 和 $ g_{i}\left(x^{*}\right) $ 至少有一个是 0 。
\end{proof}

\begin{theorem}
    设 $ x^{*} \in \mathbb{R}^{n}, \alpha^{*} \in \mathbb{R}^{m}, \beta^{*} \in \mathbb{R}^{p} $, 下列条件为 $ \mathrm{KKT} $ 条件:

    \begin{enumerate}
        \item (主可行) $ g_{i}\left(x^{*}\right) \leqslant 0, i=1, \ldots, m, h_{j}\left(x^{*}\right)=0, j=1, \ldots, p $
        \item (对偶可行) $ \alpha_{i}^{*} \geqslant 0, i=1, \ldots, m $
        \item (互补松弛性) $ \alpha_{i}^{*} g\left(x_{i}^{*}\right)=0, i=1, \ldots, m $
        \item (Lagrangian Stationary) $ \nabla_{x} \mathcal{L}\left(x^{*}, \alpha^{*}, \beta^{*}\right)=0 $
    \end{enumerate}

\end{theorem}

\begin{theorem}
    对于凸优化问题, 有:

$ x^{*} $ 原始最优, $ \left(\alpha^{*}, \beta^{*}\right) $ 对偶最优,且有$$ 强对偶性 \Leftrightarrow  满足  K K T  条件$$
\end{theorem}

\subsection{总结}


给定一个\term{凸优化}问题, \term{拉格朗日算子}将凸优化问题的目标函数和限制考虑进一个函数中, 在拉格朗日算子基础上可以定义\term{主问题}和\term{对偶问题}。

\term{主问题}是先调整 $ (\alpha, \beta) $ 使拉格朗日算子最大(变为主目标), 再调整 $ x $ 使主目标最小。 $ x $ 主可行时, 主目标等于原凸优化问题的目标函数, 主目标的最小值等于原凸优化问题的最小值。

\term{对偶问题}是先调整 $ x $ 使拉格朗日算子最小(变为对偶目标, 因为拉格朗日算子是关于 $ x $ 的没有限制 的凸函数, 所以变为对偶目标时其对 $ x $ 的偏导数为 0 ), 再调整使对偶目标最大。 $ (\alpha, \beta) $ 对偶可行 时, 对偶目标小于等于主目标的最小值。

\term{弱对偶性}指对偶目标的最大值小于等于主目标的最小值。强对偶性指对偶目标的最大值等于主目标的最小值。

$$对偶目标的最大值=主目标的最小值=原凸优化问题的最小值 \Leftrightarrow^{当且仅当}满足KKT条件$$

因此,当对偶问题比原凸优化问题容易求解时,可以通过求解对偶问题来解原凸优化问题。
