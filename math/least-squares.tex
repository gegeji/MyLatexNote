\chapter{Least Squares}

\section{An Example: Measurement Problem}

\begin{problem}
    已知测量量路段长度: $ A D=89, A C=67, B D=53, A B=35, C D=20 $ $ , x_{1}, x_{2} $ 和 $ x_{3} $ 的长度是多少?
\end{problem}

% todo (2021-11-11 18:19): figure

由 $ x_{1}, x_{2} $ 和 $ x_{3} $ 的关系可得方程组:
$$
\left\{\begin{array}{r}
x_{1}+x_{2}+x_{3}=89 \\
x_{1}+x_{2}=67 \\
x_{2}+x_{3}=53 \\
x_{1}=35 \\
x_{3}=20
\end{array} \Leftrightarrow A x=b, A=\left[\begin{array}{lll}
1 & 1 & 1 \\
1 & 1 & 0 \\
0 & 1 & 1 \\
1 & 0 & 0 \\
0 & 0 & 1
\end{array}\right], b=\left[\begin{array}{l}
89 \\
67 \\
53 \\
35 \\
20
\end{array}\right]\right.
$$

取后三个式子求解方程组,回代前两个式子

$$\displaystyle  \begin{array}{{>{\displaystyle}l}}
\left\{\begin{array}{ r }
x_{2} +x_{3} =53\\
x_{1} =35\\
x_{3} =20
\end{array} \Rightarrow x_{1} =35,x_{2} =33,x_{3} =20.\right. \\
\left\{\begin{array}{ r }
x_{1} +x_{2} +x_{3} =88\neq 89\\
x_{1} +x_{2} =68\neq 67
\end{array}\right. 
\end{array}$$
     

由于测量存在误差,方程组之间相互矛盾,该超定方程组无解。



\begin{problem}[最小二乘问题]
    寻找该方程组的近似解,并尽可能逼近方程组的目标$b$, 即残差向量 $ r=A x-b $ 某种度量下尽可能小

    $$ \min _{x}\|A x-b\|_{2}^{2}=\|r\|_{2}^{2} \quad (\ell_2范数度量残差) $$
\end{problem}


使用$\ell_2$、$\ell_\infty$等也可以度量误差,但是函数在零点处不光滑,不能求导。

\begin{problem}[求解最小二乘解]
    给定 $ \mathrm{A} \in \mathbb{R}^{m \times n}, \mathrm{~b} \in \mathbb{R}^{m} $, 求解 $ x \in \mathbb{R}^{n} $ 让目标函数最小

$$ \min _{x}\|A x-b\|_{2}^{2}=\min _{x} \sum_{i=1}^{m}\left(\sum_{j=1}^{n} A_{i j} x_{j}-b_{i}\right)^{2} $$
\end{problem}

\begin{notation}[最小二乘法的解]
    最小二乘法的解为 $ \hat{x} $

    $$
    \hat{x}=\arg \underset{x}{\min}\|A x-b\|_{2}^{2}=\arg \underset{x}{\min} \sum_{i=1}^{m}\left(\sum_{j=1}^{n} A_{i j} x_{j}-b_{i}\right)^{2}
    $$
\end{notation}

\begin{example}
    $ f(x)=\|A x-b\|_{2}^{2} $, $ \mathrm{A}=\left[\begin{array}{cc}2 & 0 \\ -1 & 1 \\ 0 & 2\end{array}\right], b=\left[\begin{array}{c}1 \\ 0 \\ -1\end{array}\right] $

    求解$$\hat{x} = \arg \underset{x}{\min} \|A x-b\|_{2}^{2}$$

    解:
    $$ f(x)=\|A x-b\|_{2}^{2}=\left(2 x_{1}-1\right)^{2}+\left(-x_{1}+x_{2}\right)^{2}+\left(2 x_{2}+1\right)^{2} $$

    $$ \frac{\partial f}{\partial x_{1}}=10 x_{1}-2 x_{2}-4 , \frac{\partial f}{\partial x_{2}}=-2 x_{1}+10 x_{2}+4 $$

    $$ \nabla f(x)=\left[\begin{array}{l}\dfrac{\partial f}{\partial x_{1}} \\ \dfrac{\partial f}{\partial x_{2}}\end{array}\right]=0 \Rightarrow \hat{x}=\left(\frac{1}{3},-\frac{1}{3}\right)^{T} $$
\end{example}


\begin{theorem}
    设最小二乘法的解为 $ \hat{x} $ ,满足:
    $$
    \|A \hat{x}-b\|_{2}^{2} \leq\|A x-b\|_{2}^{2}, \forall x \in \mathbf{R}^{n}
    $$

    当残差 $ \hat{r}=A \hat{x}-b=0 $ 时,则 $ \hat{x} $ 是线性方程组 $ A x=b $ 的解; 否则其为误差最小平方和意义下方程组的近似解。
\end{theorem}

\section{投影与$A$列空间的关系}

矩阵 $ \mathrm{A} \in \mathbb{R}^{m \times n} $ 的列 $ a_{1}, a_{2}, \ldots, a_{n} \in \mathbb{R}^{m} $ 的最小二乘法问题
$$
\hat{x}=\underset{x}{\arg \underset{x}{\min}}\|A x-b\|_{2}^{2} ,\|A x-b\|_{2}^{2}=\left\|\sum_{j=1}^{n} a_{j} x_{j}-b\right\|_{2}^{2}
$$

% todo (2021-11-11 18:30): figure

\begin{theorem}[投影与$A$列空间的关系]
    $ A \hat{x} \in \operatorname{range}(A) $是$A$的列空间中最接近$b$的向量。 
    
    $ \hat{r}=A \hat{x} -b$正交于$A$的列空间(值域空间) $ \operatorname{range}(A) $。
\end{theorem}

\section{求解最小二乘法}

给定 $ A \in \mathbb{R}^{m \times n}, b \in \mathbb{R}^{m}, x \in \mathbb{R}^{n} $ 目标函数:
$$
f(x)=\|A x-b\|_{2}^{2}=\sum_{i=1}^{m}\left(\sum_{j=1}^{n} A_{i j} x_{j}-b_{i}\right)^{2}
$$

为使目标函数最小,求最优解 $ \hat{x}:\hat{x}=\arg \underset{x}{\min} f(x) $

\begin{theorem}
    可微函数 $ f(x) $ 的最优解 $ \hat{x} $ 满足条件:梯度 $ \nabla f(\hat{x})=\mathbf{0} $ , 即:
$$
\nabla f(\hat{x})=\left[\begin{array}{c}
\dfrac{\partial f}{\partial x_{1}}(\hat{x}) \\
\vdots \\
\dfrac{\partial f}{\partial x_{n}}(\hat{x})
\end{array}\right]=2 A^{T}(A \hat{x}-b)=0
$$
\end{theorem}

\begin{theorem}[正规方程与最小二乘解]
    $$ A^{T} A x=A^{T} b $$

    $A$的列向量\textbf{线性无关}时,则 $ \hat{x}=\left(A^{T} A\right)^{-1} A^{T} b $。 
\end{theorem}

\begin{proof}
    设函数 $ g_{i}(x)=\sum_{j=1}^{n} A_{i j} x_{j}-b_{i} $ ,则有

    
        $$g_{i}( x) =\sum\limits _{j=1}^{n} A_{ij} x_{j} -b_{i} \Rightarrow \left(\begin{array}{ c c c c c }
        A_{1,1} & \cdots  & A_{1,k} & \cdots  & A_{1,n}\\
        \vdots  &  & \vdots  &  & \vdots \\
        \boldsymbol{\textcolor[rgb]{0.72,0.33,0.31}{A}\textcolor[rgb]{0.72,0.33,0.31}{_{j,1}}} & \boldsymbol{\textcolor[rgb]{0.72,0.33,0.31}{\cdots }} & \boldsymbol{\textcolor[rgb]{0.72,0.33,0.31}{A}\textcolor[rgb]{0.72,0.33,0.31}{_{j,k}}} & \boldsymbol{\textcolor[rgb]{0.72,0.33,0.31}{\cdots }} & \boldsymbol{\textcolor[rgb]{0.72,0.33,0.31}{A}\textcolor[rgb]{0.72,0.33,0.31}{_{j,n}}}\\
        \vdots  &  & \vdots  &  & \vdots \\
        A_{m,1} & \cdots  & A_{m,k} & \cdots  & A_{m,n}
        \end{array}\right)\left(\begin{array}{ c }
        \boldsymbol{\textcolor[rgb]{0.72,0.33,0.31}{x}\textcolor[rgb]{0.72,0.33,0.31}{_{1}}}\\
        \boldsymbol{\textcolor[rgb]{0.72,0.33,0.31}{\vdots }}\\
        \boldsymbol{\textcolor[rgb]{0.72,0.33,0.31}{x}\textcolor[rgb]{0.72,0.33,0.31}{_{j}}}\\
        \boldsymbol{\textcolor[rgb]{0.72,0.33,0.31}{\vdots }}\\
        \boldsymbol{\textcolor[rgb]{0.72,0.33,0.31}{x}\textcolor[rgb]{0.72,0.33,0.31}{_{n}}}
        \end{array}\right) -\left(\begin{array}{ c }
        \textcolor[rgb]{0,0,0}{b_{1}}\\
        \textcolor[rgb]{0,0,0}{\vdots }\\
        \textcolor[rgb]{0.72,0.33,0.31}{b\boldsymbol{_{j}}}\\
        \textcolor[rgb]{0,0,0}{\vdots }\\
        \textcolor[rgb]{0,0,0}{b_{n}}
        \end{array}\right)$$
    
    $$\begin{aligned} 
        f(x)&=\|A x-b\|_{2}^{2}
        &=\sum_{i=1}^{m}\left(\sum_{j=1}^{n} A_{i j} x_{j}-b_{i}\right)^{2}
        &=\sum_{i=1}^{m}\left(g_{i}(x)\right)^{2} 
    \end{aligned}$$

    函数 $ f(x) $ 对变量 $ x_{k} $ 偏导为
    
    $$ \frac{\partial f(x)}{\partial x_{k}}=\sum_{i=1}^{m}\left(\left(2 g_{i}(x)\right)\left(\frac{\partial g_{i}(x)}{\partial x_{k}}\right)\right) $$


    又因为
    $$ \frac{\partial g_{i}(x)}{\partial x_{k}}=A_{i k} $$


    所以
    $$ \begin{aligned} 
        \frac{\partial f}{\partial x_{k}}(x) 
        &=\sum_{i=1}^{m} 2\left(g_{i}(x)\right)\left(A_{i k}\right) \\
        &=2 \sum_{i=1}^{m}\left(\left(\sum_{j=1}^{n} A_{i j} x_{j}-b_{i}\right)\left(A_{i k}\right)\right) 
        \\ &=2 \sum_{i=1}^{m}\left(\left(\sum_{j=1}^{n} A_{i j} x_{j}\right)\left(A_{i k}\right)\right)-2 \sum_{i=1}^{m}\left(\left(b_{i}\right)\left(A_{i k}\right)\right) \end{aligned} $$

    注意有
        

    $$\sum\limits _{j=1}^{n} A_{ij} x_{j} =\left(\begin{array}{ c c c c c }
    A_{1,1} & \cdots  & A_{1,k} & \cdots  & A_{1,n}\\
    \vdots  &  & \vdots  &  & \vdots \\
    \boldsymbol{\textcolor[rgb]{0.72,0.33,0.31}{A}\textcolor[rgb]{0.72,0.33,0.31}{_{j,1}}} & \boldsymbol{\textcolor[rgb]{0.72,0.33,0.31}{\cdots }} & \boldsymbol{\textcolor[rgb]{0.72,0.33,0.31}{A}\textcolor[rgb]{0.72,0.33,0.31}{_{j,k}}} & \boldsymbol{\textcolor[rgb]{0.72,0.33,0.31}{\cdots }} & \boldsymbol{\textcolor[rgb]{0.72,0.33,0.31}{A}\textcolor[rgb]{0.72,0.33,0.31}{_{j,n}}}\\
    \vdots  &  & \vdots  &  & \vdots \\
    A_{m,1} & \cdots  & A_{m,k} & \cdots  & A_{m,n}
    \end{array}\right)\left(\begin{array}{ c }
    \boldsymbol{\textcolor[rgb]{0.72,0.33,0.31}{x}\textcolor[rgb]{0.72,0.33,0.31}{_{1}}}\\
    \boldsymbol{\textcolor[rgb]{0.72,0.33,0.31}{\vdots }}\\
    \boldsymbol{\textcolor[rgb]{0.72,0.33,0.31}{x}\textcolor[rgb]{0.72,0.33,0.31}{_{j}}}\\
    \boldsymbol{\textcolor[rgb]{0.72,0.33,0.31}{\vdots }}\\
    \boldsymbol{\textcolor[rgb]{0.72,0.33,0.31}{x}\textcolor[rgb]{0.72,0.33,0.31}{_{n}}}
    \end{array}\right) =\left(\begin{array}{ c }
    Result_{1}\\
    \vdots \\
    \boldsymbol{\textcolor[rgb]{0.72,0.33,0.31}{Result}\textcolor[rgb]{0.72,0.33,0.31}{_{k}}}\\
    \vdots \\
    Result_{m}
    \end{array}\right) =Ax$$

    $$\displaystyle \sum _{i=1}^{m}\left(\underbrace{\left(\sum _{j=1}^{n} A_{ij} x_{j}\right)}_{Result}( A_{ik})\right) =\begin{array}{ c }
    A_{1,\textcolor[rgb]{0.29,0.56,0.89}{\boldsymbol{k}}} \times Result_{1}\\
    +\\
    \vdots \\
    +\\
    \boldsymbol{\textcolor[rgb]{0.72,0.33,0.31}{A_{i,\textcolor[rgb]{0.29,0.56,0.89}{\boldsymbol{k}}} \times Result}\textcolor[rgb]{0.72,0.33,0.31}{_{i}}}\\
    +\\
    \vdots \\
    +\\
    A\textcolor[rgb]{0.29,0.56,0.89}{\boldsymbol{_{\textcolor[rgb]{0,0,0}{m,} k}}} \times Result_{m}
    \end{array} =\left(\begin{array}{ c }
    A_{1,\textcolor[rgb]{0.29,0.56,0.89}{\boldsymbol{k}}}\\
    \vdots \\
    A_{i,\textcolor[rgb]{0.29,0.56,0.89}{\boldsymbol{k}}}\\
    \vdots \\
    A_{m,\textcolor[rgb]{0.29,0.56,0.89}{\boldsymbol{k}}}
    \end{array}\right)^{T} Ax=a_{\textcolor[rgb]{0.29,0.56,0.89}{\boldsymbol{k}}}^{T} Ax$$
        
    类似地,有 
    
    $$\displaystyle \sum _{i=1}^{m}(( b_{i})( A_{ik})) =\begin{array}{ c }
    A_{1,\textcolor[rgb]{0.29,0.56,0.89}{\boldsymbol{k}}} \times b_{1}\\
    +\\
    \vdots \\
    +\\
    \boldsymbol{\textcolor[rgb]{0.72,0.33,0.31}{A}\textcolor[rgb]{0.72,0.33,0.31}{_{i,\textcolor[rgb]{0.29,0.56,0.89}{\boldsymbol{k}}}}\textcolor[rgb]{0.72,0.33,0.31}{\times b}\textcolor[rgb]{0.72,0.33,0.31}{_{i}}}\\
    +\\
    \vdots \\
    +\\
    A\textcolor[rgb]{0.29,0.56,0.89}{\boldsymbol{_{\textcolor[rgb]{0,0,0}{m,} k}}} \times b_{m}
    \end{array} =a_{\textcolor[rgb]{0.29,0.56,0.89}{\boldsymbol{k}}}^{T} b$$
        
    所以
    $$
    \begin{aligned}
        \frac{\partial f}{\partial x_{k}}(x)
        &=2 a_{k}^{T} A x-2 a_{k}^{T} b\\
        &=2 a_{k}^{T}(A x-b)
    \end{aligned}
    $$

    所以函数 $ f(x) $ 的梯度
$$
\begin{aligned}
    \nabla f(x)&=\left[\begin{array}{c}
    \dfrac{\partial f}{\partial x_{1}}(x) \\
    \vdots \\
    \dfrac{\partial f}{\partial x_{n}}(x)
    \end{array}\right]\\
    &=2\left[\begin{array}{c}
    a_{1}^{T}(A x-b) \\
    a_{2}^{T}(A x-b) \\
    \vdots \\
    a_{n}^{T}(A x-b)
    \end{array}\right] \\
    &= 2\left[a_{1}, a_{2}, \cdots, a_{n}\right]^{T}(A x-b) \\
    &=2 A^{T}(A x-b) 
\end{aligned}
$$

$$\nabla f(x)=2\left(A^{T} A x-A^{T} b\right)=0 \Rightarrow A^{T} A x=A^{T} b$$

$A$的列向量无关时,则 $ \hat{x}=\left(A^{T} A\right)^{-1} A^{T} b $。

\end{proof}

\section{几何解释}

残余向量 $ \hat{r}=A \hat{x}-b $ 满足 $ A^{T} \hat{r}=A^{T}(A \hat{x}-b)=0 $.

残余向量 $ \hat{r} $ 正交于 $ A $ 的每一列,因此正交于$ \operatorname{range}(A) $

向量 $ b $ 在 $ \operatorname{range}(A) $ 上的投影是 $ A\left(A^{T} A\right)^{-1} A^{T} b $

\section{正规方程}

\begin{theorem}[最小二乘法问题的正规方程]
    $$
A^{T} A x=A^{T} b
$$

等价于

$$ \nabla f(x)=0, f(x)=\|A x-b\|_{2}^{2 \prime \prime} $$
\end{theorem}

系数矩阵 $ A^{T} A $ 是 $ A $ 的Gram矩阵,最小二乘法问题所有的解都满足正规方程。

\begin{theorem}
    如果A的列线性无关,则

    $ A^{T} A $ 为非奇异矩阵,正规方程此时有唯一解。
\end{theorem}


\section{QR分解求解最小二乘法}

\begin{theorem}[QR分解求解最小二乘法]
    若 $ \mathrm{A} \in \mathbb{R}^{m \times n} $ 的列向量线性无关,则存在 $ \mathrm{A}=\mathrm{QR} $ 分解, $ Q \in \mathbb{R}^{m \times n} $ , $ R \in \mathbb{R}^{\mathrm{n} \times n} $ 
    
    最小二乘法问题的解
$$
\begin{aligned}
\hat{x}=\left(A^{T} A\right)^{-1} A^{T} b &=\left((Q R)^{T}(Q R)\right)^{-1}(Q R)^{T} b \\
&=\left(R^{T} Q^{T} Q R\right)^{-1} R^{T} Q^{T} b \\
&=\left(R^{T} R\right)^{-1} R^{T} Q^{T} b \\
&=R^{-1} Q^{T} b
\end{aligned}
$$
\end{theorem}

\subsection{The Complexity of Solving Least Square Problem via QR Decomposition}

算法复杂度:

\begin{itemize}
    \item 首先对A进行QR分解 $ A=Q R\left(2 m n^{2}\right. $ flops $ ) $
    \item 计算矩阵向量乘积 $ d=Q^{T} b(2 \mathrm{mn} $ flops $ ) $
    \item 通过回代求解 $ R x=d\left(n^{2}\right. $ flops $ ) $
    \item 复杂度: $ 2 m n^{2} $ flops
\end{itemize}

\begin{example}
    $$
A=\left[\begin{array}{cc}
3 & -6 \\
4 & -8 \\
0 & 1
\end{array}\right], \quad b=\left[\begin{array}{c}
-1 \\
7 \\
0
\end{array}\right]
$$
首先对$A$进行QR分解
$$
Q=\left[\begin{array}{cc}
3 / 5 & 0 \\
4 / 5 & 0 \\
0 & 1
\end{array}\right], \quad R=\left[\begin{array}{cc}
5 & -10 \\
0 & 1
\end{array}\right]
$$

计算 $ d=Q^{T} b=(5,2) $

求解 $ R x=d $
$$
\left[\begin{array}{cc}
5 & -10 \\
0 & 1
\end{array}\right]\left[\begin{array}{l}
x_{1} \\
x_{2}
\end{array}\right]=\left[\begin{array}{l}
5 \\
2
\end{array}\right]
$$

解得 $ x_{1}=5, x_{2}=2 $

\end{example}

\section{求解正规方程可能带来的输入误差}

直接求解正规方程组求解:
$$
A^{T} A x=A^{T} b
$$

可能会造成严重的舍入误差。

\begin{example}
    一个列向量“几乎”线性相关的矩阵
$$
A=\left[\begin{array}{cc}
1 & -1 \\
0 & 10^{-5} \\
0 & 0
\end{array}\right], \quad b=\left[\begin{array}{c}
0 \\
10^{-5} \\
1
\end{array}\right]
$$

将中间结果四舍五入到小数点后8位.

方法 1 :通过Gram矩阵求解
$$
A^{T} A=\left[\begin{array}{cc}
1 & -1 \\
-1 & 1+10^{-10}
\end{array}\right] \approx\left[\begin{array}{cc}
1 & -1 \\
-1 & 1
\end{array}\right], \quad A^{T} b=\left[\begin{array}{c}
0 \\
10^{-10}
\end{array}\right] \Rightarrow x=\left[\begin{array}{c}
10^{-10} \\
10^{-10}
\end{array}\right]
$$
经过四舍五入之后,Gram矩阵为奇异矩阵。


方法 2 : 通过对 $A$进行QR分解
$$
Q=\left[\begin{array}{ll}
1 & 0 \\
0 & 1 \\
0 & 0
\end{array}\right], \quad R=\left[\begin{array}{cc}
1 & -1 \\
0 & 10^{-5}
\end{array}\right]
$$

$$\begin{aligned}
    \hat{x}=\left(A^{T} A\right)^{-1} A^{T} b=R^{-1} Q^{T} b \\
    \Rightarrow& R x=Q^{T} b\\
    \Rightarrow& \left[\begin{array}{cc}1 & -1 \\ 0 & 10^{-5}\end{array}\right]\left[\begin{array}{l}x_{1} \\ x_{2}\end{array}\right]=\left[\begin{array}{l}0 \\ 10^{-5}\end{array}\right] \\
    \Rightarrow& x=\left[\begin{array}{l}1 \\ 1\end{array}\right]
\end{aligned}$$

\end{example}

方法2 比方法1更稳定,因为它避免构造Gram矩阵。



\section{梯度下降法}

给定 $ A \in \mathbb{R}^{m \times n}, \mathrm{~b} \in \mathbb{R}^{m}, x \in \mathbb{R}^{n} $ 目标函数:
$$
f(x)=\|A x-b\|_{2}^{2}=\sum_{i=1}^{m}\left(\sum_{j=1}^{n} A_{i j} x_{j}-b_{i}\right)^{2}
$$
为使目标函数最小, 令最优解 $ \hat{x}: \quad \hat{x}=\arg \underset{x}{ \min } f(x) $

\begin{problem}
    $ A \in \mathbb{R}^{\mathrm{m} \times n} $ 列向量线性相关或n非常大$
    A^{T} A \in \mathbb{R}^{n \times n}$不可逆,无法直接代入求得最小二乘解。
\end{problem}

通过迭代求解目标的最优解过程: $$ x^{(1)}, x^{(2)}, \cdots, x^{(k)} \rightarrow \hat{x} $$ 

设 $ x^{(k)} $ 是第$k$步迭代,期望更新 $ x^{(k+1)} $ ,满足 $ f\left(x^{(k+1)}\right)<f\left(x^{(k)}\right) $.

设函数 $ f(x) $ 可微,根据泰勒公式,在 $ x^{(k)} $ 的一阶公式为
$$
f\left(x^{(k+1)}\right)=f\left(x^{(k)}\right)+\left\langle\nabla f\left(x^{(k)}\right), x^{(k+1)}-x^{(k)}\right\rangle+o\left(\left\|x^{(k+1)}-x^{(k)}\right\|\right)
$$

如果 $ \left\|x^{(k+1)}-x^{(k)}\right\|_{2} $ 足够小, 则有
$$
f\left(x^{(k+1)}\right)-f\left(x^{(k)}\right) \approx\left\langle\nabla f\left(x^{(k)}\right), x^{(k+1)}-x^{(k)}\right\rangle
$$

\begin{corollary}
    根据柯西不等式 \ref{thm:cauchy-schwartz=inequality}

    $ \left|\left\langle\nabla f\left(x^{(k)}\right), x^{(k+1)}-x^{(k)}\right\rangle\right| \leq\left\|\nabla f\left(x^{(k)}\right)\right\|_{2}\left\|x^{(k+1)}-x^{(k)}\right\|_{2} $

    所以有
    $$
\left\langle\nabla f\left(x^{(k)}\right), x^{(k+1)}-x^{(k)}\right\rangle \geq-\left\|\nabla f\left(x^{(k)}\right)\right\|_{2}\left\|x^{(k+1)}-x^{(k)}\right\|_{2}
$$

当 $ x^{(k+1)}-x^{(k)}=-\alpha_{k} \nabla f\left(x^{(k)}\right), \alpha_{k}>0 $ 时,等式成立。
\end{corollary}

此时$f\left(x^{(k+1)}\right)-f\left(x^{(k)}\right) \le 0$。

迭代公式为 

$$  x^{(k+1)}=x^{(k)}-\alpha_{k} \nabla f\left(x^{(k)}\right) , f\left(x^{(k+1)}\right)<f\left(x^{(k)}\right) $$

\begin{definition}[梯度下降法]
    $$
\min _{x \in \mathbb{R}^{n}} \frac{1}{2}\|A x-b\|_{2}^{2}, \quad A \in \mathbb{R}^{m \times n}, b \in \mathbb{R}^{m}
$$

令 $ f(x)=\frac{1}{2}\|A x-b\|_{2}^{2} $, 则 $ f $ 为凸函数, 并有 $ \nabla f(x)=A^{T}(A x-b) $

则 $ A^{T} A \in \mathbb{R}^{n \times n} $ ,列向量\textbf{线性相关}导致其\textbf{不可逆}或$n$非常大。通过梯度下降法迭代求解

$$ x^{(k+1)}=x^{(k)}-\alpha^{(k)} A^{T}\left(A x^{(k)}-b\right) $$
\end{definition}

\section{估计学习率(步长)$\alpha$}

\begin{problem}
    $$ x^{(k+1)}=x^{(k)}-\alpha^{(k)} A^{T}\left(A x^{(k)}-b\right) $$

    需要估计 $ \alpha^{(k)} $。
\end{problem}

为了估计 $ \alpha^{(k)} $, 通过线性搜索估计:
$$
\alpha^{(k)}=\arg \min _{\alpha \in \Re} f\left(x^{(k)}-\alpha A^{T}\left(A x^{(k)}-b\right)\right)
$$

即 $ \alpha^{(k)} $ 是最优步长。在上面的优化式中$x^{(k)}$、$A$、$b$均视为定值。

\begin{theorem}[线性搜索估计的最优步长]
    $$\alpha^{(k)}=\frac{\left\|A^{T}\left(A x^{(k)}-b\right)\right\|_{2}^{2}}{\left\|A A^{T}\left(A x^{(k)}-b\right)\right\|_{2}^{2}}$$
\end{theorem}

\begin{proof}
    令 $ g(\alpha)=f\left(x^{(k)}-\alpha A^{T}\left(A x^{(k)}-b\right)\right) $ 是关于 $ \alpha $ 的 凸函数, 则有

    % todo (2021-11-12 07:39): complete proof
$$
\min _{\alpha} g(\alpha) \Rightarrow g^{\prime}(\alpha)=0 \Rightarrow \alpha^{(k)}=\frac{\left\|A^{T}\left(A x^{(k)}-b\right)\right\|_{2}^{2}}{\left\|A A^{T}\left(A x^{(k)}-b\right)\right\|_{2}^{2}}
$$
\end{proof}


\begin{algorithm}[htbp]
    \caption{使用线性搜索估计步长的梯度下降法}
    初始 $ x^{(0)} $, $k=0$\;
    \While(){Not Convergent}{
        $p^{(k)}=A^{T}\left(A x^{(k)}-b\right)$\;
        $\alpha^{(k)}=\dfrac{\left\|p^{(k)}\right\|_{2}^{2}}{\left\|A p^{(k)}\right\|_{2}^{2}}$\;
        $x^{(k+1)}=x^{(k)}-\alpha^{(k)} p^{(k)}$\;
    }
\end{algorithm}