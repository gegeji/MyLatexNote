\chapter{Multi-objective Least Squares}

\section{Definition of Multi-objective Least Squares}

\begin{problem}
    假设有以下多个目标

$$
J_{1}(x)=\left\|A_{1} x-b_{1}\right\|_{2}^{2}, \cdots, J_{k}(x)=\left\|A_{k} x-b_{k}\right\|_{2}^{2}
$$

矩阵 $ A_{i} \in \mathbb{R}^{m_{i} \times n} $, 向量 $ b_{i} \in \mathbb{R}^{m_{i}} $;

寻找一个向量 $ x \in \mathbb{R}^{n} $ 使得这 $ k $ 个目标 $ _{i}(x), i=1, \cdots, k $最小。
\end{problem}

可以将上述多目标规划问题转换为加权最小二乘法问题。

\begin{problem}[加权最小二乘法问题]
    $$ 
    \min _{x} J(x)
$$

$$\begin{aligned}
    J(x)&=\lambda_{1}\left\|A_{1} x-b_{1}\right\|_{2}^{2}+\cdots+\lambda_{k}\left\|A_{k} x-b_{k}\right\|_{2}^{2} \\
    &=\left\|\sqrt{\lambda_{1}}\left(A_{1} x-b_{1}\right)\right\|_{2}^{2}+\cdots+\left\|\sqrt{\lambda_{k}}\left(A_{k} x-b_{k}\right)\right\|_{2}^{2}
    \end{aligned}$$

$ \lambda_{i}>0, i=1, \cdots, k $, 表示不同目标的相对重要程度。
\end{problem}

利用 $ \ell_{2} $ 范数平方的可加性, 目标函数 $ J(x) $ 可以写成紧密形式:

\begin{problem}[加权最小二乘法问题紧密形式]
    $$
J(x)=\left\|\left[\begin{array}{c}
\sqrt{\lambda_{1}}\left(A_{1} x-b_{1}\right) \\
\vdots \\
\sqrt{\lambda_{k}}\left(A_{k} x-b_{k}\right)
\end{array}\right]\right\|_{2}^{2}
$$
\end{problem}


进一步可简化为

\begin{problem}[加权最小二乘法问题矩阵形式]

    $$ J(x)=\|\tilde{A} x-\tilde{b}\|_{2}^{2} $$

其中
$$
\tilde{A}=\left[\begin{array}{c}
\sqrt{\lambda_{1}} A_{1} \\
\vdots \\
\sqrt{\lambda_{k}} A_{k}
\end{array}\right], \quad \tilde{b}=\left[\begin{array}{c}
\sqrt{\lambda_{1}} b_{1} \\
\vdots \\
\sqrt{\lambda_{k}} b_{k}
\end{array}\right]
$$
\end{problem}


因此将多目标问题转化为单目标问题,使用最小二乘法进行求解。

\begin{problem}[双目标规划问题]
    $$ \min _{x}\left\|A_{1} x-b_{1}\right\|_{2}^{2}+\lambda\left\|A_{2} x-b_{2}\right\|_{2}^{2}, A_{1}, A_{2} \in \mathbb{R}^{10 \times 5} $$

    $\lambda$的变化会影响解的状态。
\end{problem}


\section{求解多目标最小二乘问题}

\begin{problem}
    $$
    J(x)=\|\tilde{A} x-\tilde{b}\|_{2}^{2},
\tilde{A}=\left[\begin{array}{c}
\sqrt{\lambda_{1}} A_{1} \\
\vdots \\
\sqrt{\lambda_{k}} A_{k}
\end{array}\right], \quad \tilde{b}=\left[\begin{array}{c}
\sqrt{\lambda_{1}} b_{1} \\
\vdots \\
\sqrt{\lambda_{k}} b_{k}
\end{array}\right]
$$

\end{problem}

\begin{theorem}
    如果$\tilde{A}$的列向量线性无关时,则该问题的解唯一。

$$\begin{aligned} \hat{x}&=\left(\tilde{A}^{T} \tilde{A}\right)^{-1} \tilde{A}^{T} \tilde{b} \\
&= \left(\lambda_{1} A_{1}{ }^{T} A_{1}+\cdots+\lambda_{k} A_{k}{ }^{T} A_{k}\right)^{-1}\left(\lambda_{1} A_{1}{ }^{T} b_{1}+\cdots+\lambda_{k} A_{k}{ }^{T} b_{k}\right) \end{aligned} $$
\end{theorem}



可对 $ \tilde{A} $ 进行QR分解计算 $ \hat{x} $ 。每一个矩阵$A_i$的行向量可以线性相关。



\section{正则化数据拟合}

\begin{problem}
    线性模型拟合数据 $ \left(x^{(1)}, y^{(1)}\right), \cdots,\left(x^{(N)}, y^{(N)}\right) $

    $$
\hat{f}(x)=\theta_{1} f_{1}(x)+\cdots \theta_{p} f_{p}(x), \theta=\left[\theta_{1}, \cdots, \theta_{p}\right]^{T}
$$
$ f_{1}(x) $ 为常数函数,且恒等于 $1$ 。
\end{problem}


较大的参数 $ \theta_{i} $ 会让模型对 $ f_{i}(x) $ 的变化更加敏感。 让参数 $ \theta_{2}, \ldots, \theta_{p} $ 更小,可以避免模型过拟合。

即可引出两个目标函数:

\begin{problem}
    $$
J_{1}(\theta)=\sum_{k=1}^{N}\left(\hat{f}\left(x^{(k)}\right)-y^{(k)}\right)^{2}, \quad J_{2}(\theta)=\sum_{j=2}^{p} \theta_{j}^{2}
$$
首要目标 $ J_{1}(\theta) $ 是误差的平方和。
\end{problem}

改写成加权最小二乘的形式

\begin{problem}
    $$
\min _{\theta} J_{1}(\theta)+\lambda J_{2}(\theta)=\sum_{k=1}^{N}\left(\hat{f}\left(x^{(k)}\right)-y^{(k)}\right)^{2}+\lambda \sum_{j=2}^{p} \theta_{j}^{2}
$$

正则化参数 $ \lambda>0 $。
\end{problem}




该问题等价于最小二乘法问题:

\begin{problem}
    $$
\min _{\theta}\left\|\left[\begin{array}{r}
A_{1} \\
\sqrt{\lambda} A_{2}
\end{array}\right] \theta-\left[\begin{array}{l}
y_{d \times 1} \\
0
\end{array}\right]\right\|_{2}^{2}
$$

$$ A_{1}=\left[\begin{array}{cccc}1 & f_{2}\left(x^{(1)}\right) & \ldots & f_{p}\left(x^{(1)}\right) \\ 1 & f_{2}\left(x^{(2)}\right) & \ldots & f_{p}\left(x^{(2)}\right) \\ \vdots & \vdots & \vdots & \vdots \\ 1 & f_{2}\left(x^{(N)}\right) & \cdots & f_{p}\left(x^{(N)}\right)\end{array}\right],  A_{2}=\left[\begin{array}{ccccc}0 & 1 & 0 & \cdots & 0 \\ 0 & 0 & 1 & \cdots & 0 \\ \vdots & \vdots & \vdots & \ddots & 0 \\ 0 & 0 & 0 & \cdots & 1\end{array}\right], y=\left[\begin{array}{c}y^{(1)} \\ \vdots \\ y^{(N)}\end{array}\right] $$
\end{problem}

参数$\lambda$越大,会迫使$\theta$得到接近零解。
% todo (2021-11-19 09:15): figure

\section{图像逆问题}

\begin{problem}
    $$ y=A x_{e x}+v $$

向量 $ x_{e x} \in \mathbb{R}^{n} $ 表示未知原始信息(需要估计),
向量 $ v \in \mathbb{R}^{m} $ 表示未知的误差或者噪声,
向量 $ y \in \mathbb{R}^{m} $ 为观测的已知数据,
矩阵 $ A \in \mathbb{R}^{m \times n} $ 将测量值 $ y $ 和原始信息 $ x_{e x} $ 之间的关系。


\end{problem}

使用最小二乘估计法进行估计

\begin{problem}[最小二乘法进行估计]
    $$
\min _{x}\|A x-y\|_{2}^{2}
$$
利用未知 $ x_{e x} $ 先验信息,对目标进行约束,构成多目标优化问题。 
\end{problem}


例如可以使用岭回归进行求解。


\begin{definition}[Tikhonov 正则化]
    $$
\min _{x}\|A x-y\|_{2}^{2}+\lambda\|x\|_{2}^{2}, \lambda>0
$$
\end{definition}

目标在于使 $ \|A x-y\|_{2}^{2} $ 足够小,同时 $ x $ 的能量也要小。

\begin{theorem}[岭回归问题的标准方程]
    该优化模型等价于求解
$$
\left(A^{T} A+\lambda I\right) x=A^{T} y
$$
\end{theorem}

即使矩阵$A$的列线性相关时,也有唯一解。

\subsection{差分矩阵平滑最小二乘解}

\begin{definition}[图像转换为列向量存储]
    二维图像 $ X \in \mathbb{R}^{M \times N} $ , 可按列存储成向量 $ x \in \mathbb{R}^{M N} $ :
    $$
    x=\left[\begin{array}{c}
    X_{1: M, 1} \\
    X_{1: M, 2} \\
    \vdots \\
    X_{1: M, N}
    \end{array}\right] $$ 
\end{definition}


\begin{example}[\textsf{reshape}]
    $$ 
X=\left[\begin{array}{ll}
1 & 2 \\
4 & 5
\end{array}\right] \Rightarrow x=\left[\begin{array}{l}
1 \\
4 \\
2 \\
5
\end{array}\right]
$$
\end{example}

对于前面的优化问题

$$
y=A x_{e x}+v
$$

可以假设

\begin{proposition}[图像逆问题先验假设]
    图像具有光滑性:\textbf{图像相邻两个像素值之间变化不大}。

    \begin{itemize}
    \item 水平方向: $ X\left[n_{1}, n_{2}+1\right] \approx X\left[n_{1}, n_{2}\right] $
    \item 垂直方向: $ X\left[n_{1}+1, n_{2}\right] \approx X\left[n_{1}, n_{2}\right] $
\end{itemize}
\end{proposition}


用差分矩阵进行平滑。

\begin{definition}[垂直差分矩阵]
    
    垂直差分矩阵是大小为 $ N \times N $ 的块矩阵,每块大小 $ (M-1)   \times M $。
    $$D_{v}=\left[\begin{array}{cccc}
        D & 0 & \cdots & 0 \\
        0 & D & \cdots & 0 \\
        \vdots & \vdots & \ddots & \vdots \\
        0 & 0 & \vdots & D
        \end{array}\right], D=\left[\begin{array}{ccccccc}
        -1 & 1 & 0 & \cdots & 0 & 0 & 0 \\
        0 & -1 & 1 & \cdots & 0 & 0 & 0 \\
        \vdots & \vdots & \vdots & \vdots & \vdots & \vdots & \vdots \\
        0 & 0 & 0 & \cdots & 0 & -1 & 1
        \end{array}\right] $$
\end{definition}

\begin{example}
$$
\begin{array}{c}
X=\left[\begin{array}{ll}
1 & 2 \\
4 & 5
\end{array}\right] \Rightarrow x=\left[\begin{array}{l}
1 \\
4 \\
2 \\
5
\end{array}\right], D_{v} x \Rightarrow\left[\begin{array}{l}
4-1 \\
5-2
\end{array}\right]
\end{array}
$$
\end{example}

\begin{definition}
    水平差分矩阵:大小为 $ (N-1) \times N $ 的块矩阵,每块大小 $ M \times M $ :
$$
D_{h}=\left[\begin{array}{ccccccc}
-I_{M, M} & I_{M, M} & 0 & \cdots & 0 & 0 & 0 \\
0 & -I_{M, M} & I_{M, M} & \cdots & 0 & 0 & 0 \\
\vdots & \vdots & \vdots & & \vdots & \vdots & \vdots \\
0 & 0 & 0 & \cdots & 0 & -I_{M, M} & I_{M, M}
\end{array}\right]
$$
\end{definition}

\begin{example}
    $$ X=\left[\begin{array}{ll}1 & 2 \\ 4 & 5\end{array}\right] \Rightarrow x=\left[\begin{array}{l}1 \\ 4 \\ 2 \\ 5\end{array}\right], D_{h} x \Rightarrow\left[\begin{array}{l}2-1 \\ 5-4\end{array}\right] $$
\end{example}

定义优化问题

\begin{problem}
     $$
\hat{x}=\arg \min _{x}\|A x-y\|_{2}^{2}+\lambda\left\|D_{v} x\right\|_{2}^{2}+\lambda\left\|D_{h} x\right\|_{2}^{2}, \lambda>0
$$

$ \|A x-y\|_{2}^{2} $ 称为保证项: 保证 $ A \hat{x} \approx y $.

$ \lambda\left\|D_{v} x\right\|_{2}^{2}+\lambda\left\|D_{h} x\right\|_{2}^{2} $ 为惩罚项,惩罚相邻像素值的差异变化
$$
\left\|D_{h} x\right\|_{2}^{2}+\left\|D_{v} x\right\|_{2}^{2}=\sum_{i=1}^{M} \sum_{j=1}^{N-1}\left(X_{i, j+1}-X_{i j}\right)^{2}+\sum_{i=1}^{M-1} \sum_{j=1}^{N}\left(X_{i+1, j}-X_{i j}\right)^{2}
$$
\end{problem}

求解这个模型可以得到图像逆退化的近似变换。





\section{信号去噪}

\begin{problem}[信号去噪问题]
    观察信号向量 $ y \in \mathbb{R}^{n} $ ,
$$
y=x_{e x}+v
$$
$ x_{e x} \in \mathbb{R}^{n} $ 是未知信号, $ v \in \mathbb{R}^{n} $ 是噪声。




% todo (2021-11-20 07:53): figure


    目标是找一个近似信号$x$变换缓慢,信号既有光滑性,同时逼近 $ y $ , 其优化模型为
$$
\min _{x}\|x-y\|_{2}^{2}+\lambda \sum_{i=1}^{n-1}\left(x_{i+1}-x_{i}\right)^{2}, \lambda>0
$$
\end{problem}


\begin{definition}[差分矩阵]
    令矩阵 $ D \in \mathbb{R}^{(n-1) \times n} $ 为差分矩阵:
$$ D=\left[\begin{array}{ccccccc}-1 & 1 & 0 & \cdots & 0 & 0 & 0 \\ 0 & -1 & 1 & \cdots & 0 & 0 & 0 \\ \vdots & \vdots & \vdots & \ldots & \vdots & \vdots & \vdots \\ 0 & 0 & 0 & \cdots & -1 & 1 & 0 \\ 0 & 0 & 0 & \cdots & 0 & -1 & 1\end{array}\right] $$

\end{definition}


则有 $ \sum_{i=1}^{n-1}\left(x_{i+1}-x_{i}\right)^{2}=\|D x\|_{2}^{2} $

优化模型等价于

$$ \min _{x}\left\|\left[\begin{array}{c}I \\ \sqrt{\lambda} D\end{array}\right] x-\left[\begin{array}{l}y \\ 0\end{array}\right]\right\|_{2}^{2} $$

优化模型等价于求解线性方程

$$
\left(I+\lambda D^{T} D\right) x=y
$$

当 $ \lambda \rightarrow 0, \hat{x}(\lambda) \rightarrow y $;当 $ \lambda \rightarrow \infty, \hat{x}(\lambda) \rightarrow a v g(y) 1 $
