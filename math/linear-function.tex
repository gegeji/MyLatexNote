\chapter{Linear Function}

\section{Linear Function}

\begin{definition}[Linear Function]
    $f$是一个将n维向量映射成数的函数. 

    $$ f: \mathbb{R}^{n} \rightarrow \mathbb{R} $$
\end{definition}

线性函数 $ f $ 满足以下两个性质 $ \left(k \in \mathbb{R}, x, y \in \mathbb{R}^{n}\right) $ :

\begin{itemize}
    \item 齐次性(homogeneity): $ f(k x)=k f(x) $
    \item 叠加性(Additivity): $ f(x+y)=f(x)+f(y) $
\end{itemize}

\begin{example}
    求平均值: $ f(x)=\frac{1}{n} \sum_{i=1}^{n} x_{i} $ 为线性函数. 
\end{example}

\begin{example}
    求最大值: $ f(x)=\max \left\{x_{1}, x_{2}, \ldots, x_{n}\right\} $ 并不是线性函数. 
\end{example}

\begin{proof}
   令 $ x=(1,-1), y=(-1,1), \alpha=0.5, \beta=0.5 $, 
   有 $$ f(\alpha x+\beta y)=0 \neq \alpha f(x)+\beta f(y)=1 $$.

   $$ \begin{aligned} f(x+y) 
    &=\max \left\{x_{1}+y_{1}, x_{2}+y_{2}, \ldots, x_{n}+y_{n}\right\} 
    \\ & \leq \max \left\{x_{1}, x_{2}, \ldots, x_{n}\right\}+\max \left\{y_{1}, y_{2}, \ldots, y_{n}\right\} 
    \\ & \leq f(x)+f(y) \end{aligned} $$
\end{proof}

\begin{theorem}
    设 $ \alpha_{1, \ldots,} \alpha_{m} \in \mathbb{R}, u_{1}, \ldots, u_{m} \in \mathbb{R}^{n} $, 则线性函数 $ f $ 满足

    $$ \begin{aligned} f\left(\alpha_{1} u_{1}+\alpha_{2} u_{2}+\ldots+\alpha_{m} u_{m}\right) &=f\left(\alpha_{1} u_{1}\right)+f\left(\alpha_{2} u_{2}+\ldots+\alpha_{m} u_{m}\right) \\ &=\alpha_{1} f\left(u_{1}\right)+f\left(\alpha_{2} u_{2}+\ldots+\alpha_{m} u_{m}\right) \\ &=\alpha_{1} f\left(u_{1}\right)+\alpha_{2} f\left(u_{2}\right)+\ldots+\alpha_{m} f\left(u_{m}\right) \end{aligned} $$
\end{theorem}

\begin{definition}[内积函数 (inner product function)]
    对于n维向量 $ a $,满足以下形式的函数被称为内积函数

    $$ f(x)=a^{T} x=a_{1} x_{1}+a_{2} x_{2}+\ldots+a_{n} x_{n} $$
\end{definition}

上述 $ f(x) $ 可以看作是每项 $ x_{\mathrm{i}} $ 的加权之和. 

\begin{corollary}
    内积函数都是线性的.
\end{corollary}

\begin{proof}
    $$ \begin{aligned} f(\alpha x+\beta y) &=a^{T}(\alpha x+\beta y) \\ &=a^{T}(\alpha x)+a^{T}(\beta y) \\ &=\alpha\left(a^{T} x\right)+\beta\left(a^{T} y\right) \\ &=\alpha f(x)+\beta f(y) \end{aligned} $$
\end{proof}


\begin{corollary}
    所有线性函数都是内积.

    $$ \begin{aligned} f(x) &=f\left(x_{1} e_{1}+x_{2} e_{2}+\ldots+x_{n} e_{n}\right) \\ &=x_{1} f\left(e_{1}\right)+x_{2} f\left(e_{2}\right)+\ldots+x_{n} f\left(e_{n}\right) \end{aligned} $$
\end{corollary}

\begin{proof}
    假设 $ f: \mathbb{R}^{n} \rightarrow \mathbb{R} $ 是线性函数, 那么可用 $ f(x)=a^{T} x $ 来表示, $ a $ 为常量.

    $$ \begin{aligned} f(x) &=f\left(x_{1} e_{1}+x_{2} e_{2}+\ldots+x_{n} e_{n}\right) \\ &=x_{1} f\left(e_{1}\right)+x_{2} f\left(e_{2}\right)+\ldots+x_{n} f\left(e_{n}\right) \end{aligned} $$
\end{proof}





\begin{definition}[仿射函数 (affine function)]
    其一般形式为 $ f(x)=a^{T} x+\mathrm{b} $, 其中 $ a \in \mathbb{R}^{n}, \quad b \in \mathbb{R} $ 为标量. 
\end{definition}

\begin{theorem}
    函数 $ f: \mathbb{R}^{n} \rightarrow \mathbb{R} $ 为仿射函数需要满足

$ f(\alpha x+\beta y)=\alpha f(x)+\beta f(y), \alpha+\beta=1, \alpha, \beta \in \mathbb{R}, x, y \in \mathbb{R}^{n} $
\end{theorem}

\section{泰勒展开}

\begin{definition}[函数$f$第$i$个分量的一阶偏导数]
    假设 $ f: \mathbb{R}^{n} \rightarrow \mathbb{R} $ , 函数 $ f $ 在 $ z $ 点可微

    $$ \begin{aligned} \frac{\partial f}{\partial z_{i}}(z) &=\lim _{t \rightarrow 0} \frac{f\left(z_{1}, \cdots, z_{i-1}, z_{i}+t, z_{i+1}, \cdots, z_{n}\right)-f(z)}{t} \\ &=\lim _{t \rightarrow 0} \frac{f\left(z+t e_{i}\right)-f(z)}{t} \end{aligned} $$
\end{definition}

\begin{definition}[$f$在点$z$的梯度]
    $$ \nabla f(z)=\left[\begin{array}{c}\frac{\partial f}{\partial z_{1}}(z) \\ \vdots \\ \frac{\partial f}{\partial z_{n}}(z)\end{array}\right] $$
\end{definition}

\begin{definition}[Taylor's Approximation]
    假设 $ f: \mathbb{R}^{n} \rightarrow \mathbb{R} $ , 函数 $ f $ 在 $ z $ 点充分光滑, 即处处可导.

    $$\begin{aligned} f(x)&=f(z)+\frac{\partial f}{\partial x_{1}}(z)\left(x_{1}-z_{1}\right)+\frac{\partial f}{\partial x_{2}}(z)\left(x_{2}-z_{2}\right)+\cdots+\frac{\partial f}{\partial x_{n}}(z)\left(x_{n}-z_{n}\right) 
    \\ & +\frac{1}{2 !} \sum_{j=1}^{n} \sum_{i=1}^{n} \frac{\partial f^{2}}{\partial x_{i} \partial x_{j}}(z)\left(x_{i}-z_{i}\right)\left(x_{\mathrm{j}}-z_{j}\right)+\cdots \end{aligned}$$
\end{definition}

\begin{example}
    泰勒公式利用多项式在一点附近逼近函数

    $$ \sin x= x-\frac{x^{3}}{3 !}+\frac{x^{5}}{5 !}-\cdots+(-1)^{k-1} \frac{x^{2 k-1}}{(2 k-1) !}+\frac{\sin \left[\xi+(2 k+1) \frac{\pi}{2}\right]}{(2 k+1) !} x^{2 k+1} $$

    一次逼近: $  \sin x \approx x $

    三次逼近: $  \sin x \approx x-\frac{x^{3}}{3 !} $
\end{example}

\begin{proof}
    $$ f(x)=P_{n}(x)+R_{n}(x) $$
    
    $ P_{n}(x)=a_{0}+a_{1}\left(x-x_{0}\right)+a_{2}\left(x-x_{0}\right)^{2}+\cdots+a_{n}\left(x-x_{0}\right)^{n}  $

    $ R_{n}(x)=o\left(x-x_{0}\right)^{n} $

    $ f(x) \approx P_{n}(x) $

    $\therefore  P_{n}\left(x_{0}\right)=f\left(x_{0}\right) $,
    $ P_{n}^{\prime}\left(x_{0}\right)=f^{\prime}\left(x_{0}\right) $,
    $ P_{n}^{\prime \prime}\left(x_{0}\right)=f^{\prime \prime}\left(x_{0}\right) $,
    $ \cdots  $,
    $ P_{n}^{(n)}\left(x_{0}\right)=f^{(n)}\left(x_{0}\right) $

    要求 $ P_{n}\left(x_{0}\right)=f\left(x_{0}\right) \Rightarrow  a_{0}=f\left(x_{0}\right)  $

    $$ P_{n}^{\prime}(x)=a_{1}+2 a_{2}\left(x-x_{0}\right)+\cdots+n a_{n}\left(x-x_{0}\right)^{n-1} \Rightarrow  a_{1}=f^{\prime}\left(x_{0}\right)  $$

    依此类推. $a_{n}=\frac{f^{(n)}\left(x_{0}\right)}{n !} $
\end{proof}

\begin{corollary}[n阶泰勒多项式]
    $$ \begin{aligned} P_{n}(x) &=f\left(x_{0}\right)+f^{\prime}\left(x_{0}\right)\left(x-x_{0}\right)+\frac{f^{\prime \prime}\left(x_{0}\right)}{2 !}\left(x-x_{0}\right)^{2} \\ &+\cdots+\frac{f^{(n)}\left(x_{0}\right)}{n !}\left(x-x_{0}\right)^{n} \end{aligned} $$
    where $a_{n}=\frac{f^{(n)}\left(x_{0}\right)}{n !} $
\end{corollary}

\begin{corollary}[对于高阶余项的公式]
    带拉格朗日余项的泰勒公式
    $$ \begin{aligned} f(x) &=f\left(x_{0}\right)+f^{\prime}\left(x_{0}\right)\left(x-x_{0}\right)+\frac{f^{\prime \prime}\left(x_{0}\right)}{2 !}\left(x-x_{0}\right)^{2} \\ &+\cdots+\frac{f^{(n)}\left(x_{0}\right)}{n !}\left(x-x_{0}\right)^{n}+\frac{f^{(n+1)}(\xi)}{(n+1) !}\left(x-x_{0}\right)^{n+1} \end{aligned} $$

    $ R_{n}(x)=\frac{f^{(n+1)}(\xi)}{(n+1) !}\left(x-x_{0}\right)^{n+1}\left(\xi\right. $ 在 $ x_{0} $ 与 $ x  $ 之间 $ ) $
\end{corollary}

\begin{corollary}[麦克劳林(Maclaurin)公式]在零点展开麦克劳林(Maclaurin)公式
    $$ \begin{aligned} f(x)=& f(0)+f^{\prime}(0) x+\frac{f^{\prime \prime}(0)}{2 !} x^{2}+\cdots+\frac{f^{(n)}(0)}{n !} x^{n} \\ &+\frac{f^{(n+1)}(\theta x)}{(n+1) !} x^{n+1} &(0<\theta<1) \end{aligned} $$
    
\end{corollary}

\begin{definition}[一阶泰勒公式]
    假设$ f: \mathbb{R}^{n} \rightarrow \mathbb{R} $, 函数$f$在$z$点可导

    $$ \hat{f}(x)=f(z)+\frac{\partial f}{\partial x_{1}}(z)\left(x_{1}-z_{1}\right)+\ldots+\frac{\partial f}{\partial x_{1}}(z)\left(x_{n}-z_{n}\right) $$
\end{definition}

当$x$非常接近$z$时, $ \hat{f}(x) $ 也非常接近 $ f(z) $.  
$ \hat{f}(x) $ 是关于 $ x $ 的一个仿射函数. 

\begin{corollary}[一阶泰勒公式的内积形式]
    $$ \hat{f}(x)=f(z)+\nabla f(z)^{T}(x-z) \quad (\nabla f(z)=\left[\begin{array}{c}\frac{\partial f}{\partial x_{1}}(z) \\ \vdots \\ \frac{\partial f}{\partial x_{n}}(z)\end{array}\right]) $$
\end{corollary} 

一维时, $ \hat{f}(x)=f(z)+f^{\prime}(z)(x-z) $.

\begin{example}
    $$ f(x)=x_{1}-3 x_{2}+e^{2 x_{1}+x_{2}-1} $$

    $$ \nabla f(x)=\left[\begin{array}{l}\frac{\partial f}{\partial x_{1}}(x) \\ \frac{\partial f}{\partial x_{2}}(x)\end{array}\right]=\left[\begin{array}{l}1+2 e^{2 x_{1}+x_{2}-1} \\ -3+e^{2 x_{1}+x_{2}-1}\end{array}\right] $$

    函数 $ f $ 在 0 点的一阶泰勒公式为:
    $$ \hat{f}(x)=f(0)+\nabla f(0)^{T}(x-0)=e^{-1}+\left(1+2 e^{-1}\right) x_{1}+\left(-3+e^{-1}\right) x_{2} $$
\end{example}

\section{Regression Model}
\begin{definition}[Regression Model]
    回归模型(regression model)为关于$x$的仿射函数
    $$ \hat{y}=x^{T} \beta+v $$

    $x$是\textit{特征向量(feature vector)}, 它的元素$x_i$称为\textit{回归元(regressors)}. n维向量 $ \beta $ 是\textit{权重向量(weight vector)}. 标量 $ v $ 是\textit{偏移量(offset)}. 标量 $ \hat{y} $ 是\textit{预测值(prediction)}. 表示某个实际结果或因变量, 用$y$表示. 
\end{definition}