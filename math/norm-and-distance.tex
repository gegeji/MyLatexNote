\chapter{Norm and Distance}

\section{Norm}

\begin{definition}[Vector Norm]
    在向量空间中存在一个函数 $ \|\cdot\|: \mathbb{R}^{n} \rightarrow \mathbb{R} $, 且满足以下条件

\begin{itemize}
    \item 齐次性: $ \|\alpha x\|=|\alpha|\|x\|, \quad \alpha \in \mathbb{R} $ 且 $ x \in \mathbb{R}^{n} $;
    \item 三角不等式: $ \|x+y\| \leq\|x\|+\|y\|, \quad x, y \in \mathbb{R}^{n} $;
    \item 非负性: $ \|x\| \geq {0}, \boldsymbol{x} \in \mathbb{R}^{n} $ 且 $ \|\boldsymbol{x}\|=0 \Leftrightarrow \boldsymbol{x}=0 $;
\end{itemize}
则称$\|\cdot\|$为向量范数。
\end{definition}

\begin{example}[$ \ell_{1} $-范数(曼哈顿范数, Manhattan norm)]
    $$ \|x\|_{1}=\left|x_{1}\right|+\left|x_{2}\right|+\ldots+\left|x_{n}\right| \quad x, y \in \mathbb{R}^{n}, \alpha \in \mathbb{R} $$
\end{example}

\begin{proof}
    $$ \|\alpha x\|_{1}=\left|\alpha x_{1}\right|+\left|\alpha x_{2}\right|+\cdots+\left|\alpha x_{n}\right|=|\alpha|\|x\|_{1} \geq 0 $$

    $$ \|x+y\|_{1}=\left|x_{1}+y_{1}\right|+\cdots+\left|x_{n}+y_{n}\right| \leq\left|x_{1}\right|+\left|y_{1}\right|+\cdots+\left|x_{n}\right|+\left|y_{n}\right|=\|x\|_{1}+\|y\|_{1} $$
\end{proof}

\begin{example}[$ \ell_{2} $-范数(欧几里得范数,Euclidean norm)]
    $$ \|x\|_{2}=\sqrt{\left(x_{1}^{2}+x_{2}^{2}+\cdots+x_{n}^{2}\right)}=\sqrt{x^{T} x}=(\langle x, x\rangle)^{\frac{1}{2}} $$
\end{example}

\begin{proof}
    $$ \|\alpha x\|_{2}=(\langle\alpha x, \alpha x\rangle)^{\frac{1}{2}}=|\alpha|(\langle x, x\rangle)^{\frac{1}{2}}=|\alpha|\|x\|_{2} $$
    
    $$\begin{aligned} \|x+y\|_{2}^{2}&=\langle x+y, x+y\rangle=\langle x, x\rangle+\langle x, y\rangle+\langle y, x\rangle+\langle y, y\rangle  \\
    &=\|x\|_{2}^{2}+2\langle x, y\rangle+\|y\|_{2}^{2} \leq\|x\|_{2}^{2}+2\|x\|_{2}\|y\|_{2}+\|y\|_{2}^{2}\\
    &=\left(\|x\|_{2}+\|y\|_{2}\right)^{2} \end{aligned}$$

    $$ \|x+y\|_{2} \leq\|x\|_{2}+\|y\|_{2} $$
\end{proof}

\begin{corollary}[柯西—施瓦茨不等式]
    $$ |\langle x, y\rangle|^{2} \leq\langle x, x\rangle\langle y, y\rangle=\|x\|_{2}^{2}\|y\|_{2}^{2} $$
\end{corollary}

\begin{definition}[$ \ell_{\infty} $-范数]
    $$ \|x\|_{\infty}=\max _{1 \leq i \leq n}\left|x_{i}\right|, x \in \mathbb{R}^{n} $$
\end{definition}

\begin{proof}
    $$ \begin{aligned} \max _{1 \leq i \leq n}\left|x_{i}\right| 
        &\leq \left(\left|x_{1}\right|^{p}+\cdots+\left|x_{i}\right|^{p}+\cdots+\left|x_{n}\right|^{p}\right)^{1 / p} \\
        &\leq \left(n \max _{1 \leq i \leq n}\left|x_{i}\right|^{p}\right)^{1 / p}\\
        &  =n^{1 / p} \max _{1 \leq i \leq n}\left|x_{i}\right| \\ &\rightarrow \max _{1 \leq i \leq n}\left|x_{i}\right| \quad(p \rightarrow \infty)\end{aligned}
    $$
\end{proof}

\begin{definition}[$ \ell_{\mathrm{p}} $-范数]
    $$ \|x\|_{p}=\left(x_{1}^{\mathrm{p}}+x_{2}^{p}+\cdots+x_{n}^{p}\right)^{\frac{1}{p}}, \quad x \in \mathbb{R}^{n}, p \ge 1 $$

    $ \ell_{1} $ 范数 $ \|x\|_{1}$,$ \ell_{2} $-范数 $ \|x\|_{2} $, $ \ell_{\infty} $-范数是 $ \ell_{p} $-范数的特例。
\end{definition}

证明可以使用以下两条不等式

\begin{theorem}[Minkowshi Inequality]
    $$ \left(\sum_{i=1}^{n}\left|x_{i}+y_{i}\right|^{p}\right)^{\frac{1}{p}} \leq\left(\sum_{i=1}^{n}\left|x_{i}\right|^{p}\right)^{\frac{1}{p}}+\left(\sum_{i=1}^{n}\left|y_{i}\right|^{p}\right)^{\frac{1}{p}}, p \geq 1, x, y \in \mathbb{R}^{n} $$
\end{theorem}

\begin{theorem}[Hölder Inequality]
    $$ \sum_{i=1}^{n}\left|x_{i} y_{i}\right| \leq\left(\sum_{i=1}^{n}\left|x_{i}\right|^{p}\right)^{1 / p}\left(\sum_{i=1}^{n}\left|y_{i}\right|^{q}\right)^{1 / q}, \frac{1}{p}+\frac{1}{q}=1,1<p, q<\infty $$
\end{theorem}

\section{Root Mean Square Value (RMS)}

\begin{definition}[向量 $ x $的均方值 (mean-square value)]
    向量 $ x \in \mathbb{R}^n $的均方值 (mean-square value)

    $$ \frac{x_{1}^{2}+x_{2}^{2}+\cdots+x_{n}^{2}}{n}=\frac{\|x\|_{2}^{2}}{n} $$
\end{definition}

\begin{definition}[n维向量 $ x $ 的均方根(root-mean-square value, RMS)]
    $$ \operatorname{rms}(x)=\sqrt{\frac{x_{1}^{2}+x_{2}^{2}+\cdots+x_{n}^{2}}{n}}=\frac{\|x\|_{2}}{\sqrt{n}} $$
\end{definition}

$ \operatorname{rms}(x) $ 给出了 $ \left|x_{i}\right| $ 的 “典型" (typical)值。例如, $ \mathrm{rms}(\mathbf{1})=1 $ (与$n$无关)。均方根(RMS)值对于比较不同长度的向量大小是比较有用的。

\section{Chebyshev's Inequality}

\begin{theorem}[Chebyshev's Inequality]
$$
\begin{aligned}
    P(|X-\mu| \ge \varepsilon) \le \frac{\sigma^2}{\varepsilon^2}
\end{aligned}
$$

$$
 P(|X-\mu| < \varepsilon) \ge 1 - \frac{\sigma^2}{\varepsilon^2}
$$
\end{theorem}

\begin{theorem}[Chebyshev's Inequality]
    假设$k$为向量 $ x $ 分量满足条件 $ \left|x_{i}\right| \geq a $ 的个数,即 $ x_{i}^{2} \geq a^{2} $ 的个数。

    因此: $ \|x\|_{2}^{2}=x_{1}^{2}+x_{2}^{2}+\cdots+x_{n}^{2} \geq k a^{2} $

    将 $ a^{2} $ 移项,可得到 $ k \leq \frac{\|x\|_{2}^{2}}{a^{2}} $

    满足 $ \left|x_{i}\right| \geq a $ 的 $ x_{i} $ 数量不会超过 $ \frac{\|\| \|_{2}^{2}}{a^{2}} $
\end{theorem}

\begin{corollary}[Chebyshev's Inequality Using RMS]
    $$ \operatorname{rms}(x)=\sqrt{\frac{x_{1}^{2}+x_{2}^{2}+\cdots+x_{n}^{2}}{n}}=\frac{\|x\|_{2}}{\sqrt{n}} $$

    $ \left|x_{i}\right| \geq a $ 的项数占整体的比例不会超过 $ \left(\frac{\operatorname{rms}(x)}{a}\right)^{2} $, 即 $ \frac{k}{n} \leq\left(\frac{\operatorname{rms}(x)}{a}\right)^{2} $
\end{corollary}

\section{Distance}

\begin{definition}[Euclidean distance]
    $n$维向量$a$和$b$之间的欧氏距离
    $$ \operatorname{dist}(a, b)=\|a-b\|_{2} $$
\end{definition}

\begin{definition}[RMS deviation]
    $ \operatorname{rms}(a-b) $ 是a和b之间的均方根偏差.
\end{definition}

\begin{theorem}[Trianglar Inequality]
    $$ \|a-c\|_{2}=\|(a-b)+(b-c)\|_{2} \leq\|a-b\|_{2}+\|b-c\|_{2} $$
\end{theorem}

\subsection{Feature Distance and Nearest Neighbor}

\begin{definition}[Feature Distance]
    如果 $ x $ 和y分别为两个实体的特征向量, 那么它们的特征距离(feature distance)为 $ \|x-y\|_{2} $
\end{definition}

\begin{definition}
    给定向量$x$,一个组向量$ Z_{1}, \ldots, Z_{m} $, 当$ \hat{q}_{j} $满足:

    $$ \left\|x-z_{j}\right\|_{2} \leq\left\|x-z_{i}\right\|_{2}, \quad i=1, \ldots, m $$

    则称 $ z_{j} $ 是 $ x $ 的最近邻(nearest neighbor)
\end{definition}

\section{Standard Derivation}

\begin{definition}[算术平均值]
    对于$n$维向量$x$

    $$ \operatorname{avg}(x)=\frac{\mathbf{1}^{T} x}{n} $$
\end{definition}

\begin{definition}[De-meaned Vector]
    $$ \tilde{x}=x-\operatorname{avg}(x) \mathbf{1} $$

    因此 $ \operatorname{avg} \boldsymbol{g}(\tilde{x})=0 $
\end{definition}

\begin{definition}[$x$的标准差]
    $$ \operatorname{std}(x)=\operatorname{rms}(\tilde{x})=\frac{\left\|x-\left(1^{T} x / n\right) 1\right\|_{2}}{\sqrt{n}} $$
\end{definition}

$\operatorname{std}(x)$表示数据元素的变化程度。对于常数$\alpha$,当且仅当$ x=\alpha \mathbf{1} $时,$ \operatorname{std}(x)=0 $.

\begin{theorem}
    $$ \operatorname{rms}(x)^{2}=\operatorname{avg}(x)^{2}+\operatorname{std}(x)^{2} $$
\end{theorem}

\section{Angle}

\begin{definition}[两个非零向量 $ a $ 和$b$之间的角(angle)]
    $$ \angle(a, b)=\arccos \left(\frac{a^{T} b}{\|a\|_{2}\|b\|_{2}}\right) $$

    $ \angle(a, b) $ 的取值范围为 $ [0, \pi] $, 且满足$$ a^{T} b=\|a\|_{2}\|b\|_{2} \cos (\angle(a, b)) $$
\end{definition}

在二维和三维向量之中,这里的角与普通角度(ordinary angle)是一致的。

\begin{itemize}
    \item $\theta =\frac{\pi}{2}=90°$:a和b为正交,写作$a \perp b (a ^T b  =0)$。
    \item $\theta =0$:a和b为同向的 $(a ^T  b=‖a ‖‖b  ‖)$。
    \item $\theta =\pi =180°$: a和b为反向的$(a ^T   b  = - ‖a ‖‖b ‖)$。
    \item $\theta <\frac{\pi}{2}=90°$:a和b成锐角$(a ^T b >0)$。
    \item $\theta >\frac{\pi}{2}=90°$:a和b成钝角$(a ^T b <0)$。
\end{itemize}

\begin{definition}[球面的距离]
    $$  \mathrm{R} \angle(a, b) $$
\end{definition}

\subsection{相关系数}

给定向量$a$和$b$,其去均值向量为:

$$ \tilde{a}=a-\operatorname{avg}(a) 1,  \tilde{b}=b-\operatorname{avg}(b) 1 $$

\begin{definition}[$a$和$b$的相关系数]
    $$ \rho=\frac{\tilde{a}^{T} \tilde{b}}{\|\tilde{a}\|_{2}\|\tilde{b}\|_{2}} = \cos \angle (\tilde{a}, \tilde{b}) $$

    where  $ \tilde{a} \neq 0 $,  $ \tilde{b} \neq 0 $.
\end{definition}

\begin{example}
    高度相关的向量:
\begin{itemize}
    \item 邻近地区的降雨时间序列。
    \item 类型密切相关文档的单词计数向量。
    \item 同行业中类似公司的日收益。
\end{itemize}

比较不相关的向量:
\begin{itemize}
    \item 无关的向量。
    \item 音频信号(比如,在多轨录音中的不同轨)。
\end{itemize}

负相关的向量:
\begin{itemize}
    \item 深圳与墨尔本的每天气温变化
\end{itemize}
\end{example}
