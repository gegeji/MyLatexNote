\section{统计量}

不含任何未知参数的样本的函数称为统计量。
它是完全由样本决定的量。

\section{常见统计量}

\begin{definition}[样本平均值]
    \begin{equation}\overline{X}=\frac{1}{n} \sum_{i=1}^{n} X_{i}\end{equation}
\end{definition}

\begin{definition}[样本方差]
    \begin{equation}S^{2}=\frac{1}{ n-1} \sum_{i=1}^{n}\left(X_{i}-\bar{X}\right)^{2}\end{equation}
\end{definition}

\begin{definition}[样本标准差]
    \begin{equation}S=\sqrt{\frac{1}{ { n-1} } \sum_{i=1}^{n}\left(X_{i}-\bar{X}\right)^{2}}\end{equation}
\end{definition}

\begin{definition}[样本 $k$ 阶原点矩]
    \begin{equation}A_{k}=\frac{1}{n} \sum_{i=1}^{n} X_{i}^{k}\end{equation}
\end{definition}

\begin{definition}[样本 $k$ 阶中心矩]
    \begin{equation}{B}_{{k}}=\frac{1}{{n}} \sum_{i=1}^{n}\left({X}_{i}-\overline{{X}}\right)^{k}\end{equation}
\end{definition}

\begin{definition}[$X$和$Y$的$k+p$阶混合原点矩]
    若
    \begin{equation}E \left\{\left(X^ k\right)\left(Y^ p\right)\right\}, k, p=1,2, \ldots \end{equation}
    存在, 则称它为$X$和$Y$的$k+p$阶混合原点矩
\end{definition}

\begin{definition}[$X$和$Y$的$k+l$阶混合中心矩]
    若
    \begin{equation} E\left\{[X-E(X)]^ k[Y-E(Y)]^l \right\},  k,l=1, \quad 2, \ldots \end{equation}
    存在, 则称它为$X$和$Y$的$k+l$阶混合中心矩
\end{definition}



\section{卡方分布}

\begin{definition}[卡方分布]
    设$X_1,X_2,...X_n$相互独立, 都服从正态分布 $N(0,1)$ (都是来自总体 $N(0,1)$ 的样本), 则称随机变量:

    \begin{equation}
        \chi^2 = X_1^2 + X_2^2 + ... + X_n^2
    \end{equation}

    所服从的分布为自由度为 $n$ 的$\chi^2$分布。 记作$\chi^2 \sim \chi^2(n)$
\end{definition}

\subsection{卡方分布性质}

\begin{corollary}
    $\chi^2(1) \sim X^2(1)$
\end{corollary}

\begin{corollary}[$\chi^2$分布的可加性]
    若 $ X_{1} \sim \chi^{2}\left(n_{1}\right), X_{2} \sim \chi^{2}\left(n_{2}\right) $, 且 $ X_{1} $ 与 $ X_{2} $ 相互独立, 则
    \begin{equation}
        X_{1}+X_{2} \sim \chi^{2}\left(n_{1}+n_{2}\right)
    \end{equation}
\end{corollary}

\begin{proof}
    事实上, 卡方分布是Gamma分布的特殊情况。 自由度为 $ n $ 的卡方分布 $ \chi_{n}^{2} $ 其实就是 $ \Gamma\left(\frac{n}{2}, \frac{1}{2}\right) $.

    根据Gamma分布的矩生成函数(Moment Generating Function, MGF),若$ X_{1} \sim \chi_{n_{1}}^{2} $, $ X_{2} \sim \chi_{n_{2}}^{2} $,

    那么 $ X_{1}, X_{2} $ 对应的MGF分别为
    \begin{equation} M_{X_{1}}(t)=(1-2 t)^{-n_{1} / 2} \end{equation}
    \begin{equation} M_{X_{2}}(t)=(1-2 t)^{-n_{2} / 2} \end{equation}
    因为 $ X_{1}, X_{2} $ 相互独立, 那么 $ Y=X_{1}+X_{2} $ 的MGF为
    \begin{equation} M_{Y}(t)=M_{X_{1}}(t) M_{X_{2}}(t)=(1-2 t)^{-\left(n_{1}+n_{2}\right) / 2} \end{equation}

    由MGF的唯一性,
    \begin{equation} \quad Y \sim \Gamma\left(\frac{n_{1}+n_{2}}{2}, \frac{1}{2}\right)=\chi_{n_{1}+n_{2}}^{2} \end{equation}
\end{proof}

\begin{corollary}[$\chi^2$分布的数学期望和方差]
    \begin{equation}E(\chi^2) = n\end{equation}
    \begin{equation}D(\chi^2) = 2n\end{equation}
\end{corollary}

\begin{proof}
    \begin{equation}
        E(X) = 0,
        D(X) = 1,
        E(X_i^2) = 1,
        D(X_i^2) = 2
    \end{equation}
\end{proof}

\begin{corollary}[卡方分布中心极限定理]
    设$X_1,X_2,...X_n$相互独立, 都服从正态分布 N(0,1), $E(X)=\mu$, $D(X)=\sigma^2$, 则有
    \begin{equation}
        P(\frac{\chi^2(n)-n}{\sqrt{2n}} \le x) \sim \Phi(x)
    \end{equation}
\end{corollary}

\begin{proof}
    \begin{equation}
        E(X) = 0\\
        D(X) = 1\\
        E(X_i^2) = 1\\
        D(X_i^2) = 2
    \end{equation}
\end{proof}

\begin{corollary}
    \begin{equation}
        \frac{\chi^2(n)}{n} \leftarrow  \frac{1}{n} \sum^n_{i=1} X_i^2 = 1
    \end{equation}
\end{corollary}

\begin{proof}
    \begin{equation}
        E(X_i^2) = 1
    \end{equation}
\end{proof}

\begin{definition}[卡方分布的上分位点]
    \begin{equation}
        p(\chi^2 > \chi^2_\alpha(n)) = \alpha
    \end{equation}
\end{definition}

\section{t 分布}

\begin{definition}[自由度为 $t$ 的 t 分布]
    设$X \sim N(0,1), Y \sim \chi^2(n)$,且 X,Y 相互独立,则称随机变量

    \begin{equation}
        t = \frac{X}{\sqrt{Y/n}}
    \end{equation}

    为自由度为 $n$ 的 t 分布, 记作$t \sim t(n)$

\end{definition}

\subsection{t 分布性质}

\begin{corollary}[t 分布数学期望和方差]
    \begin{equation}
        E(t) = 0
    \end{equation}
    \begin{equation}
        D\big(t(n)\big) = \frac{n}{n-2}
    \end{equation}
\end{corollary}

\begin{corollary}[t 分布的概率密度函数]
    $n = 1$时 \begin{equation}f(t)=\frac{1}{\pi (1+t^2)}(柯西密度)\end{equation} 数学期望不存在。
    $n>1$时, $h(t)$的图形关于 $t=0$ 对称。
\end{corollary}

\begin{corollary}
    $n$ 足够大时 t 分布近似于 $N(0,1)$ 分布。 由于卡方分布
    \begin{equation}
        n\to \infty 时, \frac{\chi^2(n)}{n} \to 1
    \end{equation}
    所以
    \begin{equation}
        \begin{aligned}
            t(n) =
            {\frac{N(0,1)}{\sqrt{\frac{\chi^2(n)}{n}}}
            \rightarrow{n \to \infty}
            N(0,1)}
        \end{aligned}
    \end{equation}
\end{corollary}

\begin{corollary}[与 F 分布的关系]
    \begin{equation}
        \begin{aligned}
            t^2(n) & = \frac{N(0,1)^2}{\chi^2 (n)/n}                    \\
                   & = \frac{\chi ^2(1)/1}{ \chi ^2(n) / n} \sim F(1,n)
        \end{aligned}
    \end{equation}

    \begin{equation}
        \frac{1}{t^2(n)} \sim F(n,1)
    \end{equation}
\end{corollary}

\begin{definition}[t 分布的上分位点]
    \begin{equation}
        t_{1-\alpha} (n) = -t_\alpha(n)
    \end{equation}
\end{definition}

\section{F 分布}

\begin{definition}[F 分布]
    设$U \sim \chi^2(n_1), V \sim \chi^2(n_2)$, 且 U, V 相互独立, 则称随机变量

    \begin{equation}
        F = \frac{U/n_1}{V/n_2}
    \end{equation}

    服从自由度为$n_1,n_2$的 F 分布。 记作$F \sim F(n_1,n_2)$
\end{definition}

\subsection{F 分布性质}

\begin{corollary}[F 分布数学期望]
    \begin{equation}
        E(F) =\frac{{ n_2} }{{ n_2}  - 2}
    \end{equation}

    即与$n_1$无关。
\end{corollary}

\begin{corollary}[$F(n_1,n_2),F(n_2,n_1)$的关系]
    \begin{equation}
        \frac{1}{F(n_1,n_2)} \sim F(n_2,n_1)
    \end{equation}
\end{corollary}

\begin{corollary}[与 t 分布的关系]
    \begin{equation}
        \begin{aligned}
            t^2(n) & = \frac{N(0,1)^2}{\chi^2 (n)/n}                    \\
                   & = \frac{\chi ^2(1)/1}{ \chi ^2(n) / n} \sim F(1,n)
        \end{aligned}
    \end{equation}

    \begin{equation}
        \frac{1}{t^2(n)} \sim F(n,1)
    \end{equation}
\end{corollary}

\begin{corollary}[F 分布上 α 分位点的性质]
    \begin{equation}
        F_{1-\alpha} (n_1,n_2) = \frac{1}{F_\alpha(n_2,n_1)}
    \end{equation}
\end{corollary}

\section{正态总体的样本均值和样本方差的分布}

\begin{definition}[正态总体的样本均值的数学期望、方差和样本方差的数学期望]
    设总体$X\sim N(\mu,\sigma^2)$的均值为$\mu$, 方差为$σ^2$.

    $X_1,X_2,...X_n$是来自$X$的一个样本, 样本均值是$\overline{X}$,样本方差是$S^2$,则有

    \begin{equation}
        E(\overline{X}) = { \mu}
    \end{equation}

    \begin{equation}
        D(\overline{ X}) =  {\frac{\sigma^2}{n}}
    \end{equation}

    \begin{equation}
        E(S^2) = { \sigma^2}
    \end{equation}
\end{definition}

\begin{corollary}[矩估计法原理]
    \begin{equation}\overline{X}=\frac{1}{n} \sum_{i=1}^{n} X_{i} \to E(X)\end{equation}

    \begin{equation}\frac{1}{n} \sum X_1^2 - \overline{X}^2 \to D(X) = E(X^2) - E(X)^2\end{equation}
\end{corollary}

\begin{theorem}[样本均值的分布]
    设总体 X~N($\mu,\sigma^2$)的均值为$\mu$, 方差为$σ^2$, $X_1,X_2,...X_n$是来自$X$的一个样本, 样本均值是$\overline{X}$,则有

    \begin{equation}
        {\overline{X} \sim N(\mu,\sigma^2/n) }
    \end{equation}
\end{theorem}

\begin{proof}
    \begin{equation}
        \sum^n_{i=1} X_i = X_1 + X_2 + ...\sim N(n\mu, n\sigma^2)
    \end{equation}
\end{proof}

\begin{corollary}
    \begin{equation}
        \frac{\overline{X}-\mu}{\sigma^2/n} = \frac{\sqrt{n}(\overline{X}-\mu)}{\sigma} \sim N(0,1)
    \end{equation}
\end{corollary}

\begin{theorem}[样本方差的分布]
    设总体 $X\sim N(\mu,\sigma^2)$的均值为$\mu$, 方差为$σ^2$, $X_1,X_2,...X_n$是来自$X$的一个样本, 样本均值是$\overline{X}$,样本方差是$S^2$,则有

    \begin{equation}
        { \frac{(n-1)S^2}{\sigma^2} \sim \chi^2(n-1)}
    \end{equation}

    而且$\overline{X}$与$S^2$相互独立。
\end{theorem}

\begin{theorem}[样本均值和样本方差的关系]
    设$X_1,X_2,...X_n$是总体 X~N($\mu,\sigma^2$)的样本,样本均值是$\overline{X}$,样本方差是$S^2$,则有

    \begin{equation}
        {\frac{\overline{X}-\mu}{S / \sqrt{n}} = \frac{\sqrt{n}(\overline{X} - \mu)}{{ S}} \sim t(n-1)}
    \end{equation}

    对比

    \begin{equation}
        \frac{\overline{X}-\mu}{\sigma^2/n} = \frac{\sqrt{n}(\overline{X}-\mu)}{{ \sigma}} \sim N(0,1)
    \end{equation}

    但是 n 很大时

    \begin{equation}
        S^2 = \frac{1}{n} \sum_{i=1}^n X_i - (\overline{X})^2
    \end{equation}

    \begin{equation}
        S^2 \to E(X^2) - E(X)^2= \sigma^2
    \end{equation}

    \begin{equation}
        \therefore S \to \sigma
    \end{equation}
\end{theorem}

\begin{theorem}[两总体样本均值差、样本方差比的分布]
    设 $ X \sim N\left(\mu_{1}, \sigma_{1}^{2}\right), \quad Y \sim N\left(\mu_{2}, \sigma_{2}^{2}\right) $, 且 $ X $ 与Y独立,

    $ X_{1}, X_{2}, \ldots, X_{n} $ 是来自 $ X $ 的样本, $ Y_{1}, Y_{2}, \ldots, Y_{n_{2}} $ 是取自 $ Y $ 的样本,

    $ \overline{{X}} $ 和 $ \overline{{Y}} $ 分别是这两个样本的样本均值, $ {S}_{1}^{2} $ 和 $ {S}_{2}^{2} $ 分别是这两个样本的样本方差,则有

    \begin{enumerate}
        \item $\frac{S_1^2/S^2_2}{\sigma_1^2 / \sigma_2^2}$的分布
              \begin{equation}
                  {
                          \frac{S_1^2/S^2_2}{\sigma_1^2 / \sigma_2^2} \sim F(n_1 - 1, n_2 -1)}
              \end{equation}
        \item \begin{equation}
                  \frac{\overline{X} - \overline{Y}- (\mu_1 - \mu_2 )}{\sqrt{ \frac{(n_1 - 1) S_1^2 + (n_2 - 1 ) S_2^2 }{\sigma^2 (n_1+n_2-2)} }} \sim t(n_1+n_2-2)
              \end{equation}
    \end{enumerate}

\end{theorem}

\begin{proof}

    1. $\frac{S_1^2/S^2_2}{\sigma_1^2 / \sigma_2^2}$的分布
    \begin{equation}
        {
                \frac{S_1^2/S^2_2}{\sigma_1^2 / \sigma_2^2} \sim F(n_1 - 1, n_2 -1)}
    \end{equation}

    当$\sigma^2_1 = \sigma^2_2= \sigma^2$时
    \begin{equation}
        \overline{X} - \overline{Y} = N(\mu_1-\mu_2, \sigma^2 / n_1 + \sigma^2 /n_2)
    \end{equation}

    \begin{equation}
        \therefore U = \frac{\overline{X} - \overline{Y}- (\mu_1 - \mu_2 )}{\sigma \sqrt{ \frac{1}{n_1} + \frac{1}{n_2} }} \sim N(0,1)
    \end{equation}


    又因为

    \begin{equation}
        \frac{n_1 -1}{\sigma^2}S_1^2 \sim \chi^2(n_1 - 1)
    \end{equation}

    \begin{equation} \frac{n_2 -1}{\sigma^2}S_2^2 \sim \chi^2(n_2 - 1) \end{equation}

    \begin{equation}
        \therefore
        V=\frac{n_1 -1}{\sigma^2}S_1^2 + \frac{n_2 -1}{\sigma^2}S_2^2 \sim \chi^2(n_1 + n_2 -2)
    \end{equation}

    2 和 3 相互独立,根据 t 分布定义,因此有

    \begin{equation}
        \frac{\overline{X} - \overline{Y}- (\mu_1 - \mu_2 )}{S_w \sqrt{ \frac{1}{n_1} + \frac{1}{n_2} }} \sim t(n_1+n_2-2)
    \end{equation}

    代入$S_w^2 = \frac{(n_1 - 1) S_1^2 + (n_2 - 1 ) S_2^2  }{ (n_1+n_2-2)}$

    \begin{equation}
        \frac{\overline{X} - \overline{Y}- (\mu_1 - \mu_2 )}{\sqrt{ \frac{(n_1 - 1) S_1^2 + (n_2 - 1 ) S_2^2 }{\sigma^2 (n_1+n_2-2)} }} \sim t(n_1+n_2-2)
    \end{equation}
\end{proof}