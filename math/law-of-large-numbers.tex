\section{辛钦大数定律}

\begin{theorem}[辛钦大数定律]
    设随机变量序列$X_1,X_2, …$   相互独立, 服从同一分布, 具有数学期$E(X_i)=\mu, i=1,2,…$,  则对于任意正数ε , 有

\begin{equation}\lim _{n \rightarrow \infty} P\left\{\left|\frac{1}{n} \sum_{i=1}^{n} X_{i}-\mu\right|<\varepsilon\right\}=1\end{equation}

则序列$\bar{X}=\frac{1}{n} \sum_{i=1}^{n} X_{i}$ 依概率收敛于$\mu$(很可能接近于$\mu$)。
\end{theorem}

另一表述:
\begin{equation}\overline{X} \rightarrow^{P} \mu\end{equation}

\subsection{辛钦大数定律条件}

\begin{itemize}
    \item $X_1,X_2, …$ 相互独立
    \item $X_1,X_2, …$ 服从同一分布
    \item 不要求方差存在。 
\end{itemize}

\begin{corollary}
设${X_n \rightarrow^{P} a}$, ${Y_n \rightarrow^{P} b}$则

\begin{equation}
   g(X_n, Y_n) \rightarrow^{P} g(a,b)
\end{equation}
\end{corollary}

\section{伯努利大数定理}

\begin{theorem}[伯努利大数定理]
设$f_A$是n次独立重复试验中事件A的发生次数, p是每次试验中发生的概率, 则对于任意的正数$\varepsilon$.

\begin{equation}
 \lim_{n \to \infty}\{ |f_A/n - p| < \varepsilon\} = 1  
\end{equation}
\end{theorem}

贝努里大数定律表明, 当重复试验次数n充分大时, 事件A发生的频率nA/n与事件A的概率p有较大偏差的概率很小。

\section{中心极限定理}

\begin{theorem}[Lyapunov中心极限定理]
    设$X_1,X_2,...$相互独立,他们拥有数学期望和方差

\begin{equation}
 E(X_k) = \mu_k
\end{equation}

\begin{equation}
  D(X_k) = \sigma_k^2
\end{equation}
则有

\begin{equation}
\begin{aligned} \lim _{n \rightarrow \infty} F_{n}(x) &=\lim _{n \rightarrow \infty} P\left\{\frac{\sum_{k=1}^{n} X_{k}-\sum_{k=1}^{n} \mu_{k}}{B_{n}} \leq x\right\}
\\ &=\int_{-\infty}^{x} \frac{1}{\sqrt{2 \pi}} {e}^{-{t}^{2} / 2} {~d} {t}
\\&=\Phi(x)
 \end{aligned}
 \end{equation}

其中$B_n=\sqrt{D\left(\sum_{k=1}^{n} X_{k}\right)}$
\end{theorem}

\begin{theorem}
    设$X_1,X_2,...$相互独立,他们服从同一分布,拥有数学期望和方差

\begin{equation}
 E(X_k) = \mu
\end{equation}

\begin{equation}
  D(X_k) = \sigma^2
\end{equation}
则有

\begin{equation}\begin{aligned}
\lim _{n \rightarrow \infty} F_{n}(x)&=\lim _{n \rightarrow \infty} P\left\{\frac{\sum_{i=1}^{n} X_{i}-n \mu}{\sigma \sqrt{n}} \leq x\right\}
\\ &=\int_{-\infty}^{x} \frac{1}{\sqrt{2 \pi}} {e}^{-{t}^{2} / 2} {~d} {t}
\\&=\Phi(x)
\end{aligned}
\end{equation}
\end{theorem}

\begin{theorem}[De Moivre-Laplace定理]
设$\eta_{n}$服从参数为n,p的二项分布, 则对于任意$x$,有
\begin{equation}
\lim _{n \rightarrow \infty} P\left\{\frac{\eta_{n}-n p}{\sqrt{n p(1-p)}} \leq x\right\}=\int_{-\infty}^{x} \frac{1}{\sqrt{2 \pi}} e^{-\frac{t^{2}}{2}} d t=\Phi(x)
\end{equation}
\end{theorem}

\begin{proof}
    将$\eta_{n}$拆分成$n$个相互独立,服从同一分布的随机变量$X_1,X_2,...$之和
\end{proof}


