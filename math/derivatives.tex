\chapter{Derivatives}

\section{Trace}

\begin{equation} \operatorname{Tr}(\mathbf{A})=\sum_{i} A_{i i} \end{equation}

\begin{equation} \operatorname{Tr}(\mathbf{A})=\operatorname{Tr}\left(\mathbf{A}^{T}\right) \end{equation}

\begin{equation} \operatorname{Tr}(\mathbf{A B})=\operatorname{Tr}(\mathbf{B A}) \end{equation}

\begin{equation} \operatorname{Tr}(\mathbf{A}+\mathbf{B})=\operatorname{Tr}(\mathbf{A})+\operatorname{Tr}(\mathbf{B}) \end{equation}

\begin{equation} \operatorname{Tr}(\mathbf{A B C})=\operatorname{Tr}(\mathbf{B C A})=\operatorname{Tr}(\mathbf{C A B}) \end{equation}

\begin{equation} \mathbf{a}^{T} \mathbf{a}=\operatorname{Tr}\left(\mathbf{a a}^{T}\right) \end{equation}

\section{Derivatives}

This section is covering differentiation of a number of expressions with respect to a matrix $ \mathbf{X} $. 

Note that it is always assumed that $ \mathbf{X} $ has \textbf{no special structure}, i.e. that the elements of $ \mathbf{X} $ are independent (e.g. not symmetric, Toeplitz, positive definite). See section $ 2.8 $ for differentiation of structured matrices. 

\begin{theorem}
    The basic assumptions can be written in a formula as
\begin{equation}
\frac{\partial X_{k l}}{\partial X_{i j}}=\delta_{i k} \delta_{l j}
\end{equation}
\end{theorem}


that is for e.g. vector forms,
\begin{equation}
\left[\frac{\partial \mathbf{x}}{\partial y}\right]_{i}=\frac{\partial x_{i}}{\partial y} \quad\left[\frac{\partial x}{\partial \mathbf{y}}\right]_{i}=\frac{\partial x}{\partial y_{i}} \quad\left[\frac{\partial \mathbf{x}}{\partial \mathbf{y}}\right]_{i j}=\frac{\partial x_{i}}{\partial y_{j}}
\end{equation}

\begin{theorem}
    The following rules are general and very useful when deriving the differential of an expression: 
\begin{equation}
\begin{aligned}
\partial \mathbf{A} &=0 \quad(A\text{ is a constant}) \\
\partial(\alpha \mathbf{X}) &=\alpha \partial \mathbf{X} \\
\partial(\mathbf{X}+\mathbf{Y}) &=\partial \mathbf{X}+\partial \mathbf{Y} \\
\partial(\operatorname{Tr}(\mathbf{X})) &=\operatorname{Tr}(\partial \mathbf{X}) \\
\partial(\mathbf{X Y}) &=(\partial \mathbf{X}) \mathbf{Y}+\mathbf{X}(\partial \mathbf{Y}) \\
\partial(\mathbf{X} \circ \mathbf{Y}) &=(\partial \mathbf{X}) \circ \mathbf{Y}+\mathbf{X} \circ(\partial \mathbf{Y}) \\
\partial(\mathbf{X} \otimes \mathbf{Y}) &=(\partial \mathbf{X}) \otimes \mathbf{Y}+\mathbf{X} \otimes(\partial \mathbf{Y}) \\
\partial\left(\mathbf{X}^{-1}\right) &=-\mathbf{X}^{-1}(\partial \mathbf{X}) \mathbf{X}^{-1} \\
\partial(\operatorname{det}(\mathbf{X})) &=\operatorname{Tr}(\operatorname{adj}(\mathbf{X}) \partial \mathbf{X}) \\
\partial(\operatorname{det}(\mathbf{X})) &=\operatorname{det}(\mathbf{X}) \operatorname{Tr}\left(\mathbf{X}^{-1} \partial \mathbf{X}\right) \\
\partial(\ln (\operatorname{det}(\mathbf{X}))) &=\operatorname{Tr}\left(\mathbf{X}^{-1} \partial \mathbf{X}\right) \\
\partial \mathbf{X}^{T} &=(\partial \mathbf{X})^{T} \\
\partial \mathbf{X}^{H} &=(\partial \mathbf{X})^{H}
\end{aligned}
\end{equation}

\end{theorem}

\subsection{First Order}




\begin{equation} \begin{aligned} \frac{\partial \mathbf{x}^{T} \mathbf{a}}{\partial \mathbf{x}} &=\frac{\partial \mathbf{a}^{T} \mathbf{x}}{\partial \mathbf{x}}=\mathbf{a} \\ \frac{\partial \mathbf{a}^{T} \mathbf{X} \mathbf{b}}{\partial \mathbf{X}} &=\mathbf{a b}^{T} \\ \frac{\partial \mathbf{a}^{T} \mathbf{X}^{T} \mathbf{b}}{\partial \mathbf{X}} &=\mathbf{b a}^{T} \\ \frac{\partial \mathbf{a}^{T} \mathbf{X} \mathbf{a}}{\partial \mathbf{X}} &=\frac{\partial \mathbf{a}^{T} \mathbf{X}^{T} \mathbf{a}}{\partial \mathbf{X}}=\mathbf{a a}^{T} \\ \frac{\partial \mathbf{X}}{\partial X_{i j}} &=\mathbf{J}^{i j} \\ \frac{\partial(\mathbf{X} \mathbf{A})_{i j}}{\partial X_{m n}} &=\delta_{i m}(\mathbf{A})_{n j}=\left(\mathbf{J}^{m n} \mathbf{A}\right)_{i j} \\ \frac{\partial\left(\mathbf{X}^{T} \mathbf{A}\right)_{i j}}{\partial X_{m n}} &=\delta_{i n}(\mathbf{A})_{m j}=\left(\mathbf{J}^{n m} \mathbf{A}\right)_{i j} \end{aligned} \end{equation}

\section{Derivatives of Traces}

\begin{equation} \begin{aligned} \frac{\partial}{\partial \mathbf{X}} \operatorname{Tr}(\mathbf{X}) &=\mathbf{I} \\ \frac{\partial}{\partial \mathbf{X}} \operatorname{Tr}(\mathbf{X} \mathbf{A}) &=\mathbf{A}^{T} \\ \frac{\partial}{\partial \mathbf{X}} \operatorname{Tr}(\mathbf{A X B}) &=\mathbf{A}^{T} \mathbf{B}^{T} \\ \frac{\partial}{\partial \mathbf{X}} \operatorname{Tr}\left(\mathbf{A X}^{T} \mathbf{B}\right) &=\mathbf{B} \mathbf{A} \\ \frac{\partial}{\partial \mathbf{X}} \operatorname{Tr}\left(\mathbf{X}^{T} \mathbf{A}\right) &=\mathbf{A} \\ \frac{\partial}{\partial \mathbf{X}} \operatorname{Tr}\left(\mathbf{A X}^{T}\right) &=\mathbf{A} \\ \frac{\partial}{\partial \mathbf{X}} \operatorname{Tr}(\mathbf{A} \otimes \mathbf{X}) &=\operatorname{Tr}(\mathbf{A}) \mathbf{I} \end{aligned} \end{equation}

\subsection{Second Order}


\begin{equation}
\begin{aligned}
\frac{\partial}{\partial \mathbf{X}} \operatorname{Tr}\left(\mathbf{X}^{2}\right) &=2 \mathbf{X}^{T} \\
\frac{\partial}{\partial \mathbf{X}} \operatorname{Tr}\left(\mathbf{X}^{2} \mathbf{B}\right) &=(\mathbf{X B}+\mathbf{B X})^{T} \\
\frac{\partial}{\partial \mathbf{X}} \operatorname{Tr}\left(\mathbf{X}^{T} \mathbf{B} \mathbf{X}\right) &=\mathbf{B X}+\mathbf{B}^{T} \mathbf{X} \\
\frac{\partial}{\partial \mathbf{X}} \operatorname{Tr}\left(\mathbf{B X X}^{T}\right) &=\mathbf{B X}+\mathbf{B}^{T} \mathbf{X} \\
\frac{\partial}{\partial \mathbf{X}} \operatorname{Tr}\left(\mathbf{X} \mathbf{X}^{T} \mathbf{B}\right) &=\mathbf{B X}+\mathbf{B}^{T} \mathbf{X} \\
\frac{\partial}{\partial \mathbf{X}} \operatorname{Tr}\left(\mathbf{X B X}^{T}\right) &=\mathbf{X B}^{T}+\mathbf{X B} \\
\frac{\partial}{\partial \mathbf{X}} \operatorname{Tr}\left(\mathbf{B X}^{T} \mathbf{X}\right) &=\mathbf{X B}^{T}+\mathbf{X B} \\
\frac{\partial}{\partial \mathbf{X}} \operatorname{Tr}\left(\mathbf{X}^{T} \mathbf{X B}\right) &=\mathbf{X B}^{T}+\mathbf{X B} \\
\frac{\partial}{\partial \mathbf{X}} \operatorname{Tr}(\mathbf{A X B X}) &=\mathbf{A}^{T} \mathbf{X}^{T} \mathbf{B}^{T}+\mathbf{B}^{T} \mathbf{X}^{T} \mathbf{A}^{T} \\
\frac{\partial}{\partial \mathbf{X}} \operatorname{Tr}\left(\mathbf{X}^{T} \mathbf{X}\right) &=\frac{\partial}{\partial \mathbf{X}} \operatorname{Tr}\left(\mathbf{X X}^{T}\right)=\mathbf{A}=\mathbf{X}^{T} \mathbf{C}^{T} \mathbf{X B}^{T}+\mathbf{C A X B} \\
\frac{\partial}{\partial \mathbf{X}} \operatorname{Tr}\left[(\mathbf{A X B}+\mathbf{C})(\mathbf{A X B}+\mathbf{C})^{T}\right] &=2 \mathbf{A}^{T}(\mathbf{A X B}+\mathbf{C}) \mathbf{B}^{T} \\
\frac{\partial}{\partial \mathbf{X}} \operatorname{Tr}(\mathbf{X} \otimes \mathbf{X}) &=\frac{\partial}{\partial \mathbf{X}} \operatorname{Tr}(\mathbf{X}) \operatorname{Tr}(\mathbf{X})=2 \operatorname{Tr}(\mathbf{X}) \mathbf{I}(1120)
\end{aligned}
\end{equation}

\section{The Chain Rule}

Sometimes the objective is to find the derivative of a matrix which is a function of another matrix. 

\begin{theorem}
    Let $ \mathbf{U}=f(\mathbf{X}) $, the goal is to find the derivative of the function $ \mathrm{g}(\mathbf{U}) $ with respect to $ \mathbf{X} $ :
\begin{equation}
\frac{\partial g(\mathbf{U})}{\partial \mathbf{X}}=\frac{\partial g(f(\mathbf{X}))}{\partial \mathbf{X}}
\end{equation}
Then the Chain Rule can then be written the following way:
\begin{equation}
\frac{\partial g(\mathbf{U})}{\partial \mathbf{X}}=\frac{\partial g(\mathbf{U})}{\partial x_{i j}}=\sum_{k=1}^{M} \sum_{l=1}^{N} \frac{\partial g(\mathbf{U})}{\partial u_{k l}} \frac{\partial u_{k l}}{\partial x_{i j}}
\end{equation}
\end{theorem}

\begin{corollary}
    Using matrix notation, this can be written as:
\begin{equation}
\frac{\partial g(\mathbf{U})}{\partial X_{i j}}=\operatorname{Tr}\left[\left(\frac{\partial g(\mathbf{U})}{\partial \mathbf{U}}\right)^{T} \frac{\partial \mathbf{U}}{\partial X_{i j}}\right]
\end{equation}
\end{corollary}

