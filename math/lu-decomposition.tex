\chapter{LU分解}

\section{Solving Linear Equation Systems}

\subsection{Linear Equation Systems}
\begin{example}
    \begin{equation}
         A\boldsymbol{x} =\boldsymbol{b} \Leftrightarrow { \left\{\begin{matrix}
        x & + & 2y & + & 3z & = & 6\\
        2x & + & 5y & + & 2z & = & 4\\
        6x & - & 3y & + & z & = & 2
        \end{matrix}\right. }
         \end{equation}
\end{example}

见Row Picture,Column Picture的概念。

\subsection{Elimination}

\begin{example}[Elimination]
    Before

        \begin{equation} \left\{\begin{matrix}
        x & - & 2y & = & 1\\
        3x & + & 2y & = & 11
        \end{matrix}\right. \end{equation}
        
        
   
    After
    
    \begin{equation} \left\{\begin{matrix}
        x & - & 2y & = & 1\\
         &  & 8y & = & 8
        \end{matrix}\right. \end{equation}
    

\end{example}

\begin{definition}[Pivot]
    The first nonzero in the row that does elimination. \textbf{Zero is not allowed as a pivot.}
\end{definition}

\begin{definition}[Multiplier]
    (Entry to eliminate) divided by (pivot)
\end{definition}

消元法通过elimination matrices $E$进行消元操作,使得主对角线以下的元素为0. Multiply the $ j^{\text {th }} $ equation by $ \ell_{i j} $ and subtract from the $ i^{\text {th }} $ equation. (This eliminates $ x_{j} $ from equation $ i $.) We need a lot of these simple matrices $ E_{i j} $, one for every nonzero to be eliminated below the main diagonal.

\subsection{消元法的本质}

\begin{definition}[Elementary matrix, Elimination matrix]
    The elementary matrix or elimination matrix $ E_{i j} $ has the extra nonzero entry $ -\ell $ in the $ i, j $ position. Then $ E_{i j} $ subtracts a multiple $ \ell $ of row $ j $ from row $ i $.
\end{definition}

\begin{theorem}
    消元法的本质是

    \begin{equation}Ax= b \Rightarrow EAx = Eb\end{equation}
\end{theorem}

\begin{example}[Inverse of an elimination matrix]
    If $ E $ subtracts 5 times row 1 from row 2 , then $ E^{-1} $ adds 5 times row 1 to row 2 :
$$
\begin{array}{c}
\boldsymbol{E} \text { subtracts } \\
\boldsymbol{E}^{-1} \text { adds }
\end{array} \quad E=\left[\begin{array}{rll}
1 & 0 & 0 \\
-\mathbf{5} & 1 & 0 \\
0 & 0 & 1
\end{array}\right] \quad \text { and } \quad E^{-1}=\left[\begin{array}{lll}
1 & 0 & 0 \\
\mathbf{5} & 1 & 0 \\
0 & 0 & 1
\end{array}\right]
$$
Multiply $ E E^{-1} $ to get the identity matrix $ I $. Also multiply $ E^{-1} E $ to get $ I $. We are adding and subtracting the same 5 times row 1 . If $ A C=I $ then automatically $ C A=I $.
\end{example}

\begin{example}
    Suppose $ F $ subtracts 4 times row 2 from row 3 , and $ F^{-1} $ adds it back:
$$
F=\left[\begin{array}{rrr}
1 & 0 & 0 \\
0 & 1 & 0 \\
0 & -4 & 1
\end{array}\right] \text { and } F^{-1}=\left[\begin{array}{lll}
1 & 0 & 0 \\
0 & 1 & 0 \\
0 & 4 & 1
\end{array}\right] \text {. }
$$

Now multiply $ F $ by the matrix $ E $ in Example 2 to find $ F E $. Also multiply $ E^{-1} $ times $ F^{-1} $ to find $ (F E)^{-1} $. Notice the orders $ F E $ and $ E^{-1} F^{-1} $ !
$$
F E=\left[\begin{array}{rrr}
1 & 0 & 0 \\
-5 & 1 & 0 \\
20 & -4 & 1
\end{array}\right] \quad \text { is inverted by } \quad E^{-1} F^{-1}=\left[\begin{array}{ccc}
1 & 0 & 0 \\
5 & 1 & 0 \\
0 & 4 & 1
\end{array}\right]
$$


\end{example}

The result is beautiful and correct. The product $ F E $ contains  ``20'' but its inverse doesn't. $ E $ subtracts 5 times row 1 from row 2 . Then $ F $ subtracts 4 times the new row 2 (changed by row 1) from row 3 . In this order $ F E $, row 3 feels an effect from row 1 .

In the order $ E^{-1} F^{-1} $, that effect does not happen. First $ F^{-1} $ adds 4 times row 2 to row 3 . After that, $ E^{-1} $ adds 5 times row 1 to row 2 . There is no 20 , because row 3 doesn't change again. In this order $ E^{-1} F^{-1} $, row 3 feels no effect from row 1 .

This is why the next section chooses $ A=L U $, to go back from the triangular $ U $ to $ A $. \textbf{The multipliers fall into place perfectly in the lower triangular $ L $}.

\begin{definition}[Permutation matrices $P$]
    $  P_{i j} $ is the identity matrix with rows $ i $ and $ j $ reversed. When this \term{permutation matrix} $ P_{i j} $ multiplies a matrix, it exchanges rows $ i $ and $ j $.
\end{definition}


\subsection{Computing $A^{-1}$ by Gauss-Jordan Elimination}

\begin{definition}[Augmented matrix]
    \begin{equation} \left[\begin{matrix}
                A & b
            \end{matrix}\right]\end{equation}
\end{definition}

\begin{center}
    Multiply $ \left[\begin{matrix}
                A & I
            \end{matrix}\right]$ by $ A^{-1}$ to get $ \left[\begin{matrix}
                I & A^{-1}
            \end{matrix}\right]$
\end{center}

\begin{remark}
    The inverse exists if and only if elimination produces $n$ pivots. Row exchanges are allowed. Elimination solves $Ax = b$ without explicitly using the matrix $A^{-1}$.
\end{remark}

\begin{remark}
    (Important) Suppose there is a nonzero vector $ x $ such that $ A x=0 . $ Then $ A $ cannot have an inverse. No matrix can bring $ \mathbf{0} $ back to $ \boldsymbol{x} $.

If $ A $ is invertible, then $ A x=0 $ can only have the \textbf{zero solution} $ x=A^{-1} 0=0 $.
\end{remark}


\begin{example}
    \begin{equation} \begin{aligned}\left[\begin{array}{llll}K & e_{1} & e_{2} & e_{3}\end{array}\right] & =\left[\begin{array}{rrrrrr}\mathbf{2} & -\mathbf{1} & \mathbf{0} & 1 & 0 & 0 \\ -\mathbf{1} & \mathbf{2} & -\mathbf{1} & 0 & 1 & 0 \\ \mathbf{0} & -\mathbf{1} & \mathbf{2} & 0 & 0 & 1\end{array}\right] \quad \text { Start Gauss-Jordan on } K                                                  \\ & \rightarrow\left[\begin{array}{rrrrrr}2 & -1 & 0 & 1 & 0 & 0 \\ \mathbf{0} & \frac{3}{2} & -\mathbf{1} & \frac{1}{2} & \mathbf{1} & \mathbf{0} \\ 0 & -1 & 2 & 0 & 0 & 1\end{array}\right] \quad\left(\frac{\mathbf{1}}{\mathbf{2}} \text { row } 1+\text { row 2 }\right) \\ & \rightarrow\left[\begin{array}{rrrrrr}2 & -1 & 0 & 1 & 0 & 0 \\ 0 & \frac{3}{2} & -1 & \frac{1}{2} & 1 & 0 \\ \mathbf{0} & \mathbf{0} & \frac{4}{3} & \frac{1}{3} & \frac{2}{3} & \mathbf{1}\end{array}\right] \quad \left(\frac{\mathbf{2}}{3} \text { row 2 + row 3}\right)\\
               \left(\begin{array}{c}\text { Zero above } \\ \text { third pivot }\end{array}\right) & \rightarrow\left[\begin{array}{rrrrrr}2 & -1 & 0 & 1 & 0 & 0 \\ 0 & \frac{3}{2} & 0 & \frac{3}{4} & \frac{3}{2} & \frac{3}{4} \\ 0 & 0 & \frac{4}{3} & \frac{1}{3} & \frac{2}{3} & 1\end{array}\right] \quad\left(\frac{3}{4} \text{ row } \mathbf{3}+ \text{ row } \mathbf{2}\right) \\
               \left(\begin{array}{l}\text { Zero above } \\ \text { second pivot }\end{array}\right) & \rightarrow\left[\begin{array}{cccccc}2 & 0 & 0 & \frac{3}{2} & 1 & \frac{1}{2} \\ 0 & \frac{3}{2} & 0 & \frac{3}{4} & \frac{3}{2} & \frac{3}{4} \\ 0 & 0 & \frac{4}{3} & \frac{1}{3} & \frac{2}{3} & 1\end{array}\right] \quad\left(\frac{2}{3} \text{ row } 2+ \text{ row }1 \right)
        \end{aligned} \end{equation}


    然后将它变成Reduced echelon form $R$.

   
    \begin{equation} \begin{matrix}
                (\text{divided by }2)                      \\
                \left(\text{divided by }\dfrac{3}{2}\right) \\
                \left(\text{divided by }\dfrac{4}{3}\right)
            \end{matrix}\left[\begin{matrix}
                    1 & 0 & 0 & \dfrac{3}{4} & \dfrac{1}{2} & \dfrac{1}{4} \\
                    0 & 1 & 0 & \dfrac{1}{2} & 1           & \dfrac{1}{2} \\
                    0 & 0 & 1 & \dfrac{1}{4} & \dfrac{1}{2} & \dfrac{3}{4}
                \end{matrix}\right] =\left[\begin{matrix}
                    I & x_{1} & x_{2} & x_{3}
                \end{matrix}\right] =\ \left[\begin{matrix}
                    I & K^{-1}
                \end{matrix}\right]\end{equation}
    


    \begin{enumerate}
        \item $ K $ is symmetric across its main diagonal. Then $ K^{-1} $ is also symmetric.
        \item $ K $ is tridiagonal (only three nonzero diagonals). But $ K^{-1} $ is a dense matrix with no zeros. That is another reason we don't often compute inverse matrices. The inverse of a band matrix is generally a dense matrix.
        \item The product of pivots is $ 2\left(\frac{3}{2}\right)\left(\frac{4}{3}\right)=4 $. This number 4 is the determinant of $ K $. $ K^{-1} $ involves division by the determinant of $ K $.
        \begin{equation} K^{-1}=\frac{1}{4}\left[\begin{array}{lll}3 & 2 & 1 \\ 2 & 4 & 2 \\ 1 & 2 & 3\end{array}\right] \end{equation}
    \end{enumerate}


    This is why an \textbf{invertible matrix} cannot have a \textbf{zero determinant}: we need to \textbf{divide}.

\end{example}

\section{LU分解}

$ \left(E_{32} E_{31} E_{21}\right) A=U \quad $ becomes $ \quad A=\left(E_{21}^{-1} E_{31}^{-1} E_{32}^{-1}\right) U \quad $ which is $ \quad A=L U $

\begin{theorem}
    When a row of $A$ starts with zeros, so does that row of $L$.

    When a column of $A$ starts with zeros, so does that column of $U$.
\end{theorem}

\begin{example}[The key reason why $ A $ equals $ L U$]
    Ask yourself about the pivot rows that are subtracted from lower rows. Are they the original rows of $ A ? $ No, elimination probably changed them.

    Are they rows of $ U ? $ Yes, the pivot rows never change again.

    When computing the third row of $ U $, we subtract multiples of earlier rows of $ U $ (not rows of $ A ! $ ):
    \begin{equation} \text{Row 3 of }  U=(\text{Row 3 of }  A)-\ell_{31}(
        \text{Row 1 of } U)-\ell_{32}(\text{Row 2 of }  of  U) \end{equation}

    Rewrite this equation to see that the row $ \left[\begin{array}{lll}\ell_{31} & \ell_{32} & 1\end{array}\right] $ is multiplying the matrix $ U $ :

    \begin{equation} (\text{Row 3 of } A)=\ell_{31}(\text{Row 1 of }  U)+\ell_{32}(\text{Row 2 of } U)+1(\text{Row 3 of }  U) \end{equation}

    This is exactly row 3 of $ A=L U . $

    That row of $
        L $ holds $ \ell_{31}, \ell_{32}, 1 . $ All rows look like this, whatever the size of $ A $. With no row exchanges, we have $ A=L U $.
\end{example}

\begin{definition}[$A$的LU分解]
    \begin{equation} A=\left(\begin{array}{ccccc}a_{11} & \cdots & a_{1 k} & \cdots & a_{1 n} \\ \vdots & \ddots & \vdots & & \vdots \\ a_{k 1} & \cdots & a_{k k} & \cdots & a_{k n} \\ \vdots & & \vdots & \ddots & \vdots \\ a_{n 1} & \cdots & a_{n k} & \cdots & a_{n n}\end{array}\right) =LU \end{equation}

    where $ L=\left(\begin{array}{cccc}1 & 0 & \cdots & 0 \\ l_{21} & 1 & \ddots & 0 \\ \vdots & \vdots & \ddots & 0 \\ l_{n 1} & l_{n 2} & \cdots & 1\end{array}\right) , U=\left(\begin{array}{cccc}u_{11} & u_{12} & \cdots & u_{1 n} \\ 0 & u_{22} & \cdots & u_{2 n} \\ \vdots & 0 & \ddots & \vdots \\ 0 & 0 & 0 & u_{n n}\end{array}\right) $
\end{definition}

所以
\begin{equation}
    \begin{aligned}
        A & =\left(\begin{array}{ccccc}a_{11} & \cdots & a_{1 r} & \cdots & a_{1 n} \\ \vdots & \ddots & \vdots & & \vdots \\ a_{r 1} & \cdots & a_{r r} & \cdots & a_{r n} \\ \vdots & & \vdots & \ddots & \vdots \\ a_{n 1} & \cdots & a_{n r} & \cdots & a_{n n}\end{array}\right)                                               \\
          & =\left(\begin{array}{ccccc}1 & 0 & 0 & \cdots & 0 \\ \vdots & \ddots & 0 & \ddots & 0 \\ l_{r 1} & \cdots & 1 & \ddots & \vdots \\ \vdots & & \vdots & \ddots & 0 \\ l_{n 1} & \cdots & l_{n r} & \cdots & 1\end{array}\right) \cdot \left(\begin{array}{cccc}u_{11} & u_{12} & \cdots & u_{1 n} \\ 0 & u_{22} & \cdots & u_{2 n} \\ \vdots & 0 & \ddots & \vdots \\ 0 & 0 & 0 & u_{n n}\end{array}\right)
    \end{aligned}
\end{equation}

\begin{remark}
    First point: Every inverse matrix $ E^{-1} $ is lower triangular. Its off-diagonal entry is $ \ell_{i j} $, to undo the subtraction produced by $ -\ell_{i j} . $ The main diagonals of $ E $ and $ E^{-1} $ contain 1's.
\end{remark}

\begin{remark}
    Second point: Equation (2) shows a lower triangular matrix (the product of the $ E_{i j} $ ) multiplying $ A . $ It also shows all the $ E_{i j}^{-1} $ multiplying $ U $ to bring back $ A . $ This lower triangular product of inverses is $ L $.
\end{remark}

One reason for working with the inverses is that we want to factor $ A $, not $ U $. The "inverse form" gives $ A=L U $. Another reason is that we get something extra, almost more than we deserve. This is the third point, showing that $ L $ is exactly right.

\begin{remark}
    Third point: Each multiplier $ \ell_{i j} $ goes directly into its $ i, j $ position-\textbf{unchanged}-in the product of inverses which is $ L $. Usually matrix multiplication will mix up all the numbers. Here that doesn't happen. The order is right for the inverse matrices, to keep the $ \ell $'s unchanged. The reason is given below in equation (2).
Since each $ E^{-1} $ has 1's down its diagonal, the final good point is that $ L $ does too.
\end{remark}

Since each $ E^{-1} $ has 1's down its diagonal, the final good point is that $ L $ does too.

This is \textbf{elimination without row exchanges}. The upper triangular $ U $ has the pivots on its diagonal. The lower triangular $ L $ has all 1 's on its diagonal. \textbf{The multipliers $ \ell_{i j} $ are below the diagonal of $ L $}.

\subsection{$A = LDU$}

$A=LU$是不对称的。但是可以改写为对称形式。

Split $ U $ into $ \left[\begin{array}{cccc}d_{1} & & & \\ & d_{2} & & \\ & & \ddots & \\ & & & d_{n}\end{array}\right]\left[\begin{array}{cccc}1 & u_{12} / d_{1} & u_{13} / d_{1} & \cdot \\ & 1 & u_{23} / d_{2} & \cdot \\ & & \ddots & \vdots \\ & & & 1\end{array}\right] $.

\begin{example}
    
    \begin{equation}(A= LU) \left[\begin{array}{ll}1 & 0 \\ 3 & 1\end{array}\right]\left[\begin{array}{ll}2 & 8 \\ 0 & 5\end{array}\right] \end{equation} 
    
    splits further into
    
    \begin{equation}(A= LDU) \left[\begin{array}{ll}1 & 0 \\ 3 & 1\end{array}\right]\left[\begin{array}{ll}2 & \\ & 5\end{array}\right]\left[\begin{array}{ll}1 & 4 \\ 0 & 1\end{array}\right] \end{equation}.
    
\end{example}

\begin{theorem}
    当$A$是对称矩阵的时候,且消元的时候不需要行交换:

    \begin{equation}S = LDL^T\end{equation}
\end{theorem}


\begin{example}
    \begin{equation} \left[\begin{array}{ll}1 & 2 \\ 2 & 7\end{array}\right]=\left[\begin{array}{ll}\mathbf{1} & 0 \\ \mathbf{2} & \mathbf{1}\end{array}\right] \quad\left[\begin{array}{ll}1 & 0 \\ 0 & 3\end{array}\right] \quad\left[\begin{array}{ll}\mathbf{1} & \mathbf{2} \\ 0 & \mathbf{1}\end{array}\right] \end{equation}
\end{example}


\subsection{$L$、$U$矩阵的性质}

回顾矩阵乘法的定义

\begin{definition}[矩阵乘法]
    设矩阵 $ A \in \mathfrak{R}^{m \times p}, B \in \mathfrak{R}^{p \times n} $,那么矩阵$A$与$B$的乘积, 记 作C $ =A B $ ,  则矩阵 $ C \in \mathfrak{R}^{m \times n} $ 的第 $i$行第 $ j $ 列元素 $ C_{i j} $


    \begin{equation}
{C}_{i j}=\sum_{k=1}^{p} A_{i k} B_{k j}
\end{equation}
\end{definition}

对于$L$、$U$有

\begin{theorem}
    $ A $ 的第一行元素 $ a_{1 j} $ 为
    \begin{equation}
        a_{1 j}=u_{1 j}, j=1, \cdots, n
    \end{equation}
\end{theorem}

\begin{corollary}
    $ U $ 的第一行元素 $ u_{1 j} $ 为
    \begin{equation}
        u_{1 j}=a_{1 j}, j=1, \cdots, n
    \end{equation}
\end{corollary}

\begin{proof}
    \begin{equation} \begin{aligned}A & =\left(\begin{array}{ c c c c c }
                \boldsymbol{\textcolor[rgb]{0.72,0.33,0.31}{a_{11}}} & \boldsymbol{\textcolor[rgb]{0.72,0.33,0.31}{\cdots }} & \boldsymbol{\textcolor[rgb]{0.72,0.33,0.31}{a_{1r}}} & \boldsymbol{\textcolor[rgb]{0.72,0.33,0.31}{\cdots }} & \boldsymbol{\textcolor[rgb]{0.72,0.33,0.31}{a_{1n}}} \\
                \vdots                                               & \ddots                                                & \vdots                                               &                                                       & \vdots                                               \\
                a_{r1}                                               & \cdots                                                & a_{rr}                                               & \cdots                                                & a_{rn}                                               \\
                \vdots                                               &                                                       & \vdots                                               & \ddots                                                & \vdots                                               \\
                a_{n1}                                               & \cdots                                                & a_{nr}                                               & \cdots                                                & a_{nn}
            \end{array}\right)
               \\ &=\left(\begin{array}{ c c c c c }
                \boldsymbol{\textcolor[rgb]{0.72,0.33,0.31}{1}} & 0      & 0      & \cdots & 0      \\
                \vdots                                          & \ddots & 0      & \ddots & 0      \\
                l_{r1}                                          & \cdots & 1      & \cdots & \vdots \\
                \vdots                                          & \ddots & \vdots & \ddots & 0      \\
                l_{n1}                                          & \cdots & l_{nr} & \cdots & 1
            \end{array}\right) \cdot \left(\begin{array}{ c c c c c }
                \boldsymbol{\textcolor[rgb]{0.72,0.33,0.31}{u_{11}}} & \boldsymbol{\textcolor[rgb]{0.72,0.33,0.31}{\cdots }} & \boldsymbol{\textcolor[rgb]{0.72,0.33,0.31}{u_{1r}}} & \boldsymbol{\textcolor[rgb]{0.72,0.33,0.31}{\cdots }} & \boldsymbol{\textcolor[rgb]{0.72,0.33,0.31}{u_{1n}}} \\
                0                                                    & \ddots                                                & \vdots                                               & \ddots                                                & \vdots                                               \\
                0                                                    & 0                                                     & u_{rr}                                               & \cdots                                                & u_{rn}                                               \\
                \vdots                                               & \ddots                                                & 0                                                    & \ddots                                                & \vdots                                               \\
                0                                                    & \cdots                                                & 0                                                    & 0                                                     & u_{nn}
            \end{array}\right)\end{aligned}\end{equation}
\end{proof}

\begin{corollary}
    $ L $ 的第一列元素 $ l_{i1} $ 为

    \begin{equation} l_{i 1}=\frac{a_{i 1}}{u_{11}} , i=2,3, \cdots, n \end{equation}
\end{corollary}

\begin{proof}
    \begin{equation} \begin{aligned} A & =\left(\begin{array}{ c c c c c }
                \boldsymbol{\textcolor[rgb]{0.72,0.33,0.31}{a_{11}}}  & \cdots & a_{1r} & \cdots & a_{1n} \\
                \boldsymbol{\textcolor[rgb]{0.72,0.33,0.31}{\vdots }} & \ddots & \vdots &        & \vdots \\
                \boldsymbol{\textcolor[rgb]{0.72,0.33,0.31}{a_{r1}}}  & \cdots & a_{rr} & \cdots & a_{rn} \\
                \boldsymbol{\textcolor[rgb]{0.72,0.33,0.31}{\vdots }} &        & \vdots & \ddots & \vdots \\
                \boldsymbol{\textcolor[rgb]{0.72,0.33,0.31}{a_{n1}}}  & \cdots & a_{nr} & \cdots & a_{nn}
            \end{array}\right)                                               \\
                  & =\left(\begin{array}{ c c c c c }
                \boldsymbol{\textcolor[rgb]{0.72,0.33,0.31}{1}}       & 0      & 0      & \cdots & 0      \\
                \boldsymbol{\textcolor[rgb]{0.72,0.33,0.31}{\vdots }} & \ddots & 0      & \ddots & 0      \\
                \boldsymbol{\textcolor[rgb]{0.72,0.33,0.31}{l_{r1}}}  & \cdots & 1      & \ddots & \vdots \\
                \boldsymbol{\textcolor[rgb]{0.72,0.33,0.31}{\vdots }} &        & \vdots & \ddots & 0      \\
                \boldsymbol{\textcolor[rgb]{0.72,0.33,0.31}{l_{n1}}}  & \cdots & l_{nr} & \cdots & 1
            \end{array}\right) \cdot \left(\begin{array}{ c c c c c }
                \boldsymbol{\textcolor[rgb]{0.72,0.33,0.31}{u_{11}}} & \cdots & u_{1r} & \cdots & u_{1n} \\
                0                                                    & \ddots & \vdots & \ddots & \vdots \\
                0                                                    & 0      & u_{rr} & \cdots & u_{rn} \\
                \vdots                                               & \ddots & 0      & \ddots & \vdots \\
                0                                                    & \cdots & 0      & 0      & u_{nn}
            \end{array}\right)
        \end{aligned}\end{equation}
\end{proof}

\begin{theorem}
    $ A $ 的第 $ r $ 行主对角线以右元素 $ a_{r j}(j=1, \cdots, n, j \ge r) $ 为

    \begin{equation}a_{r j}=\sum_{k=1}^{r} l_{r k} u_{k j}, r=1,2, \cdots, n,j=r, \cdots, n \end{equation}
\end{theorem}

\begin{proof}
    \begin{equation}
        \begin{aligned}
            A & =\left(\begin{array}{ c c c c c c c }
            a_{11} & \cdots  & a_{1r} & \cdots  & a_{1j} & \cdots  & a_{1n}\\
            \vdots  & \ddots  & \vdots  &  & \vdots  &  & \vdots \\
            a_{r1} & \cdots  & a_{rr} & \cdots  & \boldsymbol{\textcolor[rgb]{0.72,0.33,0.31}{a}\textcolor[rgb]{0.72,0.33,0.31}{_{rj}}} & \cdots  & a_{rn}\\
            \vdots  &  & \vdots  & \ddots  & \vdots  & \ddots  & \vdots \\
            a_{n1} & \cdots  & a_{nr} & \cdots  &  & \cdots  & a_{nn}
            \end{array}\right)\\
             & =\left(\begin{array}{ c c c c c c }
            1 & 0 & 0 & 0 & \cdots  & 0\\
            \vdots  & \ddots  & 0 & 0 & \ddots  & \vdots \\
            \boldsymbol{\textcolor[rgb]{0.72,0.33,0.31}{l}\textcolor[rgb]{0.72,0.33,0.31}{_{r1}}} & \boldsymbol{\textcolor[rgb]{0.72,0.33,0.31}{\cdots }} & \boldsymbol{\textcolor[rgb]{0.72,0.33,0.31}{1}} & 0 & \cdots  & 0\\
            \vdots  & \ddots  & \vdots  & 1 & \ddots  & \vdots \\
            l_{n1} & \cdots  & l_{nr} & l_{n,r+1} & \cdots  & 1
            \end{array}\right) \cdot \left(\begin{array}{ c c c c c c c }
            u_{11} & \cdots  & u_{1r} & \cdots  & \boldsymbol{\textcolor[rgb]{0.72,0.33,0.31}{u}\textcolor[rgb]{0.72,0.33,0.31}{_{1j}}} & \cdots  & u_{1n}\\
            0 & \ddots  & \vdots  & \ddots  & \boldsymbol{\textcolor[rgb]{0.72,0.33,0.31}{\vdots }} & \cdots  & \vdots \\
            0 & 0 & u_{rr} & \cdots  & \boldsymbol{\textcolor[rgb]{0.72,0.33,0.31}{u}\textcolor[rgb]{0.72,0.33,0.31}{_{rj}}} & \cdots  & u_{rn}\\
            \vdots  & \ddots  & 0 & \ddots  & u_{r+1,j} & \cdots  & \vdots \\
            0 & \cdots  & 0 & 0 & 0 & \ddots  & u_{nn}
            \end{array}\right)
            \end{aligned}
    \end{equation}

    由于$l_{rj}, j > r$都是0,所以求和只需加和到第$r$项。
\end{proof}

\begin{corollary}
    $U$第 $ r $ 行主对角线以右元素 $ u_{r j} $

    \begin{equation} u_{r j}=a_{r j}-\sum_{k=1}^{r-1} l_{r k} u_{k j}, j = r, \cdots, n \end{equation}
\end{corollary}

\begin{proof}
    \begin{equation}\begin{aligned}
                        & a_{r j}=\sum_{k=1}^{r} l_{r k} u_{k j}, r=1,2, \cdots, n,j=r, \cdots, n                     \\
            \Rightarrow & a_{r j}=\sum_{k=1}^{r - 1} l_{r k} u_{k j} + l_{rr} u_{rj}, r=1,2, \cdots, n,j=r, \cdots, n \\
            \Rightarrow & u_{r j}=a_{r j}-\sum_{k=1}^{r-1} l_{r k} u_{k j}, j = r, \cdots, n
        \end{aligned}\end{equation}
\end{proof}


\begin{corollary}
    $U$的对角线元素$u_{r r}$

    \begin{equation} u_{r r}=a_{r r}-\sum_{k=1}^{r-1} l_{r k} u_{k r} \end{equation}
\end{corollary}


\begin{theorem}
    $ A $ 的第 $ r $ 列元素主对角线以下元素 $ a_{i r}(i=r+1, \cdots, n) $ 为

    \begin{equation}a_{i r}=\sum_{k=1}^{r} l_{i k} u_{k r}, i=r+1, \cdots, n, r=1,2, \cdots, n-1 \end{equation}
\end{theorem}

\begin{proof}
    \begin{equation}
        \begin{aligned}
            A & =\left(\begin{array}{ c c c c c c c }
            a_{11} & \cdots  & a_{1j} & \cdots  & a_{1r} & \cdots  & a_{1n}\\
            \vdots  & \ddots  & \vdots  &  & \vdots  &  & \vdots \\
            a_{r1} & \cdots  & \boldsymbol{\textcolor[rgb]{0.72,0.33,0.31}{a_{rj}}} & \cdots  & a_{rr} & \cdots  & a_{rn}\\
            \vdots  &  & \vdots  &  & \vdots  & \ddots  & \vdots \\
            a_{n1} & \cdots  & a_{nj} & \cdots  & a_{nr} & \cdots  & a_{nn}
            \end{array}\right)\\
             & =\left(\begin{array}{ c c c c c c }
            1 & 0 & 0 & 0 & \cdots  & 0\\
            \vdots  & \ddots  & 0 & 0 & \ddots  & \vdots \\
            \boldsymbol{\textcolor[rgb]{0.72,0.33,0.31}{l}\textcolor[rgb]{0.72,0.33,0.31}{_{r1}}} & \boldsymbol{\textcolor[rgb]{0.72,0.33,0.31}{\cdots }} & \boldsymbol{\textcolor[rgb]{0.72,0.33,0.31}{1}} & 0 & \cdots  & 0\\
            \vdots  & \ddots  & \vdots  & 1 & \ddots  & \vdots \\
            l_{n1} & \cdots  & l_{nr} & l_{n,r+1} & \cdots  & 1
            \end{array}\right) \cdot \left(\begin{array}{ c c c c c c c }
            u_{11} & \cdots  & \boldsymbol{\textcolor[rgb]{0.72,0.33,0.31}{u_{1j}}} & \cdots  & u_{1r} & \cdots  & u_{1n}\\
            0 & \ddots  & \boldsymbol{\textcolor[rgb]{0.72,0.33,0.31}{\vdots }} & \ddots  & \vdots  & \cdots  & \vdots \\
            0 & 0 & \boldsymbol{\textcolor[rgb]{0.72,0.33,0.31}{u_{jj}}} & \cdots  & u_{rr} & \cdots  & u_{rn}\\
            \vdots  & \ddots  & 0 & \ddots  & 0 & \cdots  & \vdots \\
            0 & \cdots  & 0 & 0 & 0 & \ddots  & u_{nn}
            \end{array}\right)
            \end{aligned}
    \end{equation}

    由于$u_{rj}, r > j$都是0,所以求和只需加和到第$r$项。
\end{proof}

\begin{corollary}
    显然, $ r=1 $ 时

    \begin{equation} a_{i 1}=l_{i 1} u_{11} , i=2,3, \cdots, n \end{equation}
\end{corollary}

\begin{corollary}
    $L$第 $ r $ 列主对角线以下元素 $ l_{i r} $

    \begin{equation} l_{i r}=\frac{a_{i r}-\sum_{k=1}^{r-1} l_{i k} u_{k r}}{u_{r r}}, i = r + 1, \cdots, n \end{equation}
\end{corollary}






\subsection{Solving $Ax = b$ Using LU Decomposition and its Complexity}
\label{Ax-eqs-b-LU}

求解 $ A x=b, A $ 为非奇异矩阵,LU算法为求解方程组 $ A x=b $ 的标准解法。

复杂度: $ \frac{2}{3} n^{3}+2 n^{2} \approx \frac{2}{3} n^{3} $ flops

\begin{algorithm}[htbp]
    \caption{Solving $Ax = b$ Using LU Decomposition}
    对矩阵 $ A $ 进行LU分解 $ \left(\frac{2}{3} n^{3} \text{flops} \right) $\;
    回代法: 求解 $ L y=b\left(n^{2}\text{flops} \right) $\;
    回代法: 求解 $ U x=y\left(n^{2}\text{flops} \right) $\;
\end{algorithm}


\subsection{Example of LU Decomposition}

\begin{example}
    对矩阵 $ A $ 进行 $ L U $ 分解
    \begin{equation}
        A=\left[\begin{array}{lll}
                8 & 2 & 9 \\
                4 & 9 & 4 \\
                6 & 7 & 9
            \end{array}\right]
    \end{equation}


    \begin{equation} A=\left[\begin{array}{lll}8 & 2 & 9 \\ 4 & 9 & 4 \\ 6 & 7 & 9\end{array}\right]=\left[\begin{array}{ccc}1 & 0 & 0 \\ l_{21} & 1 & 0 \\ l_{31} & l_{32} & 1\end{array}\right]\left[\begin{array}{ccc}u_{11} & u_{12} & u_{13} \\ 0 & u_{22} & u_{23} \\ 0 & 0 & u_{33}\end{array}\right] \end{equation}

    计算$U$的第一行和 $ L $ 的第一列

    \begin{equation} \left(u_{11}, u_{12}, u_{13}\right)=(8,2,9) , \left(l_{21}, l_{31}\right)=\left(\frac{1}{2}, \frac{3}{4}\right) \end{equation}

    然后计算$U$的第二行和$L$的第二列

    \begin{equation} u_{22}=a_{22}-l_{21} u_{12}=8 ,
        u_{23}=a_{23}-l_{21} u_{13}=-\frac{1}{2} , l_{32}=\frac{a_{32}-l_{31} u_{12}}{u_{22}}=\frac{11}{16} \end{equation}


    最后计算$U$的第三行

    \begin{equation} u_{33}=a_{33}-l_{31}  u_{13}-l_{32} u_{23}=-\frac{83}{32} \end{equation}

\end{example}

\section{Problem of LU Decomposition}


\begin{example}
    \begin{equation} A=\left[\begin{array}{ccc}1 & 0 & 0 \\ 0 & 0 & 2 \\ 0 & 1 & -1\end{array}\right]=\left[\begin{array}{ccc}1 & 0 & 0 \\ l_{21} & 1 & 0 \\ l_{31} & L_{32} & 1\end{array}\right]\left[\begin{array}{ccc}u_{11} & u_{12} & u_{13} \\ 0 & u_{22} & u_{23} \\ 0 & 0 & u_{33}\end{array}\right] \end{equation}

    计算$U$的第一行和$L$的第一列
    \begin{equation}
        \left[\begin{array}{ccc}
                1 & 0 & 0  \\
                0 & 0 & 2  \\
                0 & 1 & -1
            \end{array}\right]=\left[\begin{array}{ccc}
                1 & 0      & 0 \\
                0 & 1      & 0 \\
                0 & l_{32} & 1
            \end{array}\right]\left[\begin{array}{ccc}
                1 & 0      & 0      \\
                0 & u_{22} & u_{23} \\
                0 & 0      & u_{33}
            \end{array}\right]
    \end{equation}

    然后计算$U$的第二行和$L$的第二列
    \begin{equation}
        \begin{array}{l}
            u_{22}=a_{22}-l_{21} u_{12}=0 \\
            u_{23}=a_{23}-l_{21} u_{13}=2
        \end{array} \quad l_{32}=\frac{a_{32}-l_{31} u_{12}}{u_{22}}=\frac{1}{\boldsymbol{0}}
    \end{equation}
    即该矩阵无法 $ {LU} $ 分解!

    通过$PA=LU$分解,可以得到LU分解

    \begin{equation}P=\left[\begin{matrix}
        1 & 0 & 0\\
        0 & 1 & 0\\
        0 & 0 & 1
        \end{matrix}\right] ,L=\left[\begin{matrix}
        1 & 0 & 0\\
        0 & 1 & -1\\
        0 & 0 & 2
        \end{matrix}\right] ,U=\left[\begin{matrix}
        1 & 0 & 0\\
        0 & 0 & 1\\
        0 & 1 & 0
        \end{matrix}\right]\end{equation}
\end{example}

\section{\texorpdfstring{$PA=LU$}{PA=LU}}

\begin{theorem}
    非奇异矩阵 $ {A} \in \mathfrak{R}^{n \times n} $ ,则可分解为 $ A=P^{T} L U $

    $ P $ 是一个置换矩阵, $ L $ 为下三角矩阵并且对角线元素全为 $ 1 , U $ 为 上三角矩阵。
\end{theorem}


$PA=LU$分解方法不唯一,随着 $ P $ 的选择不同, $L$、$U$也不同。

\begin{example}[$PA=LU$]

    假设
    \begin{equation} A=\left[\begin{array}{lll}0 & 5 & 5 \\ 2 & 9 & 0 \\ 6 & 8 & 8\end{array}\right], P_{1}=\left[\begin{array}{lll}0 & 0 & 1 \\ 0 & 1 & 0 \\ 1 & 0 & 0\end{array}\right], P_{2}=\left[\begin{array}{lll}0 & 1 & 0 \\ 1 & 0 & 0 \\ 0 & 0 & 1\end{array}\right] \end{equation}

    易知
    \begin{equation}P_{1}^{T}=P_{1}^{-1}=P_{1}, P_{2}^{T}=P_{2}^{-1}=P_{2}\end{equation}

    计算可得
    \begin{equation} P_{1} A=\left[\begin{array}{lll}6 & 8 & 8 \\ 2 & 9 & 0 \\ 0 & 5 & 5\end{array}\right], P_{2} A=\left[\begin{array}{lll}2 & 9 & 0 \\ 0 & 5 & 5 \\ 6 & 8 & 8\end{array}\right] \end{equation}

    LU分解不唯一:

    \begin{equation}
        P_{1} A=\left[\begin{array}{lll}
                6 & 8 & 8 \\
                2 & 9 & 0 \\
                0 & 5 & 5
            \end{array}\right]=\left[\begin{array}{ccc}
                1           & 0             & 0 \\
                \frac{1}{3} & 1             & 0 \\
                0           & \frac{15}{19} & 1
            \end{array}\right]\left[\begin{array}{ccc}
                6 & 8            & 8              \\
                0 & \frac{19}{3} & \frac{-8}{3}   \\
                0 & 0            & \frac{135}{19}
            \end{array}\right]=L_{1} U_{1} \Rightarrow A=P_{1} L_{1} U_{1}
    \end{equation}


    \begin{equation} P_{2} A=\left[\begin{array}{lll}2 & 9 & 0 \\ 0 & 5 & 5 \\ 6 & 8 & 8\end{array}\right]=\left[\begin{array}{ccc}1 & 0 & 0 \\ 0 & 1 & 0 \\ 3 & \frac{-19}{5} & 1\end{array}\right]\left[\begin{array}{ccc}2 & 9 & 2 \\ 0 & 5 & 5 \\ 0 & 0 & 27\end{array}\right]=L_{2} U_{2} \Rightarrow A=P_{2} L_{2} U_{2} \end{equation}

\end{example}


\begin{theorem}
    这个方法等价于对$A$进行行初等变换然后对 $ P A $ 进行分解 $ P A=L U $.
\end{theorem}

\subsubsection{The Complexity of $PA = LU$}

\label{complexity:PA-eqs-LU}

复杂度: $ \frac{2}{3} n^{3} $ flops



\section{舍入误差的影响}

\begin{example}
    \begin{equation} \left[\begin{array}{cc}10^{-5} & 1 \\ 1 & 1\end{array}\right]\left[\begin{array}{l}x_{1} \\ x_{2}\end{array}\right]=\left[\begin{array}{l}1 \\ 0\end{array}\right] \end{equation}

    解得: \begin{equation} x_{1}=-\frac{1}{1-10^{-5}}, x_{2}=\frac{1}{1-10^{-5}} \end{equation}

    使用LU分解求解上述上述方程,并且使用以下两个置换矩阵:
    \begin{equation}
        P_{1}=\left[\begin{array}{ll}
                1 & 0 \\
                0 & 1
            \end{array}\right] \quad \text { or } \quad P_{2}=\left[\begin{array}{ll}
                0 & 1 \\
                1 & 0
            \end{array}\right]
    \end{equation}

    计算过程中,中间结果四舍五入到小数点后四位。

    选择1: $  P_{1}=I $.

    \begin{equation}
        \left[\begin{array}{cc}
                10^{-5} & 1 \\
                1       & 1
            \end{array}\right]=\left[\begin{array}{cc}
                1      & 0 \\
                10^{5} & 1
            \end{array}\right]\left[\begin{array}{cc}
                10^{-5} & 1        \\
                0       & 1-10^{5}
            \end{array}\right]
    \end{equation}

    $L$和$U$四舍五入到小数点后四位
    \begin{equation}
        L=\left[\begin{array}{cc}
                1      & 0 \\
                10^{5} & 1
            \end{array}\right], \quad U=\left[\begin{array}{cc}
                10^{-5} & 1       \\
                0       & -10^{5}
            \end{array}\right]
    \end{equation}

    向前回代
    \begin{equation}
        \left[\begin{array}{cc}
                1      & 0 \\
                10^{5} & 1
            \end{array}\right]\left[\begin{array}{l}
                z_{1} \\
                z_{2}
            \end{array}\right]=\left[\begin{array}{l}
                1 \\
                0
            \end{array}\right] \Rightarrow z_{1}=1, z_{2}=-10^{5}
    \end{equation}

    向后回代
    \begin{equation}
        \left[\begin{array}{cc}
                10^{-5} & 1       \\
                0       & -10^{5}
            \end{array}\right]\left[\begin{array}{l}
                x_{1} \\
                x_{2}
            \end{array}\right]=\left[\begin{array}{l}
                1 \\
                -10^{-5}
            \end{array}\right] \Rightarrow x_{1}=0, x_{2}=1
    \end{equation}

    \begin{remark}
        $ x_{1} $ 的误差为 $ 100 \% $.
    \end{remark}



    选择2:行进行交换。

    \begin{equation} \left[\begin{array}{cc}1 & 1 \\ 10^{-5} & 1\end{array}\right]=\left[\begin{array}{cc}1 & 0 \\ 10^{-5} & 1\end{array}\right]\left[\begin{array}{cc}1 & 1 \\ 0 & 1-10^{-5}\end{array}\right] \end{equation}

    $L$和$U$四舍五入到小数点后四位
    \begin{equation}
        L=\left[\begin{array}{cc}
                1       & 0 \\
                10^{-5} & 1
            \end{array}\right], \quad U=\left[\begin{array}{ll}
                1 & 1 \\
                0 & 1
            \end{array}\right]
    \end{equation}

    向前回代
    \begin{equation}
        \left[\begin{array}{cc}
                1       & 0 \\
                10^{-5} & 1
            \end{array}\right]\left[\begin{array}{l}
                z_{1} \\
                z_{2}
            \end{array}\right]=\left[\begin{array}{l}
                0 \\
                1
            \end{array}\right] \Rightarrow z_{1}=0, z_{2}=1
    \end{equation}

    向后回代
    \begin{equation}
        \left[\begin{array}{ll}
                1 & 1 \\
                0 & 1
            \end{array}\right]\left[\begin{array}{l}
                x_{1} \\
                x_{2}
            \end{array}\right]=\left[\begin{array}{l}
                0 \\
                1
            \end{array}\right] \Rightarrow x_{1}=-1, x_{2}=1
    \end{equation}

    \begin{remark}
        $ x_{1}, x_{2} $ 的误差约为 $ 10^{-5} $.
    \end{remark}

\end{example}


不同置换矩阵 $ P $ ,算法可能导致产生不同的误差的结果; 由于数值存储存在误差:
第一种 $ P_{1} $ 行交换,算法不稳定;
第二种 $ P_{2} $ 行交换, 算法是稳定得到 “准确” 近似解;

在数值分析中,一些比较简单的规则去挑选置换矩阵 $ P  $, 使 得算法结果比较稳定。
\begin{equation}
    \left[\begin{array}{cc}
            10^{-5} & 1 \\
            1       & 1
        \end{array}\right]\left[\begin{array}{l}
            x_{1} \\
            x_{2}
        \end{array}\right]=\left[\begin{array}{l}
            1 \\
            0
        \end{array}\right] \quad\left[\begin{array}{cc}
            1       & 1 \\
            10^{-5} & 1
        \end{array}\right]\left[\begin{array}{l}
            x_{1} \\
            x_{2}
        \end{array}\right]=\left[\begin{array}{l}
            0 \\
            1
        \end{array}\right]
\end{equation}





\section{稀疏线性方程组}

\begin{theorem}
    如果矩阵 $ {A} $ 是稀疏矩阵, 则它一般可以被分解为
    \begin{equation}
        A=P_{1} L U P_{2}
    \end{equation}

    矩阵 $ P_{1}, P_{2} $ 都为置换矩阵。
\end{theorem}

\begin{corollary}
    对矩阵 $ {A} $ 进行行变换和列变换得到: $ \tilde{A}=P_{1}^{T} A P_{2}^{T} $

    然后进行分解: $ \tilde{{A}}=L U $.
\end{corollary}



$ P_{1} $ 和 $ P_{2} $ 的选择会影响 $ {L} $ 和$U$的稀疏度。

