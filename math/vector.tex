\chapter{对向量的介绍}

\section{Vector}

\begin{definition}[Vector]
    一个有序的数字列表.

    \( \left[\begin{array}{c}-1.1 \\ 0.0 \\ 3.6 \\ -7.2\end{array}\right] \) 或者 \( \quad\left(\begin{array}{c}-1.1 \\ 0.0 \\ 3.6 \\ -7.2\end{array}\right) \) 或者 \( \quad(-1.1,0,3.6,-7.2) \)
\end{definition}

表中的数字是\textit{元素}(\textit{项、系数、分量})。元素的数量是向量的\textit{大小}(\textit{维数,长度})。大小为n的向量称为\textit{$n$维向量}。
向量中的数字通常被称作\textit{标量}。

用符号来表示向量,比如$\alpha$ ,$b$,一般小写字母表示. 其它表示形式 $\boldsymbol{g}, \vec{a}$

\begin{definition}[n维向量 \( a \) 的第 \( i \) 元素]
    n维向量 \( a \) 的第i \( i \) 元素表示为 \( a_{i} \).

    有时i指的是向量列表中的第i个向量.
\end{definition}

\begin{definition}[$a=b$]
    对于所有$i$,如果有$a_i = b_i$,则称两个相同大小的向量$a$和$b$是相等的,可写成$a = b$
\end{definition}

\begin{definition}[stacked vector]
    假设$b$、$c$、$d$是大小为$m$、$n$、$p$的向量
    
    $$ a=\left[\begin{array}{l}b \\ C \\ d\end{array}\right] $$

    $$ a=\left(b_{1}, b_{2}, \ldots, b_{m}, c_{1}, c_{2}, \ldots, c_{n}, d_{1}, d_{2}, \ldots, d_{p}\right) $$
\end{definition}

\begin{definition}[零向量]
    所有项为0的n维向量表示为$0_n$或者$0$
\end{definition}

\begin{definition}[全一向量]
      所有项为1的n维向量表示为$𝟏_n$或者$1$
\end{definition}

\begin{definition}[单位向量]
    当第i项为1,其余项为0时表示为$e_i$

    $$ {e}_{1}=\left[\begin{array}{l}1 \\ 0 \\ 0\end{array}\right], \quad e_{2}=\left[\begin{array}{l}0 \\ 1 \\ 0\end{array}\right], \quad e_{3}=\left[\begin{array}{l}0 \\ 0 \\ 1\end{array}\right] $$
\end{definition}

\begin{definition}[稀疏向量]
    如果一个向量的许多项都是0,该向量为稀疏(Sparse)的。稀疏向量能在计算机上高效地存储和操作。

$\operatorname{nnz}(x)$是指向量$x$中非零的项数(number of non-zeros),有时用 $\ell_0$表示 。

\end{definition}

向量 \( x=\left(x_{1}, x_{2}\right) \) 可以在二维中表示一个位置或一个位移、 图像、 单词统计等。

\section{Vector Space}

\begin{definition}[向量空间$V$]
    设 \( V \) 是非空子集, \( P \) 是一数域, 向量空间$V$满足:

    \begin{enumerate}
        \item 向量加法: \( V+V \rightarrow V \), 记作 \( \forall x, y \in V \), 则 \( x+y \in V( \) 加法封闭)
        \item 标量乘法: \( F \times V \rightarrow V \), 记作 \( \forall x \in V, \lambda \in P \), 则 \( \lambda x \in V( \) 乘法封闭)
    \end{enumerate}

上述两个运算满足下列八条规则 \( (\forall x, y, z \in V, \lambda, \mu \in P) \) 
\begin{enumerate}
    \item \( x+y=y+x \) (交换律) 
    \item \( x+(y+z)=(x+y)+z \) (结合律)
    \item \( V \) 存在一个零元素, 记作0, \( x+0=x \)
    \item 存在 \( x \) 的负元素,记作 \( -x \), 满足 \( x+(-x)=0 \)
    \item \( \forall x \in V \), 都有 \( 1 x=x, 1 \in P \)
    \item \( \lambda(\mu x)=(\lambda \mu) x \)
    \item \( (\lambda+\mu) x=\lambda x+\mu x \)
    \item \(  \lambda(x+y)=\lambda x+\lambda y \)
\end{enumerate}
\end{definition}

\begin{corollary}
    向量空间也称为线性空间.
\end{corollary}

\begin{corollary}
    如果 \( x, y \in \mathbb{R}^{2} \), 则 \( x+y \in \mathbb{R}^{2}, \lambda x \in \mathbb{R}^{2}(\lambda \in \mathbb{R}) \)
\end{corollary}

\section{向量运算}

\begin{definition}[向量加法]
    $n$维向量$a$和$b$可以相加,求和形式表示为$a + b$

    设向量 \( a, {b}, {C} \) 是向量空间 \( V \) 的元素,即 \( a, {b}, {c} \in V_{\text {。 }} \)

\begin{enumerate}
    \item 交换律: \( a+b=b+a \)
    \item 结合律: \( (a+b)+c=a+(b+c) \) (因此可写成 \( a+{b}+{c}) \)
    \item \( a+0=0+a=a \)
    \item \( a-a=0 \)
\end{enumerate}
\end{definition}

\begin{corollary}[向量位移相加]
    如果二维向量$a$和$b$都表示位移,则它们的位移之和为$a + b$
\end{corollary}

\begin{definition}[标量与向量的乘法]
    $$ \beta a=\left[\begin{array}{c}\beta a_{1} \\ \vdots \\ \beta a_{n}\end{array}\right] $$

    标量 \( \beta, \gamma \) 与向量 \( a 、 b \)
\begin{enumerate}
    \item 结合律: \( (\beta \gamma) a=\beta(\gamma a) \)
    \item 左分配律: \( (\beta+\gamma) a=\beta a+\gamma a \)
    \item 右分配律: \( \beta(a+b)=\beta a+\beta b \)
\end{enumerate}
\end{definition}

\begin{definition}[线性组合]
    对于向量 \( a_{1}, \ldots, a_{m} \) 和标量 \( \beta_{1}, \ldots, \beta_{m} \),
    $$ \beta_{1} a_{1}+\cdots+\beta_{m} a_{m} $$
    是向量的线性组合。\( \beta_{1}, \ldots, \beta_{m} \) 是该向量的\textit{系数}。
\end{definition}

\begin{example}
    对于任何向量 \( b \in \mathbb{R}^{n} \), 有如下等式
    $$ b=b_{1} e_{1}+\cdots+b_{n} e_{n}, b=\left[\begin{array}{c}b_{1} \\ b_{2} \\ \vdots \\ b_{n}\end{array}\right] $$
\end{example}

\section{内积}

\begin{definition}[内积]
    在数域 \( \mathbb{R} \) 上的向量空间 \( V \), 定义函数 \( \langle\cdot,\cdot\rangle:V \times V \rightarrow \mathbb{R} \), 满足:

    \begin{enumerate}
        \item $ \langle{a}, {a}\rangle \geq 0, \forall {a} \in V $, 当且仅当 $a=0$ 时 $ \langle a, a\rangle=0 $
        \item \( \langle\alpha {a}+\beta {b}, c\rangle=\alpha\langle{a}, c\rangle+\beta\langle{b}, c\rangle, \forall \alpha, \beta \in \mathbb{R} \), 且 \( {a}, {b}, c \in V \)
        \item \( \langle{a}, {b}\rangle=\langle{b}, {a}\rangle, \forall {a}, {b} \in V \)
    \end{enumerate}

    函数 \( \langle\cdot,\cdot\rangle:V \times V \rightarrow \mathbb{R} \)成为内积。
\end{definition}

\begin{example}
    在向量空间 \( \mathbb{R}^{n} \) 上, 计算两个向量对应项相乘之后求和函数

    \[ \langle a, b\rangle=a_{1} b_{1}+a_{2} b_{2}+\cdots+a_{n} b_{n}=a^{T}_{b} \]
\( a=\left[\begin{array}{c}a_{1} \\ a_{2} \\ \vdots \\ a_{n}\end{array}\right], b=\left[\begin{array}{c}b_{1} \\ b_{2} \\ \vdots \\ b_{n}\end{array}\right] \in \mathbb{R}^{n} \)
\end{example}

\begin{proof}
    \( \langle a, a\rangle=a_{1} a_{1}+a_{2} a_{2}+\cdots+a_{n} a_{n}=\sum_{i=1}^{n} a_{i}^{2} \geq 0,\langle a, a\rangle=0 \), 则 \( a=0 \)

    \(\begin{aligned} \langle\alpha a+\beta {b}, {c}\rangle &=\left(\alpha a_{1}+\beta b_{1}\right) c_{1}+\left(\alpha a_{2}+\beta b_{2}\right) c_{2}+\cdots+\left(\alpha a_{n}+\beta b_{n}\right) c_{n} 
    \\ &=\alpha \sum_{i=1}^{n} a_{i} c_{i}+\beta \sum_{i=1}^{n} b_{i} c_{i}
    \\ &=\alpha\langle a, c\rangle+\beta\langle b, c\rangle\end{aligned} \)

    $$ \langle a, b\rangle=a^{{T}} b=b^{{T}} a=\langle b, a\rangle $$
\end{proof}

内积的性质:交换律、结合律、分配律。

交换律: \( a^{T} b=b^{T} a \)

结合律: \( (\gamma a)^{T} b=\gamma\left(a^{T} b\right) \)

分配律: \( (a+b)^{T} c=a^{T} c+b^{T} c \)

\subsection{常用的内积等式}
\begin{corollary}[选出第$i$项]
    $$ e_{i}^{T} a=a_{i} $$
\end{corollary}

\begin{corollary}[向量每一项之和]
    $$ \mathbf{1}^{T} a=a_{1}+\cdots+a_{n} $$
\end{corollary}

\begin{corollary}[向量每一项的平方和]
    $$ a^{T} a=a_{1}^{2}+\cdots+a_{n}^{2} $$
\end{corollary}

\section{Cauchy-Schwartz Inequality}
\begin{theorem}[Cauchy-Schwartz Inequality]
    设 \( \langle \cdot,\cdot \rangle \) 是向量空间 \( V \) 上的内积, \( \forall x, y \in V \), 则有

    $$
|\langle x, y\rangle|^{2} \leq\langle x, x\rangle\langle y, y\rangle
$$
\end{theorem}

\begin{proof}
    令 \( \lambda \in \mathbb{R} \), 则有 \( 0 \leq\langle x+\lambda y, x+\lambda y\rangle \) \( =\langle x, x\rangle+\lambda\langle y, x\rangle+\lambda\langle x, y\rangle+\lambda^{2}\langle y, y\rangle \) \( =\langle x, x\rangle+2 \lambda\langle y, x\rangle+\lambda^{2}\langle y, y\rangle \) 
    
    则有 \( \lambda^{2}\langle y, y\rangle+2 \lambda\langle y, x\rangle+\langle x, x\rangle \geq 0, \forall \lambda \in \mathbb{R} . \)

    \( \nabla=(2\langle y, x\rangle)^{2}-4\langle y, y\rangle\langle x, x\rangle \leq 0 \)
    
\( |\langle x, y\rangle|^{2} \leq\langle x, x\rangle\langle y, y\rangle \)

当$ |\langle x, y\rangle|^{2}=\langle x, x\rangle\langle y, y\rangle $ 时, 有 $ \left\langle x, x\rangle^2 +2 \lambda\langle y, x\rangle+\lambda^{2}\langle y, y\rangle=0\right. $

也即 \( \langle x+\lambda y, x+\lambda y\rangle=0 \), 因此 \( x+\lambda y=0 \), 即 \( x=-\lambda y \).
\end{proof}

\section{浮点运算}

计算机以浮点格式存储(实)数值。

基本的算术运算(加法,乘法等)被称为浮点运算(flop)。

算法或操作的时间复杂度:作为输入维数的函数所需要的浮点运算总数。

算法复杂度通常以非常粗略地近似估算。

(程序)执行时间的粗略估计:计算机速度/flops

目前的计算机大约是$1$Gflops/秒($10^9$flops/秒)

\begin{corollary}
    假设有$n$维向量$x$和$y$:

    \begin{itemize}
        \item $x+y$需要$n$次加法,所以时间复杂度为 ($n$)flops。
        \item $x^T y$ 需要$n$次乘法和$n - 1$次加法,所以时间复杂度为$(2n - 1)$flops。
        \item 对于$x^T y$,通常将其时间复杂度简化为$2n$,甚至为$n$。
        \item 当$x$或$y$是稀疏的时候,算法的实际运算时间会比理论时间更少。
    \end{itemize}
\end{corollary}



