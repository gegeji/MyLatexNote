\chapter{Complexity}

\section{Vector Operations}

In the table below, $ x $ and $ y $ are $ n $-vectors and $ a $ is a scalar.

\begin{equation} \begin{array}{ll}a x & n \\ x+y & n \\ x^{T} y & 2 n \\ \|x\| & 2 n \\ \|x-y\| & 3 n \\ \operatorname{rms}(x) & 2 n \\ \operatorname{std}(x) & 4 n \\ \angle(x, y) & 6 n\end{array} \end{equation}


The convolution $ a * b $ of an $ n $-vector $ a $ and $ m $-vector $ b $ can be computed by a special algorithm that requires $ 5(m+n) \log _{2}(m+n) $ flops.

\section{Matrix Operations}

In the table below, $ A $ and $ B $ are $ m \times n $ matrices, $ C $ is an $ m \times p $ matrix, $ x $ is an $ n $-vector, and $ a $ is a scalar.

\begin{equation} \begin{array}{ll}a A & m n \\ A+B & m n \\ A x & 2 m n \\ A C & 2 m n p \\ A^{T} A & m n^{2} \\ \|A\| & 2 m n\end{array} \end{equation}

\section{Factorization and Inverses}

In the table below, $ A $ is a tall or square $ m \times n $ matrix, $ R $ is an $ n \times n $ triangular matrix, and $ b $ is an $ n $-vector. We assume the factorization or inverses exist; in particular in any expression involving $ A^{-1}, A $ must be square.

\begin{equation} \begin{array}{ll}{QR} \text { factorization of } A & 2 m n^{2} \\ R^{-1} b & n^{2} \\ A^{-1} b & 2 n^{3} \\ A^{-1} & 3 n^{3} \\ A^{\dagger} & 3 m n^{2}\end{array} \end{equation}

The pseudo-inverse $ A^{\dagger} $ of a wide $ m \times n $ matrix (with linearly independent rows) can be computed in $ 3 m^{2} n $ flops.

\section{Solving Least Squares Problems}

In the table below, $ A $ is an $ m \times n $ matrix, $ C $ is a wide $ p \times n $ matrix, and $ b $ is an $ m $-vector. We assume the associated independence conditions hold.

\begin{equation} \begin{array}{lll}\operatorname{minimize} & \|A x-b\|^{2} & 2 m n^{2} \\ \operatorname { minimize } & \|A x-b\|^{2} \text { subject to } C x=d & 2(m+p) n^{2}+2 n p^{2} \\ \operatorname { minimize } &\|x\|^{2} \text { subject to } C x=d & 2 n p^{2}\end{array} \end{equation}

\section{Big-Times-Small-Squared Mnemonic}

Many of the complexities listed above that involve two dimensions can be remembered using a simple mnemonic: The cost is order
\begin{equation}
\text { (big) } \times(\text { small })^{2} \text { flops, }
\end{equation}
where `big' and `small' refer to big and small problem dimensions. We list some examples below.

\begin{itemize}
    \item Computing the Gram matrix of a tall $ m \times n $ matrix requires $ m n^{2} $ flops. Here $ m $ is the big dimension and $ n $ is the small dimension.
    \item In the QR factorization of an $ m \times n $ matrix, we have $ m \geq n $, so $ m $ is the big dimension and $ n $ is the small dimension. The complexity is $ 2 m n^{2} $ flops.
    \item Computing the pseudo-inverse $ A^{\dagger} $ of an $ m \times n $ matrix $ A $ when $ A $ is tall (and has independent columns) costs $ 3 m n^{2} $ flops. When $ A $ is wide (and has independent rows), it is $ 3 n m^{2} $ flops.
    \item For least squares, we have $ m \geq n $, so $ m $ is the big dimension and $ n $ is the small dimension. The cost of computing the least squares approximate solution is $ 2 m n^{2} $ flops.
    \item For the least norm problem, we have $ p \leq n $, so $ n $ is the big dimension and $ p $ is the small dimension. The cost is $ 2 n p^{2} $ flops.
    \item The constrained least squares problem involves two matrices $ A $ and $ C $, and three dimensions that satisfy $ m+p \geq n $. The numbers $ m+p $ and $ n $ are the big and small dimensions of the stacked matrix $ \left[\begin{array}{c}A \\ C\end{array}\right] $. The cost of solving the constrained least squares problem is $ 2(m+p) n^{2}+2 n p^{2} $ flops, which is between $ 2(m+p) n^{2} $ flops and $ 4(m+p) n^{2} $ flops, since $ n \leq m+p $.
\end{itemize}

\section{本书中其他列出时间复杂度的章节、定理}

\begin{center}
    \begin{table}[htbp]
    \begin{tabular}{|l|l|l|}
    \hline
    \textbf{算法} & \textbf{页码} & \textbf{章节(定理、推论)编号} \\ \hline
        向量运算        &   \pageref{complexity:vector-operation} &  \cref{complexity:vector-operation}      \\ \hline
        矩阵运算        &     \pageref{complexity:matrix-operations} &  \cref{complexity:matrix-operations}     \\ \hline
        前向回代        &   \pageref{complexity:Backward-Substitution} &   \cref{complexity:Backward-Substitution}       \\ \hline
        后向回代        &   \pageref{complexity:forward-substitution} & \cref{complexity:forward-substitution}         \\ \hline
        用Gram-Schmidt求解QR分解        &    \pageref{complexity:gram-schmidt-qr}  &  \cref{complexity:gram-schmidt-qr}     \\ \hline
        Householder变换求解$Hx$    &    \pageref{complexity:Hx} &  \cref{complexity:Hx}      \\ \hline
        Householder算法中求解$H_k x$        &    \pageref{complexity:Hkx} &   \cref{complexity:Hkx}     \\ \hline
        Householder算法    &    \pageref{complexity:householder}  &   \cref{complexity:householder}     \\ \hline
        求解上、下三角矩阵的逆        &    \pageref{complexity:inverse-of-triangular}&  \cref{complexity:inverse-of-triangular}       \\ \hline
        QR分解求解最小二乘法        &    \pageref{complexity:least-square-using-qr} &  \cref{complexity:least-square-using-qr}      \\ \hline
        求解$PA=LU$        &    \pageref{complexity:PA-eqs-LU} &  \cref{complexity:PA-eqs-LU}     \\ \hline
        用QR分解求解$Ax=b$        &    \pageref{complexity:qr-solves-Ax-eqs-b} &    \cref{complexity:qr-solves-Ax-eqs-b}    \\ \hline
        用QR分解求解$A^{-1}$         &    \pageref{complexity:qr-solves-inverse-of-A} & \cref{complexity:qr-solves-inverse-of-A}      \\ \hline
        用QR分解求解最小范数优化问题        &    \pageref{complexity:qr-solves-min-norm} & \cref{complexity:qr-solves-min-norm}      \\ \hline
        计算列正交矩阵$A$的矩阵向量乘法$Ax$        &    \pageref{complexity:Qx}     &  \cref{complexity:Qx}  \\ \hline
        QR分解求解线性方程组        &    \pageref{complexity:linear-equation-qr} &  \cref{complexity:linear-equation-qr}      \\ \hline
            用LU分解求解KKT    &      \pageref{complexity:kkt-lu}  & \cref{complexity:kkt-lu}    \\ \hline
        KKT        &    \pageref{complexity:kkt} &   \cref{complexity:kkt}     \\ \hline
    \end{tabular}
    \end{table}
\end{center}
