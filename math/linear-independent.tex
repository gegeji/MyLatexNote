\chapter{Linear Independence}

\section{线性相关、线性无关}

\begin{definition}[线性相关(linearly dependent)]
    对于向量 $ a_{1}, \ldots a_{m} \in \mathbb{R}^{n} $, 如果存在不全为零的数 $ \beta_{1}, \ldots \beta_{m} \in \mathbb{R} $, 使得
\begin{equation}
\beta_{1} a_{1}+\cdots+\beta_{m} a_{m}=0
\end{equation}

则称向量 $ a_{1}, \ldots a_{m} $ 是\term{线性相关(linearly dependent)}. 
\end{definition}

\begin{corollary}
    线性相关等价于至少有一个向量 $ a_{i} $ 是其它向量的线性组合。 

    Equivalently, at least one vector $ a_{i} $ is a linear combination of the other vectors:
\begin{equation}
a_{i}=-\frac{x_{1}}{x_{i}} a_{1}-\cdots-\frac{x_{i-1}}{x_{i}} a_{i-1}-\frac{x_{i+1}}{x_{i}} a_{i+1}-\cdots-\frac{x_{n}}{x_{i}} a_{n}
\end{equation}
if $ x_{i} \neq 0 $.
\end{corollary}


\begin{corollary}
    the vector $0$ can be written as a nontrivial linear combination of $ a_{1}, \ldots, a_{n} $.
\end{corollary}


\begin{corollary}
    向量集 $ \left\{a_{1}\right\} $ 是线性相关的, 当且仅当 $ a_{1}=0 $ . 
\end{corollary}
\begin{corollary}
    向量集 $ \left\{a_{1}, a_{2}\right\} $ 是线性相关的,  当且仅当其中一个 $ a_{1}=\beta a_{2}, \beta \neq 0 $ . 
\end{corollary}

\begin{definition}[线性独立 (linearly independent)]
    \label{Def:LinearIndependence}
    如果$n$维向量集 $ \left\{a_{1}, \ldots, a_{m}\right\} $ 不是线性相关的, 即\term{线性独立 (linearly dependent)}, 也称\term{线性无关},  即:
\begin{equation}
\beta_{1} a_{1}+\cdots+\beta_{m} a_{m}=0
\end{equation}
当且仅当 $ \beta_{1}=\cdots=\beta_{m}=0 $ , 上述等式成立。 
\end{definition}

线性无关等价于不存在一个向量 $ a_{i} $ 是其它向量的线性组合。 

\begin{corollary}
    一个$n$维向量集最多有$n$个线性无关的向量。
\end{corollary}

\begin{corollary}
    如果 $ {n} $ 维向量集有 $ {n}+1 $ 个向量, 那它们必线性相关。
\end{corollary}

\begin{example}
    $n$维单位向量 $ e_{1}, \ldots, e_{n} $ 是线性独立的。 
\end{example}

\begin{example}
    \begin{equation} a_{1}=\left[\begin{array}{c}1 \\ -2 \\ 0\end{array}\right], \quad a_{2}=\left[\begin{array}{c}-1 \\ 0 \\ 1\end{array}\right], \quad a_{3}=\left[\begin{array}{l}0 \\ 1 \\ 1\end{array}\right] \end{equation}

    \begin{equation} \beta_{1} a_{1}+\beta_{2} a_{2}+\beta_{3} a_{3}=\left[\begin{array}{c}\beta_{1}-\beta_{2} \\ -2 \beta_{1}+\beta_{3} \\ \beta_{2}+\beta_{3}\end{array}\right]=0 \end{equation}

    \begin{equation} \beta_{1}=\beta_{2}=\beta_{3}=0 \end{equation}
\end{example}


\begin{corollary}
    \begin{equation}
A=\left[\begin{array}{llll}
a_{1} & a_{2} & \cdots & a_{n}
\end{array}\right]
\end{equation}
has linearly independent columns if
\begin{equation}
A x=0 \quad \Longrightarrow \quad x=0
\end{equation}
\end{corollary}

\begin{theorem}
    假设 $ x $ 是线性无关向量 $ a_{1}, \ldots, a_{k} $ 的线性组合:
\begin{equation}
x=\beta_{1} a_{1}+\cdots \beta_{k} a_{k}
\end{equation}
则其系数 $ \beta_{1}, \ldots \beta_{k} $ 是唯一的, 即如果有:

\begin{equation}
x=\gamma_{1} a_{1}+\cdots \gamma_{k} a_{k}
\end{equation}
则对于 $ i=1, \ldots k $, 有 $ \beta_{i}=\gamma_{i} $ . 
\end{theorem}

\begin{proof}
\begin{equation}
\left(\beta_{1}-\gamma_{1}\right) a_{1}+\cdots\left(\beta_{k}-\gamma_{k}\right) a_{k}=x-x=0
\end{equation}

由于向量 $ a_{1}, \ldots, a_{k} $ 线性无关, 有 $ \beta_{1}-\gamma_{1}=\beta_{k}-\gamma_{k}=0 $ . 

所以线性组合的系数是唯一的。
\end{proof}

\section{Basis}

\begin{definition}[基 (Basis)]
    $n$个线性独立的$n$维向量 $ a_{1}, \ldots, a_{n} $ 的集合。
\end{definition}

\begin{definition}[向量 $ b $ 在基底 $ a_{1}, \ldots, a_{n} $ 下的分解]
    任何一个$n$维向量 $ b $ 都可以用它们的线性组合来表示

\begin{equation}
b=\beta_{1} a_{1}+\cdots+\beta_{n} a_{n}
\end{equation}
\end{definition}

\begin{proof}
    同一向量的系数是唯一的。 
\end{proof}

\begin{example}
    $ e_{1}, \ldots, e_{n} $ 是一组基, 那么 $ b $ 在此基底下的分解为

    \begin{equation} b=b_{1} e_{1}+\cdots+b_{n} e_{n} ,b=\left[\begin{array}{c}b_{1} \\ \vdots \\ b_{n}\end{array}\right] \in \mathbb{R}^{n} \end{equation}
\end{example}

\section{标准正交向量}

\begin{definition}[Orthogonal Vectors]
    \label{Def:OrthogonalVectors}
    在$n$维向量集 $ a_{1}, \ldots, a_{k} $ 中,  如果对于 $ i \neq j $, 都有 $ a_{i} \perp a_{j} $ ,  则称它们相互\term{正交(orthogonal)}. 
\end{definition}

\begin{definition}[Orthonormal Vectors]
    \label{Def:OrthonormalVectors}
    如果$n$维向量集 $ a_{1}, \ldots, a_{k} $ 相互正交, 且每个向量的模长都为单位长度 1 ,  即对于 $ i=1, \ldots k $, 有 $ \left\|a_{i}\right\|_{2}^{2}=1 $, 则称它们是\term{标准正交 (orthonormal)}的。 

    \begin{equation} a_{i}^{T} a_{j}=\left\{\begin{array}{ll}1 & i=j \\ 0 & i \neq j\end{array}\right. \end{equation}
\end{definition}

\begin{corollary}
    标准正交的向量集是线性无关的。 
\end{corollary}

\begin{theorem}
    if $ n $ vectors $ a_{1}, a_{2}, \ldots, a_{k} $ of length $ n $ are linearly independent, then
\begin{equation}
n \leq m
\end{equation}

    (根据线性无关的性质, 必有向量集向量个数 $ k \leq n $.)
\end{theorem}

\begin{proof}
    The proof is by induction on the dimension $n$.

    First consider a linearly independent collection $ a_{1}, \ldots, a_{k} $ of 1-vectors. We must have $ a_{1} \neq 0 $. This means that every element $ a_{i} $ of the collection can be expressed as a multiple $ a_{i}=\left(a_{i} / a_{1}\right) a_{1} $ of the first element $ a_{1} $. This contradicts linear independence unless $ k=1 $.

    Next suppose $ n \geq 2 $ and the independence-dimension inequality holds for dimension $ n-1 $. 
    
    Let $ a_{1}, \ldots, a_{k} $ be a linearly independent list of $ n $-vectors. We need to show that $ k \leq n $. We partition the vectors as

    \begin{equation}
    a_{i}=\left[\begin{array}{r}
    b_{i} \\
    \alpha_{i}
    \end{array}\right], \quad i=1, \ldots, k
    \end{equation}
    where $ b_{i} $ is an $ (n-1) $-vector and $ \alpha_{i} $ is a scalar.

    First suppose that $ \alpha_{1}=\cdots=\alpha_{k}=0 $. 
    
    Then the vectors $ b_{1}, \ldots, b_{k} $ are linearly independent: $ \sum_{i=1}^{k} \beta_{i} b_{i}=0 $ holds if and only if $ \sum_{i=1}^{k} \beta_{i} a_{i}=0 $, which is only possible for $ \beta_{1}=\cdots=\beta_{k}=0 $ because the vectors $ a_{i} $ are linearly independent. The vectors $ b_{1}, \ldots, b_{k} $ therefore form a linearly independent collection of $ (n-1) $ -vectors. By the induction hypothesis we have $ k \leq n-1 $, so certainly $ k \leq n $.

    Next suppose that the scalars $ \alpha_{i} $ are not all zero.
    
    Assume $ \alpha_{j} \neq 0 . $ We define a collection of $ k-1 $ vectors $ c_{i} $ of length $ n-1 $ as follows:

    \begin{equation}c_{i}=\left\{\begin{matrix} 
        b_{i}-\frac{\alpha_{i}}{\alpha_{j}} b_{j}, \quad i=1, \ldots, j-1  \\  
       b_{i+1}-\frac{\alpha_{i+1}}{\alpha_{j}} b_{j}, \quad i=j, \ldots, k-1  
     \end{matrix}\right. \end{equation}

     These $ k-1 $ vectors are linearly independent: If $ \sum_{i=1}^{k-1} \beta_{i} c_{i}=0 $ then

    \begin{equation}
        \label{eqn:k-leq-n}
        \sum_{i=1}^{j-1} \beta_{i}\left[\begin{array}{c}
        b_{i} \\
        \alpha_{i}
        \end{array}\right]+\gamma\left[\begin{array}{c}
        b_{j} \\
        \alpha_{j}
        \end{array}\right]+\sum_{i=j+1}^{k} \beta_{i-1}\left[\begin{array}{c}
        b_{i} \\
        \alpha_{i}
        \end{array}\right]=0
    \end{equation}

    with
    \begin{equation}
    \gamma=-\frac{1}{\alpha_{j}}\left(\sum_{i=1}^{j-1} \beta_{i} \alpha_{i}+\sum_{i=j+1}^{k} \beta_{i-1} \alpha_{i}\right)
    \end{equation}

    Since the vectors $ a_{i}=\left(b_{i}, \alpha_{i}\right) $ are linearly independent, the \cref{eqn:k-leq-n} only holds when all the coefficients $ \beta_{i} $ and $ \gamma $ are all zero. This in turns implies that the vectors $ c_{1}, \ldots, c_{k-1} $ are linearly independent. By the induction hypothesis $ k-1 \leq n-1 $ so we have established that $ k \leq n $.
\end{proof}

\begin{corollary}
    If an $ m \times n $ matrix has linearly independent columns then $ m \geq n $. 
\end{corollary}

\begin{corollary}
    If an $ m \times n $ matrix has linearly independent rows then $ m \leq n $.
\end{corollary}

\begin{definition}[$n$维向量的一个标准正交基]
    当 $ k=n $ 时,  $ a_{1}, \ldots, a_{n} $ 是 $ n $ 维向量的一个\term{标准正交基}. 
\end{definition}

\begin{definition}[ $ x $ 在标准正交基下的标准正交分解]
    如果 $ a_{1}, \ldots, a_{n} $ 是一个标准正交基, 对于任意维向量 $ x $
\begin{equation}
x=\left(a_{1}^{T} x\right) a_{1}+\cdots+\left(a_{n}^{T} x\right) a_{n}
\end{equation}
则称其为 \term{$ x $ 在标准正交基下的标准正交分解}. 
\end{definition}

    这个分解可以用于计算不同标准正交基下的系数。 

\begin{proof}
    由于正交向量的性质
    \begin{equation} a_{i}^{T} a_{j}=\left\{\begin{array}{ll}1 & i=j \\ 0 & i \neq j\end{array}\right. \end{equation}

    所以

    \begin{equation} a_{i}^{T} x=\left(a_{1}^{T} x\right) a_{i}^{T} a_{1}+\cdots+\left(a_{i}^{T} x\right) a_{i}^{T} a_{i}+\cdots+\left(a_{n}^{T} x\right) a_{i}^{T} a_{n}=a_{i}^{T} x \end{equation}
\end{proof}

\section{Gram-Schmidt Algorithm}
\label{Chap:Gram-Schmidt Algorithm}
\begin{algorithm}[htbp]
    \caption{Gram-Schmidt Algorithm}
    \KwIn{$ \mathrm{n} $ 维向量 $ a_{1}, \ldots, a_{k} $}
    \KwOut{若这些向量线性无关,返回标准正交基$ q_{1}, \ldots, q_{k} $;若线性相关时判断 $a_j$ 是 $ a_{1}, \ldots, a_{j-1} $ 的线性组合 }
    $ q_{1}= \dfrac{a_{1}}{\left\|a_{1}\right\|_{2}}   $\;
    \While(){$i=2,\cdots,k$}{
        正交化: $ \tilde{q}_{i}=a_{i}-\left(q_{1}^{T} a_{i}\right) q_{1}-\cdots-\left(q_{i-1}^{T} a_{i}\right) q_{i-1} $\;
        检验线性相关:如果 $ \tilde{q}_{i}=0 $, 提前退出迭代\;
        单位化: $
        q_{i}=\dfrac{\tilde{q}_{i}}{\left\|\tilde{q}_{i}\right\|_{2}}$\;
    }
\end{algorithm}

如果步骤2中未提前结束迭代, 那么 $ a_{1}, \ldots, a_{k} $ 是线性独立的, 而且 $ q_{1}, \ldots, q_{k} $ 是标准正交基。 

如果在第$j$次迭代中提前结束, 说明 $ a_{j} $ 是 $ a_{1}, \ldots, a_{j-1} $ 的线性组合, 因此 $ a_{1}, \ldots, a_{k} $ 是线性相关的。 

\begin{theorem}
    \label{thm: qs-are-orthogonal}
    Gram-Schmidt正交化算法得到的
    $q_{1}, \ldots, q_{i-1}, q_{i} $ 是标准正交的。 
\end{theorem}

\begin{proof}
    假设第 $ i-1 $ 次迭代成立,  即: \begin{equation} \quad q_{r} \perp q_{s}, \forall r, s<i \end{equation}

    正交化步骤保证有以下关系成立
    \begin{equation} \tilde{q}_{i}=a_{i}-\left(q_{1}^{T} a_{i}\right) q_{1}-\cdots-\left(q_{i-1}^{T} a_{i}\right) q_{i-1} \end{equation}

    等式两边同时乘以 $ q_{j}^{T}, j=1, \ldots, i-1 $
    \begin{equation} \begin{aligned} q_{j}^{T} \tilde{q}_{i} &=q_{j}^{T} a_{i}-\left(q_{1}^{T} a_{i}\right)\left(q_{j}^{T} q_{1}\right)-\cdots-\left(q_{i-1}^{T} a_{i}\right)\left(q_{j}^{T} q_{i-1}\right) \\ &=q_{j}^{T} a_{i}-q_{j}^{T} a_{i}
        \\ &=0  \end{aligned} \end{equation}

    $ \because q_{j}^{T} q_{r}=0, j \neq r, q_{j}^{T} q_{j}=1 $

     $\therefore \tilde{q}_{i} \perp q_{1}, \ldots, \tilde{q}_{i} \perp q_{i-1} $.

    单位化步骤保证了 $
    q_{i}=\dfrac{\tilde{q}_{i}}{\left\|\tilde{q}_{i}\right\|_{2}}$, 即 $ q_{1}, \ldots, q_{i} $ 是标准正交。 
\end{proof}

\begin{algorithm}[htbp]
    \caption{Gram-Schmidt Algorithm for Three Vectors}
    \KwIn{Three independent vectors $ \boldsymbol{a}, \boldsymbol{b}, \boldsymbol{c} $}
    \KwOut{Three orthonormal vectors $ \boldsymbol{q}_{1}=\boldsymbol{A} /\|\boldsymbol{A}\|, \boldsymbol{q}_{2}=\boldsymbol{B} /\|\boldsymbol{B}\|, \boldsymbol{q}_{3}=\boldsymbol{C} /\|\boldsymbol{C}\| $.}
    Choose $ \boldsymbol{A}=\boldsymbol{a} $\;
    $ \boldsymbol{B}=\boldsymbol{b}-\frac{\boldsymbol{A}^{\mathrm{T}} \boldsymbol{b}}{\boldsymbol{A}^{\mathrm{T}} \boldsymbol{A}} \boldsymbol{A} $\;
    $ \boldsymbol{C}=\boldsymbol{c}-\frac{\boldsymbol{A}^{\mathrm{T}} \boldsymbol{c}}{\boldsymbol{A}^{\mathrm{T}} \boldsymbol{A}} \boldsymbol{A}-\frac{\boldsymbol{B}^{\mathrm{T}} \boldsymbol{c}}{\boldsymbol{B}^{\mathrm{T}} \boldsymbol{B}} \boldsymbol{B}   $\;
    单位化\;
\end{algorithm}

\subsection{The Analysis of Gram-Schmidt Algorithm}

假设Gram-Schmidt 正交法未在第$i$次迭代提前终止。

\begin{corollary}
    $ a_{i} $ 是 $ q_{1}, \ldots, q_{i} $ 的一个线性组合。
    
    \begin{equation} a_{i}=\left\|\tilde{q}_{i}\right\|_{2} q_{i}+\left(q_{1}^{T} a_{i}\right) q_{1}+\cdots+\left(q_{i-1}^{T} a_{i}\right) q_{i-1} \end{equation}
\end{corollary}

\begin{proof}
    由
    \begin{equation} \tilde{q}_{i}=a_{i}-\left(q_{1}^{T} a_{i}\right) q_{1}-\cdots-\left(q_{i-1}^{T} a_{i}\right) q_{i-1} \end{equation}

    所以
    \begin{equation}a_{i}= \tilde{q}_{i}+\left(q_{1}^{T} a_{i}\right) q_{1}+\cdots+\left(q_{i-1}^{T} a_{i}\right) q_{i-1} \end{equation}

    注意有性质: $
    q_{i}=\dfrac{\tilde{q}_{i}}{\left\|\tilde{q}_{i}\right\|_{2}}$, 即 $ q_{1}, \ldots, q_{i} $.

    \begin{equation} a_{i}=\left\|\tilde{q}_{i}\right\|_{2} q_{i}+\left(q_{1}^{T} a_{i}\right) q_{1}+\cdots+\left(q_{i-1}^{T} a_{i}\right) q_{i-1} \end{equation}
\end{proof}


则有 

\begin{corollary}
    \begin{equation}q_{i} = \frac{a_{i}-\left(q_{1}^{T} a_{i}\right) q_{1}-\cdots-\left(q_{i-1}^{T} a_{i}\right) q_{i-1}}{\left\|\tilde{q}_{i}\right\|_{2}}\end{equation}
\end{corollary}


\begin{corollary}
    $ q_{i} $ 是 $ a_{1}, \ldots, a_{i} $ 的一个线性组合。
\end{corollary}

\begin{proof}
    归纳假设, 每个 $ q_{i-1} $ 都是 $ a_{1}, \ldots, a_{i-1} $ 的线性组合:

    \begin{equation} \begin{aligned}q_{2}&= \frac{a_{2}-\left(q_{1}^{T} a_{2}\right) q_{1}}{\left\|\tilde{q}_{2}\right\|_{2}}
        \\ &=
        \frac{a_{2}-\left(q_{1}^{T} a_{2}\right) \frac{ a_{1} }{\left\|a_{1}\right\|_{2}} }{\left\|\tilde{q}_{2}\right\|_{2}}
    \end{aligned} \end{equation}

\begin{equation} q_{3}=
\frac{a_{3}-\left(q_{1}^{T} a_{3}\right) q_{1}-\left(q_{2}^{T} a_{3}\right) q_{2}}{\left\|\tilde{q}_{3}\right\|_2 }  \end{equation}

通过对 $ i $ 的归纳证明,可得$ q_{i} $ 是 $ a_{1}, \ldots, a_{i} $ 的线性组合。
\end{proof}


假设Schmidt正交法在第$j$次迭代提前终止。

\begin{corollary}
    $ a_{j} $ 是 $ q_{1}, \ldots, q_{j-1} $ 的一个线性组合。

\begin{equation} a_{j}=\left(q_{1}^{T} a_{j}\right) q_{1}+\cdots+\left(q_{j-1}^{T} a_{j}\right) q_{j-1} \end{equation}
\end{corollary}


\begin{proof}
    \begin{equation}\begin{aligned}
        \tilde{q}_{i} &=a_{i}-\left(q_{1}^{T} a_{i}\right) q_{1}-\cdots-\left(q_{i-1}^{T} a_{i}\right) q_{i-1} \\
        0 &=a_{i}-\left(q_{1}^{T} a_{i}\right) q_{1}-\cdots-\left(q_{i-1}^{T} a_{i}\right) q_{i-1}
    \end{aligned}\end{equation}

    所以
    \begin{equation} a_{i}=\left\|\tilde{q}_{i}\right\|_{2} q_{i}+\left(q_{1}^{T} a_{i}\right) q_{1}+\cdots+\left(q_{i-1}^{T} a_{i}\right) q_{i-1} \end{equation}
\end{proof}

\begin{corollary}
    $ a_{j} $ 是 $ a_{1}, \ldots, a_{j-1} $ 的线性组合。
\end{corollary}

\begin{proof}
    每一个 $ q_{1}, \ldots, q_{j-1} $ 都是 $ a_{1}, \ldots, a_{j-1} $ 的线性组合。

    因此 $ a_{j} $ 是 $ a_{1}, \ldots, a_{j-1} $ 的线性组合。
\end{proof}
