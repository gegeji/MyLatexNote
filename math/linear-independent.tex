\chapter{Linear Independence}

\section{线性相关、线性无关}

\begin{definition}[线性相关(linearly dependent)]
    定义:对于向量 $ a_{1}, \ldots a_{m} \in \mathbb{R}^{n} $, 如果存在不全为零的数 $ \beta_{1}, \ldots \beta_{m} \in \mathbb{R} $, 使得
$$
\beta_{1} a_{1}+\cdots+\beta_{m} a_{m}=0
$$

则称向量 $ a_{1}, \ldots a_{m} $ 是线性相关(linearly dependent). 
\end{definition}

线性相关等价于至少有一个向量 $ a_{i} $ 是其它向量的线性组合. 

\begin{corollary}
    向量集 $ \left\{a_{1}\right\} $ 是线性相关的,当且仅当 $ a_{1}=0 $ . 

    向量集 $ \left\{a_{1}, a_{2}\right\} $ 是线性相关的, 当且仅当其中一个 $ a_{1}=\beta a_{2}, \beta \neq 0 $ . 
\end{corollary}

\begin{definition}[线性独立 (linearly dependent)]
    如果n维向量集 $ \left\{a_{1}, \ldots, a_{m}\right\} $ 不是线性相关的,即线性独立 (linearly dependent), 也称线性无关, 即:
$$
\beta_{1} a_{1}+\cdots+\beta_{m} a_{m}=0
$$
当且仅当 $ \beta_{1}=\cdots=\beta_{m}=0 $ ,上述等式成立. 
\end{definition}

线性无关等价于不存在一个向量 $ a_{i} $ 是其它向量的线性组合. 

\begin{corollary}
    注:一个n维向量集最多有n个线性无关的向量, 也就是说如果 $ \mathrm{n} $ 维 向量集有 $ \mathrm{n}+1 $ 个向量,那它们必线性相关
\end{corollary}

\begin{example}
    n维单位向量 $ e_{1}, \ldots, e_{n} $ 是线性独立的. 
\end{example}

\begin{example}
    $$ a_{1}=\left[\begin{array}{c}1 \\ -2 \\ 0\end{array}\right], \quad a_{2}=\left[\begin{array}{c}-1 \\ 0 \\ 1\end{array}\right], \quad a_{3}=\left[\begin{array}{l}0 \\ 1 \\ 1\end{array}\right] $$

    $$ \beta_{1} a_{1}+\beta_{2} a_{2}+\beta_{3} a_{3}=\left[\begin{array}{c}\beta_{1}-\beta_{2} \\ -2 \beta_{1}+\beta_{3} \\ \beta_{2}+\beta_{3}\end{array}\right]=0 $$

    $$ \beta_{1}=\beta_{2}=\beta_{3}=0 $$
\end{example}

\begin{theorem}
    假设 $ x $ 是线性无关向量 $ a_{1}, \ldots, a_{k} $ 的线性组合:
$$
x=\beta_{1} a_{1}+\cdots \beta_{k} a_{k}
$$
则其系数 $ \beta_{1}, \ldots \beta_{k} $ 是唯一的,即如果有:

$$
x=\gamma_{1} a_{1}+\cdots \gamma_{k} a_{k}
$$
则对于 $ i=1, \ldots k $, 有 $ \beta_{i}=\gamma_{i} $ . 
\end{theorem}

\begin{proof}
    系数是唯一的原因:
$$
\left(\beta_{1}-\gamma_{1}\right) a_{1}+\cdots\left(\beta_{k}-\gamma_{k}\right) a_{k}=x-x=0
$$

由于向量 $ a_{1}, \ldots, a_{k} $ 线性无关,有 $ \beta_{1}-\gamma_{1}=\beta_{k}-\gamma_{k}=0 $ . 
\end{proof}

\section{basis}

\begin{definition}[基 (basis)]
    n个线性独立的n维向量 $ a_{1}, \ldots, a_{n} $ 的集合
\end{definition}

\begin{definition}[向量 $ b $ 在基底 $ a_{1}, \ldots, a_{n} $ 下的分解]
    任何一个n维向量 $ b $ 都可以用它们的线性组合来表示

$$
b=\beta_{1} a_{1}+\cdots+\beta_{n} a_{n}
$$
\end{definition}

\begin{proof}
    同一向量的系数是唯一的. 
\end{proof}

\begin{example}
    $ e_{1}, \ldots, e_{n} $ 是一组基,那么 $ b $ 在此基底下的分解为

    $$ b=b_{1} e_{1}+\cdots+b_{n} e_{n} ,b=\left[\begin{array}{c}b_{1} \\ \vdots \\ b_{n}\end{array}\right] \in \mathbb{R}^{n} $$
\end{example}

\section{标准正交向量}

\begin{definition}[Orthogonal Vectors]
    在n维向量集 $ a_{1}, \ldots, a_{k} $ 中, 如果对于 $ i \neq j $, 都有 $ a_{i} \perp a_{j} $ , 则称它们相互正交(orthogonal). 
\end{definition}

\begin{definition}[Orthonormal Vectors]
    如果n维向量集 $ a_{1}, \ldots, a_{k} $ 相互正交,且每个向量的模长都为单位长度 1 , 即对于 $ i=1, \ldots k $, 有 $ \left\|a_{i}\right\|_{2}^{2}=1 $, 则称它们是标准正交 (orthonormal)的. 

    $$ a_{i}^{T} a_{j}=\left\{\begin{array}{ll}1 & i=j \\ 0 & i \neq j\end{array}\right. $$
\end{definition}

\begin{corollary}
    标准正交的向量集是线性无关的. 
\end{corollary}

\begin{corollary}
    根据线性无关的性质,必有向量集向量个数 $ k \leq n $
\end{corollary}

\begin{definition}[$n$维向量的一个标准正交基]
    当 $ k=n $ 时, $ a_{1}, \ldots, a_{n} $ 是 $ n $ 维向量的一个标准正交基. 
\end{definition}

\begin{definition}[ $ x $ 在标准正交基下的标准正交分解]
    如果 $ a_{1}, \ldots, a_{n} $ 是一个标准正交基, 对于任意维向量 $ x $;
$$
x=\left(a_{1}^{T} x\right) a_{1}+\cdots+\left(a_{n}^{T} x\right) a_{n}
$$
则称其为 $ x $ 在标准正交基下的标准正交分解. 
\end{definition}

    这个分解可以用于计算不同标准正交基下的系数. 

\begin{proof}
    $$ a_{i}^{T} x=\left(a_{1}^{T} x\right) a_{i}^{T} a_{1}+\cdots+\left(a_{i}^{T} x\right) a_{i}^{T} a_{i}+\cdots+\left(a_{n}^{T} x\right) a_{i}^{T} a_{n}=a_{i}^{T} x $$
\end{proof}

\section{Gram-Schmidt Algorithm}

\begin{algorithm}[htbp]
    \caption{Gram-Schmidt Algorithm}
    \KwIn{$ \mathrm{n} $ 维向量 $ a_{1}, \ldots, a_{k} $}
    \KwOut{若这些向量线性无关时 $ q_{1}, \ldots, q_{k} $(标准正交基);若线性相关时判断 $a_j$ 是 $ a_{1}, \ldots, a_{j-1} $ 的线性组合 }
    $ q_{1}=a_{1} /\left\|a_{1}\right\|_{2} $\;
    \While(){$i=2,\cdots,k$}{
        正交化: $ \widetilde{q}_{i}=a_{i}-\left(q_{1}^{T} a_{i}\right) q_{1}-\cdots-\left(q_{i-1}^{T} a_{i}\right) q_{i-1} $\;
        检验线性相关:如果 $ \widetilde{q}_{i}=0 $, 提前退出迭代\;
        单位化: $ q_{i}=\widetilde{q}_{i} /\left\|\widetilde{q}_{i}\right\|_{2} $\;
    }
\end{algorithm}

如果步骤2中未提前结束迭代,那么 $ a_{1}, \ldots, a_{k} $ 是线性独立的,而且 $ q_{1}, \ldots, q_{k} $ 是标准正交基. 

如果在第$j$次迭代中提前结束,说明 $ a_{j} $ 是 $ a_{1}, \ldots, a_{j-1} $ 的线性组合, 因此 $ a_{1}, \ldots, a_{k} $ 是线性相关的. 

\begin{theorem}
    $q_{1}, \ldots, q_{i-1}, q_{i} $ 是标准正交的. 
\end{theorem}

\begin{proof}
    假设第 $ i-1 $ 次迭代成立, 即: $ \quad q_{r} \perp q_{s}, \forall r, s<i $.

    正交化步骤保证有以下关系成立
    $$ \widetilde{q}_{i}=a_{i}-\left(q_{1}^{T} a_{i}\right) q_{1}-\cdots-\left(q_{i-1}^{T} a_{i}\right) q_{i-1} $$

    等式两边同时乘以 $ q_{j}^{T}, j=1, \ldots, i-1 $
    $$ \begin{aligned} q_{j}^{T} \tilde{q}_{i} &=q_{j}^{T} a_{i}-\left(q_{1}^{T} a_{i}\right)\left(q_{j}^{T} q_{1}\right)-\cdots-\left(q_{i-1}^{T} a_{i}\right)\left(q_{j}^{T} q_{i-1}\right) \\ &=q_{j}^{T} a_{i}-q_{j}^{T} a_{i}=0  \end{aligned} $$

    $ \because q_{j}^{T} q_{r}=0, j \neq r, q_{j}^{T} q_{j}=1 $

     $\therefore \widetilde{q}_{i} \perp q_{1}, \ldots, \widetilde{q}_{i} \perp q_{i-1} $.

    单位化步骤保证了 $ q_{i}=\widetilde{q}_{i} /\left\|\widetilde{q}_{i}\right\|_{2} $, 即 $ q_{1}, \ldots, q_{i} $ 是标准正交. 
\end{proof}

\begin{algorithm}
    \caption{Gram-Schmidt Algorithm (Another Algorithm)}
    \KwIn{Three independent vectors $ \boldsymbol{a}, \boldsymbol{b}, \boldsymbol{c} $}
    \KwOut{Three orthonormal vectors $ \boldsymbol{q}_{1}=\boldsymbol{A} /\|\boldsymbol{A}\|, \boldsymbol{q}_{2}=\boldsymbol{B} /\|\boldsymbol{B}\|, \boldsymbol{q}_{3}=\boldsymbol{C} /\|\boldsymbol{C}\| $.}
    Choose $ \boldsymbol{A}=\boldsymbol{a} $\;
    $$ \boldsymbol{B}=\boldsymbol{b}-\frac{\boldsymbol{A}^{\mathrm{T}} \boldsymbol{b}}{\boldsymbol{A}^{\mathrm{T}} \boldsymbol{A}} \boldsymbol{A} $$\;
    $$ \boldsymbol{C}=\boldsymbol{c}-\frac{\boldsymbol{A}^{\mathrm{T}} \boldsymbol{c}}{\boldsymbol{A}^{\mathrm{T}} \boldsymbol{A}} \boldsymbol{A}-\frac{\boldsymbol{B}^{\mathrm{T}} \boldsymbol{c}}{\boldsymbol{B}^{\mathrm{T}} \boldsymbol{B}} \boldsymbol{B}   $$\;
    单位化\;
\end{algorithm}

\subsection{The Analysis of Gram-Schmidt Algorithm}

假设Gram-Schmidt 正交法未在第$i$次迭代提前终止:

\begin{corollary}
    $ a_{i} $ 是 $ q_{1}, \ldots, q_{i} $ 的一个线性组合.
    
    $$ a_{i}=\left\|\tilde{q}_{i}\right\|_{2} q_{i}+\left(q_{1}^{T} a_{i}\right) q_{1}+\cdots+\left(q_{i-1}^{T} a_{i}\right) q_{i-1} $$
\end{corollary}

\begin{proof}
    $$ \widetilde{q}_{i}=a_{i}-\left(q_{1}^{T} a_{i}\right) q_{1}-\cdots-\left(q_{i-1}^{T} a_{i}\right) q_{i-1} $$

    $$a_{i}= \widetilde{q}_{i}+\left(q_{1}^{T} a_{i}\right) q_{1}+\cdots+\left(q_{i-1}^{T} a_{i}\right) q_{i-1} $$

    注意有性质: $ q_{i}=\widetilde{q}_{i} /\left\|\widetilde{q}_{i}\right\|_{2} $.

    $$ a_{i}=\left\|\tilde{q}_{i}\right\|_{2} q_{i}+\left(q_{1}^{T} a_{i}\right) q_{1}+\cdots+\left(q_{i-1}^{T} a_{i}\right) q_{i-1} $$
\end{proof}


则有 

\begin{corollary}
    $$q_{i} = \frac{a_{i}-\left(q_{1}^{T} a_{i}\right) q_{1}-\cdots-\left(q_{i-1}^{T} a_{i}\right) q_{i-1}}{\left\|\tilde{q}_{i}\right\|_{2}}$$
\end{corollary}


\begin{corollary}
    $ q_{i} $ 是 $ a_{1}, \ldots, a_{i} $ 的一个线性组合.
\end{corollary}

\begin{proof}
    归纳假设,每个 $ q_{i-1} $ 都是 $ a_{1}, \ldots, a_{i-1} $ 的线性组合:

    $$ \begin{aligned}q_{2}&= \frac{a_{2}-\left(q_{1}^{T} a_{2}\right) q_{1}}{\left\|\tilde{q}_{2}\right\|_{2}}
        \\ &=
        \frac{a_{2}-\left(q_{1}^{T} a_{2}\right) \frac{ a_{1} }{\left\|a_{1}\right\|_{2}} }{\left\|\tilde{q}_{2}\right\|_{2}}
    \end{aligned} $$

$$ q_{3}=
\frac{a_{3}-\left(q_{1}^{T} a_{3}\right) q_{1}-\left(q_{2}^{T} a_{3}\right) q_{2}}{\left\|\tilde{q}_{3}\right\|_2 }  $$

通过对 $ i $ 的归纳证明,可得$ q_{i} $ 是 $ a_{1}, \ldots, a_{i} $ 的一个线性组合.
\end{proof}


假设Schmidt正交法在第$j$次迭代提前终止:

\begin{corollary}
    $ a_{j} $ 是 $ q_{1}, \ldots, q_{j-1} $ 的一个线性组合.

$$ a_{j}=\left(q_{1}^{T} a_{j}\right) q_{1}+\cdots+\left(q_{j-1}^{T} a_{j}\right) q_{j-1} $$
\end{corollary}


\begin{proof}
    $$ \widetilde{q}_{i}=a_{i}-\left(q_{1}^{T} a_{i}\right) q_{1}-\cdots-\left(q_{i-1}^{T} a_{i}\right) q_{i-1} $$

    $$ 0=a_{i}-\left(q_{1}^{T} a_{i}\right) q_{1}-\cdots-\left(q_{i-1}^{T} a_{i}\right) q_{i-1} $$

    $$ a_{i}=\left\|\tilde{q}_{i}\right\|_{2} q_{i}+\left(q_{1}^{T} a_{i}\right) q_{1}+\cdots+\left(q_{i-1}^{T} a_{i}\right) q_{i-1} $$
\end{proof}

\begin{corollary}
    $ a_{j} $ 是 $ a_{1}, \ldots, a_{j-1} $ 的线性组合.
\end{corollary}

\begin{proof}
    每一个 $ q_{1}, \ldots, q_{j-1} $ 都是 $ a_{1}, \ldots, a_{j-1} $ 的线性组合.

    因此 $ a_{j} $ 是 $ a_{1}, \ldots, a_{j-1} $ 的线性组合.
\end{proof}
