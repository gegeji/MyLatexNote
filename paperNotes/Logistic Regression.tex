\chapter{Logistic Regression}

\begin{definition}[Logistic Distribution]
    设 $ X $ 是连续随机变量, $ X $ 服从逻辑斯谛分布是指 $ X $ 具有下列分布函数和密度函数:

    $$
\begin{array}{l}
F(x)=P(X \leqslant x)=\frac{1}{1+\mathrm{e}^{-(x-\mu) / \gamma}} \\
f(x)=F^{\prime}(x)=\frac{\mathrm{e}^{-(x-\mu) / \gamma}}{\gamma\left(1+\mathrm{e}^{-(x-\mu) / \gamma}\right)^{2}}
\end{array}
$$
式中, $ \mu $ 为位置参数, $ \gamma>0 $ 为形状参数.
\end{definition}



\begin{definition}[Logistic Regression]
    $$ P(Y=1 \mid x)=\frac{\exp (w \cdot x+b)}{1+\exp (w \cdot x+b)} $$
$$ P\left(Y=0\{x)=\frac{1}{1+\exp (w \cdot x+b)}\right. $$

这里, $\quad x \in \mathbf{R}^{n}$ 是输入, $Y \in\{0,1\}$ 是输出, $w \in \mathbf{R}^{n}$ 和 $b \in \mathbf{R}$ 是参数, $w$ 称为权值向 量, $b$ 称为偏置, $w \cdot x$ 为 $w$ 和 $x$ 的内积.
\end{definition}

有时为了方便, 将权值向量和输入向量加以扩充, 仍记作 $ w, x $, 即 $ w=\left(w^{(1)}\right. $, $ \left.w^{(2)}, \cdots, w^{(n)}, b\right)^{\mathrm{T}}, x=\left(x^{(1)}, x^{(2)}, \cdots, x^{(n)}, 1\right)^{\mathrm{T}} $. 这时, 逻辑斯谛回归模型如下:

$$ P(Y=1 \mid x)=\frac{\exp (w \cdot x)}{1+\exp (w \cdot x)} $$

$$ P(Y=0 \mid x)=\frac{1}{1+\exp (w \cdot x)} $$

\begin{definition}[几率]
    如果一个事件发生的的概率是$p$, 那么这个事件的几率是 $ \frac{p}{1-p} $ (发生与不发生的概率之比)
\end{definition}

该事件的对数几率(log odds)或 logit 函数是

$$ \operatorname{logit}(p)=\log \frac{p}{1-p} $$

$$ \log \frac{P(Y=1 \mid x)}{1-P(Y=1 \mid x)}=w \cdot x $$

考虑对输入 $ x $ 进行分类的线性函数 $ w \cdot x $, 其值域为实数域.

注意, 这里 $ x \in \mathbf{R}^{n+1}, w \in \mathbf{R}^{n+1} $. 通过逻辑斯谛回归模型定义式可以将线性函 数 $ w \cdot x $ 转换为概率:

$$\log \frac{P(Y=1 \mid x)}{1-P(Y=1 \mid x)}= P(Y=1 \mid x)=\frac{\exp (w \cdot x)}{1+\exp (w \cdot x)} $$