\chapter{Hidden Markov Map Matching}

Source: \cite{newson2009hidden}.

\section{论文动机}

\subsection{原始数据}

the raw input
data consists of vehicle \textit{locations} measured by GPS, Each measured point consists of a time-stamped
latitude/longitude pair. 

The \textit{roads} are also represented in the
conventional way, as a graph of nodes and edges.
The \textit{nodes} are at
intersections, dead ends, and road name changes, and the edges
represent road segments between the nodes. Some \textit{edges} are
directional to indicate one-way roads. Each node has an associated
latitude/longitude to indicate its location, and each edge has a
polyline (折线) of latitude/longitude pairs to represent its geometry.

\section{其他论文的方法}

create a (possibly smoothed) curve
from the location measurements and attempt to find matching
roads with similar geometry

\begin{example}
White et al. 
present four algorithms, starting with the simple, nearest match
scheme. 
Their second algorithm \textbf{adds orientation information to
the nearest match approach}, comparing the measured heading to
the angle of the road. Their third algorithm evolves the second
algorithm to \textbf{include connectivity constraints}, and their fourth
algorithm does \textbf{curve matching}. 

\begin{remark}
    their most sophisticated algorithm, the fourth one, was
outperformed by the simpler second algorithm when tested on a
total of about 17 km of driving data.
\end{remark}
\end{example}

builds up a
topologically feasible path through the road network. Matches are
determined by a similarity measure that weights errors based on
distance and orientation. The algorithm was found to perform flawlessly, even though the GPS data was collected while
\textit{Selective Availability} was turned on, leading to noisier location
measurements than are available now.

Kim and Kim [10] look at a
way to measure how much each GPS point belongs to any given
road, taking into account its distance from the road, the shape of
the road segment, and the continuity of the path. The measure is
used in a fuzzy matching scheme with learned parameters to
optimize performance.

Brakatsoulas et al. [3]. Their
algorithm uses variations of the Fréchet distance to match the
curve of the GPS trace to candidate paths in the road network.



\section{论文贡献}

maintaining a principled approach to the problem while simultaneously making
the algorithm robust to location data that is both \textbf{geometrically noisy and temporally sparse}.

a test of
our map matching algorithm where we vary the levels of noise
and sparseness of the sensed location data over a 50 mile urban
drive

\section{模型}

