\chapter{tract (to draw/to drag)}

\begin{word}{tractor}
    a strong vehicle with large wheels, used for pulling farm machinery 拖拉机
\end{word}

\begin{word}{traction}
\end{word}

\begin{word}{attract, attraction, attractive}
\end{word}

\begin{word}{distract, dictracting, distracted, distraction}
    distracting: taking your attention away from what you are trying to do
    \textit{distracting thoughts}

    distracted: somebody/something because you are worried or thinking about something else
    \textit{Luke looked momentarily distracted.}
\end{word}

\begin{word}{contract, contraction, contractor}
    /ˈkɒntrækt/ noun, /kənˈtrækt/ verb

    an official agreement between two or more people, stating what each will do

    \textbf{contract with/between}
    \textit{Tyler has agreed a seven-year contract with a Hollywood studio.}

    \textbf{contract to do sth}
    \textit{a three-year contract to provide pay telephones at local restaurants}

    \textbf{on a contract/under contract}
    \textit{Employees who refuse to relocate are in breach of contract} (=have done something not allowed by their contracts).


    \textbf{subject to contract}: if an agreement is subject to contract, it has not yet been agreed formally by a contract

    to become smaller or narrower
    \textit{Metal contracts as it cools.}

    to get an illness
    \textit{Two-thirds of the adult population there have contracted AIDS.}

    contraction: a very strong and painful movement of a muscle, especially the muscles around the womb (the part of a woman’s or female animal’s body where her baby grows before it is born 子宫 SYN  uterus) during birth
\end{word}

\begin{word}{extract, extraction, extractor}
    \label{extract}
    extract sth from sth

    extraction: the process of removing or obtaining something from something else
    \textit{the extraction of salt from seawater}

    \textbf{be of French/Russian/Italian etc extraction} to be from a French, Russian etc family even though you were not born in that country
\end{word}

\begin{word}{abstract, abstraction, abstractly}
    based on general ideas or principles rather than specific examples or real events (syn theoretical)

    existing only as an idea or quality rather than as something real that you can see or touch (opp concerte)

    a painting, design etc which contains shapes or images that do not look like real things or people抽象画;抽象设计;抽象派作品

    a short written statement containing only the most important ideas in a speech, article etc
    
    \textbf{in the abstract} considered in a general way rather than being based on specific details and examples

    to write a document containing the most important ideas or points from a speech, article etc

    (formal) to remove something from somewhere
\end{word}

\begin{word}{detract, detraction}
    \label{detract}
    \textbf{detract from something}
    to make something seem less good (OPP  enhance):
    \textit{One mistake is not going to detract from your achievement.}
\end{word}

\begin{word}{protract}
    \label{protract}
    to draw out or lengthen, especially in time; extend the duration of; prolong.
\end{word}

\begin{word}{retract, retractable}
    if you retract something that you said or agreed, you say that you did not mean it (SYN  withdraw):
    \textit{He confessed to the murder but later retracted his statement.}

    if part of a machine or an animal’s body retracts or is retracted, it moves back into the main part
    \textit{The sea otter (海獭) can retract the claws on its front feet.}
\end{word}

\begin{word}{subtract, subtraction, subtractive}
    subtract sth from sth
\end{word}