\chapter{cur, curs, cours(to run)}

\begin{word}{occur, occurence}
    it occurs to somebody to do something 
    \textit{It never seems to occur to my children to contact me.}
    
    it occurs to somebody (that)
    \textit{It had never occurred to him that he might be falling in love with her.}
\end{word}

\begin{word}{excursion, excurse}
    a short journey arranged so that a group of people can visit a place, especially while they are on holiday
    \textit{Included in the tour is an excursion \textbf{to} the Grand Canyon.}

    a short journey made for a particular purpose
    \textit{a shopping excursion}

    excursion into something: an attempt to experience or learn about something that is new to you
    \textit{the company's excursion into new markets}

    excurse: to digress (\textit{move away from the subject you are talking or writing about and talk or write about something different for a while}), to wander

    to go on an excursion
\end{word}

\begin{remark}
    The use of excurse is rare.
\end{remark}

\begin{word}{concur}
    to agree with someone or have the same opinion as them
    \textit{The committee largely concurred with these views.}
\end{word}

\begin{word}{concurrent, concurrency}
    existing or happening at the same time
    \textit{The exhibition reflected concurrent developments abroad.}
    concurrent with
    \textit{My opinions are concurrent with yours.}
\end{word}

\begin{word}{course}
    if a liquid or electricity courses somewhere, it flows there quickly
    \textit{Tears coursed down his cheeks.}

    if a feeling courses through you, you feel it suddenly and strongly
    \textit{His smile sent waves of excitement coursing through her.}

    a period of time or process during which something happens
    \textit{During the course of our conversation, it emerged that Bob had been in prison.}

    take/run its course: the usual or natural way that something changes, develops, or is done
    \textit{It seems the boom in World Music has run its course.}
\end{word}
