\section{D, d}

\begin{DefWord}{disconnect}
\end{DefWord}

\begin{DefWord}{distract, dictracting, distracted, distraction}[distract]
    distracting: taking your attention away from what you are trying to do
    \textit{distracting thoughts}

    distracted: somebody/something because you are worried or thinking about something else
    \textit{Luke looked momentarily distracted.}
\end{DefWord}

\begin{DefWord}{dialogue}
\end{DefWord}

\begin{DefWord}{department}
\end{DefWord}

\begin{DefWord}{doctor}
\end{DefWord}

\begin{DefWord}{detractor}
    See \ref{detract}.
\end{DefWord}

\begin{DefWord}{department}
    one of the groups of people who work together in a particular part of a large organization such as a hospital, university, company, or government
    \textit{the personnel department}

    an area in a large shop where a particular type of product is sold
    \textit{the toy department}
\end{DefWord}

\begin{DefWord}{depart, departure}[depart]
    to leave, especially when you are starting a journey

    depart this life formal to die

    to start to use new ideas or do something in a different way
    \textit{It's revolutionary music; it departs from the old form and structures.}

    to leave an organization or job
    \textit{the company's departing chairman}

    \textbf{departure from one place to another place}
\end{DefWord}

\begin{DefWord}{dissect, dissection, dissectible}[dissect]
    to cut up the body of a dead animal or person in order to study it

    to examine something carefully in order to understand it
    \textit{books in which the lives of famous people are dissected}

    to divide an area of land into several smaller pieces
    \textit{fields dissected by small streams}
\end{DefWord}

\begin{DefWord}{detain, detainer, detainee, detainment}[detain]
    to force someone officially to stay in a place
    \textit{A suspect has been detained by the police for questioning.}

    to delay someone for a short length of time
\textit{I'm sorry I'm late - I was unavoidably detained.}
\end{DefWord}

\begin{DefWord}{detract, detraction}[detract]
    \textbf{detract from something}
    to make something seem less good (OPP  enhance):
    \textit{One mistake is not going to detract from your achievement.}
\end{DefWord}

\begin{DefWord}{decide}
    de 加强
\end{DefWord}

\begin{DefWord}{dispel}
    make something go away, especially a belief, idea, or feeling
    \textit{We want to \textbf{dispel} the \textbf{myth} that you cannot eat well in Britain.}

    Syn $\rightarrow$ \ref{expel}.
\end{DefWord}

\begin{DefWord}{devolve}
    if you devolve responsibility, power etc to a person or group at a lower level, or if it devolves on them, it is given to them(将)〔责任、权力等〕下放[转交, 委派]

    \textbf{devolve something to somebody/something}
    \textit{The federal government has devolved responsibility for welfare to the states.}

    \textbf{devolve on/upon}
    \textit{Half of the cost of the study will devolve upon the firm.}

    if land, money etc devolves to someone, it becomes their property when someone else dies(将)〔土地、钱等在某人死后〕转移, 转让〔给某人〕 SYN  pass
\end{DefWord}

\begin{DefWord}{diverse, diversify}[diverse]
    if a business, company, country etc diversifies, it increases the range of goods or services it produces
    \textit{farmers forced to \textbf{diversify} away \textbf{from} their core business}

    \textit{The company is planning to \textbf{diversify} \textbf{into} other mining activities.}

    to change something or to make it change so that there is more variety
    \textit{User requirements have diversified over the years.}

    to put money into several different types of investment instead of only one or two 投资多元化, 进行分散化投资
    \textit{Spread the risk by diversifying into dollar bonds. 购买美元债券进行多种投资, 以分散风险. }

\end{DefWord}

\begin{DefWord}{divert, diversion, diverting}[divert]
    to change the use of something such as time or money 改变…的用途
    \textbf{divert something into/to/(away) from etc something}
    \textit{The company should divert more resources into research.}

    to change the direction in which something travels 改变…的方向, 使转向
    \textbf{divert a river/footpath/road etc}
    Canals divert water from the Truckee River into the lake.

    if you divert your telephone calls, you arrange for them to go directly to another number, for example because you are not able to answer them yourself for some time 转移〔电话〕:
    \textit{Remember to divert your phone when you are out of the office.}

    to deliberately take someone's attention from something by making them think about or notice other things〔故意〕转移, 分散〔别人的注意力〕
    \textbf{divert (somebody's) attention (away from somebody/something)}
    T\textit{he crime crackdown is an attempt to divert attention from social problems.}
 
    to amuse or entertain someone 使消遣, 给…解闷, 供…娱乐

\end{DefWord}


\begin{DefWord}{disclaim}
    to state, especially officially, that you are not responsible for something, that you do not know about it, or that you are not involved with it (deny)
    \textit{Martin disclaimed any responsibility for his son's actions.}
\end{DefWord}