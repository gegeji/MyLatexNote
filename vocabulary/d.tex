\section{D, d}

\begin{word}{disconnect}
\end{word}

\begin{word}{distract, dictracting, distracted, distraction}[distract]
    distracting: taking your attention away from what you are trying to do
    \textit{distracting thoughts}

    distracted: somebody/something because you are worried or thinking about something else
    \textit{Luke looked momentarily distracted.}
\end{word}

\begin{word}{dialogue}
\end{word}

\begin{word}{department}
\end{word}

\begin{word}{doctor}
\end{word}

\begin{word}{detractor}
    See \ref{detract}.
\end{word}

\begin{word}{department}
    one of the groups of people who work together in a particular part of a large organization such as a hospital, university, company, or government
    \textit{the personnel department}

    an area in a large shop where a particular type of product is sold
    \textit{the toy department}
\end{word}

\begin{word}{depart, departure}[depart]
    to leave, especially when you are starting a journey

    depart this life formal to die

    to start to use new ideas or do something in a different way
    \textit{It's revolutionary music; it departs from the old form and structures.}

    to leave an organization or job
    \textit{the company's departing chairman}

    \textbf{departure from one place to another place}
\end{word}

\begin{word}{dissect, dissection, dissectible}[dissect]
    to cut up the body of a dead animal or person in order to study it

    to examine something carefully in order to understand it
    \textit{books in which the lives of famous people are dissected}

    to divide an area of land into several smaller pieces
    \textit{fields dissected by small streams}
\end{word}

\begin{word}{detain, detainer, detainee, detainment}[detain]
    to force someone officially to stay in a place
    \textit{A suspect has been detained by the police for questioning.}

    to delay someone for a short length of time
\textit{I'm sorry I'm late - I was unavoidably detained.}
\end{word}

\begin{word}{detract, detraction}[detract]
    \textbf{detract from something}
    to make something seem less good (OPP  enhance):
    \textit{One mistake is not going to detract from your achievement.}
\end{word}