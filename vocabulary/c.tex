\section{C, c}

\begin{word}{concur}
    to agree with someone or have the same opinion as them
    \textit{The committee largely concurred with these views.}
\end{word}

\begin{word}{concurrent, concurrency}[concurrent]
    existing or happening at the same time
    \textit{The exhibition reflected concurrent developments abroad.}
    concurrent with
    \textit{My opinions are concurrent with yours.}
\end{word}

\begin{word}{course}
    if a liquid or electricity courses somewhere, it flows there quickly
    \textit{Tears coursed down his cheeks.}

    if a feeling courses through you, you feel it suddenly and strongly
    \textit{His smile sent waves of excitement coursing through her.}

    a period of time or process during which something happens
    \textit{During the course of our conversation, it emerged that Bob had been in prison.}

    take/run its course: the usual or natural way that something changes, develops, or is done
    \textit{It seems the boom in World Music has run its course.}
\end{word}

\begin{word}{contradict, contradiction, contradictory}[contradict]
\end{word}

\begin{word}{contractor}
    See \ref{contract}.

    a person or company that agrees to do work or provide goods for another company
\end{word}


\begin{word}{counterpart}
    someone or something that has the same job or purpose as someone or something else in a different place
    \textit{Belgian officials are discussing this with their French counterparts.}
\end{word}

\begin{word}{compass}
\end{word}

\begin{word}{comprehend, comprehension, comprehensible}[comprehend]
    comprehensible input
\end{word}

\begin{word}{comprehensive}
    comprehensive introduction
\end{word}

\begin{word}{comprise}
    \textit{The house \textbf{comprises} two bedrooms, a kitchen, and a living room.} 
    
    \textit{The committee \textbf{is comprised of} well-known mountaineers. }
    
    \textit{Women \textbf{comprise} a high proportion of part-time workers.} 
    
    \textit{Food exports are very important, \textbf{comprising} 74\% of the total.}
\end{word}

\begin{word}{consequent, consequence}[consequent]
    something that happens as a result of a particular action or set of conditions; 
    
    (\textit{rare}) people of consequence; 
    
    \textbf{of little/no/any etc consequence} formal not very important or valuable
\end{word}

\begin{word}{consecutive}
   \textit{ It had rained for four consecutive days.}

    \textit{Can they win the title for the third consecutive season?}
\end{word}

\begin{word}{contract, contraction}[contract]
    /ˈkɒntrækt/ noun, /kənˈtrækt/ verb

    an official agreement between two or more people, stating what each will do

    \textbf{contract with/between}
    \textit{Tyler has agreed a seven-year contract with a Hollywood studio.}

    \textbf{contract to do sth}
    \textit{a three-year contract to provide pay telephones at local restaurants}

    \textbf{on a contract/under contract}
    \textit{Employees who refuse to relocate are in breach of contract} (=have done something not allowed by their contracts).


    \textbf{subject to contract}: if an agreement is subject to contract, it has not yet been agreed formally by a contract

    to become smaller or narrower
    \textit{Metal contracts as it cools.}

    to get an illness
    \textit{Two-thirds of the adult population there have contracted AIDS.}

    contraction: a very strong and painful movement of a muscle, especially the muscles around the womb (the part of a woman's or female animal's body where her baby grows before it is born 子宫 SYN  uterus) during birth
\end{word}