\section{R, r}

\begin{DefWord}{retain}
    \textit{We hope to \textbf{retain} all these beautiful moments in our memory.}
\end{DefWord}

\begin{DefWord}{retract, retractable}[retract]
    if you retract something that you said or agreed, you say that you did not mean it (SYN  withdraw):
    \textit{He confessed to the murder but later retracted his statement.}

    if part of a machine or an animal's body retracts or is retracted, it moves back into the main part
    \textit{The sea otter (海獭) can retract the claws on its front feet.}
\end{DefWord}

\begin{DefWord}{repel, repulsive, repellent}[repel]
    if something repels you, it is so unpleasant that you do not want to be near it, or it makes you feel ill使厌恶, 使反感 $\rightarrow$ :
    \textit{The smell repelled him}

    to make someone who is attacking you go away, by fighting them
    \textit{The army was ready to repel an attack.}

    to keep something or someone away from you:
    \textit{a lotion that repels mosquitoes}

    if two things repel each other, they push each other away with an electrical force (OPP attract)
    \textit{Two positive charges repel each other.}

    repellent: very unpleasant (repulsive)
    \textit{She found him physically repellent}

    \textbf{repellent to}
    \textit{The sight of blood is repellent to some people.}

\end{DefWord}

\begin{DefWord}{revolve, revolving, revolution}[revolve]
    to move around like a wheel, or to make something move around like a wheel, e.g. the windmill

    \textit{revolving door}

    \textbf{revolve around somebody/something}:
    to have something as a main subject or purpose
    
    \textit{She seems to think that the world \textbf{revolves around} her}

    to move in circles around something
    \textit{The Moon revolves around the Earth.}

    a \textbf{revolving} object is designed so that it turns with a circular movement

\end{DefWord}

\begin{DefWord}{revolt}
    反抗
\end{DefWord}

\begin{DefWord}{revert}
\end{DefWord}

\begin{DefWord}{rebel}
\end{DefWord}

\begin{DefWord}{retrospect}
    \textbf{in retrospect} thinking back to a time in the past, especially with the advantage of knowing more now than you did then

    \textit{In retrospect, I wonder if we should have done more.}
\end{DefWord}

\begin{DefWord}{respire, respiratory, respiration, artificial respiration}[respire]
    respiration /ˌrespəˈreɪʃən/ 

    to breathe呼吸

    artificial respiration
\end{DefWord}

\begin{DefWord}{revoke, revocable, irrevocable}[revoke]
\end{DefWord}

\begin{RefWord}{reclaim, reclamation}[reclaim]
    to get back an amount of money that you have paid = claim back
    \textit{You may be entitled to reclaim some tax.}

    to make an area of desert, wet land etc suitable for farming or building
    \textit{This land will be reclaimed for a new airport.}

    to get back something that you have lost or that has been taken away from you
    \textit{I want to reclaim the championship that I lost in 1999.}

    to obtain useful products from waste material
    \textit{You can reclaim old boards and use them as shelves.}

    \textbf{reclaim somebody (from something)} to rescue somebody from a bad or criminal way of life
\end{RefWord}