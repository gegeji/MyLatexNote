\section{E, e}

\begin{DefWord}{excessive}
\end{DefWord}

\begin{DefWord}{exceed, excess, excessive, excessively}[exceed]
\end{DefWord}

\begin{DefWord}{excursion, excurse}[excursion]
    a short journey arranged so that a group of people can visit a place, especially while they are on holiday
    \textit{Included in the tour is an excursion \textbf{to} the Grand Canyon.}

    a short journey made for a particular purpose
    \textit{a shopping excursion}

    excursion into something: an attempt to experience or learn about something that is new to you
    \textit{the company's excursion into new markets}

    excurse: to digress (\textit{move away from the subject you are talking or writing about and talk or write about something different for a while}), to wander

    to go on an excursion
\end{DefWord}

\begin{remark}
    The use of excurse is rare.
\end{remark}

\begin{DefWord}{excessive}[excess]
\end{DefWord}

\begin{DefWord}{exchange}
\end{DefWord}

\begin{DefWord}{ex-boyfriend}
\end{DefWord}

\begin{DefWord}{external}
\end{DefWord}

\begin{DefWord}{excurse, excursion}[excurse]
\end{DefWord}

\begin{DefWord}{extract}
\end{DefWord}

\begin{DefWord}{epilogue}
    a concluding section that rounds out the design of a literary work.
\end{DefWord}

\begin{DefWord}{eloquent, eloquence}[eloquent]
    able to express your ideas and opinions well, especially in a way that influences people;

\textit{an eloquent appeal for support}
showing a feeling or meaning without using words
\end{DefWord}

\begin{DefWord}{extractor}
    a machine for removing air that is hot or smells unpleasant from a kitchen, factory etc
\end{DefWord}

\begin{DefWord}{expedite}
    expedite /ˈekspədaɪt/

    to make a process or action happen more quickly, speed up
    \textit{strategies to expedite the decision-making process.}
    \textit{Please expedite the shipment of mangoes, as they are perishable (food that is perishable is likely to decay quickly).}
\end{DefWord}

\begin{DefWord}{expedition}
    a long and carefully organized journey, especially to a dangerous or unfamiliar place, or the people that make this journey
    \textit{another Everest expedition}

    a short journey, usually made for a particular purpose
    \textit{a shopping expedition}
\end{DefWord}

\begin{DefWord}{enterprise, enterpriser}[enterprise]
    SME (small-medium enterprise);a large and complicated project, especially one that is done with a group of other people (syn \textbf{initiative})

    enterpriser 企业家
\end{DefWord}

\begin{DefWord}{execute, execution, executive}[execute]
    a marketing executive; 
    
    a commission with executive powers;executive body/committee etc
\end{DefWord}

\begin{DefWord}{extract, extraction, extractor}[extract]
    extract sth from sth

    extraction: the process of removing or obtaining something from something else
    \textit{the extraction of salt from seawater}

    \textbf{be of French/Russian/Italian etc extraction} to be from a French, Russian etc family even though you were not born in that country
\end{DefWord}

\begin{DefWord}{evade, evasion, evasive, evasively, evasiveness}[evade]
    to avoid talking about something, especially because you are trying to hide something
    \textit{I could tell that he was trying to evade the issue.}

    to not do or deal with something that you should do
    \textit{You can't go on evading your responsibilities in this way.}

    to avoid paying money that you ought to pay, for example tax
    \textit{Employers will always try to find ways to evade tax.}

    if something evades you, you cannot do it or understand it
    \textit{The subtleties of his argument evaded me.}

    not willing to answer questions directly
    \textit{Direct questions would almost certainly result in evasive answers.}

    \textbf{take evasive action}: to move or do something quickly to avoid someone being hurt
    \textit{Both pilots took evasive action and a collision was avoided.}
\end{DefWord}

\begin{DefWord}{excise, excision}[excise]
    /ˈeksaɪz/ the government tax that is put on the goods that are produced and used inside a country
    \textit{excise duty on tobacco}

    /ɪkˈsaɪz/ formal to remove or get rid of something, especially by cutting it out
    \textit{The tumour was excised.}
\end{DefWord}

\begin{DefWord}{evolve, evolution}[evolve]
    if an animal or plant evolves, it changes gradually over a long period of time
    \textit{Fish \textbf{evolved from} prehistoric sea creatures.}

    to develop and change gradually over a long period of time

    \textit{The school has evolved its own style of teaching.}
    \textit{The group gradually \textbf{evolved into} a political party.}
    \textit{The idea \textbf{evolved out of} work done by British scientists.}
\end{DefWord}

\begin{DefWord}{expel, expulsion}[expel]
    Syn $\rightarrow$ \ref{dispel}.

    to officially force someone to leave a school or organization
    
    \textbf{expel somebody from something}
    \textit{Two girls were expelled from school for taking drugs.}

    \textbf{expel somebody for doing something}
    \textit{He was expelled for making racist remarks.}

    to force a foreigner to leave a country, especially because they have broken the law or for political reasons
    \textit{Foreign priests were expelled from the country.}

    \textbf{expel somebody for something}
    \textit{Three diplomats were expelled for spying.}
\end{DefWord}

\begin{DefWord}{expire}
    expiration date
\end{DefWord}

\begin{DefWord}{exhale}
    to breathe air, smoke etc out of your mouth
\end{DefWord}

\begin{DefWord}{evoke, evocation, evocable}[evoke]
    to produce a strong feeling or memory in someone

    \textit{The photographs evoked strong memories of our holidays in France.}

\end{DefWord}