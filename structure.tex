%%%%%%%%%%%%%%%%%%%%%%%%%%%%%%%%%%%%%%%%%
% The Legrand Orange Book
% Structural Definitions File
% Version 2.1 (26/09/2018)
%
% Original author:
% Mathias Legrand (legrand.mathias@gmail.com) with modifications by:
% Vel (vel@latextemplates.com)
% 
% This file was downloaded from:
% http://www.LaTeXTemplates.com
%
% License:
% CC BY-NC-SA 3.0 (http://creativecommons.org/licenses/by-nc-sa/3.0/)
%
%%%%%%%%%%%%%%%%%%%%%%%%%%%%%%%%%%%%%%%%%

%----------------------------------------------------------------------------------------
%	VARIOUS REQUIRED PACKAGES AND CONFIGURATIONS
%----------------------------------------------------------------------------------------

\usepackage{graphicx} % Required for including pictures
\graphicspath{{Pictures/}} % Specifies the directory where pictures are stored

\usepackage{lipsum} % Inserts dummy text

\usepackage{tikz} % Required for drawing custom shapes

\usepackage[main=english, provide=*]{babel} % English language/hyphenation
% \babelfont[japanese]{rm}
%          [Renderer=HarfBuzz]{IPAMincho}


\usepackage{enumitem} % Customize lists
\setlist{nolistsep} % Reduce spacing between bullet points and numbered lists

\usepackage{booktabs} % Required for nicer horizontal rules in tables

\usepackage{xcolor} % Required for specifying colors by name
% \definecolor{ocre}{RGB}{243,102,25} % Define the orange color used for highlighting throughout the book
\definecolor{violet}{RGB}{76, 50, 168}
\definecolor{lightViolet}{RGB}{242, 240, 255}
\definecolor{coral}{RGB}{184, 84, 80}
\definecolor{grass}{RGB}{130,179,102}
\definecolor{lightCoralRed}{RGB}{250, 223, 216}
\usepackage{calc}

%----------------------------------------------------------------------------------------
%	My Custom Packages
%----------------------------------------------------------------------------------------

\usepackage{algorithm2e}
\usepackage{tabularx} % adjustable width tables
\usepackage{svg}

% for sub figure
\usepackage{caption}
\usepackage{subcaption}

% for scaling tikz tool
% \usepackage{environ}
\usepackage{adjustbox}

% in companion with babel/polyglossia
\usepackage{csquotes}

% code listing
\usepackage{listings}
% basic style, sourced from overleaf tutorial
\definecolor{codegreen}{rgb}{0,0.6,0}
\definecolor{codegray}{rgb}{0.5,0.5,0.5}
\definecolor{codepurple}{rgb}{0.58,0,0.82}
\definecolor{backcolour}{rgb}{0.95,0.95,0.92}

\lstdefinestyle{mystyle}{
    backgroundcolor=\color{backcolour},   
    commentstyle=\color{codegreen},
    keywordstyle=\color{magenta},
    numberstyle=\tiny\color{codegray},
    stringstyle=\color{codepurple},
    basicstyle=\ttfamily\footnotesize,
    breakatwhitespace=false,         
    breaklines=true,                 
    captionpos=b,                    
    keepspaces=true,                 
    numbers=left,                    
    numbersep=5pt,                  
    showspaces=false,                
    showstringspaces=false,
    showtabs=false,                  
    tabsize=2
}

\lstset{style=mystyle}
%----------------------------------------------------------------------------------------
%	MARGINS
%----------------------------------------------------------------------------------------

\usepackage{geometry} % Required for adjusting page dimensions and margins

\geometry{
	paper=a4paper, % Paper size, change to letterpaper for US letter size
	top=3cm, % Top margin
	bottom=3cm, % Bottom margin
	left=3cm, % Left margin
	right=3cm, % Right margin
	headheight=14pt, % Header height
	footskip=1.4cm, % Space from the bottom margin to the baseline of the footer
	headsep=10pt, % Space from the top margin to the baseline of the header
	%showframe, % Uncomment to show how the type block is set on the page
}


%----------------------------------------------------------------------------------------
%	BIBLIOGRAPHY AND INDEX
%----------------------------------------------------------------------------------------

\usepackage[style=numeric,citestyle=numeric,sorting=nyt,sortcites=true,autopunct=true,babel=hyphen,hyperref=true,abbreviate=false,backref=true,backend=biber]{biblatex}
\addbibresource{E:/OneDrive/Library/library.bib}
% \addbibresource{bibliography.bib} % BibTeX bibliography file
\defbibheading{bibempty}{}

\usepackage{calc} % For simpler calculation - used for spacing the index letter headings correctly
\usepackage{makeidx} % Required to make an index
\makeindex % Tells LaTeX to create the files required for indexing

%----------------------------------------------------------------------------------------
%	MAIN TABLE OF CONTENTS
%----------------------------------------------------------------------------------------

\usepackage{titletoc} % Required for manipulating the table of contents
% \usepackage{tocloft} % custom list of definitions

\contentsmargin{0cm} % Removes the default margin

% Part text styling (this is mostly taken care of in the PART HEADINGS section of this file)
\titlecontents{part}
	[0cm] % Left indentation
	{\addvspace{20pt}\bfseries} % Spacing and font options for parts
	{}
	{}
	{}

% Chapter text styling
\titlecontents{chapter}
	[1.25cm] % Left indentation
	{\addvspace{12pt}\large\sffamily\bfseries} % Spacing and font options for chapters
	{\color{violet!60}\contentslabel[\Large\thecontentslabel]{1.25cm}\color{violet}} % Formatting of numbered sections of this type
	{\color{violet}} % Formatting of numberless sections of this type
	{\color{violet!60}\normalsize\;\titlerule*[.5pc]{.}\;\thecontentspage} % Formatting of the filler to the right of the heading and the page number

% Section text styling
\titlecontents{section}
	[1.25cm] % Left indentation
	{\addvspace{3pt}\sffamily\bfseries} % Spacing and font options for sections
	{\contentslabel[\thecontentslabel]{1.25cm}} % Formatting of numbered sections of this type
	{} % Formatting of numberless sections of this type
	{\hfill\color{black}\thecontentspage} % Formatting of the filler to the right of the heading and the page number

% Subsection text styling
\titlecontents{subsection}
	[1.25cm] % Left indentation
	{\addvspace{1pt}\sffamily\small} % Spacing and font options for subsections
	{\contentslabel[\thecontentslabel]{1.25cm}} % Formatting of numbered sections of this type
	{} % Formatting of numberless sections of this type
	{\ \titlerule*[.5pc]{.}\;\thecontentspage} % Formatting of the filler to the right of the heading and the page number

% Figure text styling
\titlecontents{figure}
	[1.25cm] % Left indentation
	{\addvspace{1pt}\sffamily\small} % Spacing and font options for figures
	{\thecontentslabel\hspace*{1em}} % Formatting of numbered sections of this type
	{} % Formatting of numberless sections of this type
	{\ \titlerule*[.5pc]{.}\;\thecontentspage} % Formatting of the filler to the right of the heading and the page number

% Table text styling
\titlecontents{table}
	[1.25cm] % Left indentation
	{\addvspace{1pt}\sffamily\small} % Spacing and font options for tables
	{\thecontentslabel\hspace*{1em}} % Formatting of numbered sections of this type
	{} % Formatting of numberless sections of this type
	{\ \titlerule*[.5pc]{.}\;\thecontentspage} % Formatting of the filler to the right of the heading and the page number

%----------------------------------------------------------------------------------------
%	MINI TABLE OF CONTENTS IN PART HEADS
%----------------------------------------------------------------------------------------

% Chapter text styling
\titlecontents{lchapter}
	[0em] % Left indentation
	{\addvspace{15pt}\large\sffamily\bfseries} % Spacing and font options for chapters
	{\color{violet}\contentslabel[\Large\thecontentslabel]{1.25cm}\color{violet}} % Chapter number
	{}  
	{\color{violet}\normalsize\sffamily\bfseries\;\titlerule*[.5pc]{.}\;\thecontentspage} % Page number

% Section text styling
\titlecontents{lsection}
	[0em] % Left indentation
	{\sffamily\small} % Spacing and font options for sections
	{\contentslabel[\thecontentslabel]{1.25cm}} % Section number
	{}
	{}

% Subsection text styling (note these aren't shown by default, display them by searchings this file for tocdepth and reading the commented text)
\titlecontents{lsubsection}
	[.5em] % Left indentation
	{\sffamily\footnotesize} % Spacing and font options for subsections
	{\contentslabel[\thecontentslabel]{1.25cm}}
	{}
	{}

%----------------------------------------------------------------------------------------
%	HEADERS AND FOOTERS
%----------------------------------------------------------------------------------------

\usepackage{fancyhdr} % Required for header and footer configuration

\pagestyle{fancy} % Enable the custom headers and footers

\renewcommand{\chaptermark}[1]{\markboth{\sffamily\normalsize\bfseries\chaptername\ \thechapter.\ #1}{}} % Styling for the current chapter in the header
\renewcommand{\sectionmark}[1]{\markright{\sffamily\normalsize\thesection\hspace{5pt}#1}{}} % Styling for the current section in the header

\fancyhf{} % Clear default headers and footers
\fancyhead[LE,RO]{\sffamily\normalsize\thepage} % Styling for the page number in the header
\fancyhead[LO]{\rightmark} % Print the nearest section name on the left side of odd pages
\fancyhead[RE]{\leftmark} % Print the current chapter name on the right side of even pages
%\fancyfoot[C]{\thepage} % Uncomment to include a footer

\renewcommand{\headrulewidth}{0.5pt} % Thickness of the rule under the header

\fancypagestyle{plain}{% Style for when a plain pagestyle is specified
	\fancyhead{}\renewcommand{\headrulewidth}{0pt}%
}

% Removes the header from odd empty pages at the end of chapters
\makeatletter
\renewcommand{\cleardoublepage}{
\clearpage\ifodd\c@page\else
\hbox{}
\vspace*{\fill}
\thispagestyle{empty}
\newpage
\fi}

%----------------------------------------------------------------------------------------
%	THEOREM STYLES
%----------------------------------------------------------------------------------------

\usepackage{amsmath,amsfonts,amssymb,amsthm} % For math equations, theorems, symbols, etc
% \usepackage{thmtools}


\newcommand{\intoo}[2]{\mathopen{]}#1\,;#2\mathclose{[}}
\newcommand{\ud}{\mathop{\mathrm{{}d}}\mathopen{}}
\newcommand{\intff}[2]{\mathopen{[}#1\,;#2\mathclose{]}}
\renewcommand{\qedsymbol}{$\blacksquare$}
\newtheorem{notation}{Notation}[chapter]

% Boxed/framed environments
\newtheoremstyle{violetnumbox}% Theorem style name
{0pt}% Space above
{0pt}% Space below
{\normalfont}% Body font
{}% Indent amount
{\small\bf\sffamily\color{violet}}% Theorem head font
{\;}% Punctuation after theorem head
{0.25em}% Space after theorem head
{\small\sffamily\color{violet}\thmname{#1}\nobreakspace\thmnumber{\@ifnotempty{#1}{}\@upn{#2}}% Theorem text (e.g. Theorem 2.1)
\thmnote{\nobreakspace\the\thm@notefont\sffamily\bfseries\color{black}---\nobreakspace#3.}% Optional theorem note
\addcontentsline{lod}{section}{\quad(#1)\quad\protect\numberline{#2}{#3}}} % add line to list of theorems; #1 is theorem name, #2 #3 is the name of theorem


\newtheoremstyle{blacknumex}% Theorem style name
{5pt}% Space above
{5pt}% Space below
{\normalfont}% Body font
{} % Indent amount
{\small\bf\sffamily}% Theorem head font
{\;}% Punctuation after theorem head
{0.25em}% Space after theorem head
{\small\sffamily{\tiny\ensuremath{\blacksquare}}\nobreakspace\thmname{#1}\nobreakspace\thmnumber{\@ifnotempty{#1}{}\@upn{#2}}% Theorem text (e.g. Theorem 2.1)
\thmnote{\nobreakspace\the\thm@notefont\sffamily\bfseries---\nobreakspace#3.}}% Optional theorem note


\newtheoremstyle{blacknumbox} % Theorem style name
{0pt}% Space above
{0pt}% Space below
{\normalfont}% Body font
{}% Indent amount
{\small\bf\sffamily}% Theorem head font
{\;}% Punctuation after theorem head
{0.25em}% Space after theorem head
{\small\sffamily\thmname{#1}\nobreakspace\thmnumber{\@ifnotempty{#1}{}\@upn{#2}}% Theorem text (e.g. Theorem 2.1)
\thmnote{\nobreakspace\the\thm@notefont\sffamily\bfseries---\nobreakspace#3.}% Optional theorem note
\addcontentsline{lod}{section}{\quad(#1)\quad\protect\numberline{#2}{#3}}} % add line to list of theorems


\newtheoremstyle{blacknumboxVocabulary} % Theorem style name
{0pt}% Space above
{0pt}% Space below
{\normalfont}% Body font
{}% Indent amount
{\small\bf\sffamily}% Theorem head font
{\;}% Punctuation after theorem head
{0.25em}% Space after theorem head
{\small\sffamily\thmname{#1}\nobreakspace\thmnumber{\@ifnotempty{#1}{}\@upn{#2}}% Theorem text (e.g. Theorem 2.1)
\thmnote{\nobreakspace\the\thm@notefont\sffamily\bfseries---\nobreakspace#3.}% Optional theorem note
}


% Non-boxed/non-framed environments
\newtheoremstyle{violetnum}% Theorem style name
{5pt}% Space above
{5pt}% Space below
{\normalfont}% Body font
{}% Indent amount
{\small\bf\sffamily\color{violet}}% Theorem head font
{\;}% Punctuation after theorem head
{0.25em}% Space after theorem head
{\small\sffamily\color{violet}\thmname{#1}\nobreakspace\thmnumber{\@ifnotempty{#1}{}\@upn{#2}}% Theorem text (e.g. Theorem 2.1)
\thmnote{\nobreakspace\the\thm@notefont\sffamily\bfseries\color{black}---\nobreakspace#3.}} % Optional theorem note

% Coral numbering (for definition)
\newtheoremstyle{coralnum}% Theorem style name
{5pt}% Space above
{5pt}% Space below
{\normalfont}% Body font
{}% Indent amount
{\small\bf\sffamily\color{coral}}% Theorem head font
{\;}% Punctuation after theorem head
{0.25em}% Space after theorem head
{\small\sffamily\color{coral}\thmname{#1}\nobreakspace\thmnumber{\@ifnotempty{#1}{}\@upn{#2}}% Theorem text (e.g. Theorem 2.1)
\thmnote{\nobreakspace\the\thm@notefont\sffamily\bfseries\color{black}---\nobreakspace#3.}
\addcontentsline{lod}{section}{\quad(#1)\quad\protect\numberline{#2}{#3}}} % Optional theorem note


\makeatother

% Defines the theorem text style for each type of theorem to one of the three styles above
\newcounter{dummy} 
\numberwithin{dummy}{section}

\theoremstyle{violetnumbox}
\newtheorem{theoremeT}[dummy]{Theorem}
\newtheorem{problemT}{Problem}[chapter]
\newtheorem{exerciseT}{Exercise}[chapter]

\theoremstyle{blacknumex}
% \newtheorem{exampleT}{Example}[chapter]

\theoremstyle{blacknumboxVocabulary} 
\newtheorem{vocabulary}{Vocabulary}[chapter]

\theoremstyle{blacknumbox}
\newtheorem{corollaryT}[dummy]{Corollary}

\theoremstyle{violetnum}
\newtheorem{proposition}[dummy]{Proposition}
\newtheorem{exampleT}{Example}[chapter]

\theoremstyle{coralnum}
\newtheorem{definitionT}{Definition}[section]

% define new environment: RefWord and DefWord for vocabulary.
\usepackage{xparse}
\NewDocumentEnvironment{RefWord}{ m o}{
	% Start Code
	\IfNoValueTF{#2}{
		% if optional parameter is null
		\begin{vocabulary}[#1]
			\index{#1}
			% See \textbf{#1} (Vocabulary \ref{#1}).
			\hyperref[#1]{  $\rightarrow$ V \ref{#1}}.
		}{
		\begin{vocabulary}[#1]
			\index{#2}
			\hyperref[#2]{$\rightarrow$ \texttt{#2} (V \ref{#2})}.
			
	}
}{	% End Code
		\end{vocabulary}
}
\NewDocumentEnvironment{DefWord}{mo}{
	% Start Code
	\IfNoValueTF{#2}{
		\begin{vocabulary}[#1]
			\label{#1}
			\index{#1}
		}{
		\begin{vocabulary}[#1]
			\label{#2}
			\index{#2}
	}
}{% End Code
		\end{vocabulary}
}
\newcommand{\term}[1]{\textit{#1}\index{#1}}

\NewDocumentEnvironment{FigureCenter}{mo}{
	\IfNoValueTF{#2}{
		\begin{figure}[htbp]
			\centering
			\caption{#1}
	}{
		\begin{figure}[htbp]
			\centering
			\caption{#1}
			\label{#2}
	}
}{
	\end{figure}
}




%----------------------------------------------------------------------------------------
%	DEFINITION OF COLORED BOXES
%----------------------------------------------------------------------------------------

\RequirePackage[framemethod=default]{mdframed} % Required for creating the theorem, definition, exercise and corollary boxes
%\RequirePackage[framemethod=TikZ]{mdframed} % for supporting rounder corners, warning: 'framemethod=pstricks' will not work with XeLatex

\mdfsetup{
	roundcorner=5pt
}

% Theorem box
\newmdenv[skipabove=7pt,
skipbelow=7pt,
% backgroundcolor=black!5, % grey
rightline=false,
leftline=true,
topline=false,
bottomline=false,
backgroundcolor=lightViolet,
linecolor=violet,
innerleftmargin=5pt,
innerrightmargin=5pt,
innertopmargin=5pt,
leftmargin=0cm,
rightmargin=0cm,
linewidth=4pt,
innerbottommargin=5pt]{tBox}

% Exercise box	  
\newmdenv[skipabove=7pt,
skipbelow=7pt,
rightline=false,
leftline=true,
topline=false,
bottomline=false,
backgroundcolor=violet!10,
linecolor=violet,
innerleftmargin=5pt,
innerrightmargin=5pt,
innertopmargin=5pt,
innerbottommargin=5pt,
leftmargin=0cm,
rightmargin=0cm,
linewidth=4pt]{eBox}	

% Definition box
\newmdenv[skipabove=7pt,
skipbelow=7pt,
rightline=false,
leftline=true,
topline=false,
bottomline=false,
linecolor=coral,
innerleftmargin=5pt,
innerrightmargin=5pt,
innertopmargin=0pt,
leftmargin=0cm,
rightmargin=0cm,
linewidth=4pt,
innerbottommargin=5pt]{dBox}	

\mdfsetup{
	roundcorner=5pt
}

% Corollary box
\newmdenv[skipabove=7pt,
skipbelow=7pt,
rightline=false,
leftline=true,
%leftline=false,
topline=false,
bottomline=false,
linecolor=gray,
backgroundcolor=black!5,
innerleftmargin=5pt,
innerrightmargin=5pt,
innertopmargin=5pt,
leftmargin=0cm,
rightmargin=0cm,
linewidth=4pt,
innerbottommargin=5pt]{cBox}

% \usepackage{etoolbox}

% Creates an environment for each type of theorem and assigns it a theorem text style from the "Theorem Styles" section above and a colored box from above
\newenvironment{theorem}{\begin{tBox}\begin{theoremeT}}{\end{theoremeT}\end{tBox}}
\newenvironment{exercise}{\begin{eBox}\begin{exerciseT}}{\hfill{\color{violet}\tiny\ensuremath{\blacksquare}}\end{exerciseT}\end{eBox}}				  
\newenvironment{definition}{\begin{dBox}\begin{definitionT}}{\end{definitionT}\end{dBox}}	
\newenvironment{problem}{\begin{problemT}}{\hfill{\tiny\ensuremath{\blacksquare}}\end{problemT}}
\newenvironment{example}{\begin{exampleT}}{\hfill{\tiny\ensuremath{\blacksquare}}\end{exampleT}}		
\newenvironment{corollary}{\begin{cBox}\begin{corollaryT}}{\end{corollaryT}\end{cBox}}	

% lod: List of Definition

\makeatletter
% A command to create the new List of Definitions
\newcommand\listofdefinitions{%
\chapter{List Of Definitions, Theorems and Corollaries}\@starttoc{lod}}
%\newcommand\listofdefinitions{%
%\null\hfill\textbf{\Large\contentsname}\hfill\null\par
%  \@mkboth{\MakeUppercase\contentsname}{\MakeUppercase\contentsname}%
%\@starttoc{lod}}

%\usepackage{etoc}
%\makeatletter
%    \etocsettocstyle
%    {\section *{\Huge\contentsname%
% FOLLOWING TWO LINES OPTIONAL DEPENDING ON YOUR NEED
%                \@mkboth {\MakeUppercase \contentsname}
%                         {\MakeUppercase \contentsname}%
%    }}
%    {}%
%\def\etocstandarddisplaystyle{\etocarticlestyle}
%\makeatother

% initial definitions to save the chapter info (name and number)
\def\thischaptertitle{}
\def\thischapternumber{}
\newtoggle{noDefs}

\apptocmd{\@chapter}%
  {\gdef\thischaptertitle{#1}\gdef\thischapternumber{\thechapter}%
   \global\toggletrue{noDefs}}{}{}

% the defn environment does the job: the first time it is used after a \chapter command, 
% it writes the information of the chapter to the LoD
% todo
\AtBeginDocument{%
  \AtBeginEnvironment{definition}{%
    \iftoggle{noDefs}{
      \addcontentsline{lod}{chapter}{\chaptername~\thischapternumber\hspace{0.5em}\thischaptertitle}{}
      \global\togglefalse{noDefs}
    }{}
  }%
}
\makeatother

% candidate: use thmtools to print list of definitions
%\usepackage{thmtools}
%\colorlet{shadecolor}{lightgray!25}

%\newtheorem{thm}{Theorem}[chapter]

%\newtheoremstyle{definitionsty}{3pt}{3pt}{\slshape}{}{\bfseries}{.}{.5em}{}
%\theoremstyle{definitionsty}
%\newtheorem{tdefn}{Definition}[chapter]
%\newenvironment{defn}
%  {\begin{shaded}\begin{tdefn}}
%  {\end{tdefn}\end{shaded}}

%\usepackage{etoolbox}
%\makeatletter
%\patchcmd\thmtlo@chaptervspacehack
%  {\addtocontents{loe}{\protect\addvspace{10\p@}}}
%  {\addtocontents{loe}{\protect\thmlopatch@endchapter\protect\thmlopatch@chapter{\thechapter}}}
%  {}{}
%\AtEndDocument{\addtocontents{loe}{\protect\thmlopatch@endchapter}}
%\long\def\thmlopatch@chapter#1#2\thmlopatch@endchapter{%
%  \setbox\z@=\vbox{#2}%
%  \ifdim\ht\z@>\z@
%    \hbox{\bfseries\chaptername\ #1}\nobreak
%    #2
%    \addvspace{10\p@}
%  \fi
%}
%\def\thmlopatch@endchapter{}

%\makeatother
%\renewcommand{\thmtformatoptarg}[1]{ #1}
%\renewcommand{\listtheoremname}{List of definitions}

% candidate: use tocloft to define custom list of definitions
%\newcommand{\listmytheoremsname}{List of Definitions, Theorems, Corollaries}
%\newlistof{myentry}{ety}{\listmytheoremsname}
%\newcommand{\mydefinition}[1]{%
%\refstepcounter{mydefinition}%
%\par\noindent\textbf{Definition \themydefinition. #1}
%\addcontentsline{ety}{mydefinition}
%{\protect\numberline{\thechapter.\themyentry}#1}\par}


%----------------------------------------------------------------------------------------
%	REMARK ENVIRONMENT
%----------------------------------------------------------------------------------------

% Remark box
\newmdenv[skipabove=7pt,
skipbelow=7pt,
% backgroundcolor=black!5, % grey
backgroundcolor=lightCoralRed,
linecolor=coral,
innerleftmargin=5pt,
innerrightmargin=5pt,
innertopmargin=5pt,
leftmargin=0cm,
rightmargin=0cm,
innerbottommargin=5pt]{rBox}

\newenvironment{remark}{ % \par
\vspace{0.125pt}\small % Vertical white space above the remark and smaller font size
\begin{list}{}{
\leftmargin=35pt % Indentation on the left
\rightmargin=25pt}\item\ignorespaces % Indentation on the right

% coral red circle, with ! insides it	
\begin{tikzpicture}[overlay]
\node[draw=coral!60,line width=1pt,circle,fill=lightCoralRed,% fill=coral!25,
font=\sffamily\bfseries,inner sep=2pt,outer sep=0pt] at (-15pt,-6pt){\textcolor{coral}{!}};\end{tikzpicture}

\advance\baselineskip -0.5pt\begin{rBox}}{\end{rBox}\end{list}\vskip0.125pt} % Tighter line spacing and white space after remark



%----------------------------------------------------------------------------------------
%	SECTION NUMBERING IN THE MARGIN
%----------------------------------------------------------------------------------------

\makeatletter
\renewcommand{\@seccntformat}[1]{\llap{\textcolor{violet}{\csname the#1\endcsname}\hspace{1em}}}                    
\renewcommand{\section}{\@startsection{section}{1}{\z@}
{-4ex \@plus -1ex \@minus -.4ex}
{1ex \@plus.2ex }
{\normalfont\large\sffamily\bfseries}}
\renewcommand{\subsection}{\@startsection {subsection}{2}{\z@}
{-3ex \@plus -0.1ex \@minus -.4ex}
{0.5ex \@plus.2ex }
{\normalfont\sffamily\bfseries}}
\renewcommand{\subsubsection}{\@startsection {subsubsection}{3}{\z@}
{-2ex \@plus -0.1ex \@minus -.2ex}
{.2ex \@plus.2ex }
{\normalfont\small\sffamily\bfseries}}                        
\renewcommand\paragraph{\@startsection{paragraph}{4}{\z@}
{-2ex \@plus-.2ex \@minus .2ex}
{.1ex}
{\normalfont\small\sffamily\bfseries}}

%----------------------------------------------------------------------------------------
%	PART HEADINGS
%----------------------------------------------------------------------------------------

% Numbered part in the table of contents
\newcommand{\@mypartnumtocformat}[2]{%
	\setlength\fboxsep{0pt}%
	\noindent\colorbox{violet!20}{\strut\parbox[c][.7cm]{\ecart}{\color{violet!70}\Large\sffamily\bfseries\centering#1}}\hskip\esp\colorbox{violet!40}{\strut\parbox[c][.7cm]{\linewidth-\ecart-\esp}{\Large\sffamily\centering#2}}%
}

% Unnumbered part in the table of contents
\newcommand{\@myparttocformat}[1]{%
	\setlength\fboxsep{0pt}%
	\noindent\colorbox{violet!40}{\strut\parbox[c][.7cm]{\linewidth}{\Large\sffamily\centering#1}}%
}

\newlength\esp
\setlength\esp{4pt}
\newlength\ecart
\setlength\ecart{1.2cm-\esp}
\newcommand{\thepartimage}{}%
\newcommand{\partimage}[1]{\renewcommand{\thepartimage}{#1}}%
\def\@part[#1]#2{%
\ifnum \c@secnumdepth >-2\relax%
\refstepcounter{part}%
\addcontentsline{toc}{part}{\texorpdfstring{\protect\@mypartnumtocformat{\thepart}{#1}}{\partname~\thepart\ ---\ #1}}
\else%
\addcontentsline{toc}{part}{\texorpdfstring{\protect\@myparttocformat{#1}}{#1}}%
\fi%
\startcontents%
\markboth{}{}%
{\thispagestyle{empty}%
\begin{tikzpicture}[remember picture,overlay]%
\node at (current page.north west){\begin{tikzpicture}[remember picture,overlay]%	
\fill[violet!20](0cm,0cm) rectangle (\paperwidth,-\paperheight);
\node[anchor=north] at (4cm,-3.25cm){\color{violet!40}\fontsize{220}{100}\sffamily\bfseries\thepart}; 
\node[anchor=south east] at (\paperwidth-1cm,-\paperheight+1cm){\parbox[t][][t]{8.5cm}{
\printcontents{l}{0}{\setcounter{tocdepth}{1}}% The depth to which the Part mini table of contents displays headings; 0 for chapters only, 1 for chapters and sections and 2 for chapters, sections and subsections
% for displaying section in TOC
}};
\node[anchor=north east] at (\paperwidth-1.5cm,-3.25cm){\parbox[t][][t]{15cm}{\strut\raggedleft\color{white}\fontsize{30}{30}\sffamily\bfseries#2}};
\end{tikzpicture}};
\end{tikzpicture}}%
\@endpart}
\def\@spart#1{%
\startcontents%
\phantomsection
{\thispagestyle{empty}%
\begin{tikzpicture}[remember picture,overlay]%
\node at (current page.north west){\begin{tikzpicture}[remember picture,overlay]%	
\fill[violet!20](0cm,0cm) rectangle (\paperwidth,-\paperheight);
\node[anchor=north east] at (\paperwidth-1.5cm,-3.25cm){\parbox[t][][t]{15cm}{\strut\raggedleft\color{white}\fontsize{30}{30}\sffamily\bfseries#1}};
\end{tikzpicture}};
\end{tikzpicture}}
\addcontentsline{toc}{part}{\texorpdfstring{%
\setlength\fboxsep{0pt}%
\noindent\protect\colorbox{violet!40}{\strut\protect\parbox[c][.7cm]{\linewidth}{\Large\sffamily\protect\centering #1\quad\mbox{}}}}{#1}}%
\@endpart}
\def\@endpart{\vfil\newpage
\if@twoside
\if@openright
\null
\thispagestyle{empty}%
\newpage
\fi
\fi
\if@tempswa
\twocolumn
\fi}

%----------------------------------------------------------------------------------------
%	CHAPTER HEADINGS
%----------------------------------------------------------------------------------------

% A switch to conditionally include a picture, implemented by Christian Hupfer
\newif\ifusechapterimage
\usechapterimagetrue
\newcommand{\thechapterimage}{}%
\newcommand{\chapterimage}[1]{\ifusechapterimage\renewcommand{\thechapterimage}{#1}\fi}%
\newcommand{\autodot}{.}
\def\@makechapterhead#1{%
{\parindent \z@ \raggedright \normalfont
\ifnum \c@secnumdepth >\m@ne
\if@mainmatter
\begin{tikzpicture}[remember picture,overlay]
\node at (current page.north west)
{\begin{tikzpicture}[remember picture,overlay]
\node[anchor=north west,inner sep=0pt] at (0,0) {\ifusechapterimage\includegraphics[width=\paperwidth]{\thechapterimage}\fi};
\draw[anchor=west] (\Gm@lmargin,-9cm) node [line width=2pt,rounded corners=15pt,draw=violet,fill=white,fill opacity=0.5,inner sep=15pt]{\strut\makebox[22cm]{}};
\draw[anchor=west] (\Gm@lmargin+.3cm,-9cm) node {\huge\sffamily\bfseries\color{black}\thechapter\autodot~#1\strut};
\end{tikzpicture}};
\end{tikzpicture}
\else
\begin{tikzpicture}[remember picture,overlay]
\node at (current page.north west)
{\begin{tikzpicture}[remember picture,overlay]
\node[anchor=north west,inner sep=0pt] at (0,0) {\ifusechapterimage\includegraphics[width=\paperwidth]{\thechapterimage}\fi};
\draw[anchor=west] (\Gm@lmargin,-9cm) node [line width=2pt,rounded corners=15pt,draw=violet,fill=white,fill opacity=0.5,inner sep=15pt]{\strut\makebox[22cm]{}};
\draw[anchor=west] (\Gm@lmargin+.3cm,-9cm) node {\huge\sffamily\bfseries\color{black}#1\strut};
\end{tikzpicture}};
\end{tikzpicture}
\fi\fi\par\vspace*{270\p@}}}

%-------------------------------------------

\def\@makeschapterhead#1{%
\begin{tikzpicture}[remember picture,overlay]
\node at (current page.north west)
{\begin{tikzpicture}[remember picture,overlay]
\node[anchor=north west,inner sep=0pt] at (0,0) {\ifusechapterimage\includegraphics[width=\paperwidth]{\thechapterimage}\fi};
\draw[anchor=west] (\Gm@lmargin,-9cm) node [line width=2pt,rounded corners=15pt,draw=violet,fill=white,fill opacity=0.5,inner sep=15pt]{\strut\makebox[22cm]{}};
\draw[anchor=west] (\Gm@lmargin+.3cm,-9cm) node {\huge\sffamily\bfseries\color{black}#1\strut};
\end{tikzpicture}};
\end{tikzpicture}
\par\vspace*{270\p@}}
\makeatother

%----------------------------------------------------------------------------------------
%	LINKS
%----------------------------------------------------------------------------------------

\usepackage{hyperref}
% \hypersetup{hidelinks,backref=true,pagebackref=true,hyperindex=true,colorlinks=false,breaklinks=true,urlcolor=violet,bookmarks=true,bookmarksopen=false}
\hypersetup{hidelinks,colorlinks=false,breaklinks=true,urlcolor=violet,bookmarksopen=false}


\usepackage{bookmark}
\bookmarksetup{
open,
numbered,
addtohook={%
\ifnum\bookmarkget{level}=0 % chapter
\bookmarksetup{bold}%
\fi
\ifnum\bookmarkget{level}=-1 % part
\bookmarksetup{color=violet,bold}%
\fi
}
}



% for the long section
% \section[Hidden Markov Map Matching Through Noise and Sparseness]{\texorpdfstring{Hidden Markov Map Matching\\ Through Noise and Sparseness}{Hidden Markov Map Matching Through Noise and Sparseness}}
% the optional parameter is for TOC display, the second one is for TeX display, the third one is for PDF string

%----------------------------------------------------------------------------------------
%	FONTS
%----------------------------------------------------------------------------------------


% \usepackage{avant} % Use the Avantgarde font for headings
%\usepackage{times} % Use the Times font for headings
% \usepackage{mathptmx} % Use the Adobe Times Roman as the default text font together with math symbols from the Sym­bol, Chancery and Com­puter Modern fonts

\usepackage{microtype} % Slightly tweak font spacing for aesthetics
%\usepackage[utf8]{inputenc} % Required for including letters with accents

\usepackage[T1]{fontenc} % Use 8-bit encoding that has 256 glyphs

% support CJK characters
\usepackage{xeCJK}

% use Noto Serif as italic font
% \setmainfont[ItalicFont=NotoSerif-Italic.ttf]{RobotoSlab-VariableFont_wght.ttf}
% \setsansfont{Fira Sans}
\usepackage{biolinum}
% \setsansfont{biolinum}
% \setsansfont[BoldFont=SegUIVar.ttf, ItalicFont=SegUIVar.ttf]{SegUIVar.ttf}
%\usepackage{tgbonum}
% \usepackage{tgbonum}

% support chemical equation arrows, deprecated
% This may cause error sometimes.
%\usepackage[version=4]{mhchem}

% use baskerville
%\usepackage[baskerville,vvarbb]{newtxmath}
%\usepackage[cal=boondoxo]{mathalfa}
%\usepackage{baskervillef}

% use fira sans
%\usepackage[sfdefault,lining]{FiraSans} %% option 'sfdefault' activates Fira Sans as the default text font
% \usepackage[fakebold]{firamath-otf}

%use scholax
\usepackage[p,osf]{scholax}
\usepackage[scaled=1.075,ncf,vvarbb]{newtxmath}% need to scale up math package
% vvarbb selects the STIX version of blackboard bold.

% set Chinese font
% \setmainfont[BoldFont=NotoSerif-Bold.ttf, ItalicFont=NotoSerif-Italic.ttf]{NotoSerif-Regular.ttf}
\setmonofont{Cascadia Code}
\setCJKmainfont[BoldFont=PingFang Heavy.ttf,ItalicFont=FZKTK.TTF]{FZPingXYSJW.TTF}
\setCJKsansfont[BoldFont=PingFang Heavy.ttf,ItalicFont=PingFang Heavy.ttf]{PingFang Heavy.ttf}
\setCJKmonofont[BoldFont=PingFang Heavy.ttf,ItalicFont=PingFang Heavy.ttf]{FZFSK.TTF}
\xeCJKsetup{CJKmath = true}

%----------------------------------------------------------------------------------------
% Special packages which is needed to loaded last
%----------------------------------------------------------------------------------------
% cleveref must be loaded after amsmath
\usepackage{cleveref}

% warning: unicode-math may need to be load last
% \usepackage{unicode-math}

%----------------------------------------------------------------------------------------
% renew list of contents
\makeatletter
%----------------------------------------------------------------------------------------

% \renewcommand\listoftables{%
%     \section{\listtablename}%
%     \@mkboth{\MakeUppercase\listtablename}%
%         {\MakeUppercase\listtablename}%
%     \@starttoc{lot}%
% }
\renewcommand\listoftables{%
    \chapter{\listtablename}%
    \@mkboth{\listtablename}%
        {\listtablename}%
    \@starttoc{lot}%
}

\renewcommand\listofalgorithms{
	\chapter{\listalgorithmcfname}%
    \@mkboth{\listalgorithmcfname}%
        {\listalgorithmcfname}%
    \@starttoc{loa}%
}

\renewcommand\listoffigures{%
    \chapter{\listfigurename}%
    \@mkboth{\listfigurename}%
        {\listfigurename}%
    \@starttoc{lof}%
}
\makeatother
