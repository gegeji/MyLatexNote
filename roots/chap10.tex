\chapter{Law, Judgement and Equality}

\section{leg, legis (law)}

\begin{DefWord}{legal}
\end{DefWord}

\begin{DefWord}{illegal}
\end{DefWord}

\begin{DefWord}{legality}
    /lɪˈɡæləti/

    the fact of being allowed by law 合法性
    \textit{Several ministers have questioned the legality of the ban.}

    \textbf{legalities} [plural] the legal parts of an agreement
    \textit{You need a lawyer to explain all the legalities of the contracts.}
\end{DefWord}

\begin{DefWord}{delegate, delegation}[delegate]
    noun /ˈdeləɡət/, verb /ˈdelɪɡeɪt/

    someone who has been elected or chosen to speak, vote, or take decisions for a group
    \textit{Around 350 delegates attended the conference.}

    to give part of your power or work to someone in a lower position than you
    \textit{A good manager knows when to delegate. 一个好的经理知道何时该放权。}
    \textit{Minor tasks should be \textbf{delegated to your assistant}.}

    to choose someone to do a particular job, or to be a representative of a group, organization etc
    \textbf{delegate somebody to do something}
    \textit{I was delegated to find a suitable conference venue.}

    delegation /ˌdeləˈɡeɪʃən/:
    a group of people who represent a company, organization etc
    \textit{the head of the American \textbf{delegation to} the United Nations}
    \textit{a \textbf{delegation of} government officials}

    the process of giving power or work to someone else so that they are responsible for part of what you normally do
    \textit{the \textbf{delegation of} authority 授权}
\end{DefWord}

\begin{DefWord}{relegate, relegation}[relegate]
    to give someone or something a less important position than before
    \textbf{relegate somebody/something to something}
    \textit{Women tended to be relegated to typing and filing jobs. 妇女往往被安排去做打字、归档等较不重要的工作。}

    if a sports team is relegated, it is moved into a lower division (opp  promote)
    \textbf{relegate something/somebody to something}
    \textit{We were relegated to the Fourth Division last year. 去年我们被降为丁级队。}
\end{DefWord}

\begin{DefWord}{legislate}
    to make a law about something
    \textit{Only Parliament has
     the power to \textbf{legislate on} national matters.}
     \textit{The government has promised to \textbf{legislate against} discrimination.}
     \textit{We must \textbf{legislate to control drugs}.}
    \textit{Parliament \textbf{legislated for} a higher minimum wage.}
    \textit{to legislate morality}

    \textbf{legislate for something} to think about how something may affect what you are doing, and do something to prepare for it 预防某事物:
    \textit{You can't legislate for bad luck.}
    
\end{DefWord}

\begin{DefWord}{legislative}
   [only before noun] concerned with making laws
   \textit{The new assemblies will have no legislative power. 新的议会将没有立法权。}
    \textbf{legislative assembly/council/body etc}

    \textbf{legislative elections}
\end{DefWord}

\begin{DefWord}{legislation}
    a law or a set of laws passed by a parliament
    \textit{an important piece of legislation 一条重要的法规}

    \textit{the \textbf{legislation on} abortion}

    \textit{new \textbf{legislation to }protect children}

    \textit{The government is bringing in \textbf{legislation to} combat this problem.}

    \textit{The company can be prosecuted \textbf{under the new legislation}. 根据新法规的规定﹐这家公司有可能遭到起诉。}

    \textbf{banking/environmental/financial/agricultural/international legislation}

    the process of making and passing laws
    \textit{Legislation will be difficult and will take time.}
\end{DefWord}

\begin{DefWord}{legislature}
    an institution that has the power to make or change laws

    \textbf{the national/state legislature}
\end{DefWord}

\begin{DefWord}{legitimate, legitimately, legitimacy, illegitimate, illegitimacy, illegitimately}[legitimate]
    /ləˈdʒɪtəmət/

    fair or reasonable
    \textit{That's a perfectly legitimate question. 那个问题完全合乎情理。}
    \textit{Most scientists believe it is legitimate to use animals in medical research.}

    acceptable or allowed by law
    \textit{Their business operations are perfectly legitimate.}

    a legitimate child is born to parents who are legally married to each other

    \textbf{legitimate power/interest/person (获确立婚生地位人士)/authority}
    \textit{The idea is not people can learn that there is white privilege and be considered to have learned it, and learn some other things. \# The idea is you are to learn that you're a privileged white person; you are to learn it over and over; really what you're supposed to learn is to feel guilty about it; and to express that on a regular basis, understanding that at no point in your lifetime will you ever be a morally legitimate person, because you have this privilege. It becomes a kind of Christian teaching, and it seems to serve a certain purpose -- -- I have to say this, I hope it doesn't hurt anybody's feelings. For white people, it is a great way to show that you understand racism is real. For black people and Latino people, it is a great way to assuage how bad a self-image a race can have after hundreds of years of torture. }
\end{DefWord}

\begin{DefWord}{legitimize (also legitimise)}
    to make something that is unfair or morally wrong seem acceptable and right
    \textit{There is a danger that these films legitimize violence.}

    to make something official or legal
    \textit{Acceptance by the UN would effectively legitimize the regime.}

    when parents legitimize a child, they get married so that the child becomes \textbf{legitimate} 赋予〔非婚生子女〕合法地位

\end{DefWord}

\section{jur (law; to sware), juris (law; oath)}

\begin{DefWord}{jury}
    a group of often 12 ordinary people who listen to the details of a case in court and decide whether someone is guilty or not

    the \textbf{members of the jury} 陪审团成员
    The jury found him not guilty. 陪审团裁定他无罪。
    the right to \textbf{trial by jury} 由陪审团进行审理的权利
    \textbf{sit/serve on a jury} be part of a jury

    a group of people chosen to judge a competition


    \textbf{the jury is (still) out on something} used to say that something has not been finally decided
    \textit{Is it good value? The jury is still out on that.}

\end{DefWord}

\begin{DefWord}{jural}
    relating to the law.
\end{DefWord}

\begin{DefWord}{injury}
    a wound or damage to part of your body caused by an accident or attack
    \textit{The driver of the truck suffered injuries to his legs and arms.卡车司机的腿和手臂受了伤。}
    \textit{Beckham has missed several games \textbf{through injury} (=because of injury).}

    damage to someone's reputation, career, or feelings
    \textit{He says that the allegations caused serious \textbf{injury to} his reputation.}

    \textbf{serious/terrible/fatal/minor/permanent/bodily/severe injury}

    \textbf{a nasty} (quite bad) \textbf{head/leg/shoulder/spinal/facial/sports/industrial injury}

    \textbf{internal/multiple injuries}

    \textit{Tom was OK, and \textbf{had} just a few minor \textbf{injuries}.}

    \textit{He \textbf{suffered} a serious leg \textbf{injury} in a motorcycle accident.}

    \textit{He couldn't take the chance of \textbf{getting an injury.}}

    \textbf{sustain/receive an injury} \textit{formal} (=suffer an injury)
     \textit{She sustained an injury to her hip. 她臀部受了伤。}

    \textbf{escape/avoid injury}
    \textit{Two workmen narrowly escaped injury when a wall collapsed.}

    \textbf{inflict an injury on somebody}  (=make someone have an injury)
    \textit{Jenkins was accused of inflicting a head injury on one of his former colleagues. 詹金斯被指控打伤以前同事的头部。}

    \textbf{be prone to injury}
    \textit{She was rather prone to injury and often missed matches as a result.}

    \textit{ It took her six months to \textbf{recover from} the injury.}

    \textit{Be careful with that knife or you'll \textbf{do yourself an injury.}}

    \textbf{protect from injury 防止受伤}
\end{DefWord}

\begin{DefWord}{injurious}
    causing injury, harm, or damage 有害的﹐致伤的
    
    \textbf{injurious to}
    \textit{Smoking is injurious to health.}
    \textit{injurious cargo}
\end{DefWord}


\begin{DefWord}{perjury}
    the crime of telling a lie after promising to tell the truth in a court of law, or a lie told in this way

    \textbf{to commit perjury}

    \textit{The defence witnesses were found guilty of perjury.}
\end{DefWord}


\begin{DefWord}{conjure}
    to perform clever tricks in which you seem to make things appear, disappear, or change by magic
    \textit{The magician conjured a rabbit out of his hat.}

    to make something appear or happen in a way which is not expected
    \textit{He has conjured victories from worse situations than this.}

    \textbf{a name to conjure with/a name to reckon with}
    \textit{Miyazaki is still a name to conjure with among anime fans. 宫崎骏在日本动漫迷中仍是一个大名鼎鼎的名字。}

    conjure something $\leftrightarrow$ up
    to bring a thought, picture, idea, or memory to someone's mind
    \textbf{conjure up images/pictures/thoughts etc (of something)}
    \textit{Dieting always seems to conjure up images of endless salads.}

    to make something appear when it is not expected, as if by magic
    \textit{Somehow we have to conjure up another \$10,000.}

    to make the soul of a dead person appear by saying special magic words

    to charge or entreat earnestly or solemnly
    \textit{"I conjure you … to weigh my case well … "}
\end{DefWord}

\begin{DefWord}{abjure}
    to state publicly that you will give up a particular belief or way of behaving (renounce)
    \textit{to abjure one's religion/errors/a bad habit}
\end{DefWord}


\begin{DefWord}{jurisdiction}
    the right to use an official power to make legal decisions, or the area where this right exists

    \textbf{jurisdiction over somebody/something}
    \textit{The committee has \textbf{jurisdiction over} all tax measures.}

\end{DefWord}




\begin{DefWord}{jurisprudence}
    the science or study of law 法学
    \textit{In American jurisprudence this is called judicial legislation.}
    \textit{international/natural/criminal jurisprudence}
\end{DefWord}





\section{judg, judic}


\begin{DefWord}{prejudice}
    an unreasonable dislike of or preference for a person, group, custom, etc., especially when it is based on their race, religion, sex, etc.
    \textit{a victim of racial prejudice}

    \textit{prejudice against somebody/something There is little prejudice against workers from other EU states. 对来自其他欧盟国家的劳工可说并无偏见。}
    
    \textbf{prejudice in favour of somebody/something} \textit{I must admit to a prejudice in favour of British universities. 我得承认我对英国大学有所偏爱。}

    \textbf{without prejudice (to something)}
    (law法律) without affecting any other legal matter
    \textit{They agreed to pay compensation without prejudice (= without admitting guilt).}

    \textbf{prejudice somebody (against somebody/something)} to influence somebody so that they have an unfair or unreasonable opinion about somebody/something
    \textit{The prosecution lawyers have been trying to prejudice the jury against her.}

    \textbf{prejudice something} (formal) to have a harmful effect on something
    \textit{Any delay will prejudice the child's welfare.}

    \textbf{arouse/break down/have/with/without/racial prejudice}

    \textit{Pride and Prejudice}

\end{DefWord}

\begin{DefWord}{judicial}
    connected with a court, a judge or legal judgement

    \textit{judicial powers 司法权}

    \textit{the judicial process/system 司法程序/系统}

    \textit{Greenpeace applied for a judicial review to challenge the court's decision.}

    \textit{judicial inquiry(问询)/mind(公正的心)/robe(法官袍)/survey(法庭调查)}
\end{DefWord}

\begin{DefWord}{judiciary}
    [countable + singular or plural verb]
    (plural \textbf{judiciaries})
    \textbf{the judiciary}
    all the judges in a country who, as a group, form part of the system of government; the part of a country's government that is responsible for its legal system and that consists of all the judges in its courts of law
    \textit{an independent judiciary}
\end{DefWord}

\begin{DefWord}{judicious, injudicious}[judicious]
    careful and sensible; showing good judgement (wise)
    \textit{It is curable with judicious use of antibiotics.}

    \textit{a judicious choice}
\end{DefWord}

\begin{DefWord}{adjudicate}
    /əˈdʒuːdɪkeɪt/

    to make an official decision about who is right between two groups or organizations that disagree

    \textbf{adjudicate (on/upon/in something)} \textit{A special subcommittee adjudicates on planning applications.}

    \textbf{adjudicate (something) (between A and B)} \textit{Their purpose is to adjudicate disputes between employers and employees.}

    \textit{The court has adjudicated the bankrupt (of the company).}

    to be a judge in a competition
    \textit{Who is adjudicating at this year's contest?}


\end{DefWord}

\begin{DefWord}{judge}
\end{DefWord}

\begin{DefWord}{judgement, judgment}[judgment]
    the ability to make sensible decisions after carefully considering the best thing to do
    
    \textbf{good/poor/sound judgement}

    \textbf{lack one's judgment}

    \textit{He achieved his aim \textbf{more by luck than judgement}.}

    \textbf{error of judgment} a mistake in the way that you examine a situation and decide what to do
    \textit{The accident was caused by an \textbf{error of judgement} on the part of the pilot.}

    an opinion that you form about something after thinking about it carefully; the act of making this opinion known to others

    \textbf{cool/mature/wise/solid/subjective judgment}

    \textit{In his portrait of the dictator he avoids any \textbf{moral judgements}.}

    \textbf{judgement about something} \textit{He refused to \textbf{make a judgement} about the situation.}

    \textbf{judgement on something} \textit{Who am I to pass judgement on her behaviour? (= to criticize it)}

    \textit{It will probably take some time for history to give its \textbf{final judgement} on his legacy.}

    \textbf{in somebody's judgement} \textit{It was, in her judgement, the wrong thing to do.}

    \textit{I did it \textbf{against my better judgement} (= although I thought it was perhaps the wrong thing to do).}

    \textbf{pass judgment (on something)}
    (=give your opinion, especially a negative one)
     \textit{Our aim is to help him, not to pass judgment on what he has done.}



    the decision of a court or a judge
    \textit{The \textbf{judgment will be given} tomorrow. 此案将于明日宣判。}
    \textit{The court has yet to \textbf{pass judgment} (= say what its decision is) in this case. 此案还有待法庭判决。}

    \textbf{written judgment}

    \textbf{judgement (on somebody)} (formal) something bad that happens to somebody that is thought to be a punishment from God 报应;天谴;(上帝对人的)审判

    \textbf{reserve (your) decision/judgement}
    to not decide or make a judgement about something until a later time 判决;裁决
    \textit{I'd prefer to reserve judgement until I know all the facts.}

    \textbf{sit in judgement (on/over/upon somebody)}
    to decide whether somebody's behaviour is right or wrong, especially when you have no right to do this
    \textit{How dare you sit in judgement on me?}
\end{DefWord}

\begin{DefWord}{prejudge}
    prejudge something to make a judgement about a situation before you have all the necessary information 预先判断;过早判断
    \textit{They took care not to prejudge the issue.}
\end{DefWord}

\begin{DefWord}{adjudge}
    to make a decision about somebody/something based on the facts that are available 宣判;裁决;判定

    \textbf{be adjudged + adj.} \textit{The company was adjudged bankrupt. 该公司被宣判破产。}
    \textit{The measures have since been adjudged inadequate. 自那以后,这些措施被认为是不够的。}

    \textbf{be adjudged + noun}\textit{ The tour was adjudged a success. 这次出行被认为是成功的。}

    \textbf{something is adjudged to be, have, etc. something} \textit{The reforms were generally adjudged to have failed.}

    \textit{to adjudge a man to die/death/prison}

\end{DefWord}






\section{par (being equal)}


\begin{DefWord}{compare, comparison}[compare]
    to examine people or things to see how they are similar and how they are different
    \textbf{compare A and B} \textit{It is interesting to compare their situation and ours.}

    \textit{\textbf{Compare and contrast} the characters of Jack and Ralph.比较杰克和拉尔夫的性格。}

    \textbf{compare A with/to B} 
    \textit{We \textbf{compared the results} of our study with those of other studies.}
    \textit{My own problems seem insignificant \textbf{compared with} other people's.}
    \textit{They receive just over three years of schooling, \textbf{compared to} a national \textbf{average} of 7.3.}
    \textit{an increase of over 11\% \textbf{compared to} the same \textbf{period} last year}

    \textbf{compare with/to somebody/something} to be similar to somebody/something else, either better or worse 可以匹敌,媲美
    \textit{This school compares with the best in the country (= it is as good as them).}

    \textbf{compare favorably/unfavorably} much better/worse than
    \textit{Their prices \textbf{compare favourably} to those of their competitors. 他们的价格比竞争者的要优惠。}



    \textbf{compare A to B} to show or state that somebody/something is similar to somebody/something else
    \textit{In her early career she was often compared to Ella Fitzgerald.}

    \textbf{beyond/without compare}
    (literary) better than anything else of the same kind
    \textit{Our professional service promises you a wedding without compare.}

    \textbf{compare notes (with somebody)}  
    if two or more people compare notes, they each say what they think about the same event, situation, etc.(与…)交换看法(或意见等)
    \textit{We saw the play separately and compared notes afterwards.}


    \textbf{comparison with somebody/something}
    \textit{The education system \textbf{bears/stands no comparison with} (= is not as good as) that in many Asian countries. 这种教育制度比不上许多亚洲国家的教育制度。}

    \textbf{comparison of A and B} \textit{a comparison of the rail systems in Britain and France 英国和法国铁路系统的比较}

    \textbf{comparison of A with B} \textit{a comparison of men's salaries with those of women 男女薪酬的比较}

    \textbf{comparison between A and B} \textit{comparisons between Britain and the rest of Europe 英国与欧洲其他国家之间的各种比较}

    \textbf{comparison of A to B} \textit{a comparison of the brain to a computer (= showing what is similar) 将大脑比作计算机}
 
    \textbf{comparison with somebody/something} \textit{It is difficult to \textbf{make a comparison} with her previous book—they are completely different. 这很难与她以前的书相比,两者是截然不同的。}

    You can \textit{draw comparisons} with the situation in Ireland (= say how the two situations are similar).这种情形可与爱尔兰的相比。

    There is no published information that would allow a \textit{direct comparison} with other regions or countries.没有公布的信息可以与其他地区或国家进行直接比较。

    \textit{\textbf{By comparison}, expenditure on education increased last year.}

    \textit{The tallest buildings in London are small \textbf{in comparison with} New York's skyscrapers.}

    \textbf{pale beside/next to something | pale in/by comparison (with/to something) | pale into insignificance}   
    to seem less important when compared with something else
    \textit{Our problems pale into insignificance when compared to theirs.}

    \textbf{there's no comparison} 
    used to emphasize the difference between two people or things that are being compared
    \textit{In terms of price there's no comparison (= one thing is much more expensive than the other).}

    \textbf{beyond comparison/compare}
\end{DefWord}

\begin{DefWord}{parity, disparity, imparity}[parity]
    the state of being equal, especially the state of having equal pay or status
    \textbf{parity with somebody/something} \textit{Prison officers are demanding pay parity with the police force.}

    \textbf{parity between A and B} \textit{There is a lack of parity between men and women in many areas of life.}

    (finance金融) the fact of the units of money of two different countries being equal (两国货币的)平价

    \textbf{parity with something} \textit{ to achieve parity with the dollar 取得与美元的平价}

    \textbf{parity between A and B} \textit{ Some are predicting parity between the euro and the dollar within a year.}

    disparity: a difference, especially one connected with unfair treatment(尤指因不公正对待引起的)不同,不等,差异,悬殊

    \textbf{disparity between A and B} \textit{The wide disparity between rich and poor was highlighted.特别强调了贫富差距悬殊的问题。}
    \textbf{ disparity (in something)} \textit{There are growing regional disparities in economic prosperity.}

    imparity: a less common word for disparity
    \begin{remark}
        The usage of imparity is rare. 
    \end{remark}
\end{DefWord}





\begin{DefWord}{disparage}
    \textbf{disparage somebody/something} to suggest that somebody/something is not important or valuable
    (synonym belittle)

    \textit{I don't mean to disparage your achievements.}
\end{DefWord}






\section{equ, equi (being equal)}

\begin{DefWord}{equate}
    \textbf{equate something (with something)} to think that something is the same as something else or is as important
    \textit{Some parents equate education with exam success.}

    \textbf{equate to something}
    \textit{a rate of pay which equates to £6 per hour}
\end{DefWord}

\begin{DefWord}{equation}
    a statement showing that two amounts or values are equal, for example $2x + y = 54$
    \textbf{solve/work out/derive/formulate a equation}


    the act of making something equal or considering something as equal (= of equating them)
    \textit{The equation of wealth with happiness can be dangerous. 把财富与幸福等同起来可能是危险的。}

    a problem or situation in which several things must be considered and dealt with
    \textit{When children \textbf{enter the equation,} further tensions may arise within a marriage.}
    \textit{Money also \textbf{comes into the equation.}}
    \textit{The availability of public transport is also \textbf{part of the equation.}}

    \textit{military/chemical/supply-demand equation}
    
    \textit{equation of power (势力均衡)}
\end{DefWord}

\begin{DefWord}{equity, inequity, equitable, inequitable}[equity]
    the value of a company's shares
    \textit{He plans to raise the company's return on equity to 15\%.他计划将公司的股本回报率提高到15\%.}

    the value of a property after all charges and debts have been paid (公司的)股本;资产净值
    \textit{The couple have no savings except for the equity in their house.}

    \textbf{equities} [plural] (finance金融) shares in a company that do not pay a fixed amount of interest (公司的)普通股



    a situation in which everyone is treated equally 
    \textit{a society where justice and equity prevail一个正义和公平占上风的社会}
    \textit{racial/social equity}

    a system of natural justice allowing a fair judgement in a situation which is not covered by the existing laws 衡平法〔一种原则﹐指现存法律不适用时﹐应当作出公正的裁决〕
    \textit{The rules of common law and equity are both, in essence, systems of private law. 普通法和衡平法的规则本质上都是私法体系。}

    equitable: fair and reasonable; treating everyone in an equal way (fair)
    \textit{an equitable distribution of resources}
    \textit{inequitable distribution of wealth}
\end{DefWord}

\begin{DefWord}{iniquity}
    /ɪˈnɪkwəti/ (plural \textbf{iniquities})(formal)

the fact of being very unfair or wrong; something that is very unfair or wrong 很不公正,十分错误,很不正当(的事)
\textit{the iniquity of racial prejudice}

    \begin{remark}
        拼写是 \textbf{iniquity}, 读音是/ɪˈnɪkwəti/
    \end{remark}
\end{DefWord}

\begin{DefWord}{equivalent, equivalence}[equivalent]
    a thing, amount, word, etc. that is equal in value, meaning or purpose to something else 相等的东西;等量;对应词

    \textbf{equivalent of something} \textit{the modern equivalent of the Roman baths}

    \textbf{equivalent of doing something} \textit{Breathing such polluted air is the equivalent of (= has the same effect as) smoking ten cigarettes a day.}

    \textbf{equivalent to something} \textit{The German 'Gymnasium' is the closest equivalent to the grammar school in England.德语 Gymnasium 基本上相当于英格兰的文法学校。}

    equal in value, amount, meaning, importance, etc.
    \textit{Eight kilometres is roughly equivalent to five miles.}
\end{DefWord}

\begin{DefWord}{equivocal, unequivocal, unequivocally}[equivocal]
    (of words or statements言语或陈述) not having one clear or definite meaning or intention; able to be understood in more than one way (ambiguous)
    \textit{She gave an equivocal answer, typical of a politician.}

    difficult to understand or explain clearly or easily
    \textit{The experiments produced equivocal results.}
\end{DefWord}