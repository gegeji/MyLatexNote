\chapter{Roots about Variation and Change}

\section{alter (variation)}

\begin{RefWord}{alterable}
\end{RefWord}

\begin{RefWord}{alteration}
\end{RefWord}

\begin{RefWord}{alternate}
\end{RefWord}

\begin{RefWord}{alternately}
\end{RefWord}

\begin{RefWord}{alternative}
\end{RefWord}

\begin{RefWord}{alternation}
\end{RefWord}

\section{vari (variation)}

\begin{RefWord}{various}
\end{RefWord}

\begin{RefWord}{varied}
\end{RefWord}

\begin{RefWord}{vary}
\end{RefWord}

\begin{RefWord}{variety}
\end{RefWord}

\begin{RefWord}{variable}
\end{RefWord}

\begin{RefWord}{variably}
\end{RefWord}

\begin{RefWord}{variability}
    the fact of something being likely to vary
\end{RefWord}

\section{simil (similarity)}

\begin{RefWord}{similar}
\end{RefWord}

\begin{RefWord}{similarity}
\end{RefWord}

\begin{RefWord}{assimilate}
    take in (information, ideas, or culture) and understand fully.
\textit{Marie tried to assimilate the week's events}

cause (something) to resemble; liken.
\textit{Philosophers had assimilated thought to perception.}
\textit{After studying Chinese culture, can you assimilate easily into our Chinese way of life?}

\end{RefWord}

\section{semble (similarity)}

\begin{RefWord}{semblance}
    \textbf{a/some semblance of something} a situation, condition etc that is close to or similar to a particular one, usually a good one
    \textit{She was trying to get her thoughts back into some semblance of order. 她想稍微理出个头绪来。}
    \textit{Life at last returned to some semblance of normality. 生活似乎终于恢复了正常。}
    \textit{After the war, life returned to a semblance of normality.战后﹐生活稍微恢复了正常。}

    \textbf{semblance of something} a situation in which something seems to exist although this may not, in fact, be the case表象;假象;外观;外貌
    \textit{She bared her teeth in a semblance of a smile. 她露出牙齿,显出一副笑容。}
    \textit{She struggled to bring a semblance of order to the meeting. 她尽力想使会场显得有些秩序。}
\end{RefWord}

\begin{RefWord}{resemble, resemblance}[resemble]
\end{RefWord}

\begin{RefWord}{assemble}
\end{RefWord}

\begin{RefWord}{assembly}
\end{RefWord}

\begin{RefWord}{dissemble}
\end{RefWord}

\section{simul}

\begin{RefWord}{simulate}
\end{RefWord}

\begin{RefWord}{simulation}
\end{RefWord}

\begin{RefWord}{simultaneous, simultaneously}[simultaneous]
\end{RefWord}

\section{sort (type)}

\begin{RefWord}{sort}
\end{RefWord}

\begin{RefWord}{assort, assorted, assortment}[assort]
    assort: 分类


    assorted: of various different types 各种各样的:
    \textit{paintbrushes in assorted sizes 各种大小的画笔}
    \textit{assorted vegetables 各种各样的蔬菜}
\end{RefWord}

\begin{RefWord}{consort}
    \textbf{consort with somebody} to spend time with somebody that other people do not approve of

\end{RefWord}

\section{spec}

\begin{RefWord}{specific}
\end{RefWord}

\begin{RefWord}{speciality}
    a type of food or product that a restaurant or place is famous for because it is so good
    \textit{Seafood is a speciality on the island.}
\end{RefWord}

\begin{RefWord}{specialize}
\end{RefWord}

\begin{RefWord}{specious}
    seeming to be true or correct, but actually false 似是而非的﹐貌似正确的 (SYN  misleading)
\end{RefWord}

\begin{RefWord}{specification}
\end{RefWord}

\begin{RefWord}{specimen}
\end{RefWord}

\section{type}

\begin{RefWord}{prototype}
\end{RefWord}

\begin{RefWord}{sterotype}
    a fixed idea or image that many people have of a particular type of person or thing, but which is often not true in reality  模式化观念(或形象);老一套;刻板印象
    \textit{\textbf{cultural/gender/racial stereotypes}}

    to form a fixed idea about a person or thing which may not really be true 对…形成模式化(或类型化)的看法
    \textbf{stereotype somebody} \textit{Children from certain backgrounds tend to be stereotyped by their teachers. 教师往往模式化地根据学生的某些背景把他们归类。}

    \textbf{stereotype somebody as something} \textit{Why are professors stereotyped as absent-minded?}
\end{RefWord}

