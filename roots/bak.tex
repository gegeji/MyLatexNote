\chapter{Breaking, Building, Standing}

\section{frag (to break)}

\begin{DefWord}{fragile}

    easily broken or damaged
    \textit{Porcelain (hard shiny white substance that is used for making expensive plates, cups etc =china) vase is fragile.}

    weak and uncertain; easy to destroy or harm
    \textit{a \textbf{fragile alliance/ceasefire/relationship}}

    thin or light and often beautiful
    \textit{fragile beauty 纤美}

    not strong and likely to become ill
    \textit{Her father is now 86 and in fragile health.}

    (British English, informal) \textit{I'm feeling a bit fragile after last night (= not well, perhaps because of drinking too much alcohol).}
\end{DefWord}

\begin{DefWord}{fragment}
    \textbf{fragment (of something)} \textit{Police found fragments of glass near the scene.}

    \textbf{in fragments} \textit{The shattered vase lay in fragments on the floor.}
\end{DefWord}


\begin{DefWord}{fragmented}
\end{DefWord}

\begin{DefWord}{fragmentation}
\end{DefWord}


\begin{DefWord}{fragmentary}
    made of small parts that are not connected or complete 残缺不全的;不完整的
    \textit{There is only fragmentary evidence to support this theory.}
\end{DefWord}




\section{fract}


\section{struc}


\section{stroy}



\section{sta}



\section{stitu}


\section{sist}