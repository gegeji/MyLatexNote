\begin{word}{repel, repulsive}[repel]
    if something repels you, it is so unpleasant that you do not want to be near it, or it makes you feel ill使厌恶﹐使反感 → :
    \textit{The smell repelled him}

    to make someone who is attacking you go away, by fighting them
    \textit{The army was ready to repel an attack.}

    to keep something or someone away from you:
    \textit{a lotion that repels mosquitoes}

    if two things repel each other, they push each other away with an electrical force (OPP attract)
    \textit{Two positive charges repel each other.}
\end{word}

\begin{word}{impel}
    if something impels you to do something, it makes you feel very strongly that you must do it ($\rightarrow$ \ref{compel})

    \textit{The lack of democracy and equality impelled the oppressed to fight for independence.}
\end{word}

\begin{word}{compel,  compulsion}[compel]
    to force someone to do something:
    \textit{The law will compel employers to provide health insurance.}

    (formal) to make people have a particular feeling or attitude
    \textit{His performance compelled the audience’s attention.}
\end{word}

\begin{word}{propel}
    to move, drive, or push something forward
\end{word}


\begin{word}{propeller}
    a piece of equipment consisting of two or more blades that spin around, which makes an aircraft or ship move
\end{word}

\begin{word}{dispel}
    make something go away, especially a belief, idea, or feeling
    \textit{We want to \textbf{dispel} the \textbf{myth} that you cannot eat well in Britain.}

    Syn $\rightarrow$ \ref{expel}.
\end{word}

\begin{word}{expel, expulsion}[expel]
    Syn $\rightarrow$ \ref{dispel}.

    to officially force someone to leave a school or organization
    
    \textbf{expel somebody from something}
    \textit{Two girls were expelled from school for taking drugs.}

    \textbf{expel somebody for doing something}
    \textit{He was expelled for making racist remarks.}

    to force a foreigner to leave a country, especially because they have broken the law or for political reasons
    \textit{Foreign priests were expelled from the country.}

    \textbf{expel somebody for something}
    \textit{Three diplomats were expelled for spying.}
\end{word}

\begin{word}{compel}
\end{word}


\begin{word}{devolve}
\end{word}

\begin{word}{VOLVE}
\end{word}

\begin{word}{evolve, evolution}[evolve]
    if an animal or plant evolves, it changes gradually over a long period of time
    \textit{Fish \textbf{evolved from} prehistoric sea creatures.}

    to develop and change gradually over a long period of time

    \textit{The school has evolved its own style of teaching.}
    \textit{The group gradually \textbf{evolved into} a political party.}
    \textit{The idea \textbf{evolved out of} work done by British scientists.}
\end{word}

\begin{word}{revolve, revolution}[revolve]
    to move around like a wheel, or to make something move around like a wheel

    \textit{revolving door}

    the windmill

    \textbf{revolve around somebody/something} 
    \textit{She seems to think that the world \textbf{revolves around} her}
\end{word}