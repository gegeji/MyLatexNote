\chapter{Fear, Throwing/Casting, and Education}

\section{horr (to shudder/to dread)}

\begin{DefWord}{horrify}
\end{DefWord}

\begin{DefWord}{horrible}
\end{DefWord}

\begin{DefWord}{horror}
\end{DefWord}

\begin{DefWord}{horrendous}
    frightening and terrible 可怕的﹐骇人的 (horrible)
    \textit{a horrendous experience}

    extremely unreasonable or unpleasant
    \textit{horrendous debts}
\end{DefWord}

\begin{DefWord}{abhor, abhorrent}[abhor]
    to hate a kind of behaviour or way of thinking, especially because you think it is morally wrong 厌恶﹐憎恶〔某种行为或思想方式〕
 \textit{I abhor discrimination of any kind. 我厌恶任何一种歧视。}
\end{DefWord}







\section{terr (to frighten/to fear)}

\begin{DefWord}{terrible}
\end{DefWord}

\begin{DefWord}{terrify}
\end{DefWord}

\begin{DefWord}{terrific}
    (informal) very good, especially in a way that makes you feel happy and excited
    \textit{That’s a terrific idea!}

    very large in size or degree
    \textit{He drank a terrific amount of beer.}
\end{DefWord}

\begin{DefWord}{terror}
    a feeling of extreme fear
    \textit{People \textbf{fled in terror} as fire tore through the building.}

    an event or situation that makes people feel extremely frightened, especially because they think they may die
    t 
    \textit{he terrors of war}

    violent action for political purposes 恐怖活动
    \textit{The resistance movement started a campaign of terror.}

    (informal) a child who is difficult to control
    \textit{That Johnson kid’s a real little terror!}

\end{DefWord}

\begin{DefWord}{terroist}
\end{DefWord}

\begin{DefWord}{terrorism}
\end{DefWord}

\begin{DefWord}{deter, deterrent}[deter]
    to stop someone from doing something, by making them realize it will be difficult or have bad results 威慑住﹐吓住﹐使却步 $\rightarrow$  deterrent:

    \textbf{deter somebody from (doing) something}
    \textit{The security camera was installed to deter people from stealing.}

    something that makes someone less likely to do something, by making them realize it will be difficult or have bad results 遏制物﹐威慑物﹐威慑力量:
    \textit{The small fines for this type of crime do not \textbf{act as much of a deterrent}.}

    \textbf{nuclear deterrent} the nuclear weapons that a country has in order to prevent other countries from attacking it
\end{DefWord}








\section{sper (hope)}

\begin{DefWord}{despair}
\end{DefWord}

\begin{DefWord}{desperate, desperately, desperation, desperado}
    willing to do anything to change a very bad situation, and not caring about danger
    \textit{Time was running out and we were \textbf{getting desperate}. 时间越来越少﹐我们越来越感到绝望。}

    needing or wanting something very much
    \textit{The team is \textbf{desperate for} a win.这支球队急需一场胜利。}

    \textbf{a desperate situation} is very bad or serious

    desperado:  a violent criminal who is not afraid of danger 亡命之徒﹐暴徒
\end{DefWord}


\begin{DefWord}{prosper, prosperity, prosperous}[prosper]
\end{DefWord}

\section{ject (to throw/cast)}

\begin{DefWord}{objective}
\end{DefWord}

\begin{DefWord}{subjective}
\end{DefWord}

\begin{DefWord}{deject, dejected, dejection}
    (rare) make sad or dispirited; depress.

    dejected: unhappy, disappointed, or sad
    \textit{The unemployed stood at street corners, dejected.}
\end{DefWord}

\begin{DefWord}{eject, ejection}[eject]
    to make someone leave a place or building by using force〔用武力〕驱逐﹐赶出
    \textbf{eject somebody from something}
    \textit{The demonstrators were ejected from the hall.示威者被赶出大厅。}

    to make someone leave a job or position very quickly
    \textit{420 workers have been ejected from their jobs with no warning.}

    to suddenly send something out喷射
    \textit{Two engines cut out and the plane started to eject fuel as it lost height. 两个发动机突然熄火﹐飞机在下降的同时开始喷出燃料。}

    if you eject a disk or a tape, or if it ejects, it comes out of a machine after you have pressed a particular button

    if a pilot ejects, he or she escapes from a plane, using an ejector seat because it is going to crash
\end{DefWord}

\begin{DefWord}{reject, rejection, rejective}[reject]
    \textit{The proposal was firmly rejected. Their rejective attitude and blank rejection let us down.}
\end{DefWord}

\begin{DefWord}{inject, injection}[jnject]
    \textit{The drug is injected directly into the base of the spine.}

    \textbf{inject somebody with something}
    \textit{I have to inject myself with insulin.}
    
    \textit{Traditional handbag makers are \textbf{injecting more fun into} their designs.}

    \textit{They need to inject more money into sports facilities. 他们需要在体育设施上投入更多的资金。}
\end{DefWord}


\begin{DefWord}{project, projection}[project]
noun  /ˈprɒdʒekt/, verb /prəˈdʒekt/

a carefully planned piece of work to get information about something, to build something, to improve something etc 项目﹔工程﹔计划﹔规划:

to calculate what something will be in the future, using the information you have now预计﹐推断:
 \textit{The company projected an annual growth rate of 3\%. 该公司预计每年的增长率为3\%}。

\end{DefWord}









\section{miss}

\begin{DefWord}{remit, remittance}[remit]
    to send a payment 汇(款)
    \textit{Please remit payment by cheque.}

    to free someone from a debt or punishment 免除〔债务或处罚〕 $\rightarrow$  unremitting

    \textbf{remit something to somebody/something}
    to send a proposal, plan, or problem back to someone for them to make a decision about
    \textit{The court remitted the matter to the agency for reconsideration. 法庭将此事发回该机构作重新考虑。}

    the particular piece of work that someone has been officially asked to deal with
    \textit{the remit of a senior member of staff}

    remittance: an amount of money that you send to pay for something

    when you send money
    \textit{We will forward the goods on remittance of £10.}

    \textit{bank remittance voucher} (=an official statement or receipt that is given to someone to prove that their accounts are correct or that money has been paid)
\end{DefWord}

\begin{DefWord}{unremitting}
    continuing for a long time and not likely to stop
    \textit{unremitting poverty}
\end{DefWord}

\begin{DefWord}{transmit, transmission, transmitter}[transmit]
\end{DefWord}

\begin{DefWord}{commit, commitment}[commit]
    to do something wrong or illegal

    \textbf{commit murder/rape/arson etc}

    \textbf{commit suicide} to kill yourself deliberately

    \textbf{commit adultery} if a married person commits adultery, they have sex with someone who is not their husband or wife

    to say that someone will definitely do something or must do something
    \textbf{commit somebody to doing something}
    \textit{He has clearly committed his government to continuing down the path of economic reform. 他明确地作出保证﹐他的政府会继续在经济改革的道路上走下去。}

    \textit{I’d \textbf{committed myself} and there was no turning back.}

    \textbf{commit yourself to (doing) something} 
    \textit{The banks have \textbf{committed themselves to boosting} profits by slashing costs.}

    to give someone your love or support in a serious and permanent way
    \textit{Anna wants to get married, but Bob’s not sure he wants to commit.}

    to decide to use money, time, people etc for a particular purpose
    \textit{A lot of money has been committed to this project.}

    to send someone to be tried in a court of law
    \textit{The two men were \textbf{committed for trial} at Bristol Crown Court.}

    to order someone to be put in a hospital or prison
    \textit{The judge \textbf{committed him to prison} for six months.}
\end{DefWord}

\begin{DefWord}{omit, omission}[omit]
    to not include someone or something, either deliberately or because you forget to do it (leave)
    \textit{Please don’t omit any details, no matter how trivial they may seem.}

    \textbf{omit something from something}
    \textit{Lisa’s name had been omitted from the list of honor students.}

    \textbf{omit to do something} (formal) to not do something, either because you forgot or because you deliberately didn’t do it
    \textit{Oliver omitted to mention that he was married.}
\end{DefWord}

\begin{DefWord}{submit}
    to give a plan, piece of writing etc to someone in authority for them to consider or approve
    \textit{All applications must be submitted by Monday.}

    to agree to obey someone or something or to go through a process, especially when you have no choice〔尤指无可选择时〕同意服从[遵守﹐接受]﹐顺从 (SYN  give in)
    \textit{Derek has agreed to submit to questioning.}

    formal law to suggest or say something建议﹐主张
    \textbf{submit (that)}
    \textit{I submit that the jury has been influenced by the publicity in this case.}
\end{DefWord}

\begin{DefWord}{dismiss}
\end{DefWord}

\begin{DefWord}{compromise}
\end{DefWord}

\begin{DefWord}{mission, missionary}[mission]
    an important job that involves travelling somewhere, done by a member of the air force, army etc, or by a spacecraft 任务﹐使命

    an important job that someone has been given to do, especially when they are sent to another place 〔尤指给予被派遣人员的〕重要任务﹐使命
    \textit{a group of US congressmen on a \textbf{fact-finding mission} to Northern Ireland}

    something that you feel you must do because it is your duty职责﹐天职﹔使命 (SYN  calling, vocation)
    \textit{His main \textbf{mission in life} is to earn as much money as possible.}

    the purpose or the most important aim of an organization

    a group of important people who are sent by their government to another country to discuss something or collect information〔政府派往国外的〕代表团﹐工作团﹐使团 (SYN  delegation)
    \textit{a British trade mission to Moscow}

    religious work that involves going to a foreign country in order to teach people about Christianity or help poor people〔在国外进行的基督教的〕传教﹐布道

    a building where this kind of work is done, or the people who work there 传教所﹐布道所﹔传教团﹐布道团

    missionary: someone who has been sent to a foreign country to teach people about Christianity and persuade them to become Christians

    relating to the work of missionaries
    \textit{a missionary hospital}

    \textbf{missionary zeal} if you do something with missionary zeal, you do it with great eagerness, because you believe strongly that it is a good thing to do
    \textit{a young English teacher who taught poetry with missionary zeal}
\end{DefWord}

\begin{DefWord}{dismiss, dismissal}[dismiss]
    to refuse to consider someone’s idea, opinion etc, because you think it is not serious, true, or important
    \textit{The government has dismissed criticisms that the country’s health policy is a mess.}

    \textbf{dismiss something as something}
    \textit{He just laughed and dismissed my proposal as unrealistic.}

    \textbf{dismiss somebody from something}
    \textit{ Bryant was unfairly dismissed from his post.}

    to tell someone that they are allowed to go, or are no longer needed
    \textit{The class was dismissed early today.}

    if a judge dismisses a court case, he or she stops it from continuing
    \textit{The case was dismissed owing to lack of evidence. 由于证据不足﹐该案被驳回了。}
\end{DefWord}

\begin{remark}
    In everyday British English, people usually say \textbf{sack} someone, and in everyday American English, people usually say \textbf{fire} someone, rather than use dismiss:
\end{remark}

\section{duc/duct (to teach/guide)}

\begin{DefWord}{deduce, deducible, deductive, deduction}[deduce]
    to use the knowledge and information you have in order to understand something or form an opinion about it 推论﹐推断﹐演绎

    \textbf{deduce that}
    \textit{From her son’s age, I deduced that her husband must be at least 60. 从她儿子的年龄来推测﹐我想她丈夫肯定至少有60岁了。}

    \textbf{deduce from}
    \textit{What did Darwin deduce from the presence of these species? 达尔文从这些物种的存在推断出了什么?}

    \textbf{deductive reasoning} 演绎推理

    deduction: the process of using the knowledge or information you have in order to understand something or form an opinion, or the opinion that you form
    \textit{Children will soon \textbf{make deductions} about the meaning of a word.}

    the process of taking away an amount from a total, or the amount that is taken away
    \textit{After deductions for tax etc, your salary is about £700 a month.扣除税款等之后﹐你的月薪大约有700英镑。}
\end{DefWord}

\begin{DefWord}{conduct, conductor}[conduct]
    verb /kənˈdʌkt/, noun  /ˈkɒndʌkt/ 
    to carry out a particular activity or process, especially in order to get information or prove facts 〔尤指为获取信息或证实某事时〕进行﹔实施﹔执行
    \textit{We are \textbf{conducting a survey} of consumer attitudes towards organic food.}

    \textbf{conduct an experiment/a test}
    \textit{Is it really necessary to conduct experiments on animals?}

    to stand in front of a group of musicians or singers and direct their playing or singing

    \textbf{conduct yourself} formal to behave in a particular way, especially in a situation where people judge you by the way you behave 表现﹐为人:
    \textit{The players conducted themselves impeccably, both on and off the field.}


    if something conducts electricity or heat, it allows electricity or heat to travel along or through it 传导 → \textbf{conductor}
    \textit{Aluminum, being a metal, readily conducts heat.}

    to take or lead someone somewhere
    \textit{On arrival, I was conducted to the commandant’s office. 到达以后﹐我被带到了指挥所。}

    \textbf{conducted tour (of something)}(在某地)有导游陪同的参观旅行
    \textit{a conducted tour of Berlin (=a tour of a building, city, or area with someone who tells you about that place)}

    the way someone behaves, especially in public, in their job etc〔尤指在公共场合﹑工作岗位上等的〕行为﹐举止(behaviour)
    \textit{The Senator’s conduct is being investigated by the Ethics Committee.}

    \textit{the Law Society’s \textbf{Code of Professional Conduct} 法律协会的行业行为准则}

    \textit{his arrest for \textbf{disorderly conduct} (=noisy violent behaviour)他因妨害治安行为而遭逮捕}

    \textbf{conduct of something} the way in which an activity is organized and carried out
   \textit{complaints about the conduct of the elections}
\end{DefWord}

\begin{DefWord}{misconduct}
    bad or dishonest behaviour by someone in a position of authority or trust

    \textit{a doctor who has been accused of \textbf{professional misconduct} 被指控玩忽职守的医生}

    \textit{He was fired for \textbf{serious misconduct}}

    \textit{She was found guilty of \textbf{gross misconduct} (=very serious misconduct). 她被裁定犯有严重失职罪。}
\end{DefWord}

\begin{DefWord}{produce, product, production, producer}[produce]
\end{DefWord}

\begin{DefWord}{reproduce,reproduction}[reproduce]
    if an animal or plant reproduces, or reproduces itself, it produces young plants or animals
    \textit{The turtles return to the coast to reproduce.}

    to make a photograph or printed copy of something
    \textit{Klimt’s artwork is reproduced in this exquisite book.}

    to make something happen in the same way as it happened before (repeat, copy)
    \textit{British scientists have so far been unable to reproduce these results.}

    to make something that is just like something else
    \textit{With a good set of speakers, you can reproduce the orchestra’s sound in your own home.}
\end{DefWord}

\begin{DefWord}{reduce, reduction}[reduce]
    \textbf{reduce somebody/something to something}

    \textbf{reduce somebody to tears/silence etc} to make someone cry, be silent etc
    \textit{She was reduced to tears in front of her students.}

    \textbf{reduce somebody to doing something} to make someone do something they would rather not do, especially when it involves behaving or living in a way that is not as good as before
    \textit{Eventually Charlotte was reduced to begging on the streets.}

    \textbf{reduce something to ashes/rubble/ruins}
    to destroy something, especially a building, completely
    \textit{A massive earthquake reduced the city to rubble. 一场大地震把这座城市夷为废墟。}

    to change something into a shorter simpler form
    \textit{Many jobs can be reduced to a few simple points.}
\end{DefWord}

\begin{DefWord}{abduct, abduction, abductor, abductee}[abduct]
    to take someone away by force劫持﹐绑架 (SYN  kidnap)
    \textit{The diplomat was abducted on his way to the airport.  外交官在去机场途中遭绑架。}

    abductor: a person who abducts somebody
\end{DefWord}

\begin{DefWord}{seduce}
    to persuade someone to have sex with you, especially in a way that is attractive and not too direct
    \textit{The professor was sacked for seducing female students. 这个教授因诱奸女学生而被解雇。}
    \textit{Are you trying to seduce me? 你是在勾引我吗?}

    to make someone want to do something by making it seem very attractive or interesting to them
    \textit{I was young and seduced by New York. 我当时年轻﹐抵挡不住纽约的诱惑。}

    \textbf{seduce somebody into doing something}
    \textit{Leaders are people who can seduce other people into sharing their dream. 领导人就是要能够说服别人来共同实现他们的梦想。}

    \begin{remark}
        \textbf{Seduce} is often passive in this meaning.
    \end{remark}
\end{DefWord}

\begin{DefWord}{induce,inducement}[induce]
    to persuade someone to do something, especially something that does not seem wise

    \textbf{induce somebody to do something}
    \textit{Nothing would induce me to vote for him again. 没有什么能诱使我再投他的票了。}

    to make a woman give birth to her baby, by giving her a special drug〔用药物〕为〔产妇〕引产﹐催生
    \textit{She had to be induced because the baby was four weeks late. 她的孩子晚了四星期仍未出生﹐因此要给她引产。}
    \textit{The doctor decided to \textbf{induce labour}. 医生决定引产。}

    \textbf{induce sleep} 使人入睡

    to cause a particular physical condition 诱发〔某种身体反应〕
    \textit{Patients with eating disorders may use drugs to induce vomiting.}

    \textbf{drug-induced/stress-induced etc}
    \textit{a drug-induced coma}
\end{DefWord}

\begin{DefWord}{duct}
    a pipe or tube that liquids, air, cables etc pass through
    \textit{Air is heated and then circulated through large ducts to all parts of the house.}

    a narrow tube in your body or in a plant that liquid passes through
    \textit{a tear duct 泪管}
\end{DefWord}

\section{doc (to teach/to guide)}

\begin{DefWord}{docent}
    someone who guides visitors through a museum, church etc (guide)
\end{DefWord}

\begin{DefWord}{doctoral}
    done as part of work for the university degree of doctor
    博士(学位)的:
    \textit{a doctoral thesis}
\end{DefWord}

\begin{DefWord}{doctorate}
    a university degree of the highest level 博士学位
    \textit{She received her \textbf{doctorate in history} in 1998. 她于1998年获得历史学博士学位。}
\end{DefWord}


\begin{DefWord}{document}
\end{DefWord}

\begin{DefWord}{documentary}
    documentary films, programmes, photographs etc give or show information about a particular subject
    〔电影﹑电视﹑照片等〕纪录的﹐纪实的:
    \textit{documentary films}

    consisting of or written on documents
    文件的﹐文献的
    \textbf{documentary evidence/proof}
    \textit{One of the most useful sources of documentary evidence is maps.}
\end{DefWord}

\begin{DefWord}{doctrine}
    doctrine /ˈdɒktrɪn/,doctrinal /dɒkˈtraɪnl/

    a set of beliefs that form an important part of a religion or system of ideas
    信条﹐教义﹐主义﹐学说:
    \textit{traditional doctrines of divine power 传统的神力教义}
    \textit{Marxist doctrine 马克思主义学说}

    \textbf{Doctrine}  a formal statement by a government about its future plans〔政府未来政策的〕正式声明:
    \textit{the announcement of the Truman Doctrine杜鲁门主义的宣言}
\end{DefWord}

\begin{DefWord}{docile, indocile}[docile]
\end{DefWord}