\chapter{Breaking, Building, Standing}

\section{frag (to break)}

\begin{DefWord}{fragile}

    easily broken or damaged
    \textit{Porcelain (hard shiny white substance that is used for making expensive plates, cups etc =china) vase is fragile.}

    weak and uncertain; easy to destroy or harm
    \textit{a \textbf{fragile alliance/ceasefire/relationship}}

    thin or light and often beautiful
    \textit{fragile beauty 纤美}

    not strong and likely to become ill
    \textit{Her father is now 86 and in fragile health.}

    (British English, informal) \textit{I'm feeling a bit fragile after last night (= not well, perhaps because of drinking too much alcohol).}
\end{DefWord}

\begin{DefWord}{fragment}
    a small part of something that has broken off or comes from something larger
    \textbf{fragment (of something)} \textit{Police found fragments of glass near the scene.}

    \textbf{in fragments} \textit{The shattered vase lay in fragments on the floor.}

    a single part of something; a part that is not complete
    \textit{I overheard a fragment of their conversation. 我无意中听到他们谈话的片段}

    \textbf{fragment (something)} to break or make something break into small pieces or parts
    \textit{Frequent explosions caused the chalk to fragment.}
\end{DefWord}


\begin{DefWord}{fragmented}
    broken into small pieces or parts, in a way that may have a negative effect
    \textit{a fragmented society}

    \textbf{fragmented across something} \textit{The company's efforts were fragmented across multiple product lines and businesses.}

\end{DefWord}

\begin{DefWord}{fragmentation}
    \textbf{fragmentation (of something) (into something)} the act or process of breaking or making something break into small pieces or parts
    \textit{the fragmentation of the country into small independent states}
\end{DefWord}


\begin{DefWord}{fragmentary}
    made of small parts that are not connected or complete
    \textit{There is only fragmentary evidence to support this theory.}

\end{DefWord}




\section{fra}

\begin{DefWord}{frail}
    (especially of an old person) physically weak and thin
    \textit{Mother was becoming too frail to live alone.}

    weak; easily damaged or broken
    \textit{the frail stems of the flowers}
\end{DefWord}

\begin{DefWord}{frailty}
    weakness and poor health
    \textit{Increasing frailty meant that she was more and more confined to bed.}

    \textit{the \textbf{frailty of} her thin body}

    weakness in a person’s character or moral standards (weakness)
    \textit{\textbf{human frailties}}
\end{DefWord}

\section{frang}

\begin{DefWord}{frangible}
    easily broken; breakable:
\textit{Most frangible toys are not suitable for young children.}
\end{DefWord}

\section{fring}

\begin{DefWord}{infringe, infringement}[infringe]
    infringe something (of an action, a plan, etc.) to break a law or rule
    \textit{The material can be copied without infringing copyright.}

    to limit somebody’s legal rights
    \textbf{infringe something} They said that compulsory identity cards would infringe civil liberties.

    \textbf{infringe on/upon something} \textit{She refused to answer questions that infringed on her private affairs.}
\end{DefWord}



\section{fract}

\begin{DefWord}{infract, infraction}[infract]
    an act of breaking a rule or law (infringement)
    \textit{minor \textbf{infractions of} EU regulations}
\end{DefWord}

\begin{DefWord}{fraction}
    a small part or amount of something
    \textit{Only \textbf{a small fraction of} a bank's total deposits will be withdrawn at any one time.}

    \begin{remark}
        If \textbf{fraction} is used with a plural noun, the verb is usually plural: \textit{Only a fraction of cars in the UK \textbf{use} leaded petrol.} If it is used with a singular noun that represents a group of people, the verb can be singular or plural in British English, but is usually singular in North American English: \textit{A tiny fraction of the population never \textbf{vote/votes}.}
    \end{remark}

    a division of a number, for example $\frac{5}{8}$

    a quantity of liquid that has been collected as a result of a process that separates the parts of a liquid mixture
    \textit{The third fraction contains alchohols with boiling points of 120–130℃.}
\end{DefWord}



\begin{DefWord}{fractional}
    very small; not important (minimal)
    \textit{a fractional increase}

    of or in fractions
    \textit{a fractional equation}

    connected with the process in which a liquid mixture separates or is separated into its different parts
    \textit{fractional distillation 分馏}

\end{DefWord}

\begin{DefWord}{fractionalize}
    to break up into parts or sections
\end{DefWord}


\begin{DefWord}{fracture}
    a break in a bone or other hard material

    \textit{a fracture of the leg/skull 腿骨/颅骨骨折}

    \textit{\textbf{a compound/simple fracture} (= one in which the broken bone comes/does not come through the skin) 复合(开放)骨折;单纯(闭合)骨折}

    [uncountable] the fact of something breaking, especially a bone
    \textit{Old people's bones are more prone to fracture.}

    to break or crack; to make something break or crack
    \textit{His leg fractured in two places.}

    (of a society, an organization, etc.团体、组织等) to split into several parts so that it no longer functions or exists; to split a society or an organization, etc. in this way
    \textbf{fracture something (into something)} \textit{ The company was fractured into several smaller groups.}
\end{DefWord}

\begin{DefWord}{diffract, diffraction}[diffract]
    diffract: \textbf{diffract something} to break up a stream of light into a series of dark and light bands or into the different colours of the spectrum


    diffraction:
    the action or process of breaking up a stream of light into a series of dark or light bands or into the different colours of the spectrum(使光束)衍射
\end{DefWord}

\begin{DefWord}{refract, refraction}[refract]
    \textbf{refract something} (of water, air, glass, etc.水、空气、玻璃等) to make waves, such as those of light, sound or energy, change direction when they go through at an angle 使(光波、声波、能量波等)折射;使产生折射


\end{DefWord} 

\begin{DefWord}{refracory}
    difficult to control; behaving badly (unruly)

    (of a disease or medical condition) \textbf{a refractory disease} or illness is hard to treat or cure
\end{DefWord}

\begin{DefWord}{fractious, fractiousness}[fractious]
    easily upset, especially by small things (irritable)
    \textit{Children often get fractious and tearful when tired.}

    making trouble and complaining
    \textit{The six fractious republics are demanding autonomy.}
\end{DefWord}


\section{struct (to build)}

\begin{DefWord}{construct}
    to build or make something such as a road, building or machine
    
    \textbf{construct something}
    \textit{The building \textbf{was constructed} in 1993.}
    
    \textbf{construct something out of something}
    \textit{They \textbf{constructed a shelter out of} fallen branches.}

    \textbf{construct something from something}
    The frame is \textbf{constructed from} lightweight aluminium.

    \textbf{construct something of something}
    \textit{On the smaller islands, houses are often constructed of local materials.}

    construct something to form something by putting different things together
    \textit{a \textbf{well-constructed} novel}

    \textit{his carefully constructed public image 他精心构建的公众形象}

    \textit{the socially constructed nature of gender roles}

    construct something (geometry几何) to draw a line or shape according to the rules of mathematics


    an idea or a belief that is based on various pieces of evidence that have not always been proved to be true (根据不总是真实的各种证据得出的)构想,观念,概念
    \textit{a contrast between lived reality and the construct held in the mind 现实生活与头脑所持概念之间的明显差别}

    a group of words that form a phrase

    a thing that is built or made

    
\end{DefWord}

\begin{DefWord}{construction}
    the process or method of building or making something, especially roads, buildings, bridges, etc.
    \textit{the costs of \textbf{road construction} and maintenance}

    the people and activities involved in making buildings

    the way that something has been built or made
    \textit{The bridges are similar in construction.}
    \textit{It has a basic construction of brick under a tiled roof. 这是一座简易的瓦顶砖砌建筑。}

    a thing that has been built or made
    \textit{The summer house was a simple wooden construction.}

    the way in which words are used together and arranged to form a sentence, phrase, etc.
    \textit{grammatical constructions}

    the creating of something from ideas, opinions and knowledge
    \textit{the construction of a new theory}

    the way in which words, actions, statements, etc. are understood by somebody

    \textbf{put a construction on something} (formal) to think that a statement has a particular meaning or that something was done for a particular reason
    \textit{What \textbf{construction} do you \textbf{put on} this letter (= what do you think it means)?}
    \textit{The judge put an entirely different construction on his remarks.}
\end{DefWord}

\begin{DefWord}{constructive}
    having a useful and helpful effect rather than being negative or with no purpose建设性的;有助益的;积极的
constructive criticism/suggestions/advice
\end{DefWord}

\begin{DefWord}{substructure}
    a base or structure that is below another structure and that supports 
    \textit{a substructure of timber piles}
    
\end{DefWord}

\begin{DefWord}{superstructure}
    a structure that is built on top of the main part of something such as a ship or building

    a political or social system that has developed from a simpler system
    \textit{the whole superstructure of capitalism}

\end{DefWord}


\begin{DefWord}{infrastructure}
    the basic systems and services that are necessary for a country or an organization to run smoothly, for example buildings, transport and water and power supplies
\end{DefWord}

\begin{DefWord}{structural}
    connected with the way in which something is built or organized
    \textit{Storms have caused \textbf{structural damage} to hundreds of homes.}

    \textit{a structural survey (= an examination of a building to check for any damage to the walls, roof, etc.)}

    \textit{structural changes in society}
\end{DefWord}

\begin{DefWord}{structuralism}
    (in literature, language and social science 文学、语言及社会科学) a theory that considers any text as a structure whose various parts only have meaning when they are considered in relation to each other
\end{DefWord}

\begin{DefWord}{structured}
\end{DefWord}

\begin{DefWord}{obstruct, obstruction}[obstruct]
    obstruct something to block a road, an entrance, a passage, etc. so that somebody/something cannot get through, see past, etc.
    \textit{You can't park here, you're obstructing my driveway.}

    \textbf{obstruct somebody/something} to prevent somebody/something from doing something or making progress, especially when this is done deliberately
    \textit{They were charged with obstructing the police in the course of their duty.}

    \textbf{obstruct justice/pervert the course of justice} to tell a lie or to do something in order to prevent the police, etc. from finding out the truth about a crime
\end{DefWord}

\begin{DefWord}{obstruent}
    causing obstruction, esp of the intestinal tract

    anything that causes obstruction
\end{DefWord}

\begin{remark}
    The use of \textbf{obstruent} is rare.
\end{remark}


\begin{DefWord}{instruct, instruction, instructive, instrument, instrumental, instrumentalist, instrumentally, instructor}[instruct]
    instrumental: important in making something happen
    \textbf{instrumental in (doing) something} \textit{He was instrumental in bringing about an end to the conflict.}

    made by or for musical instruments
    \textit{instrumental music}

    in the form that a noun, pronoun or adjective has when it refers to a thing that is used to do something 工具格;工具词


\end{DefWord}




\section{stroy (the variant of struct)}

\begin{DefWord}{destruct, destruction, destructive}[destroy]
\end{DefWord}

\begin{DefWord}{self-destruct}
    to destroy itself, usually by exploding
\end{DefWord}

\begin{DefWord}{destroyer}
    a small fast ship used in war, for example to protect larger ships 驱逐舰

    a person or thing that destroys 破坏者;毁灭者
    \textit{Sugar is the destroyer of healthy teeth.}
\end{DefWord}

\begin{DefWord}{destructor}
    A \textbf{destructor} is a C++ member function that is invoked automatically when the object goes out of scope or is explicitly destroyed by a call to delete

    a device for destroying a missile or a part thereof at a desired time in its flight

    a furnace for the burning of refuse; incinerator. 焚烧垃圾的火炉;焚化炉
\end{DefWord}

\begin{DefWord}{refractory}
    deliberately not obeying someone in authority and being difficult to deal with or control难驾驭的﹐不服管教的 (SYN  unruly)

    (medical) a refractory disease or illness is hard to treat or cure
\end{DefWord}


\section{sta, stitu, sist (to stand)}

\begin{DefWord}{establish}
    \textbf{establish something} to start or create an organization, a system, etc. that is meant to last for a long time
    \textit{The committee was established in 1912.}

    \textbf{establish something} to start having a relationship, especially a formal one, with another person, group or country
    \textit{to establish relations/links/contacts/connections 建立关系/联系/接触/联系}
    \textit{The school is trying to establish a relationship with the local community.}

    to hold a position for long enough or succeed in something well enough to make people accept and respect you

    \textbf{establish somebody/something/yourself} \textit{He has now firmly established his position in the organization.}
    \textit{The school has \textbf{established a reputation} for academic excellence.}
    \textit{He has just set up his own business but it will take him a while to \textit{get established.}}

    \textbf{establish somebody/something/yourself in something} \textit{Not long after that she established herself in business.}

    to make people accept a belief, claim, custom etc.
    \textbf{establish something as something} \textit{The festival has become established as an annual event.}

    \textbf{establish something} \textit{This success helped to establish the practice of vaccination.}

    to discover or prove the facts of a situation
    \textbf{establish something} \textit{Police are still trying to establish the cause of the accident.}

    \textbf{establish that…} \textit{They have established that his injuries were caused by a fall.}
\end{DefWord}

\begin{DefWord}{institute, institution}
    \foreignlanguage{japanese}{
    institute を Oxford Advanced Learner's Dictionary では次のように定義しています:

    an organization that has a particular purpose, especially one that is connected with education or a particular profession; the building used by this organization

    一方、institution の定義は、

    a large important organization that has a particular purpose, for example, a university or bank

    institution が大学を指す場合はもちろんありますが、教育関係の組織名にinstitute という語を使うものが多いという事実(※たとえば有名なMITのように)が上の説明からも納得されますね。

    institution の方は単独で使うと、日本語の「施設」と同様に「介護施設」「養老院」「孤児院」など特別なケアが必要な人たちのための施設を意味する場合が多いことも知っておくべきでしょう。
    }

    \textbf{institute something} to introduce a system, policy, etc. or start a process
    \textit{The new management intends to institute a number of changes.}
\end{DefWord}

\begin{DefWord}{constitute}


    (not used in the progressive tenses 不用于进行时) to be considered to be something
    \textit{Does such an activity constitute a criminal offence? 难道这样的活动也算刑事犯罪吗?}

    (not used in the progressive tenses 不用于进行时) to be the parts that together form something
    \textit{Female workers constitute the majority of the labour force.}

    to form a group legally or officially
    \textbf{be constituted (by somebody/something)} \textit{The committee was constituted in 1974 by an Act of Parliament.}

\end{DefWord}

\begin{DefWord}{constitution}
    the system of laws and basic principles that a state, a country or an organization is governed by
    \textit{According to the constitution…}

    \begin{quotation}
        Britain is a constitutional monarchy: it is ruled by a king or queen who accepts the advice of Parliament. It is also a parliamentary democracy (= a country whose government is controlled by a parliament that has been elected by the people). The highest positions in government are taken by elected Members of Parliament, also called MPs. The king or queen now has little real power.

        The principles and procedures by which Britain is governed have developed over many centuries. They are not written down in a single document that can be referred to in an argument. The British Constitution is made up of statute law (= laws agreed by Parliament), common law (= decisions made by judges in court and then written down) and conventions (= rules and practices that people cannot be forced to obey but which are considered necessary for efficient government). The Constitution can be altered by Acts of Parliament, or by general agreement.

        Similarly, there is no single document that lists people's rights. Some rights have been recognized by Parliament through laws, for example, the right of a person not to be discriminated against (= treated differently) because of his or her sex. The Human Rights Act 1998 made all the rights established in the European Convention on Human Rights part of British law. It is generally understood that these rights are part of the Constitution.
    \end{quotation}

    \begin{quotation}
        The US Constitution was created after the American Revolution when leaders from each state held a meeting called the Constitutional Convention to agree on a document describing the new system of government and limiting its powers, which was signed in 1787, and started being used in 1789. This established the three branches of government: the legislative branch which is Congress, the judicial branch which is the Supreme Court and lower courts created by Congress, and the executive branch which consists of the president, vice-president and government departments. The Constitution contains details about the responsibilities of each branch and who can be elected to Congress. It says that the US government is responsible for protecting individual states. Since 1787 there have been 27 amendments (= changes) to the Constitution including the bill of rights (1791) which promised citizens a number of rights such as the right to free speech and freedom of religion. People sometimes disagree about how to interpret the Constitution, some people believing that it is better to follow exactly what the Constitution says and others that it is necessary to consider what the intention of each part was and how that relates to the situation today. The Supreme Court can decide that a law is unconstitutional so that it cannot be used any more.
    \end{quotation}
\end{DefWord}

\begin{DefWord}{consist}

    \textbf{consist of}
    to be formed from the people or things mentioned
    \textit{to \textbf{consist mainly/mostly} of somebody/something}

    \textbf{consist in something}
    to have something as the main or only part or feature
    \textit{The beauty of the city consists in its magnificent buildings. 这座城市的美就在于它那些宏伟的建筑。}

    \textbf{consist in doing something} \textit{True education does not consist in simply being taught facts. 真正的教育并不在于仅仅讲授事实。}
\end{DefWord}

\begin{DefWord}{resist} to refuse to accept something and try to stop it from happening 抵制;阻挡
    \textbf{resist something} \textit{They are determined to resist pressure to change the law.}

    to fight back when attacked; to use force to stop something from happening
    \textbf{resist something} \textit{She was charged with resisting arrest.}

    to stop yourself from having something you like or doing something you very much want to do
    \textit{I found the temptation to miss the class too \textbf{hard to resist.}}

    resist something to not be harmed or damaged by something
    \textit{A healthy diet should help your body resist infection.}
\end{DefWord}

\begin{DefWord}{persist, persistence, persistent}[persist]
    to continue to do something despite difficulties or opposition, in a way that can seem unreasonable

    \textbf{persist in doing something} \textit{Why do you persist in blaming yourself for what happened? 你何必为已发生的事没完没了地自责?}

    \textbf{persist in something} \textit{She persisted in her search for the truth. 她执着地追求真理。}

    \textbf{persist with something} \textit{He persisted with his questioning. 他问个不停。}

    \textbf{+ speech} \textit{‘So, did you agree or not?’ he persisted.“那么你同意了没有?” 他叮问道。}

    to continue to exist 维持;保持;持续存在
    \textit{The belief that the earth was flat persisted for many centuries. 地球是平的这一信念持续了许多世纪。}
\end{DefWord}

\begin{DefWord}{insist}
    to demand that something happen or that somebody agree to do something
    \textit{I didn't really want to go but he insisted.}

    to state clearly that something is true, especially when other people do not believe you
    \textbf{insist on something} \textit{He insisted on his innocence.}
\end{DefWord}

\begin{DefWord}{destination}
\end{DefWord}

\begin{DefWord}{destiny}
    what happens to somebody or what will happen to them in the future, especially things that they cannot change or avoid (fate)
    \textit{The contemporary hero is one who stands out against the crowd to fulfil a personal destiny.}

    the power believed to control events (fate)
\end{DefWord}

\begin{DefWord}{restitute, restitution}[restitution]
    \textbf{restitution (of something) (to somebody/something)} (formal) \textit{the act of giving back something that was lost or stolen to its owner}
    \textit{the restitution of property seized under Communist rule归还在共产主义统治下没收的财产}

    \textbf{restitution (of something) (to somebody/something)} (law 法律) payment, usually money, for some harm or wrong that somebody has suffered
    \textit{to make restitution for the damage caused 对造成的损害进行赔偿}

    \textbf{Restitute} means to make restitution—payment or some other form of compensation to make up for loss, damage, or injury that has been caused.

    The word \textbf{restitution} is much more commonly used than the verb \textbf{restitute}.
\end{DefWord}

\begin{DefWord}{contrast}
\end{DefWord}

\begin{DefWord}{destitute}
    without money, food and the other things necessary for life
    \textit{When he died, his family was left completely destitute.}

    \textbf{destitute of something} (formal) not having something
    \textit{They seem destitute of ordinary human feelings.}
\end{DefWord}

\begin{DefWord}{distance}
\textbf{at/from a distance}
from a place or time that is not near; from far away 离一段距离;从远处;遥远地;久远地
\textit{She had loved him at a distance for years. 她曾经暗恋他好多年。}

\textbf{go the (full) distance}
to continue playing in a competition or sports contest until the end (比赛)打完全场,赛足全局
\textit{Nobody thought he would last 15 rounds, but he went the full distance. 谁都以为他坚持不到 15 个回合,可是他却打完了全场。

\textbf{in/into the distance}
far away but still able to be seen or heard 在远处;在远方
\textit{We saw lights in the distance. 我们看到了远处的点点灯光。}

\textbf{keep somebody at a distance}
to refuse to be friendly with somebody; to not let somebody be friendly towards you 对…冷淡;同…疏远;与…保持一定距离
\textit{The manager prefers to keep employees at a distance. 经理喜欢与员工保持距离。}

\textbf{keep your distance (from somebody/something)}
to make sure you are not too near somebody/something (与…)保持距离
to avoid getting too friendly or involved with a person, group, etc. 疏远;避免(与…)亲近;避免介入
\textit{She was warned to keep her distance from Charles if she didn't want to get hurt. 有人告诫她说,如果不想受到伤害,就离查尔斯远一点。}

\textbf{within touching distance (of something)}
(British English also \textbf{within spitting distance})
(also \textbf{within shouting distance} especially in North American English)
(informal) very close很近
\textit{We came within touching distance of winning the cup.我们离赢得奖杯不远了。}
\end{DefWord}

\begin{DefWord}{circumstance}
\end{DefWord}

\begin{DefWord}{desist}
    \textbf{desist (from something/from doing something)} to stop doing something
    \textit{They agreed to desist from the bombing campaign.}
\end{DefWord}

\begin{DefWord}{subsist}
    subsist (on something) to manage to stay alive, especially with limited food or money
    \textit{Old people often subsist on very small incomes.}

    to exist; to apply and be relevant
    \textit{The terms of the contract subsist.}
\end{DefWord}

\begin{DefWord}{substitute, substitution}[substitute]
\end{DefWord}

\begin{DefWord}{constant}
\end{DefWord}

\begin{DefWord}{assist, assistance}[assist]
\end{DefWord}

