\chapter{Affliction, Wisdom, Trust and Persistence}

\section{flict (苦恼)}

\begin{DefWord}{afflict, affliction}[affliction]
    be afflicted with
\end{DefWord}

\begin{quotation}
    To be, or not to be: that is the question:

    Whether it’s nobler in the mind to suffer

    The slings and arrows of outrageous fortune,

    Or to take arms against a sea of troubles,

    And by opposing end them? To die: to sleep;

    No more; and by a sleep to say we end

    The heart-ache and the thousand natural shocks

    That flesh is heir to, 'tis a consummation

    Devoutly to be wish'd. To die, to sleep:

    To sleep: perchance to dream: ay, there's the rub;

    For in that sleep of death what dreams may come

    When we have shuffled off this mortal coil,

    `Must give us pause: there's the respect

    That makes calamity of so long life;

    For who would bear the whips and scorns of time,

    The oppressor's wrong, the proud man's contumely,

    The pangs of despised love, the law's delay,

    The insolence of office and the spurns

    That patient merit of the unworthy takes,

    When he himself might his quietus make

    With a bare bodkin? who would fardels bear,

    To grunt and sweat under a weary life,

    But that the dread of something after death,

    The undiscover'd country from whose bourn

    No traveller returns, puzzles the will

    And makes us rather bear those ills we have

    Than fly to others that we know not of?

    Thus conscience does make cowards of us all;

    And thus the native hue of resolution

    Is sicklied o'er with the pale cast of thought,

    And enterprises of great pitch and moment

    With this regard their currents turn awry,

    And lose the name of action.--Soft you now!

    The fair Ophelia! Nymph, in thy orisons

    Be all my sins remember'd".


    生存还是毁灭,这是一个值得考虑的问题;

    默然忍受命运的暴虐的毒箭,

    或是挺身反抗人世的无涯的苦难,

    通过斗争把它们扫清,

    这两种行为,哪一种更高贵?

    死了;睡着了;什么都完了;

    要是在这一种睡眠之中,我们心头的创痛,

    以及其他无数血肉之躯所不能避免的打击,都可以从此消失,

    那正是我们求之不得的结局。

    死了;睡着了;睡着了也许还会做梦;

    嗯,阻碍就在这儿:因为当我们摆脱了这一具朽腐的皮囊以后,

    在那死的睡眠里,究竟将要做些什么梦,那不能不使我们踌躇顾虑。

    人们甘心久困于患难之中,也就是为了这个缘故;

    谁愿意忍受人世的鞭挞和讥嘲、压迫者的凌辱、傲慢者的冷眼、被轻蔑的爱情的惨痛、法律的迁延、官吏的横暴和费尽辛勤所换来的小人的鄙视,

    要是他只要用一柄小小的刀子,就可以清算他自己的一生?

    谁愿意负着这样的重担,在烦劳的生命的压迫下呻吟流汗,

    倘不是因为害怕不可知的死后,害怕那从来不曾有一个旅人回来过的神秘之国,

    是它迷惑了我们的意志,使我们宁愿忍受目前的折磨,

    不敢向我们所不知道的痛苦飞去?

    这样,重重的顾虑使我们全变成了懦夫,

    决心的赤热的光彩,被审慎的思维盖上了一层灰色,

    伟大的事业在这一种考虑之下,

    也会逆流而退,失去了行动的意义。

    且慢!美丽的奥菲利娅!

    ——女神,在你'的祈祷之中,不要忘记替我忏悔我的罪孽。
\end{quotation}

\begin{DefWord}{inflict, infliction}
    \textit{The wife feels her husband is \textbf{inflicting} ridiculous ideas \textbf{on} her which will keep them in poverty/}

    \textit{inflict damage on sb}
\end{DefWord}

\begin{DefWord}{conflict}
    noun /ˈkɒnflɪkt/, verb /kənˈflɪkt/

    \textit{There is a \textbf{conflict} between his dream and his responsibility to the family.}

    \textit{This \textbf{sets up a conflict }between himself and his wife.}
\end{DefWord}


\section{agon}

\begin{DefWord}{agony}
    \textit{I was in agony.}

    \textit{ He was in agonies of remorse.}
\end{DefWord}

\begin{DefWord}{agony aunt, agony uncle, agony column}
\end{DefWord}
(British English)
(North American English advice columnist, advice column)

a person who writes in a newspaper or magazine or on a website giving advice in reply to people’s emails and letters about their personal problems

\begin{DefWord}{antagonize, antagonism}
    \textbf{antagonize somebody to do something} to make somebody angry with you

    \textit{Not wishing to antagonize her further, he said no more.}

    antagonism:
   \textbf{ antagonism (to/toward(s) somebody/something) | antagonism (between A and B)} (syn hostility) feelings of opposition and hate
   
   \textit{The antagonism he felt towards his old enemy was still very strong.}
\end{DefWord}


\section{ment (mind)}

\begin{DefWord}{mental}
\end{DefWord}

\begin{DefWord}{mentality}
    = mindset (a set of attitudes or fixed ideas that somebody has and that are often difficult to change)

    (plural mentalities)
    the particular attitude or way of thinking of a person or group

    \textit{mental hospital / mental institution 精神病院}

    \textit{mental deficiency (the state of not having, or not having enough of, something that is essential)}

    \textit{of weak mentality (弱智者)}

    \textit{a get-rich-quick mentality 一夜暴富的心态}

    \textit{criminal mentality 犯罪心理学}

    \textit{mental age}

    \textbf{a mental block 对……有心理障碍}
    \textit{Personally I've got a mental block about numbers.}

    \textit{to make a mental note of 记在心里}


\end{DefWord}

\begin{DefWord}{comment}

    a fair comment
\end{DefWord}



\begin{RefWord}{commentary, commentator}
    \textbf{commentary (on something)} a spoken description of an event that is given while it is happening, especially on the radio or television

    \textbf{a running commentary 实况报道}
    \textit{He kept up \textbf{a running commentary} on everyone who came in or went out.}
\end{RefWord}





\section{soph (wisdom)}

\begin{DefWord}{sophomore}
    a student in the second year of a course of study at a college or university
\end{DefWord}

\begin{DefWord}{sophist}
    a teacher of philosophy in ancient Greece, especially one with an attitude of doubting that statements are true(古希腊的)哲学教师;(尤指怀质疑态度的)哲人,智者

a person who uses clever but wrong arguments诡辩者;诡辩家
\end{DefWord}

\begin{DefWord}{sophisticated}
    (of a machine, system, etc.) clever and complicated in the way that it works or is presented 复杂巧妙的;先进的;精密的
    \textit{Medical techniques are becoming more sophisticated all the time.}


    having a lot of experience of the world and knowing about fashion, culture and other things that people think are socially important 见多识广的;老练的;见过世面的
    \textit{Mark is a smart and sophisticated young man.}

    able to understand difficult or complicated ideas
    \textit{a sophisticated audience}
\end{DefWord}

\begin{DefWord}{philosopher, philosophy}[philosophy]

    philosopher:
    a person who studies or writes about philosophy
    \textit{We studied the writings of the Greek philosopher Aristotle.}
    
    a person who thinks deeply about things
    \textit{Many of his followers regarded him as a true philosopher. 他的许多追随者认为他是一个真正的哲学家。}
\end{DefWord}


\section{fid (trust)}

\begin{quotation}
    This left Chris profoundly lonely. He had no one to talk to about his situation. He \textbf{had no one to confide in}. He did not have an adviser, a big brother or a \textbf{confidant}. Apart from his wife, he never had a confidant in which to share his confidential information about his private life. There was no one he could speak to \textbf{in confidence}.
\end{quotation}

\begin{DefWord}{confident, confidence}[confident]
    \textit{have every confidence in sb}

    in confidence = in secret
\end{DefWord}

\begin{DefWord}{confide}
    to tell somebody secrets and personal information that you do not want other people to know(向某人)吐露(秘密、隐私等)
    \textbf{ confide something (to somebody)} \textit{She confided all her secrets to her best friend. 她向她最要好的朋友倾吐了自己所有的秘密。}
    \textbf{confide (to somebody) that…} \textit{He confided to me that he had applied for another job. 他向我透露他已申请另一份工作。}
 + speech ‘It was a lie,’ he confided.“
\end{DefWord}

\begin{DefWord}{confidant}
    a person that you trust and who you talk to about private or secret things(可吐露秘密的)知己,密友
    \textit{a close/trusted confidant of the president 总统的密友/亲信}
    \textit{There were times when a semi-stranger was a better confidante than a close friend. 有时候,一个半陌生的人比一个亲密的朋友更能成为知己。}
\end{DefWord}

\begin{DefWord}{confidential}
\end{DefWord}

\begin{DefWord}{diffident, diffidence, diffidently}[diffident]
    \textbf{diffident (about something)} not having much confidence in yourself; not wanting to talk about yourself
    \textbf{a diffident manner/smile} 畏首畏尾的态度;羞怯的一笑
    \textit{He was modest and diffident about his own success.}
\end{DefWord}

\begin{DefWord}{fidelity}
    /fɪˈdeləti/

    \textbf{fidelity (to something) (formal)} the quality of being loyal to somebody/something 忠诚;忠实;忠贞
    \textit{fidelity to your principles}

    \textbf{fidelity (to somebody)} the quality of being faithful to your husband, wife or partner by not having a sexual relationship with anyone else
    \textit{\textbf{marital/sexual fidelity}}

    \textbf{fidelity (of something) (to something)} (formal) the quality of being accurate
    \textit{The story is told with great fidelity to the original.}
\end{DefWord}

\section{cred (trust)}

\begin{DefWord}{creed}
    a set of principles or religious beliefs 信念;原则;纲领;宗教信仰
    \textit{We welcome people of all races, colours and creeds.}

    \textbf{the Creed} [singular] a statement of Christian belief that is spoken as part of some church services (基督教)信经
\end{DefWord}

\begin{DefWord}{credit}
    an arrangement that you make, with a shop for example, to pay later for something you buy
    \textbf{on credit} \textit{We bought the dishwasher on credit.}

    \textit{Your \textbf{credit limit} is now £2 000.}
    \textit{He's a bad credit risk (= he is unlikely to pay the money later).}

    praise or approval because you are responsible for something good that has happened
    \textit{to get/deserve/receive/take/claim the credit}
\end{DefWord}

\begin{DefWord}{credence}
    the acceptance of something as true
    \textit{Historical evidence \textbf{lends credence to} his theory. 史学的根据使他的理论更为可信。}

    \textit{They could \textbf{give no credence} to the findings of the survey.}

    \textit{Alternative medicine has been \textbf{gaining credence} (= becoming more widely accepted) recently。}
\end{DefWord}

\begin{DefWord}{creditable}
    of a quite good standard and deserving praise or approval
    \textit{The team produced a \textbf{creditable performance}.}

    \textit{She did \textbf{a creditable job} of impersonating the singer.}
\end{DefWord}

\begin{DefWord}{credulous}
    always believing what you are told, and therefore easily deceived (syn gullible)
    \textit{Quinn charmed credulous investors out of millions of dollars.}
\end{DefWord}

\begin{DefWord}{incredible}
    impossible or very difficult to believe
\end{DefWord}

\begin{DefWord}{accredit}
    to believe that somebody is responsible for doing or saying something
    \textbf{be accredited to somebody} \textit{The discovery of distillation is usually accredited to the Arabs of the 11th century.}

    \textbf{be accredited with something} \textit{The Arabs are usually accredited with the discovery of distillation.}
\end{DefWord}

\begin{DefWord}{discredit}
    \textbf{discredit somebody/something} to make people stop respecting somebody/something
    \textit{The photos were deliberately taken to discredit the president.}
\end{DefWord}








\section{firm (persistence)}

\begin{DefWord}{firm}
    fairly hard; not easy to press into a different shape


    not likely to change 坚定的;确定的;坚决的
    \textit{\textbf{a firm believer} in socialism 坚定信仰社会主义的人}
    \textit{\textbf{a firm agreement/date/decision/offer/promise} 巩固的协议;确定的日期;不能更改的决定;实盘;坚决的保证}
    \textit{\textbf{firm beliefs/conclusions/convictions/principles} 坚定不移的信仰;定论;坚定的信念/原则}
    \textit{She is a \textbf{firm favourite with} the children. 孩子们着实喜欢她。}
    \textit{We have no \textbf{firm evidence} to support the case. 我们没有确凿的证据支持这个论点。}
    \textit{They remained \textbf{firm friends}.}
\end{DefWord}

\begin{DefWord}{infirm}
    ill and weak, especially over a long period or as a result of being old
    \textit{Father was becoming increasingly infirm.}

    \textbf{the infirm} noun [plural] people who are weak and ill for a long period
\end{DefWord}

\begin{DefWord}{infirmary}
    (often used in names) a hospital
    \textit{Surgeons at the Radcliffe Infirmary successfully removed a blood clot from her brain.}

    a special room in a school, prison, etc. for people who are ill
    \textit{He had to remain in the college infirmary for about a fortnight.}
\end{DefWord}

\begin{DefWord}{affirm, affirmation, affirmative}[affirm]
    to state clearly or publicly that something is true or that you support something strongly
    \textbf{affirm something} \textit{Both sides affirmed their commitment to the ceasefire. 双方均申明同意停火.}
    \textbf{affirm that…} \textit{I can affirm that no one will lose their job. 我可以肯定,谁都不会丢掉工作。}

    affirmation: a definite or public statement that something is true or that you support something strongly
    \textit{She nodded in affirmation.}

    an \textbf{affirmative} word or reply means ‘yes’ or expresses agreement
    \textit{an affirmative response to the question}

    expressing something that is true, did happen, etc.; not containing words such as ‘no’, ‘not’, ‘never’, etc.
    \textit{affirmative and negative forms/sentences}
\end{DefWord}

\begin{DefWord}{confirm, confirmation, confirmed}[confirm]
    confirm: to state or show that something is definitely true or correct, especially by providing evidence

    \textbf{confirm something} \textit{His guilty expression confirmed my suspicions. 他内疚的表情证实了我的猜疑。}
    \textit{to \textbf{confirm a diagnosis/report} 确认诊断/报告}
    \textit{to \textbf{confirm results/findings} 确认结果/发现}
    \textit{Rumours of job losses were later confirmed.}
    \textit{We have yet to confirm the identities of the victims.}

    \textit{She said she could not confirm or deny the allegations. 她说她无法证实或否认这些指控。}

    \textit{The authorities refused to confirm any details. 当局拒绝证实任何细节。}

    \textbf{confirm (that)… } \textit{Police sources confirmed that ten people had been arrested at the march.}
    
    \textbf{confirm to somebody/something that…} \textit{A government official confirmed to the newspaper that Britain was pushing hard for an end to the arms embargo. 一名政府官员向该报证实,英国正在努力争取结束武器禁运。}

    \textbf{confirm what/when, etc…} \textit{Can you confirm what happened?}

    \textbf{it is confirmed that…} 
    \textit{It has been confirmed that an official complaint was made to the council.已经证实有人向委员会提出了正式投诉。}

    to make a position, an agreement, etc. more definite or official; to establish somebody/something clearly 批准(职位、协议等);确认;认可
    \textbf{confirm something} \textit{Please write to confirm your reservation (= say that it is definite). 预订后请来函确认。}
    \textit{After a six-month probationary period, her position was confirmed. 经过六个月的试用期后,她获准正式担任该职。}

    \textbf{confirm (that)…} \textit{Has everyone confirmed (that) they’re coming? 他们是不是每个人都确定了一定会来?}

    \textbf{it is confirmed that…} \textit{It has been confirmed that the meeting will take place next week. 已经确定会议将于下个星期举行。}

    \textbf{confirm somebody as something} \textit{He was confirmed as captain for the rest of the season. 他被正式任命在这个赛季剩下的时间内担任队长。}

    \textbf{confirm somebody in something} \textit{I'm very happy to confirm you in your post. 我很高兴确认你担任此职位。}

    to make somebody feel or believe something even more strongly 使感觉更强烈;使确信
    \textbf{confirm something} \textit{The walk in the mountains confirmed his fear of heights. 在山里步行使他更加确信自己有恐高症。}

    \textbf{confirm somebody in something} \textit{This latest tragedy \textbf{merely} confirms my view that the law must be tightened. 最近发生的悲剧\textbf{正好}印证了我的看法:法律必须趋严。}

    [usually passive] to perform the Christian or Jewish ceremony of confirmation (给某人)施放坚振,施坚信礼
    \textbf{be confirmed} \textit{ She was baptized when she was a month old and confirmed when she was thirteen.她出生一个月时受洗礼,十三岁时受坚信礼。}
        
    confirmed: having a particular habit or way of life and not likely to change 成习惯的;根深蒂固的
    \textit{a \textbf{confirmed bachelor} (= a man who is not likely to get married, sometimes used in newspapers to refer to a gay man) 抱定独身主义的男子(报章常用以指同性恋者)}
    \textit{This chocolate dessert is the ultimate dish for confirmed chocaholics. 这种巧克力甜点是公认的巧克力爱好者的终极菜肴。}


\end{DefWord}

\section{dur (persistence)}

\begin{DefWord}{duration}
\end{DefWord}

\begin{DefWord}{durable}
\end{DefWord}


\begin{DefWord}{endure, endurance, enduring}[endure]
    to experience and deal with something that is painful or unpleasant without giving up
\end{DefWord}
