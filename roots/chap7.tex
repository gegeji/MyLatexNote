\chapter{Energy, Being Alive and Life}

\section{vie, vit, vig, viv (life)}

\begin{DefWord}{vie}
    C'est la vie. (This is life.)
\end{DefWord}

\begin{DefWord}{vitamin}
\end{DefWord}

\begin{DefWord}{vital, vitality}[vital]
    extremely important and necessary for something to succeed or exist 极其重要的﹐必不可少的 (SYN  crucial)
    \textit{The samples could give scientists vital information about long-term changes in the Earth’s atmosphere.}

    \textit{It is vital that you keep accurate records.}

    full of energy in a way that is exciting and attractive
    \textit{Rodgers and Hart’s music sounds as fresh and vital as the day it was written.}

    vitality:/vaɪˈtæləti/
    great energy and eagerness to do things
    \textit{Despite her eighty years, Elsie was full of vitality.}

    the strength and ability of an organization, country etc to continue
    \textit{The process of restructuring has injected some much-needed vitality into the company.}
\end{DefWord}

\begin{DefWord}{survive}
\end{DefWord}

\begin{DefWord}{vivid}
\end{DefWord}

\begin{DefWord}{vivacious}
    /vəˈveɪʃəs/

    someone, especially a woman, who is vivacious has a lot of energy and a happy attractive manner – used to show approval (syn lively)
    \textit{a vivacious personality}
\end{DefWord}

\begin{DefWord}{revive}
    /rɪˈvaɪv/

    to bring something back after it has not been used or has not existed for a period of time
    \textit{Local people have decided to revive this centuries-old tradition.}

    to become healthy and strong again, or to make someone or something healthy and strong again
    \textit{The economy is beginning to revive.}

    to produce a play again after it has not been performed for a long time
    \textit{A London theatre has decided to revive the 1950s musical ‘In Town’. 伦敦的一家剧院决定重新排演20世纪50年代的音乐剧《在镇上》。}
\end{DefWord}

\begin{DefWord}{vigor, vigorous}[vigor]
    physical or mental energy and determination 活力﹐精力﹔热情
    \textbf{with vigour}
    \textit{He began working with renewed vigour.}

    vigorous: using a lot of energy and strength or determination
    \textit{Your dog needs at least 20 minutes of \textbf{vigorous exercise} every day.}
    \textit{Environmentalists have begun a vigorous campaign to oppose nuclear dumping in the area.}

    strong and healthy 强健的,精力旺盛的:
    \textit{a vigorous young man}
\end{DefWord}

\begin{DefWord}{invigorate}
    if something invigorates you, it makes you feel healthier, stronger, and have more energy
    \textit{At my age, the walk into town is enough to invigorate me.}

    to make the people in an organization or group feel excited again, so that they want to make something successful
    \textit{Carey’s hope was that the church would be renewed and invigorated.}
\end{DefWord}

\section{bio}

\begin{DefWord}{biography, biographer}[biography]
    a book that tells what has happened in someone’s life, written by someone else
\end{DefWord}

\begin{DefWord}{autobiography, autobiographical, autobiographer}[autobiography]
    autobiographical /ˌɔːtəbaɪəˈɡræfɪkəl ˌɒː-/ adjective:
    an autobiographical novel (=one based on the author’s own experiences) 自传体小说
\end{DefWord}

\begin{DefWord}{biocide}
    n. [微] 灭微生物剂;生物性农药(biocide的复数)
\end{DefWord}

\begin{DefWord}{biochemical}
\end{DefWord}

\begin{DefWord}{biodiversity}
\end{DefWord}

\begin{DefWord}{antibiotics}
    a drug that is used to kill bacteria and cure infections 抗生素
\end{DefWord}

\section{anim (soul and breath)}

\begin{DefWord}{animate}
    anim (soul) + ate (give) = give sth a soul

    to give life or energy to something 赋予…生命﹔使有生气[活力]:
    \textit{Laughter animated his face for a moment.}

    living 有生命的﹐活的 OPP  inanimate
    animate beings 生物 

    \textbf{animate something} to make models, toys, etc. seem to move in a film/movie by rapidly showing slightly different pictures of them in a series, one after another
\end{DefWord}

\begin{DefWord}{animal}
\end{DefWord}

\begin{DefWord}{animism}
    泛灵论,又称万物有灵论,是一种认为天地万物──动物、植物、环境、天气,乃至言词、画像、建筑或其他人工产物──都是有灵魂、能够思考和获取经验的主体,并且能够操纵或影响其他自然现象乃至人类社会的世界观。它亦是目前已知最古老的信仰系统,在世界各地的传统文化中都能找到其踪影。
\end{DefWord}

\begin{DefWord}{unanimous}
    a unanimous decision, vote, agreement etc is one in which all the people involved agree
    \textit{It was decided by a unanimous vote that the school should close.}

    agreeing completely about something
    \textit{The banks were unanimous in welcoming the news.}
\end{DefWord}

\begin{DefWord}{magnanimous}
    /mæɡˈnænɪməs/

    kind and generous, especially to someone that you have defeated
    \textit{a magnanimous gesture 宽宏大量的姿态}
\end{DefWord}

\begin{DefWord}{pusillanimous}
    /ˌpjuːsɪˈlænɪməs/
    
    frightened of taking even small risks
\end{DefWord}

[VBLTimestamp StimulusOnsetTime FlipTimestamp Missed Beampos] = Screen('Flip', windowPtr [, when] [, dontclear] [, dontsync] [, multiflip]);

Flip front and back display surfaces in sync with vertical retrace and return
completion timestamps.
"windowPtr" is the id of the onscreen window whose content should be shown at
flip time. "when" specifies when to flip: If set to zero (default), it will flip
on the next possible video retrace. If set to a value when > 0, it will flip at
the first video retrace after system time 'when' has been reached. "dontclear"
If set to 1, flip will not clear the framebuffer after Flip - this allows
incremental drawing of stimuli. The default is zero, which will clear the
framebuffer to background color after each flip. A value of 2 will prevent Flip
from doing anything to the framebuffer after flip. This leaves the job of
setting up the buffer to you - the framebuffer is in an undefined state after
flip. "dontsync" If set to zero (default), Flip will sync to the vertical
retrace and will pause execution of your script until the Flip has happened. If
set to 1, Flip will still synchronize stimulus onset to the vertical retrace,
but will *not* wait for the flip to happen: Flip returns immediately and all
returned timestamps are invalid. A value of 2 will cause Flip to show the
stimulus *immediately* without waiting/syncing to the vertical retrace.
"multiflip" defaults to zero: If set to a value greater than zero, Flip will
flip *all* onscreen windows instead of just the specified one. This allows to
synchronize stimulus onset on multiple displays, e.g., for multidisplay stereo
setups or haploscopes. You need to (somehow) synchronize all attached displays
for this to operate tear-free. Flip (optionally) returns a high-precision
estimate of the system time (in seconds) when the actual flip has happened in
the return argument 'VBLTimestamp'. An estimate of Stimulus-onset time is
returned in 'StimulusOnsetTime'. Beampos is the position of the monitor scanning
beam when the time measurement was taken (useful for correctness tests).
FlipTimestamp is a timestamp taken at the end of Flip's execution. Use the
difference between FlipTimestamp and VBLTimestamp to get an estimate of how long
Flips execution takes. This is useful to get a feeling for the timing error if
you try to sync script execution to the retrace, e.g., for triggering
acquisition devices like EEG, fMRI, or for starting playback of a sound.
"Missed" indicates if the requested presentation deadline for your stimulus has
been missed. A negative value means that dead- lines have been satisfied.
Positive values indicate a deadline-miss. The automatic detection of
deadline-miss is not fool-proof - it can report false positives and also false
negatives, although it should work fairly well with most experimental setups. If
you are picky about timing, please use the provided timestamps or additional
methods to exercise your own tests. 
